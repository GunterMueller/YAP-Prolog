�a\input texinfo @c -*- mode: texinfo; coding: latin-1; -*-

@c %**start of header
@setfilename yap.info
@setcontentsaftertitlepage
@settitle YAP Prolog User's Manual
@c For double-sided printing, uncomment:
@c @setchapternewpage odd
@c %**end of header

@set VERSION 6.0.7
@set EDITION 4.2.8
@set UPDATED Aug 2010

@c Index for C-Prolog compatible predicate
@defindex cy
@c Index for predicates not in C-Prolog
@defindex cn
@c Index for predicates sort of (almost) in C-Prolog
@defindex ca

@c Index for SICStus Prolog compatible predicate
@defindex sy
@c Index for predicates not in SICStus Prolog
@defindex sn
@c Index for predicates sort of (almost) in SICStus Prolog
@defindex sa


@setchapternewpage odd
@c @smallbook
@comment %** end of header

@ifnottex
@format
@dircategory The YAP Prolog System
@direntry
* YAP: (yap).           YAP Prolog User's Manual.
@end direntry
@end format
@end ifnottex

@titlepage
@title YAP User's Manual
@subtitle Version @value{VERSION}
@author V@'{@dotless{i}}tor Santos Costa,
@author Lu@'{@dotless{i}}s Damas,
@author Rog@'erio Reis, and
@author R@'uben Azevedo
@page
@vskip 2pc
Copyright @copyright{} 1989-2000 L. Damas, V. Santos Costa and Universidade
do Porto.

Permission is granted to make and distribute verbatim copies of
this manual provided the copyright notice and this permission notice
are preserved on all copies.

Permission is granted to copy and distribute modified versions of this
manual under the conditions for verbatim copying, provided that the entire
resulting derived work is distributed under the terms of a permission
notice identical to this one.

Permission is granted to copy and distribute translations of this manual
into another language, under the above conditions for modified versions.

@end titlepage

@ifnottex
@node Top, , , (dir)
@top YAP Prolog

This file documents the YAP Prolog System version @value{VERSION}, a
high-performance Prolog compiler developed at LIACC, Universidade do
Porto. YAP is based on David H. D. Warren's WAM (Warren Abstract
Machine), with several optimizations for better performance. YAP follows
the Edinburgh tradition, and is largely compatible with DEC-10 Prolog,
Quintus Prolog, and especially with C-Prolog.

This file contains extracts of the SWI-Prolog manual, as written by Jan
Wielemaker. Our thanks to the author for his kind permission in allowing
us to include his text in this document.

@menu
* Intro:: Introduction
* Install:: Installation
* Run:: Running YAP
* Syntax:: The syntax of YAP
* Loading Programs:: Loading Prolog programs
* Modules:: Using Modules in YAP
* Built-ins:: Built In Predicates
* Library:: Library Predicates
* SWI-Prolog:: SWI-Prolog emulation
* Global Variables ::  Global Variables for Prolog
* Extensions:: Extensions to Standard YAP
* Rational Trees:: Working with Rational Trees
* Co-routining:: Changing the Execution of Goals
* Attributed Variables:: Using attributed Variables
* CLPR:: The CLP(R) System
* CHR:: The CHR System
* Logtalk:: The Logtalk Object-Oriented System
* MYDDAS:: The YAP Database Interface
* Threads:: Thread Library
* Parallelism:: Running in Or-Parallel
* Tabling:: Storing Intermediate Solutions of programs 
* Low Level Profiling:: Profiling Abstract Machine Instructions
* Low Level Tracing:: Tracing at Abstract Machine Level
* Debugging:: Using the Debugger
* Efficiency:: Efficiency Considerations
* C-Interface:: Interfacing predicates written in C
* YAPLibrary:: Using YAP as a library in other programs
* Compatibility:: Compatibility with other Prolog systems
* Predicate Index:: An item for each predicate
* Concept Index:: An item for each concept

Built In Predicates
* Control:: Controlling the execution of Prolog programs
* Undefined Procedures:: Handling calls to Undefined Procedures
* Messages:: Message Handling in YAP
* Testing Terms:: Predicates on Terms
* Predicates on Atoms:: Manipulating Atoms
* Predicates on Characters:: Manipulating Characters
* Comparing Terms:: Comparison of Terms
* Arithmetic:: Arithmetic in YAP
* I/O:: Input/Output with YAP
* Database:: Modifying Prolog's Database
* Sets:: Finding All Possible Solutions
* Grammars:: Grammar Rules
* Preds:: Predicate Information
* OS:: Access to Operating System Functionality
* Term Modification:: Updating Prolog Terms
* Global Variables:: Manipulating Global Variables
* Profiling:: Profiling Prolog Execution
* Call Counting:: Limiting the Maximum Number of Reductions
* Arrays:: Supporting Global and Local Arrays
* Preds:: Information on Predicates
* Misc:: Miscellaneous Predicates


Subnodes of Running
* Running YAP Interactively:: Interacting with YAP
* Running Prolog Files:: Running Prolog files as scripts

Subnodes of Syntax
* Formal Syntax:: Syntax of Terms
* Tokens:: Syntax of Prolog tokens
* Encoding:: How characters are encoded and Wide Character Support

Subnodes of Tokens
* Numbers:: Integer and Floating-Point Numbers
* Strings:: Sequences of Characters
* Atoms:: Atomic Constants
* Variables:: Logical Variables
* Punctuation Tokens:: Tokens that separate other tokens
* Layout:: Comments and Other Layout Rules

Subnodes of Numbers
* Integers:: How Integers are read and represented
* Floats:: Floating Point Numbers

Subnodes of Encoding
* Stream Encoding:: How Prolog Streams can be coded
* BOM:: The Byte Order Mark

Subnodes of Loading Programs
* Compiling:: Program Loading and Updating
* Setting the Compiler:: Changing the compiler's parameters
* Conditional Compilation:: Compiling program fragments
* Saving:: Saving and Restoring Programs

Subnodes of Modules
* Module Concepts:: The Key Ideas in Modules
* Defining Modules:: How To Define a New Module
* Using Modules:: How to Use a Module
* Meta-Predicates in Modules:: How to Handle New Meta-Predicates
* Re-Exporting Modules:: How to Re-export Predicates From Other Modules

Subnodes of Input/Output
* Streams and Files:: Handling Streams and Files
* C-Prolog File Handling:: C-Prolog Compatible File Handling
* I/O of Terms:: Input/Output of terms
* I/O of Characters:: Input/Output of Characters
* I/O for Streams:: Input/Output using Streams
* C-Prolog to Terminal:: C-Prolog compatible Character I/O to terminal
* I/O Control:: Controlling your Input/Output
* Sockets:: Using Sockets from YAP

Subnodes of Database
* Modifying the Database:: Asserting and Retracting
* Looking at the Database:: Finding out what is in the Data Base
* Database References:: Using Data Base References
* Internal Database:: YAP's Internal Database
* BlackBoard:: Storing and Fetching Terms in the BlackBoard

Subnodes of Library
* Aggregate :: SWI and SICStus compatible aggregate library
* Apply:: SWI-Compatible Apply library.
* Association Lists:: Binary Tree Implementation of Association Lists.
* AVL Trees:: Predicates to add and lookup balanced binary  trees.
* Heaps:: Labelled binary tree where the key of each node is less
    than or equal to the keys of its children.
* Lambda:: Ulrich Neumerkel's Lambda Library
* LineUtilities:: Line Manipulation Utilities
* Lists:: List Manipulation
* MapList:: SWI-Compatible Apply library.
* matrix:: Matrix Objects
* MATLAB:: Matlab Interface
* Non-Backtrackable Data Structures:: Queues, Heaps, and Beams.
* Ordered Sets:: Ordered Set Manipulation
* Pseudo Random:: Pseudo Random Numbers
* Queues:: Queue Manipulation
* Random:: Random Numbers
* Read Utilities:: SWI inspired utilities for fast stream scanning.
* Red-Black Trees:: Predicates to add, lookup and delete in red-black binary  trees.
* RegExp:: Regular Expression Manipulation
* shlib:: SWI Prolog shlib library
* Splay Trees:: Splay Trees
* String I/O:: Writing To and Reading From Strings
* System:: System Utilities
* Terms:: Utilities on Terms
* Cleanup:: Call With registered Cleanup Calls
* Timeout:: Call With Timeout
* Trees:: Updatable Binary Trees
* Tries:: Trie Data Structure
* UGraphs:: Unweighted Graphs
* DGraphs:: Directed Graphs Implemented With Red-Black Trees
* UnDGraphs:: Undirected Graphs Using DGraphs
* LAM:: LAM MPI


Subnodes of Debugging
* Deb Preds:: Debugging Predicates
* Deb Interaction:: Interacting with the debugger

Subnodes of Compatibility
* C-Prolog:: Compatibility with the C-Prolog interpreter
* SICStus Prolog:: Compatibility with the Quintus and SICStus Prolog systems
* ISO Prolog::  Compatibility with the ISO Prolog standard

Subnodes of Attributes
* Attribute Declarations:: Declaring New Attributes
* Attribute Manipulation:: Setting and Reading Attributes
* Attributed Unification:: Tuning the Unification Algorithm
* Displaying Attributes:: Displaying Attributes in User-Readable Form
* Projecting Attributes:: Obtaining the Attributes of Interest
* Attribute Examples:: Two Simple Examples of how to use Attributes.

Subnodes of SWI-Prolog
* Invoking Predicates on all Members of a List :: maplist and friends
* SWI-Prolog Global Variables :: Emulating SWI-like attributed variables

@c Subnodes of CLP(Q,R)
@c * Introduction to CLPQ:: The CLP(Q,R) System
@c * Referencing CLPQR:: How to Reference CLP(Q,R)
@c * CLPQR Acknowledgments:: Acknowledgments for CLP(Q,R)
@c * Solver Interface:: Using the CLP(Q,R) System
@c * Notational Conventions:: The CLP(Q,R) Notation
@c * Solver Predicates:: The CLP(Q,R) Interface Predicates
@c * Unification:: Unification and CLP(Q,R)
@c * Feedback and Bindings:: Information flow in CLP(Q,R)
@c * Linearity and Nonlinear Residues:: Linear and Nonlinear Constraints
@c * How Nonlinear Residues are made to disappear:: Handling Nonlinear Residues
@c * Isolation Axioms:: Isolating the Variable to be Solved
@c * Numerical Precision and Rationals:: Reals and Rationals
@c * Projection and Redundancy Elimination:: Presenting Bindings for Query Variables
@c * Variable Ordering:: Linear Relationships between Variables
@c * Turning Answers into Terms:: using @code{call_residue/2}
@c * Projecting Inequalities:: How to project linear inequations
@c * Why Disequations:: Using Disequations in CLP(Q,R)
@c * Syntactic Sugar:: An easier syntax
@c * Monash Examples:: The Monash Library
@c * Compatibility Notes:: CLP(Q,R) and the clp(R) interpreter
@c * A Mixed Integer Linear Optimization Example:: MIP models
@c * Implementation Architecture:: CLP(Q,R) Components
@c * Fragments and Bits:: Final Last Words on CLP(Q,R)
@c * CLPQR Bugs:: Bugs in CLP(Q,R)
@c * CLPQR References:: References for CLP(Q,R)

Subnodes of CLPR
* CLPR Solver Predicates::
* CLPR Syntax::
* CLPR Unification::
* CLPR Non-linear Constraints::               

Subnodes of CHR
* CHR Introduction::            
* CHR Syntax and Semantics::
* CHR in YAP Programs::
* CHR Debugging::               
* CHR Examples::       
* CHR Compatibility::     
* CHR Guidelines::  

Subnodes of C-Interface
* Manipulating Terms:: Primitives available to the C programmer
* Manipulating Terms:: Primitives available to the C programmer
* Unifying Terms:: How to Unify Two Prolog Terms
* Manipulating Strings:: From character arrays to Lists of codes and back
* Memory Allocation:: Stealing Memory From YAP
* Controlling Streams:: Control How YAP sees Streams
* Utility Functions:: From character arrays to Lists of codes and back
* Calling YAP From C:: From C to YAP to C to YAP 
* Module Manipulation in C:: Create and Test Modules from within C
* Miscellaneous C-Functions:: Other Helpful Interface Functions
* Writing C:: Writing Predicates in C
* Loading Objects:: Loading Object Files
* Save&Rest:: Saving and Restoring
* YAP4 Notes:: Changes in Foreign Predicates Interface

Subnodes of C-Prolog
* Major Differences with C-Prolog:: Major Differences between YAP and C-Prolog
* Fully C-Prolog Compatible:: YAP predicates fully compatible with
C-Prolog
* Not Strictly C-Prolog Compatible:: YAP predicates not strictly as C-Prolog
* Not in C-Prolog:: YAP predicates not available in C-Prolog
* Not in YAP:: C-Prolog predicates not available in YAP

Subnodes of SICStus Prolog
* Major Differences with SICStus:: Major Differences between YAP and SICStus Prolog
* Fully SICStus Compatible:: YAP predicates fully compatible with
SICStus Prolog
* Not Strictly SICStus Compatible:: YAP predicates not strictly as
SICStus Prolog
* Not in SICStus Prolog:: YAP predicates not available in SICStus Prolog


Tables
* Operators:: Predefined operators

@end menu

@end ifnottex


@node Intro, Install, , Top
@unnumbered Introduction

This document provides User information on version @value{VERSION} of
YAP (@emph{Yet Another Prolog}). The YAP Prolog System is a
high-performance Prolog compiler developed at LIACC, Universidade do
Porto. YAP provides several important features:

@itemize @bullet
 @item Speed: YAP is widely considered one of the fastest available
Prolog systems.

 @item Functionality: it supports stream I/O, sockets, modules,
exceptions, Prolog debugger, C-interface, dynamic code, internal
database, DCGs, saved states, co-routining, arrays, threads.

 @item We explicitly allow both commercial and non-commercial use of YAP.
@end itemize

YAP is based on the David H. D. Warren's WAM (Warren Abstract Machine),
with several optimizations for better performance. YAP follows the
Edinburgh tradition, and was originally designed to be largely
compatible with DEC-10 Prolog, Quintus Prolog, and especially with
C-Prolog.

YAP implements most of the ISO-Prolog standard. We are striving at
full compatibility, and the manual describes what is still
missing. The manual also includes a (largely incomplete) comparison
with SICStus Prolog.

The document is intended neither as an introduction to Prolog nor to the
implementation aspects of the compiler. A good introduction to
programming in Prolog is the book @cite{The Art of Prolog}, by
L. Sterling and E. Shapiro, published by "The MIT Press, Cambridge
MA". Other references should include the classical @cite{Programming in
Prolog}, by W.F. Clocksin and C.S. Mellish, published by
Springer-Verlag.

YAP 4.3 is known to build with many versions of gcc (<= gcc-2.7.2, >=
gcc-2.8.1, >= egcs-1.0.1, gcc-2.95.*) and on a variety of Unixen:
SunOS 4.1, Solaris 2.*, Irix 5.2, HP-UX 10, Dec Alpha Unix, Linux 1.2
and Linux 2.* (RedHat 4.0 thru 5.2, Debian 2.*) in both the x86 and
alpha platforms. It has been built on Windows NT 4.0 using Cygwin from
Cygnus Solutions (see README.nt) and using Visual C++ 6.0.

The overall copyright and permission notice for YAP4.3 can be found in
the Artistic file in this directory. YAP follows the Perl Artistic
license, and it is thus non-copylefted freeware.

If you have a question about this software, desire to add code, found a
bug, want to request a feature, or wonder how to get further assistance,
please send e-mail to @email{yap-users AT lists.sourceforge.net}.  To
subscribe to the mailing list, visit the page
@url{https://lists.sourceforge.net/lists/listinfo/yap-users}.

On-line documentation is available for YAP at:

        @url{http://www.ncc.up.pt/~vsc/YAP/}

Recent versions of YAP, including both source and selected binaries,
can be found from this same URL.

This manual was written by V@'{@dotless{i}}tor Santos Costa,
Lu@'{@dotless{i}}s Damas, Rog@'erio Reis, and R@'uben Azevedo. The
manual is largely based on the DECsystem-10 Prolog User's Manual by
D.L. Bowen, L. Byrd, F. C. N. Pereira, L. M. Pereira, and
D. H. D. Warren. We have also used comments from the Edinburgh Prolog
library written by R. O'Keefe. We would also like to gratefully
acknowledge the contributions from Ashwin Srinivasian.

We are happy to include in YAP several excellent packages developed
under separate licenses. Our thanks to the authors for their kind
authorization to include these packages.

The packages are, in alphabetical order:

@itemize @bullet
@item The CHR package developed by Tom Schrijvers,
Christian Holzbaur, and Jan Wielemaker.

@item The CLP(R) package developed by Leslie De Koninck, Bart Demoen, Tom
Schrijvers, and Jan Wielemaker, based on the CLP(Q,R) implementation
by Christian Holzbaur.

@item The Logtalk Object-Oriented system is developed at the University 
of Beira Interior, Portugal, by Paulo Moura:

@url{http://logtalk.org/}

Logtalk is no longer distributed with YAP. Please use the Logtalk standalone 
installer for a smooth integration with YAP.

@item The Pillow WEB library developed at Universidad Politecnica de
Madrid by the CLIP group. This package is distributed under the FSF's
LGPL. Documentation on this package is distributed separately from
yap.tex.

@item The @code{yap2swi} library implements some of the functionality of
SWI's PL interface. Please do refer to the SWI-Prolog home page:

@url{http://www.swi-prolog.org}

for more information on SWI-Prolog and for a detailed description of its
foreign language interface.

@end itemize

@node Install, Run, Intro, Top
@chapter Installing YAP
@cindex installation


@menu
* Configuration Options:: Tuning the Functionality of YAP Machine
* Machine Options:: Tuning YAP for a Particular Machine and Compiler
@end menu

To compile YAP it should be sufficient to:

@enumerate 
@item @code{mkdir ARCH}.

@item @code{cd ARCH}.

@item @code{../configure ...options...}. 

Notice that by default @code{configure} gives you a vanilla
configuration. For instance, in order to use co-routining and/or CLP
you need to do

@example
../configure --enable-coroutining ...options...
@end example
Please @pxref{Configuration Options} for extra options.

@item check the Makefile for any extensions or changes you want to
make.

YAP uses @code{autoconf}. Recent versions of YAP try to follow GNU
conventions on where to place software.

@itemize @bullet
@item The main executable is placed at @code{BINDIR}. This executable is
actually a script that calls the Prolog engine, stored at @code{LIBDIR}.

@item @code{LIBDIR} is the directory where libraries are stored. YAPLIBDIR is a
subdirectory that contains the Prolog engine and a Prolog library.

@item @code{INCLUDEDIR} is used if you want to use YAP as a library.

@item @code{INFODIR} is where to store @code{info} files. Usually
@code{/usr/local/info}, @code{/usr/info}, or @code{/usr/share/info}.
@end itemize

@item @code{make}.

@item If the compilation succeeds, try @code{./yap}.

@item If you feel satisfied with the result, do @code{make install}.

@item @code{make install-info} will create the info files in the
standard info directory.

@item @code{make html} will create documentation in html format in the
predefined directory.

In most systems you will need to be superuser in order to do @code{make
install} and @code{make info} on the standard directories.
@end enumerate

@node Configuration Options, Machine Options, ,Install
@section Tuning the Functionality of YAP 
@cindex syntax

Compiling YAP with the standard options give you a plain vanilla
Prolog. You can tune YAP to include extra functionality by calling
@code{configure} with the appropriate options:

@itemize @bullet
 @item @code{--enable-rational-trees=yes} gives you support for infinite
rational trees.

 @item @code{--enable-coroutining=yes} gives you support for coroutining,
including freezing of goals, attributed variables, and
constraints. This will also enable support for infinite rational
trees.

 @item @code{--enable-depth-limit=yes} allows depth limited evaluation, say for
implementing iterative deepening.

 @item @code{--enable-low-level-tracer=yes} allows support for tracing all calls,
retries, and backtracks in the system. This can help in debugging your
application, but results in performance loss.

 @item @code{--enable-wam-profile=yes} allows profiling of abstract machine
instructions. This is useful when developing YAP, should not be so
useful for normal users.

 @item @code{--enable-condor=yes} allows using the Condor system that
support High Throughput Computing (HTC) on large collections of
distributively owned computing resources.

 @item @code{--enable-tabling=yes} allows tabling support. This option
is still experimental.

 @item @code{--enable-parallelism=@{env-copy,sba,a-cow@}} allows
or-parallelism supported by one of these three forms. This option is
still highly experimental.

 @item @code{--with-max-workers} allows definition of the maximum 
number of parallel processes (its value can be consulted at runtime 
using the flag @code{max_workers}).

 @item @code{--with-gmp[=DIR]} give a path to where one can find the
@code{GMP} library if not installed in the default path.

 @item @code{--enable-threads} allows using of the multi-threading 
predicates provided by YAP. Depending on the operating system, the 
option @code{--enable-pthread-locking} may also need to be used.

 @item @code{--with-max-threads} allows definition of the maximum 
number of threads (the default value is 1024; its value can be consulted 
at runtime using the flag @code{max_threads}).

@end itemize

Next section discusses machine dependent details.

@node Machine Options, , Configuration Options,Install
@section Tuning YAP for a Particular Machine and Compiler
@cindex machine optimizations

The default options should give you best performance under
@code{GCC}. Although the system is tuned for this compiler
we have been able to compile versions of YAP under lcc in Linux,
Sun's cc compiler, IBM's xlc, SGI's cc, and Microsoft's Visual C++
6.0.

@menu
* Tuning for GCC:: Using the GNUCC compiler
* Compiling Under Visual C++:: Using Microsoft's Visual C++ environment
* Tuning for SGI cc:: Compiling Under SGI's @code{cc}
@end menu


@node Tuning for GCC, Compiling Under Visual C++, , Machine Options
@section Tuning YAP for @code{GCC}.

YAP has been developed to take advantage of @code{GCC} (but not to
depend on it). The major advantage of @code{GCC} is threaded code and
explicit register reservation.

YAP is set by default to compile with the best compilation flags we
know. Even so, a few specific options reduce portability.  The option 
@itemize @bullet
 @item @code{--enable-max-performance=yes} will try to support the best
available flags for a specific architectural model. Currently, the option
assumes a recent version of @code{GCC}.
 @item @code{--enable-debug-yap} compiles YAP so that it can be debugged
by tools such as @code{dbx} or @code{gdb}.
@end itemize

Here follow a few hints:

On x86 machines the flags:

@example
YAP_EXTRAS= ... -DBP_FREE=1
@end example

tells us to use the @code{%bp} register (frame-pointer) as the emulator's
program counter. This seems to be stable and is now default.

On  Sparc/Solaris2 use:

@example
YAP_EXTRAS= ...   -mno-app-regs -DOPTIMISE_ALL_REGS_FOR_SPARC=1
@end example

and YAP will get two extra registers! This trick does not work on
SunOS 4 machines.

Note that versions of GCC can be tweaked to recognize different
processors within the same instruction set, e.g. 486, Pentium, and
PentiumPro for the x86; or Ultrasparc, and Supersparc for
Sparc. Unfortunately, some of these tweaks do may make YAP run slower or
not at all in other machines with the same instruction set, so they
cannot be made default.

Last, the best options also depends on the version of GCC you are using, and
it is a good idea to consult the GCC manual under the menus "Invoking
GCC"/"Submodel Options". Specifically, you should check
@code{-march=XXX} for recent versions of GCC/EGCS. In the case of
@code{GCC2.7} and other recent versions of @code{GCC} you can check:

@table @code

@item 486:
In order to take advantage of 486 specific optimizations in GCC 2.7.*:

@example
YAP_EXTRAS= ... -m486 -DBP_FREE=1
@end example

@item Pentium:
@example
YAP_EXTRAS= ... -m486 -malign-loops=2 -malign-jumps=2 \
                      -malign-functions=2
@end example

@item PentiumPro and other recent Intel and AMD machines:
PentiumPros are known not to require alignment. Check your version of
@code{GCC} for the best @code{-march} option.

@item Super and UltraSparcs:
@example
YAP_EXTRAS= ... -msupersparc
@end example

@item MIPS: if have a recent machine and you need a 64 bit wide address
space you can use the abi 64 bits or eabi option, as in:
@example
CC="gcc -mabi=64" ./configure --...
@end example
Be careful. At least for some versions of @code{GCC}, compiling with
@code{-g} seems to result in broken code.

@item WIN32: GCC is distributed in the MINGW32 and CYGWIN packages.

The Mingw32 environment is available from the URL:

@code{http://www.mingw.org}

You will need to install the @code{msys} and @code{mingw}
packages. You should be able to do configure, make and make install.

If you use mingw32 you may want to search the contributed packages for
the @code{gmp} multi-precision arithmetic library. If you do setup YAP
with @code{gmp} note that @code{libgmp.dll} must be in the path,
otherwise YAP will not be able to execute.

CygWin environment is available from the URL:

@code{http://www.cygwin.com}

@noindent
and mirrors. We suggest using recent versions of the cygwin shell. The
compilation steps under the cygwin shell are as follows:

@example
mkdir cyg
$YAPSRC/configure --enable-coroutining \\
                  --enable-depth-limit \\
                  --enable-max-performance
make
make install
@end example

By default, YAP will use the @code{-mno-cygwin} option to
disable the use of the cygwin dll and to enable the mingw32 subsystem
instead. YAP thus will not need the cygwin dll. It instead accesses
the system's @code{CRTDLL.DLL} @code{C} run time library supplied with
Win32 platforms through the mingw32 interface. Note that some older
WIN95 systems may not have @code{CRTDLL.DLL}, in this case it should
be sufficient to import the file from a newer WIN95 or WIN98 machine.

You should check the default installation path which is set to
@code{/YAP} in the standard Makefile. This string will usually
be expanded into @code{c:\YAP} by Windows.

The cygwin environment does not provide @t{gmp} on the MINGW
subsystem. You can fetch a dll for the @t{gmp} library from
@url{http://www.sf.net/projects/mingwrep}.

It is also possible to configure YAP to be a part of the cygwin
environment. In this case you should use:
@example
mkdir cyg
$YAPSRC/configure --enable-max-performance \\
                  --enable-cygwin=yes
make
make install
@end example
YAP will then compile using the cygwin library and will be installed
in cygwin's @code{/usr/local}. You can use YAP from a cygwin console,
or as a standalone application as long as it can find
@code{cygwin1.dll} in its path. Note that you may use to use
@code{--enable-depth-limit} for Aleph compatibility, and that you may
want to be sure that GMP is installed.

@end table

@node Compiling Under Visual C++, Tuning for SGI cc, Tuning for GCC, Machine Options
@subsection  Compiling Under Visual C++

YAP compiles cleanly under Microsoft's Visual C++ release 6.0. We next
give a step-by-step tutorial on how to compile YAP manually using this
environment.

First, it is a good idea to build YAP as a DLL:

@enumerate

@item create a project named yapdll using File.New. The project will be a
DLL project, initially empty.

Notice that either the project is named yapdll or you must replace the
preprocessors variable @var{YAPDLL_EXPORTS} to match your project names
in the files @code{YAPInterface.h} and @code{c_interface.c}.

@item add all .c files in the @var{$YAPSRC/C} directory and in the
@var{$YAPSRC\OPTYAP} directory to the Project's @code{Source Files} (use
FileView).

@item add all .h files in the @var{$YAPSRC/H} directory,
@var{$YAPSRC\include} directory and in the @var{$YAPSRC\OPTYAP}
subdirectory to the Project's @code{Header Files}.

@item Ideally, you should now use @code{m4} to generate extra  .h from .m4 files and use
@code{configure} to create a @code{config.h}. Or, you can be lazy, and
fetch these files from @var{$YAPSRC\VC\include}.

@item You may want to go to @code{Build.Set Active Configuration} and set @code{Project
Type} to @code{Release}

@item To use YAP's own include directories you have to set the Project
option  @code{Project.Project Settings.C/C++.Preprocessor.Additional
Include Directories} to include the directories @var{$YAPSRC\H},
@var{$YAPSRC\VC\include}, @var{$YAPSRC\OPTYAP} and
@var{$YAPSRC\include}.  The syntax is:

@example
$YAPSRC\H, $YAPSRC\VC\include, $YAPSRC\OPTYAP, $YAPSRC\include
@end example

@item  Build: the system should generate an @code{yapdll.dll} and an @code{yapdll.lib}.

@item Copy the file @code{yapdll.dll} to your path. The file
@code{yapdll.lib} should also be copied to a location where the linker can find it.
@end enumerate

Now you are ready to create a console interface for YAP:
@enumerate
@item create a second project  say @code{wyap} with @code{File.New}. The project will be a
WIN32 console project, initially empty.

@item add @var{$YAPSRC\console\yap.c} to the @code{Source Files}.

@item add @var{$YAPSRC\VC\include\config.h} and the files in @var{$YAPSRC\include} to
the @code{Header Files}.

@item You may want to go to @code{Build.Set Active Configuration} and set
@code{Project Type} to @code{Release}.

@item you will eventually need to bootstrap the system by booting from
@code{boot.yap}, so write:

@example
        -b $YAPSRC\pl\boot.yap
@end example

        in @code{Project.Project Settings.Debug.Program Arguments}.

@item You need the sockets and yap libraries. Add

@example
ws2_32.lib yapdll.lib to
@end example

to

to @code{Project.Project Settings.Link.Object/Library Modules}

You may also need to set the @code{Link Path} so that VC++ will find @code{yapdll.lib}.

@item set @code{Project.Project Settings.C/C++.Preprocessor.Additional
Include Directories} to include the @var{$YAPSRC/VC/include} and
@var{$YAPSRC/include}.

The syntax is:

@example
$YAPSRC\VC\include, $YAPSRC\include
@end example

@item Build the system.

@item Use @code{Build.Start Debug} to boot the system, and then create the saved state with

@example
['$YAPSRC\\pl\\init'].
save_program('startup.yss').
^Z
@end example

That's it, you've got YAP and the saved state!
@end enumerate

The $YAPSRC\VC directory has the make files to build YAP4.3.17 under VC++ 6.0.

@node Tuning for SGI cc, , Compiling Under Visual C++ ,Machine Options
@subsection  Compiling Under SGI's cc

YAP should compile under the Silicon Graphic's @code{cc} compiler,
although we advise using the GNUCC compiler, if available.

@table @code
@item 64 bit
Support for 64 bits should work by using (under Bourne shell syntax):
@example
CC="cc -64" $YAP_SRC_PATH/configure --...
@end example
@end table

@node Run, Syntax, Install, Top
@chapter Running YAP

@menu
* Running YAP Interactively:: Interacting with YAP
* Running Prolog Files:: Running Prolog files as scripts
@end menu

@cindex booting
We next describe how to invoke YAP in Unix systems.

@node Running YAP Interactively, ,Running Prolog Files,Run
@section Running YAP Interactively

Most often you will want to use YAP in interactive mode. Assuming that
YAP is in the user's search path, the top-level can be invoked under
Unix with the following command:

@example
yap [-s n] [-h n] [-a n] [-c IP_HOST port ] [filename]
@end example

@noindent
All the arguments and flags are optional and have the following meaning:
@table @code
@item -?
print a short error message.
@item -s@var{Size}
allocate @var{Size} K bytes for local and global stacks. The user may
 specify @t{M} bytes.
@item -h@var{Size}
allocate @var{Size} K bytes for heap and auxiliary stacks
@item -t@var{Size}
allocate @var{Size} K bytes for the trail stack
@item -L@var{Size} 
SWI-compatible option to allocate @var{Size} K bytes for local and global stacks, the local stack
 cannot be expanded. To avoid confusion with the load option, @var{Size}
 must immediately follow the letter @code{L}.
@item -G@var{Size}
SWI-compatible option to allocate @var{Size} K bytes for local and global stacks; the global
stack cannot be expanded
@item -T@var{Size}
SWI-compatible option to allocate @var{Size} K bytes for the trail stack; the trail cannot be expanded.
@item -l @var{YAP_FILE}
compile the Prolog file @var{YAP_FILE} before entering the top-level.
@item -L @var{YAP_FILE}
compile the Prolog file @var{YAP_FILE} and then halt. This option is
useful for implementing scripts.
@item -g @var{Goal}
run the goal @var{Goal} before top-level. The goal is converted from
an atom to a Prolog term.
@item -z @var{Goal}
run the goal @var{Goal} as top-level. The goal is converted from
an atom to a Prolog term.
@item -b @var{BOOT_FILE}
boot code is in Prolog file @var{BOOT_FILE}. The filename must define
the predicate @code{'$live'/0}.
@item -c @t{IP_HOST} @t{port}
connect standard streams to host @t{IP_HOST} at port @t{port}
@item filename
restore state saved in the given file
@item -f
do not consult initial files
@item -q
do not print informational messages
@item --
separator for arguments to Prolog code. These arguments are visible
through the @code{unix/1} built-in predicate.
@end table

Note that YAP will output an error message on the following conditions:

@itemize @bullet
@item
a file name was given but the file does not exist or is not a saved
YAP state;
@item 
the necessary amount of memory could not be allocated;
@item
the allocated memory is not enough to restore the state.
@end itemize

When restoring a saved state, YAP will allocate the
same amount of memory as that in use when the state was saved, unless a
different amount is specified by flags in the command line. By default,
YAP restores the file @samp{startup.yss} from the current directory or from
the YAP library.
@cindex environment variables

@findex YAPBINDIR
@itemize @bullet
@item
YAP usually boots from a saved state. The saved state will use the default
installation directory to search for the YAP binary unless you define
the environment variable YAPBINDIR.

@findex YAPLIBDIR
@item
YAP always tries to find saved states from the current directory
first. If it cannot it will use the environment variable YAPLIBDIR, if
defined, or search the default library directory.

@findex YAPSHAREDIR
@item
YAP will try to find library files from the YAPSHAREDIR/library
directory.
@end itemize

@node Running Prolog Files, Running YAP Interactively, , Run
@section Running Prolog Files

YAP can also be used to run Prolog files as scripts, at least in
Unix-like environments. A simple example is shown next (do not forget
that the shell comments are very important):

@example
@cartouche
#!/usr/local/bin/yap -L --
#
# Hello World script file using YAP
#
# put a dot because of syntax errors .

:- write('Hello World'), nl.

@end cartouche
@end example

The @code{#!}  characters specify that the script should call the binary
file YAP. Notice that many systems will require the complete path to the
YAP binary. The @code{-L} flag indicates that YAP should consult the
current file when booting and then halt. The remaining arguments are
then passed to YAP. Note that YAP will skip the first lines if they
start with @code{#} (the comment sign for Unix's shell). YAP will
consult the file and execute any commands.

A slightly more sophisticated example is:

@example
@cartouche
#!/usr/bin/yap -L --
#
# Hello World script file using YAP
# .

:- initialization(main).

main :- write('Hello World'), nl.

@end cartouche
@end example

The @code{initialization} directive tells YAP to execute the goal main
after consulting the file. Source code is thus compiled and @code{main}
executed at the end. The @code{.} is useful while debugging the script
as a Prolog program: it guarantees that the syntax error will not
propagate to the Prolog code.

Notice that the @code{--} is required so that the shell passes the extra
arguments to YAP.  As an example, consider the following script
@code{dump_args}:

@example
@cartouche
#!/usr/bin/yap -L --
#.

main( [] ).
main( [H|T] ) :-
        write( H ), nl,
        main( T ).

:- unix( argv(AllArgs) ), main( AllArgs ).

@end cartouche
@end example

If you this run this script with the arguments:
@example
./dump_args -s 10000
@end example
@noindent
the script will start an YAP process with stack size @code{10MB}, and
the list of arguments to the process will be empty.

Often one wants to run the script as any other program, and for this it
is convenient to ignore arguments to YAP. This is possible by using
@code{L --} as in the next version of @code{dump_args}:

@example
@cartouche
#!/usr/bin/yap -L --

main( [] ).
main( [H|T] ) :-
        write( H ), nl,
        main( T ).

:- unix( argv(AllArgs) ), main( AllArgs ).

@end cartouche
@end example

The @code{--} indicates the next arguments are not for YAP. Instead,
they must be sent directly to the @code{argv} built-in. Hence, running
@example
./dump_args test
@end example
@noindent
will write @code{test} on the standard output.


@node Syntax, Loading Programs, Run, Top
@chapter Syntax

We will describe the syntax of YAP at two levels. We first will
describe the syntax for Prolog terms. In a second level we describe
the @i{tokens} from which Prolog @i{terms} are
built.

@menu
* Formal Syntax:: Syntax of terms 
* Tokens:: Syntax of Prolog tokens
* Encoding:: How characters are encoded and Wide Character Support
@end menu

@node Formal Syntax, Tokens, ,Syntax
@section Syntax of Terms
@cindex syntax

Below, we describe the syntax of YAP terms from the different
classes of tokens defined above. The formalism used will be @emph{BNF},
extended where necessary with attributes denoting integer precedence or
operator type.

@example

@code{
 term       ---->     subterm(1200)   end_of_term_marker

 subterm(N) ---->     term(M)         [M <= N]

 term(N)    ---->     op(N, fx) subterm(N-1)
             |        op(N, fy) subterm(N)
             |        subterm(N-1) op(N, xfx) subterm(N-1)
             |        subterm(N-1) op(N, xfy) subterm(N)
             |        subterm(N) op(N, yfx) subterm(N-1)
             |        subterm(N-1) op(N, xf)
             |        subterm(N) op(N, yf)

 term(0)   ---->      atom '(' arguments ')'
             |        '(' subterm(1200)  ')'
             |        '@{' subterm(1200)  '@}'
             |        list
             |        string
             |        number
             |        atom
             |        variable

 arguments ---->      subterm(999)
             |        subterm(999) ',' arguments

 list      ---->      '[]'
             |        '[' list_expr ']'

 list_expr ---->      subterm(999)
             |        subterm(999) list_tail

 list_tail ---->      ',' list_expr
             |        ',..' subterm(999)
             |        '|' subterm(999)
}
@end example

@noindent
Notes:

@itemize @bullet

@item
@i{op(N,T)} denotes an atom which has been previously declared with type
@i{T} and base precedence @i{N}.

@item
Since ',' is itself a pre-declared operator with type @i{xfy} and
precedence 1000, is @i{subterm} starts with a '(', @i{op} must be
followed by a space to avoid ambiguity with the case of a functor
followed by arguments, e.g.:

@example
@code{ + (a,b)        [the same as '+'(','(a,b)) of arity one]}
@end example
versus
@example
@code{ +(a,b)         [the same as '+'(a,b) of arity two]}
@end example

@item
In the first rule for term(0) no blank space should exist between
@i{atom} and '('.

@item
@cindex end of term
Each term to be read by the YAP parser must end with a single
dot, followed by a blank (in the sense mentioned in the previous
paragraph). When a name consisting of a single dot could be taken for
the end of term marker, the ambiguity should be avoided by surrounding the
dot with single quotes.

@end itemize

@node Tokens, Encoding, Formal Syntax, Syntax
@section Prolog Tokens
@cindex token

Prolog tokens are grouped into the following categories:

@menu
* Numbers:: Integer and Floating-Point Numbers
* Strings:: Sequences of Characters
* Atoms:: Atomic Constants
* Variables:: Logical Variables
* Punctuation Tokens:: Tokens that separate other tokens
* Layout:: Comments and Other Layout Rules
@end menu

@node Numbers, Strings, ,Tokens
@subsection Numbers
@cindex number

Numbers can be further subdivided into integer and floating-point numbers.

@menu
* Integers:: How Integers are read and represented
* Floats:: Floating Point Numbers
@end menu

@node Integers, Floats, ,Numbers
@subsubsection Integers
@cindex integer

Integer numbers
are described by the following regular expression:

@example
@code{
<integer> := @{<digit>+<single-quote>|0@{xXo@}@}<alpha_numeric_char>+
}
@end example
@noindent
where @{...@} stands for optionality, @i{+} optional repetition (one or
more times), @i{<digit>} denotes one of the characters 0 ... 9, @i{|}
denotes or, and @i{<single-quote>} denotes the character "'". The digits
before the @i{<single-quote>} character, when present, form the number
basis, that can go from 0, 1 and up to 36. Letters from @code{A} to
@code{Z} are used when the basis is larger than 10.

Note that if no basis is specified then base 10 is assumed. Note also
that the last digit of an integer token can not be immediately followed
by one of the characters 'e', 'E', or '.'.

Following the ISO standard, YAP also accepts directives of the
form @code{0x} to represent numbers in hexadecimal base and of the form
@code{0o} to represent numbers in octal base. For usefulness,
YAP also accepts directives of the form @code{0X} to represent
numbers in hexadecimal base.

Example:
the following tokens all denote the same integer
@example
@code{10  2'1010  3'101  8'12  16'a  36'a  0xa  0o12}
@end example

Numbers of the form @code{0'a} are used to represent character
constants. So, the following tokens denote the same integer:
@example
@code{0'd  100}
@end example

YAP (version @value{VERSION}) supports integers that can fit
the word size of the machine. This is 32 bits in most current machines,
but 64 in some others, such as the Alpha running Linux or Digital
Unix. The scanner will read larger or smaller integers erroneously.

@node Floats, , Integers,Numbers
@subsubsection Floating-point Numbers
@cindex floating-point number

Floating-point numbers are described by:

@example
@code{
   <float> := <digit>+@{<dot><digit>+@}
               <exponent-marker>@{<sign>@}<digit>+
            |<digit>+<dot><digit>+
               @{<exponent-marker>@{<sign>@}<digit>+@}
}
@end example

@noindent
where @i{<dot>} denotes the decimal-point character '.',
@i{<exponent-marker>} denotes one of 'e' or 'E', and @i{<sign>} denotes
one of '+' or '-'.

Examples:
@example
@code{10.0   10e3   10e-3   3.1415e+3}
@end example

Floating-point numbers are represented as a double in the target
machine. This is usually a 64-bit number.

@node Strings, Atoms, Numbers,Tokens
@subsection Character Strings
@cindex string

Strings are described by the following rules:
@example
  string --> '"' string_quoted_characters '"'

  string_quoted_characters --> '"' '"' string_quoted_characters
  string_quoted_characters --> '\'
                          escape_sequence string_quoted_characters
  string_quoted_characters -->
                          string_character string_quoted_characters

  escape_sequence --> 'a' | 'b' | 'r' | 'f' | 't' | 'n' | 'v'
  escape_sequence --> '\' | '"' | ''' | '`'
  escape_sequence --> at_most_3_octal_digit_seq_char '\'
  escape_sequence --> 'x' at_most_2_hexa_digit_seq_char '\'
@end example
where @code{string_character} in any character except the double quote
and escape characters.

Examples:
@example
@code{""   "a string"   "a double-quote:""" }
@end example

The first string is an empty string, the last string shows the use of
double-quoting. The implementation of YAP represents strings as
lists of integers. Since YAP 4.3.0 there is no static limit on string
size.

Escape sequences can be used to include the non-printable characters
@code{a} (alert), @code{b} (backspace), @code{r} (carriage return),
@code{f} (form feed), @code{t} (horizontal tabulation), @code{n} (new
line), and @code{v} (vertical tabulation). Escape sequences also be
include the meta-characters @code{\}, @code{"}, @code{'}, and
@code{`}. Last, one can use escape sequences to include the characters
either as an octal or hexadecimal number.

The next examples demonstrates the use of escape sequences in YAP:

@example
@code{"\x0c\" "\01\" "\f" "\\" }
@end example

The first three examples return a list including only character 12 (form
feed). The last example escapes the escape character.

Escape sequences were not available in C-Prolog and in original
versions of YAP up to 4.2.0. Escape sequences can be disable by using:
@example
@code{:- yap_flag(character_escapes,off).}
@end example


@node Atoms, Variables, Strings, Tokens
@subsection Atoms
@cindex atom

Atoms are defined by one of the following rules:
@example
   atom --> solo-character
   atom --> lower-case-letter name-character*
   atom --> symbol-character+
   atom --> single-quote  single-quote
   atom --> ''' atom_quoted_characters '''


  atom_quoted_characters --> ''' ''' atom_quoted_characters
  atom_quoted_characters --> '\' atom_sequence string_quoted_characters
  atom_quoted_characters --> character string_quoted_characters

@end example

where:
@example
   <solo-character>     denotes one of:    ! ;
   <symbol-character>   denotes one of:    # & * + - . / : < 
                                           = > ? @@ \ ^ ` ~
   <lower-case-letter>  denotes one of:    a...z
   <name-character>     denotes one of:    _ a...z A...Z 0....9
   <single-quote>       denotes:           '
@end example

and @code{string_character} denotes any character except the double quote
and escape characters. Note that escape sequences in strings and atoms
follow the same rules.

Examples:
@example
@code{a   a12x   '$a'   !   =>  '1 2'}
@end example


Version @code{4.2.0} of YAP removed the previous limit of 256
characters on an atom. Size of an atom is now only limited by the space
available in the system.

@node Variables, Punctuation Tokens, Atoms, Tokens
@subsection Variables
@cindex variable

Variables are described by:
@example
   <variable-starter><variable-character>+
@end example
where
@example
  <variable-starter>   denotes one of:    _ A...Z
  <variable-character> denotes one of:    _ a...z A...Z
@end example

@cindex anonymous variable
If a variable is referred only once in a term, it needs not to be named
and one can use the character @code{_} to represent the variable. These
variables are known as anonymous variables. Note that different
occurrences of @code{_} on the same term represent @emph{different}
anonymous variables. 

@node Punctuation Tokens, Layout, Variables, Tokens
@subsection Punctuation Tokens
@cindex punctuation token

Punctuation tokens consist of one of the following characters:
@example
@center ( ) , [ ] @{ @} |
@end example

These characters are used to group terms.

@node Layout, ,Punctuation Tokens, Tokens
@subsection Layout
@cindex comment
Any characters with ASCII code less than or equal to 32 appearing before
a token are ignored.

All the text appearing in a line after the character @i).  Comments can also be
inserted by using the sequence @code{/*} to start the comment and
@code{*/} to finish it. In the presence of any sequence of comments or
layout characters, the YAP parser behaves as if it had found a
single blank character. The end of a file also counts as a blank
character for this purpose.

@node Encoding, , Tokens, Syntax
@section Wide Character Support
@cindex encodings

@menu
* Stream Encoding:: How Prolog Streams can be coded
* BOM:: The Byte Order Mark
@end menu

@cindex UTF-8
@cindex Unicode
@cindex UCS
@cindex internationalization
YAP now implements a SWI-Prolog compatible interface to wide
characters and the Universal Character Set (UCS). The following text
was adapted from the SWI-Prolog manual.

YAP now  supports wide characters, characters with character
codes above 255 that cannot be represented in a single byte.
@emph{Universal Character Set} (UCS) is the ISO/IEC 10646 standard
that specifies a unique 31-bits unsigned integer for any character in
any language.  It is a superset of 16-bit Unicode, which in turn is
a superset of ISO 8859-1 (ISO Latin-1), a superset of US-ASCII.  UCS
can handle strings holding characters from multiple languages and
character classification (uppercase, lowercase, digit, etc.) and
operations such as case-conversion are unambiguously defined.

For this reason YAP, following SWI-Prolog, has two representations for
atoms. If the text fits in ISO Latin-1, it is represented as an array
of 8-bit characters.  Otherwise the text is represented as an array of
wide chars, which may take 16 or 32 bits.  This representational issue
is completely transparent to the Prolog user.  Users of the foreign
language interface sometimes need to be aware of these issues though.

Character coding comes into view when characters of strings need to be
read from or written to file or when they have to be communicated to
other software components using the foreign language interface. In this
section we only deal with I/O through streams, which includes file I/O
as well as I/O through network sockets.


@node Stream Encoding, , BOM, Encoding
@subsection Wide character encodings on streams



Although characters are uniquely coded using the UCS standard
internally, streams and files are byte (8-bit) oriented and there are a
variety of ways to represent the larger UCS codes in an 8-bit octet
stream. The most popular one, especially in the context of the web, is
UTF-8. Bytes 0...127 represent simply the corresponding US-ASCII
character, while bytes 128...255 are used for multi-byte
encoding of characters placed higher in the UCS space. Especially on
MS-Windows the 16-bit Unicode standard, represented by pairs of bytes is
also popular.

Prolog I/O streams have a property called @emph{encoding} which
specifies the used encoding that influence @code{get_code/2} and
@code{put_code/2} as well as all the other text I/O predicates.

The default encoding for files is derived from the Prolog flag
@code{encoding}, which is initialised from the environment.  If the
environment variable @env{LANG} ends in "UTF-8", this encoding is
assumed. Otherwise the default is @code{text} and the translation is
left to the wide-character functions of the C-library (note that the
Prolog native UTF-8 mode is considerably faster than the generic
@code{mbrtowc()} one).  The encoding can be specified explicitly in
@code{load_files/2} for loading Prolog source with an alternative
encoding, @code{open/4} when opening files or using @code{set_stream/2} on
any open stream (not yet implemented). For Prolog source files we also
provide the @code{encoding/1} directive that can be used to switch
between encodings that are compatible to US-ASCII (@code{ascii},
@code{iso_latin_1}, @code{utf8} and many locales).  
@c See also
@c \secref{intsrcfile} for writing Prolog files with non-US-ASCII
@c characters and \secref{unicodesyntax} for syntax issues. 
For
additional information and Unicode resources, please visit
@uref{http://www.unicode.org/}.

YAP currently defines and supports the following encodings:

@table @code
@item  octet
Default encoding for @emph{binary} streams.  This causes
the stream to be read and written fully untranslated.

@item  ascii
7-bit encoding in 8-bit bytes.  Equivalent to @code{iso_latin_1},
but generates errors and warnings on encountering values above
127.

@item  iso_latin_1
8-bit encoding supporting many western languages.  This causes
the stream to be read and written fully untranslated.

@item  text
C-library default locale encoding for text files.  Files are read and
written using the C-library functions @code{mbrtowc()} and
@code{wcrtomb()}.  This may be the same as one of the other locales,
notably it may be the same as @code{iso_latin_1} for western
languages and @code{utf8} in a UTF-8 context.

@item  utf8
Multi-byte encoding of full UCS, compatible to @code{ascii}.
See above.

@item  unicode_be
Unicode Big Endian.  Reads input in pairs of bytes, most
significant byte first.  Can only represent 16-bit characters.

@item  unicode_le
Unicode Little Endian.  Reads input in pairs of bytes, least
significant byte first.  Can only represent 16-bit characters.
@end table 

Note that not all encodings can represent all characters. This implies
that writing text to a stream may cause errors because the stream
cannot represent these characters. The behaviour of a stream on these
errors can be controlled using @code{open/4} or @code{set_stream/2} (not
implemented). Initially the terminal stream write the characters using
Prolog escape sequences while other streams generate an I/O exception.


@node BOM, Stream Encoding, , Encoding
@subsection BOM: Byte Order Mark

@cindex BOM
@cindex Byte Order Mark
From @ref{Stream Encoding}, you may have got the impression text-files are
complicated. This section deals with a related topic, making live often
easier for the user, but providing another worry to the programmer.
@strong{BOM} or @emph{Byte Order Marker} is a technique for
identifying Unicode text-files as well as the encoding they use. Such
files start with the Unicode character @code{0xFEFF}, a non-breaking,
zero-width space character. This is a pretty unique sequence that is not
likely to be the start of a non-Unicode file and uniquely distinguishes
the various Unicode file formats. As it is a zero-width blank, it even
doesn't produce any output. This solves all problems, or ...

Some formats start of as US-ASCII and may contain some encoding mark to
switch to UTF-8, such as the @code{encoding="UTF-8"} in an XML header.
Such formats often explicitly forbid the the use of a UTF-8 BOM. In
other cases there is additional information telling the encoding making
the use of a BOM redundant or even illegal.

The BOM is handled by the @code{open/4} predicate. By default, text-files are
probed for the BOM when opened for reading. If a BOM is found, the
encoding is set accordingly and the property @code{bom(true)} is
available through @code{stream_property/2}. When opening a file for
writing, writing a BOM can be requested using the option
@code{bom(true)} with @code{open/4}.

@node Loading Programs, Modules, Syntax, Top
@chapter Loading Programs

@menu

Loading Programs
* Compiling:: Program Loading and Updating
* Setting the Compiler:: Changing the compiler's parameters
* Conditional Compilation:: Compiling program fragments
* Saving:: Saving and Restoring Programs

@end menu


@node Compiling, Setting the Compiler, , Loading Programs
@section Program loading and updating

@table @code

@item consult(@var{+F})
@findex consult/1
@snindex consult/1
@cyindex consult/1
Adds the clauses written in file @var{F} or in the list of files @var{F}
to the program.

In YAP @code{consult/1} does not remove previous clauses for
the procedures defined in @var{F}. Moreover, note that all code in YAP
is compiled.

@item reconsult(@var{+F})
@findex reconsult/1
@snindex reconsult/1
@cyindex reconsult/1
Updates the program replacing the
previous definitions for the predicates defined in @var{F}.


@item [@var{+F}]
@findex []/1
@saindex []/1
@cyindex []/1
The same as @code{consult(F)}.

@item [-@var{+F}]
@findex [-]/1
@saindex [-]/1
@cyindex [-]/1
The same as @code{reconsult(F)}

Example:

@example
?- [file1, -file2, -file3, file4].
@end example
@noindent
will consult @code{file1} @code{file4} and reconsult @code{file2} and
@code{file3}.

@item compile(@var{+F})
@findex compile/1
@syindex compile/1
@cnindex compile/1
@noindent
In YAP, the same as @code{reconsult/1}.

@item load_files(@var{+Files}, @var{+Options})
@findex load_files/2
@syindex load_files/2
@cnindex load_files/2
@noindent
General implementation of @code{consult}. Execution is controlled by the
following flags:

@table @code
@item autoload(+@var{Autoload})
SWI-compatible option where if @var{Autoload} is @code{true} predicates
are loaded on first call. Currently
not supported.
@item derived_from(+@var{File})
      SWI-compatible option to control make. Currently
      not supported.
@item encoding(+@var{Encoding})
Character encoding used in consulting files. Please @pxref{Encoding} for
supported encodings.

@item expand(+@var{Bool})
      Not yet implemented. In SWI-Prolog, if @code{true}, run the
    filenames through @code{expand_file_name/2} and load the returned
    files. Default is false, except for @code{consult/1} which is
    intended for interactive use.

@item if(+@var{Condition})
    Load the file only if the specified @var{Condition} is
    satisfied. The value @code{true} the file unconditionally,
    @code{changed} loads the file if it was not loaded before, or has
    been modified since it was loaded the last time, @code{not_loaded}
    loads the file if it was not loaded before.

@item imports(+@var{ListOrAll})
    If @code{all} and the file is a module file, import all public
    predicates. Otherwise import only the named predicates. Each
    predicate is referred to as @code{<name>/<arity>}. This option has
    no effect if the file is not a module file.

@item must_be_module(+@var{Bool})
    If true, raise an error if the file is not a module file. Used by
    @code{use_module/[1,2]}.

@c qcompile(Bool)
@c     If this call appears in a directive of a file that is compiled into Quick Load Format using qcompile/1 and this flag is true, the contents of the argument files are included in the .qlf file instead of the loading directive.

@item silent(+@var{Bool})
    If true, load the file without printing a message. The specified value is the default for all files loaded as a result of loading the specified files.

@item stream(+@var{Input})
    This SWI-Prolog extension compiles the data from the stream
    @var{Input}. If this option is used, @var{Files} must be a single
    atom which is used to identify the source-location of the loaded
    clauses as well as remove all clauses if the data is re-consulted.

    This option is added to allow compiling from non-file locations such as databases, the web, the user (see consult/1) or other servers. 

@item compilation_mode(+@var{Mode})
    This extension controls how procedures are compiled. If @var{Mode}
    is @code{compact} clauses are compiled and no source code is stored;
    if it is @code{source} clauses are compiled and source code is stored;
    if it is @code{assert_all} clauses are asserted into the data-base.
@end table

@item ensure_loaded(@var{+F}) [ISO]
@findex ensure_loaded/1
@syindex compile/1
@cnindex compile/1
When the files specified by @var{F} are module files,
@code{ensure_loaded/1} loads them if they have note been previously
loaded, otherwise advertises the user about the existing name clashes
and prompts about importing or not those predicates. Predicates which
are not public remain invisible.

When the files are not module files, @code{ensure_loaded/1} loads them
if they have not been loaded before, does nothing otherwise.

@var{F} must be a list containing the names of the files to load.

@item make [ISO]
@findex make/0
@snindex make/0
@cnindex make/0
    SWI-Prolog built-in to consult all source files that have been
    changed since they were consulted. It checks all loaded source
    files. make/0 can be combined with the compiler to speed up the
    development of large packages. In this case compile the package
    using

@example
    sun% pl -g make -o my_program -c file ...
@end example

    If `my_program' is started it will first reconsult all source files
    that have changed since the compilation. 

@item include(@var{+F}) [ISO]
@findex include/1 (directive)
@snindex compile/1 (directive)
@cnindex compile/1 (directive)
The @code{include} directive includes the text files or sequence of text
files specified by @var{F} into the file being currently consulted.

@end table

@node Setting the Compiler, Conditional Compilation, Compiling, Loading Programs
@section Changing the Compiler's Behavior

This section presents a set of built-ins predicates designed to set the 
environment for the compiler.

@table @code

@item source_mode(-@var{O},+@var{N})
@findex source_mode/2
@snindex source_mode/2
@cnindex source_mode/2
The state of source mode can either be on or off. When the source mode
is on, all clauses are kept both as compiled code and in a "hidden"
database. @var{O} is unified with the previous state and the mode is set
according to @var{N}.

@item source
@findex source/0
@snindex source/0
@cnindex source/0
After executing this goal, YAP keeps information on the source
of the predicates that will be consulted. This enables the use of
@code{listing/0}, @code{listing/1} and @code{clause/2} for those
clauses.

The same as @code{source_mode(_,on)} or as declaring all newly defined
static procedures as @code{public}.

@item no_source
@findex no_source/0
@snindex no_source/0
@cnindex no_source/0
The opposite to @code{source}.

The same as @code{source_mode(_,off)}.

@item compile_expressions
@findex compile_expressions/0
@snindex compile_expressions/0
@cnindex compile_expressions/0
After a call to this predicate, arithmetical expressions will be compiled.
(see example below). This is the default behavior.

@item do_not_compile_expressions
@findex do_not_compile_expressions/0
@snindex do_not_compile_expressions/0
@cnindex do_not_compile_expressions/0
After a call to this predicate, arithmetical expressions will not be compiled.
@example
?- source, do_not_compile_expressions.
yes
?- [user].
| p(X) :- X is 2 * (3 + 8).
| :- end_of_file.
?- compile_expressions.
yes
?- [user].
| q(X) :- X is 2 * (3 + 8).
| :- end_of_file.
:- listing.

p(A):-
      A is 2 * (3 + 8).

q(A):-
      A is 22.
@end example

@item hide(+@var{Atom})
@findex hide/1
@snindex hide/1
@cnindex hide/1
Make atom @var{Atom} invisible.

@item unhide(+@var{Atom})
@findex unhide/1
@snindex unhide/1
@cnindex unhide/1
Make hidden atom @var{Atom} visible.


@item hide_predicate(+@var{Pred})
@findex hide_predicate/1
@snindex hide_predicate/1
@cnindex hide_predicate/1
Make predicate @var{Pred} invisible to @code{current_predicate/2},
@code{listing}, and friends.

@item expand_exprs(-@var{O},+@var{N})
@findex expand_exprs/2
@snindex expand_exprs/2
@cyindex expand_exprs/2
Puts YAP in state @var{N} (@code{on} or @code{off}) and unify
@var{O} with the previous state, where @var{On} is equivalent to
@code{compile_expressions} and @code{off} is equivalent to
@code{do_not_compile_expressions}. This predicate was kept to maintain
compatibility with C-Prolog.

@item path(-@var{D})
@findex path/1
@snindex path/1
@cnindex path/1
Unifies @var{D} with the current directory search-path of YAP.
Note that this search-path is only used by YAP to find the
files for @code{consult/1}, @code{reconsult/1} and @code{restore/1} and
should not be taken for the system search path.

@item add_to_path(+@var{D})
@findex add_to_path/1
@snindex path/1
@cnindex path/1
Adds @var{D} to the end of YAP's directory search path.

@item add_to_path(+@var{D},+@var{N})
@findex add_to_path/2
@snindex path/1
@cnindex path/1
Inserts @var{D} in the position, of the directory search path of
YAP, specified by @var{N}.  @var{N} must be either of
@code{first} or @code{last}.

@item remove_from_path(+@var{D})
@findex remove_from_path/1
@snindex remove_from_path/1
@cnindex remove_from_path/1
Remove @var{D} from YAP's directory search path.

@item style_check(+@var{X})
@findex style_check/1
@snindex style_check/1
@cnindex style_check/1
Turns on style checking according to the attribute specified by @var{X},
which must be one of the following:
@table @code
@item single_var
Checks single occurrences of named variables in a clause.
@item discontiguous
Checks non-contiguous clauses for the same predicate in a file.
@item multiple
Checks the presence of clauses for the same predicate in more than one
file when the predicate has not been declared as @code{multifile}
@item all
Performs style checking for all the cases mentioned above.
@end table
By default, style checking is disabled in YAP unless we are in
@code{sicstus} or @code{iso} language mode.

The @code{style_check/1} built-in is now deprecated. Please use the
@code{set_prolog_flag/1} instead.

@item no_style_check(+@var{X})
@findex no_style_check/1
@snindex style_check/1
@cnindex style_check/1
Turns off style checking according to the attribute specified by
@var{X}, which has the same meaning as in @code{style_check/1}.

The @code{no_style_check/1} built-in is now deprecated. Please use the
@code{set_prolog_flag/1} instead.

@item multifile @var{P} [ISO]
@findex multifile/1 (directive)
@syindex multifile/1 (directive)
@cnindex multifile/1 (directive)
Instructs the compiler about the declaration of a predicate @var{P} in
more than one file. It must appear in the first of the loaded files
where the predicate is declared, and before declaration of any of its
clauses.

Multifile declarations affect @code{reconsult/1} and @code{compile/1}:
when a multifile predicate is reconsulted, only the clauses from the
same file are removed.

Since YAP4.3.0 multifile procedures can be static or dynamic.

@item discontiguous(+@var{G}) [ISO]
@findex discontiguous/1 (directive)
@syindex discontiguous/1 (directive)
@cnindex discontiguous/1 (directive)

Declare that the arguments are discontiguous procedures, that is,
clauses for discontigous procedures may be separated by clauses from
other procedures.

@item initialization(+@var{G}) [ISO]
@findex initialization/1 (directive)
@snindex initialization/1 (directive)
@cnindex initialization/1 (directive)
The compiler will execute goals @var{G} after consulting the current
file.

@item initialization(+@var{Goal},+@var{When})
@findex initialization/2 (directive)
@snindex initialization/2 (directive)
@cnindex initialization/2 (directive)
Similar to @code{initialization/1}, but allows for specifying when
@var{Goal} is executed while loading the program-text:

@table @code
@item now
    Execute @var{Goal} immediately. 
@item after_load
    Execute @var{Goal} after loading program-text. This is the same as initialization/1. 
@item restore
    Do not execute @var{Goal} while loading the program, but only when
    restoring a state (not implemented yet). 
@end table

@item library_directory(+@var{D})
@findex library_directory/1
@snindex library_directory/1
@cnindex library_directory/1
Succeeds when @var{D} is a current library directory name. Library
directories are the places where files specified in the form
@code{library(@var{File})} are searched by the predicates
@code{consult/1}, @code{reconsult/1}, @code{use_module/1} or
@code{ensure_loaded/1}.

@item file_search_path(+@var{NAME},-@var{DIRECTORY})
@findex file_search_path/2
@syindex file_search_path/2
@cnindex file_search_path/2
Allows writing file names as compound terms. The @var{NAME} and
 @var{DIRECTORY} must be atoms. The predicate may generate multiple
solutions. The predicate is originally defined as follows:

@example
file_search_path(library,A) :-
   library_directory(A).
file_search_path(system,A) :-
   prolog_flag(host_type,A).
@end example

Thus, @code{[library(A)]} will search for a file using
@code{library_directory/1} to obtain the prefix.

@item library_directory(+@var{D})
@findex library_directory/1
@snindex library_directory/1
@cnindex library_directory/1
Succeeds when @var{D} is a current library directory name. Library
directories are the places where files specified in the form
@code{library(@var{File})} are searched by the predicates
@code{consult/1}, @code{reconsult/1}, @code{use_module/1} or
@code{ensure_loaded/1}.

@item prolog_file_name(+@var{Name},-@var{FullPath})
@findex prolog_file_name/2
@syindex prolog_file_name/1
@cnindex prolog_file_name/2
Unify @var{FullPath} with the absolute path YAP would use to consult
file @var{Name}.

@item public @var{P} [ISO extension]
@findex public/1 (directive)
@snindex public/1 (directive)
@cnindex public/1 (directive)
Instructs the compiler that the source of a predicate of a list of
predicates @var{P} must be kept. This source is then accessible through
the @code{clause/2} procedure and through the @code{listing} family of
built-ins.

Note that all dynamic procedures are public. The @code{source} directive
defines all new or redefined predicates to be public.

Since YAP4.3.0 multifile procedures can be static or dynamic.

@end table

@node Conditional Compilation, Saving, Setting the Compiler, Loading Programs

@section Conditional Compilation 

@c \index{if, directive}%
Conditional compilation builds on the same principle as
@code{term_expansion/2}, @code{goal_expansion/2} and the expansion of
grammar rules to compile sections of the source-code
conditionally. One of the reasons for introducing conditional
compilation is to simplify writing portable code.
@c  See \secref{dialect}
@c for more information. Here is a simple example:

@c @table code
@c :- if(\+source_exports(library(lists), suffix/2)).

@c suffix(Suffix, List) :-
@c 	append(_, Suffix, List).

@c :- endif.
@c \end{code}

Note that these directives can only be appear as separate terms in the
input.  Typical usage scenarios include:

@itemize @bullet
    @item Load different libraries on different dialects
    @item Define a predicate if it is missing as a system predicate
    @item Realise totally different implementations for a particular
    part of the code due to different capabilities.
    @item Realise different configuration options for your software.
@end itemize


@table @code
@item if(+@var{Goal})
@findex if/1 directive
@snindex if/1
@cnindex if/1
Compile subsequent code only if @var{Goal} succeeds.  For enhanced
portability, @var{Goal} is processed by @code{expand_goal/2} before execution.
If an error occurs, the error is printed and processing proceeds as if
@var{Goal} has failed.

@item else
@findex else/0 directive
@snindex else/0
@cnindex else/0
Start `else' branch.

@item endif
@findex endif/0 directive
@snindex endif/0
@cnindex endif/0
End of conditional compilation.

@item elif(+@var{Goal})
@findex elif/1 directive
@snindex elif/1
@cnindex elif/1
Equivalent to @code{:- else. :-if(Goal) ... :- endif.}  In a sequence
as below, the section below the first matching elif is processed, If
no test succeeds the else branch is processed.

@example
:- if(test1).
section_1.
:- elif(test2).
section_2.
:- elif(test3).
section_3.
:- else.
section_else.
:- endif.
@end example

@end table

@node Saving, , Conditional Compilation, Loading Programs
@section Saving and Loading Prolog States

@table @code
@item save(+@var{F})
@findex save/1
@snindex save/1
@cyindex save/1
Saves an image of the current state of YAP in file @var{F}. From
@strong{YAP4.1.3} onwards, YAP saved states are executable
files in the Unix ports.

@item save(+@var{F},-@var{OUT})
@findex save/2
@snindex save/2
@cnindex save/2
Saves an image of the current state of YAP in file @var{F}. From
@strong{YAP4.1.3} onwards, YAP saved states are executable
files in the Unix ports.

Unify @var{OUT} with 1 when saving the file and @var{OUT} with 0 when
restoring the saved state.

@item save_program(+@var{F})
@findex save_program/1
@syindex save_program/1
@cnindex save_program/1
Saves an image of the current state of the YAP database in file
@var{F}.

@item save_program(+@var{F}, :@var{G})
@findex save_program/2
@syindex save_program/2
@cnindex save_program/2
Saves an image of the current state of the YAP database in file
@var{F}, and guarantee that execution of the restored code will start by
trying goal @var{G}.

@item restore(+@var{F})
@findex restore/1
@syindex restore/1
@cnindex restore/1
Restores a previously saved state of YAP from file @var{F}.

YAP always tries to find saved states from the current directory
first. If it cannot it will use the environment variable YAPLIBDIR, if
defined, or search the default library directory.
@end table


@node Modules, Built-ins, Loading Programs, Top
@chapter The Module System

Module systems are quite important for the development of large
applications. YAP implements a module system compatible with the Quintus
Prolog module system.

The YAP module system is predicate-based. This means a module consists
of a set of predicates (or procedures), such that some predicates are
public and the others are local to a module. Atoms and terms in general
are global to the system. Moreover, the module system is flat, meaning
that we do not support a hierarchy of modules. Modules can
automatically import other modules, though. For compatibility with other
module systems the YAP module system is non-strict, meaning both that
there is a way to access predicates private to a module and that it
is possible to declare predicates for a module from some other module.

YAP allows one to ignore the module system if one does not want to use
it. Last note that using the module system does not introduce any
significant overheads.

@menu

* Module Concepts:: The Key Ideas in Modules
* Defining Modules:: How To Define a New Module
* Using Modules:: How to Use a Module
* Meta-Predicates in Modules:: How to Handle New Meta-Predicates
* Re-Exporting Modules:: How to Re-export Predicates From Other Modules

@end menu

@node Module Concepts, Defining Modules, , Modules
@section Module Concepts

The YAP module system applies to predicates. All predicates belong to a
module. System predicates belong to the module @code{primitives}, and by
default new predicates belong to the module @code{user}. Predicates from
the module @code{primitives} are automatically visible to every module.

Every predicate must belong to a module. This module is called its
@emph{source module}.

By default, the source module for a clause occurring in a source file
with a module declaration is the declared module. For goals typed in 
a source file without module declarations, their module is the module
the file is being loaded into. If no module declarations exist, this is
the current @emph{type-in module}. The default type-in module is
@code{user}, but one can set the current module by using the built-in
@code{module/1}.

Note that in this module system one can explicitly specify the source
mode for a clause by prefixing a clause with its module, say:
@example
user:(a :- b).
@end example
@noindent
In fact, to specify the source module for a clause it is sufficient to
specify the source mode for the clause's head:
@example
user:a :- b.
@end example
@noindent

The rules for goals are similar. If a goal appears in a text file with a
module declaration, the goal's source module is the declared
module. Otherwise, it is the module the file is being loaded into or the
type-in module.

One can override this rule by prefixing a goal with the module it is
supposed to be executed in, say:
@example
nasa:launch(apollo,13).
@end example
will execute the goal @code{launch(apollo,13)} as if the current source
module was @code{nasa}.

Note that this rule breaks encapsulation and should be used with care.

@node Defining Modules, Using Modules, Module Concepts, Modules
@section Defining a New Module

A new module is defined by a @code{module} declaration:

@table @code

@item module(+@var{M},+@var{L})
@findex module/2 (directive)
@syindex module/2 (directive)
@cnindex module/2 (directive)
This directive defines the file where it appears as a module file; it
must be the first declaration in the file.
@var{M} must be an atom specifying the module name; @var{L} must be a list
containing the module's public predicates specification, in the form
@code{[predicate_name/arity,...]}.

The public predicates of a module file can be made accessible by other
files through the directives @code{use_module/1}, @code{use_module/2},
@code{ensure_loaded/1} and the predicates @code{consult/1} or
@code{reconsult/1}. The non-public predicates
of a module file are not visible by other files; they can, however, be
accessed by prefixing the module name with the
@code{:/2} operator.

@end table

The built-in @code{module/1} sets the current source module:
@table @code

@item module(+@var{M},+@var{L}, +@var{Options})
@findex module/3 (directive)
@syindex module/3 (directive)
@cnindex module/3 (directive)
Similar to @code{module/2}, this directive defines the file where it
appears in as a module file; it must be the first declaration in the file.
@var{M} must be an atom specifying the module name; @var{L} must be a
list containing the module's public predicates specification, in the
form @code{[predicate_name/arity,...]}.

The last argument @var{Options} must be a list of options, which can be:

@table @code
@item filename
 the filename for a module to import into the current module.

@item library(file)
 a library file to import into the current module.

@item hide(@var{Opt})
 if @var{Opt} is @code{false}, keep source code for current module, if
@code{true}, disable.
@end table

@item module(+@var{M})
@findex module/1
@syindex module/1
@cnindex module/1
Defines @var{M} to be the current working or type-in module. All files
which are not bound to a module are assumed to belong to the working
module (also referred to as type-in module). To compile a non-module
file into a module which is not the working one, prefix the file name
with the module name, in the form @code{@var{Module}:@var{File}}, when
loading the file.

@end table

@node Using Modules, Meta-Predicates in Modules, Defining Modules, Modules
@section Using Modules

By default, all procedures to consult a file will load the modules
defined therein. The two following declarations allow one to import a
module explicitly. They differ on whether one imports all predicate
declared in the module or not.

@table @code

@item use_module(+@var{F})
@findex use_module/1
@syindex use_module/1
@cnindex use_module/1
Loads the files specified by @var{F}, importing all their public
predicates. Predicate name clashes are resolved by asking the user about
importing or not the predicate. A warning is displayed when @var{F} is
not a module file.

@item use_module(+@var{F},+@var{L})
@findex use_module/2
@syindex use_module/2
@cnindex use_module/2
Loads the files specified by @var{F}, importing the predicates specified
in the list @var{L}. Predicate name clashes are resolved by asking the
user about importing or not the predicate. A warning is displayed when
@var{F} is not a module file.

@item use_module(?@var{M},?@var{F},+@var{L})
@findex use_module/3
@syindex use_module/3
@cnindex use_module/3
If module @var{M} has been defined, import the procedures in @var{L} to
the current module. Otherwise, load the files specified by @var{F},
importing the predicates specified in the list @var{L}. 
@end table


@node Meta-Predicates in Modules, Re-Exporting Modules, Using Modules, Modules
@section Meta-Predicates in Modules

The module system must know whether predicates operate on goals or
clauses. Otherwise, such predicates would call a goal in the module they
were defined, instead of calling it in the module they are currently
executing. So, for instance, consider a file example.pl:
@example
:- module(example,[a/1]).

a(G) :- call(G)
@end example

We import this module with @code{use_module(example)} into module
@code{user}.  The expected behavior for a goal @code{a(p)} is to
execute goal @code{p} within the module @code{user}. However,
@code{a/1} will call @code{p} within module @code{example}.

The @code{meta_predicate/1} declaration informs the system that some
arguments of a predicate are goals, clauses, clauses heads or other
terms related to a module, and that these arguments must be prefixed
with the current source module:

@table @code

@item meta_predicate @var{G1},....,@var{Gn}
@findex meta_predicate/1 (directive)
@syindex meta_predicate/1 (directive)
@cnindex meta_predicate/1 (directive)
Each @var{Gi} is a mode specification.

If the argument is @code{:} or an integer, the argument is a call and
must be expanded. Otherwise, the argument is not expanded. Note
that the system already includes declarations for all built-ins.

For example, the declaration for @code{call/1} and @code{setof/3} are:

@example
:- meta_predicate call(:), setof(?,:,?).
@end example

@end table

The previous example is expanded to the following code which explains,
why the goal @code{a(p)} calls @code{p} in @code{example} and not in
@code{user}.  The goal @code{call(G)} is expanded because of the
meta-predicate declaration for @code{call/1}.

@example
:- module(example,[a/1]).

a(G) :- call(example:G)
@end example

By adding a meta-predicate declaration for @code{a/1}, the goal
@code{a(p)} in module user will be expanded to @code{a(user:p)}
thereby preserving the module information.

@example
:- module(example,[a/1]).

:- meta_predicate a(:).
a(G) :- call(G)
@end example

An alternate mechanism is the directive @code{module_transparent/1}
offered for compatibility with SWI-Prolog.

@table @code

@item module_transparent +@var{Preds}
@findex module_transparent/1 (directive)
@syindex module_transparent/1 (directive)
@cnindex module_transparent/1 (directive)
     @var{Preds} is a comma separated sequence of name/arity predicate
     indicators (like
    @code{dynamic/1}). Each goal associated with a transparent declared
    predicate will inherit the context module from its parent goal.
@end table


@node Re-Exporting Modules, , Meta-Predicates in Modules, Modules
@section Re-Exporting Predicates From Other Modules

It is sometimes convenient to re-export predicates originally defined in
a different module. This is often useful if you are adding to the
functionality of a module, or if you are composing a large module with
several small modules. The following declarations can be used for that purpose:

@table @code

@item reexport(+@var{F})
@findex reexport/1
@snindex reexport/1
@cnindex reexport/1
Export all predicates defined in file @var{F} as if they were defined in
the current module.

@item reexport(+@var{F},+@var{Decls})
@findex reexport/2
@snindex reexport/2
@cnindex reexport/2
Export predicates defined in file @var{F} according to @var{Decls}. The
declarations may be of the form:
@itemize @bullet
@item A list of predicate declarations to be exported. Each declaration
may be a predicate indicator or of the form ``@var{PI} @code{as}
@var{NewName}'', meaning that the predicate with indicator @var{PI} is
to be exported under name @var{NewName}.
@item @code{except}(@var{List}) 
In this case, all predicates not in @var{List} are exported. Moreover,
if ``@var{PI} @code{as} @var{NewName}'' is found, the predicate with
indicator @var{PI} is to be exported under name @var{NewName}@ as
before.
@end itemize
@end table

Re-exporting predicates must be used with some care. Please, take into
account the following observations:

@itemize @bullet
@item
The @code{reexport} declarations must be the first declarations to
follow the  @code{module} declaration.
@item
It is possible to use both @code{reexport} and @code{use_module}, but
all predicates reexported are automatically available for use in the
current module.
@item
In order to obtain efficient execution, YAP compiles dependencies
between re-exported predicates. In practice, this means that changing a
@code{reexport} declaration and then @strong{just} recompiling the file
may result in incorrect execution.
@end itemize


@node Built-ins, Library, Modules, Top

@chapter Built-In Predicates

@menu

Built-ins, Debugging, Syntax, Top
* Control:: Controlling the Execution of Prolog Programs
* Undefined Procedures:: Handling calls to Undefined Procedures
* Messages:: Message Handling in YAP
* Testing Terms:: Predicates on Terms
* Predicates on Atoms:: Manipulating Atoms
* Predicates on Characters:: Manipulating Characters
* Comparing Terms:: Comparison of Terms
* Arithmetic:: Arithmetic in YAP
* I/O:: Input/Output with YAP
* Database:: Modifying Prolog's Database
* Sets:: Finding All Possible Solutions
* Grammars:: Grammar Rules
* Preds:: Predicate Information
* OS:: Access to Operating System Functionality
* Term Modification:: Updating Prolog Terms
* Global Variables:: Manipulating Global Variables
* Profiling:: Profiling Prolog Execution
* Call Counting:: Limiting the Maximum Number of Reductions
* Arrays:: Supporting Global and Local Arrays
* Preds:: Information on Predicates
* Misc:: Miscellaneous Predicates

@end menu

@node Control, Undefined Procedures, , Top
@section Control Predicates


This chapter describes the predicates for controlling the execution of
Prolog programs.

In the description of the arguments of functors the following notation
will be used:

@itemize @bullet
@item
a preceding plus sign will denote an argument as an "input argument" -
it cannot be a free variable at the time of the call; 
@item
 a preceding minus sign will denote an "output argument";
@item
an argument with no preceding symbol can be used in both ways.
@end itemize


@table @code

@item +@var{P}, +@var{Q} [ISO]
@findex ,/2
@syindex ,/2
@cyindex ,/2
Conjunction of goals (and).

@noindent
Example:
@example
 p(X) :- q(X), r(X).
@end example

@noindent
should be read as "p(@var{X}) if q(@var{X}) and r(@var{X})".

@item +@var{P} ; +@var{Q} [ISO]
@findex ;/2 
@syindex ;/2 
@cyindex ;/2 
Disjunction of goals (or).

@noindent
Example:
@example
 p(X) :- q(X); r(X).
@end example
@noindent
should be read as "p(@var{X}) if q(@var{X}) or r(@var{X})".

@item true [ISO]
@findex true/0
@syindex true/0
@cyindex true/0
Succeeds once.

@item fail [ISO]
@findex fail/0
@syindex fail/0
@cyindex fail/0
Fails always.

@item false
@findex false/0
@syindex false/0
@cnindex false/0
The same as fail

@item ! [ISO]
@findex !/0
@syindex !/0
@cyindex !/0
  Read as "cut". Cuts any choices taken in the current procedure.
When first found "cut" succeeds as a goal, but if backtracking should
later return to it, the parent goal (the one which matches the head of
the clause containing the "cut", causing the clause activation) will
fail. This is an extra-logical predicate and cannot be explained in
terms of the declarative semantics of Prolog.

example:

@example
 member(X,[X|_]).
 member(X,[_|L]) :- member(X,L).
@end example

@noindent
With the above definition

@example
 ?- member(X,[1,2,3]).
@end example

@noindent
will return each element of the list by backtracking. With the following
definition:

@example
 member(X,[X|_]) :- !.
 member(X,[_|L]) :- member(X,L).
@end example

@noindent
the same query would return only the first element of the 
list, since backtracking could not "pass through" the cut.

@item \+ +@var{P} [ISO]
@findex \+/1
@syindex \+/1
@cyindex \+/1
Goal @var{P} is not provable. The execution of this predicate fails if
and only if the goal @var{P} finitely succeeds. It is not a true logical
negation, which is impossible in standard Prolog, but
"negation-by-failure".

@noindent
This predicate might be defined as:
@example
 \+(P) :- P, !, fail.
 \+(_).
@end example
@noindent
if @var{P} did not include "cuts".

@item not +@var{P}
@findex not/1
@snindex not/1
@cyindex not/1
Goal @var{P} is not provable. The same as @code{'\+ @var{P}'}.

This predicate is kept for compatibility with C-Prolog and previous
versions of YAP. Uses of @code{not/1} should be replace by
@code{(\+)/1}, as YAP does not implement true negation.

@item  +@var{P} -> +@var{Q} [ISO]
@findex ->/2
@syindex ->/2
@cnindex ->/2
Read as "if-then-else" or "commit". This operator is similar to the
conditional operator of imperative languages and can be used alone or
with an else part as follows:

@table @code
@item +P -> +Q
"if P then Q".
@item +P -> +Q; +R
"if P then Q else R".
@end table

@noindent
These two predicates could be defined respectively in Prolog as:
@example
 (P -> Q) :- P, !, Q.
@end example
@noindent
and
@example
 (P -> Q; R) :- P, !, Q.
 (P -> Q; R) :- R.
@end example
@noindent
if there were no "cuts" in @var{P}, @var{Q} and @var{R}.

Note that the commit operator works by "cutting" any alternative
solutions of @var{P}.

Note also that you can use chains of commit operators like:
@example
    P -> Q ; R -> S ; T.
@end example
@noindent
Note that @code{(->)/2} does not affect the scope of cuts in its
arguments.

@item  +@var{Conditon} *-> +@var{Action} ; +@var{Else}
@findex ->*/2
@snindex ->*/2
@cnindex ->*/2
This construct implements the so-called @emph{soft-cut}. The control is
    defined as follows: If @var{Condition} succeeds at least once, the
    semantics is the same as (@var{Condition}, @var{Action}). If
    @var{Condition} does not succeed, the semantics is that of (\+
    @var{Condition}, @var{Else}). In other words, If @var{Condition}
    succeeds at least once, simply behave as the conjunction of
    @var{Condition} and @var{Action}, otherwise execute @var{Else}.

    The construct @var{A *-> B}, i.e. without an @var{Else} branch, is
translated as the normal conjunction @var{A}, @var{B}.

@item repeat [ISO]
@findex repeat/0
@syindex repeat/0
@cyindex repeat/0
Succeeds repeatedly.
 
In the next example, @code{repeat} is used as an efficient way to implement
a loop. The next example reads all terms in a file:

@example
 a :- repeat, read(X), write(X), nl, X=end_of_file, !.
@end example
@noindent
the loop is effectively terminated by the cut-goal, when the test-goal
@code{X=end} succeeds. While the test fails, the goals @code{read(X)},
@code{write(X)}, and @code{nl} are executed repeatedly, because
backtracking is caught by the @code{repeat} goal.

The built-in @code{repeat/1} could be defined in Prolog by:
@example
 repeat.
 repeat :- repeat.
@end example

@item call(+@var{P}) [ISO]
@findex call/1
@syindex call/1
@cyindex call/1
 If @var{P} is instantiated to an atom or a compound term, the goal
@code{call(@var{P})} is executed as if the value of @code{P} was found
instead of the call to @code{call/1}, except that any "cut" occurring in
@var{P} only cuts alternatives in the execution of @var{P}.

@item incore(+@var{P})
@findex incore/1
@syindex incore/1
@cnindex incore/1
The same as @code{call/1}.

@item call(+@var{Closure},...,?@var{Ai},...)
@findex call/n
@snindex call/n
@cnindex call/n
Meta-call where @var{Closure} is a closure that is converted into a goal by 
appending the @var{Ai} additional arguments. The number of arguments varies 
between 0 and 10.

@item call_with_args(+@var{Name},...,?@var{Ai},...)
@findex call_with_args/n
@snindex call_with_args/n
@cnindex call_with_args/n
Meta-call where @var{Name} is the name of the procedure to be called and
the @var{Ai} are the arguments. The number of arguments varies between 0
and 10.

If @var{Name} is a complex term, then @code{call_with_args/n} behaves as
@code{call/n}:

@example
call(p(X1,...,Xm), Y1,...,Yn) :- p(X1,...,Xm,Y1,...,Yn).
@end example


@item +@var{P}
 The same as @code{call(@var{P})}. This feature has been kept to provide
compatibility with C-Prolog. When compiling a goal, YAP
generates a @code{call(@var{X})} whenever a variable @var{X} is found as
a goal.

@example
 a(X) :- X.
@end example
@noindent
is converted to:
@example
 a(X) :- call(X).
@end example

@item if(?@var{G},?@var{H},?@var{I})
@findex if/3
@syindex if/3
@cnindex if/3
Call goal @var{H} once per each solution of goal @var{H}. If goal
@var{H} has no solutions, call goal @var{I}.

The built-in @code{if/3} is similar to @code{->/3}, with the difference
that it will backtrack over the test goal. Consider the following
small data-base:

@example
a(1).        b(a).          c(x).
a(2).        b(b).          c(y).
@end example

Execution of an @code{if/3} query will proceed as follows:

@example
   ?- if(a(X),b(Y),c(Z)).

X = 1,
Y = a ? ;

X = 1,
Y = b ? ;

X = 2,
Y = a ? ;

X = 2,
Y = b ? ;

no
@end example


@noindent
The system will backtrack over the two solutions for @code{a/1} and the
two solutions for @code{b/1}, generating four solutions.

Cuts are allowed inside the first goal @var{G}, but they will only prune
over @var{G}.

If you want @var{G} to be deterministic you should use if-then-else, as
it is both more efficient and more portable.

@item once(:@var{G}) [ISO]
@findex once/1
@snindex once/1
@cnindex once/1
Execute the goal @var{G} only once. The predicate is defined by:

@example
 once(G) :- call(G), !.
@end example

@noindent
Note that cuts inside @code{once/1} can only cut the other goals inside
@code{once/1}.

@item forall(:@var{Cond},:@var{Action})
@findex forall/2
@snindex forall/2
@cnindex forall/2
For all alternative bindings of @var{Cond} @var{Action} can be
proven. The example verifies that all arithmetic statements in the list
@var{L} are correct. It does not say which is wrong if one proves wrong.

@example
?- forall(member(Result = Formula, [2 = 1 + 1, 4 = 2 * 2]),
                 Result =:= Formula).
@end example

@item ignore(:@var{Goal})
@findex ignore/1
@snindex ignore/1
@cnindex ignore/1
Calls @var{Goal} as @code{once/1}, but succeeds, regardless of whether
@code{Goal} succeeded or not. Defined as:

@example
ignore(Goal) :-
        Goal, !.
ignore(_).
@end example

@item abort
@findex abort/0
@syindex abort/0
@cyindex abort/0
Abandons the execution of the current goal and returns to top level. All
break levels (see @code{break/0} below) are terminated. It is mainly
used during debugging or after a serious execution error, to return to
the top-level.


@item break
@findex break/0
@syindex break/0
@cyindex break/0
Suspends the execution of the current goal and creates a new execution
level similar to the top level, displaying the following message:

@example
 [ Break (level <number>) ]
@end example
@noindent
telling the depth of the break level just entered. To return to the
previous level just type the end-of-file character or call the
end_of_file predicate.  This predicate is especially useful during
debugging.

@item halt [ISO]
@findex halt/0
@syindex halt/0
@cyindex halt/0
Halts Prolog, and exits to the calling application. In YAP,
@code{halt/0} returns the exit code @code{0}.

@item halt(+ @var{I}) [ISO]
@findex halt/1
@syindex halt/1
@cnindex halt/1
Halts Prolog, and exits to the calling application returning the code
given by the integer @var{I}.

@item catch(+@var{Goal},+@var{Exception},+@var{Action}) [ISO]
@findex catch/3
@snindex catch/3
@cnindex catch/3
The goal @code{catch(@var{Goal},@var{Exception},@var{Action})} tries to
execute goal @var{Goal}. If during its execution, @var{Goal} throws an
exception @var{E'} and this exception unifies with @var{Exception}, the
exception is considered to be caught and @var{Action} is executed. If
the exception @var{E'} does not unify with @var{Exception}, control
again throws the exception.

The top-level of YAP maintains a default exception handler that
is responsible to capture uncaught exceptions.

@item throw(+@var{Ball}) [ISO]
@findex throw/1
@snindex throw/1
@cnindex throw/1
The goal @code{throw(@var{Ball})} throws an exception. Execution is
stopped, and the exception is sent to the ancestor goals until reaching
a matching @code{catch/3}, or until reaching top-level.

@item garbage_collect
@findex garbage_collect/0
@syindex garbage_collect/0
@cnindex garbage_collect/0
The goal @code{garbage_collect} forces a garbage collection.

@item garbage_collect_atoms
@findex garbage_collect_atoms/0
@syindex garbage_collect_atoms/0
@cnindex garbage_collect_atoms/0
The goal @code{garbage_collect} forces a garbage collection of the atoms
in the data-base. Currently, only atoms are recovered.

@item gc
@findex gc/0
@syindex gc/0
@cnindex gc/0
The goal @code{gc} enables garbage collection. The same as
@code{yap_flag(gc,on)}.

@item nogc
@findex nogc/0
@syindex nogc/0
@cnindex nogc/0
The goal @code{nogc} disables garbage collection. The same as
@code{yap_flag(gc,off)}.

@item grow_heap(+@var{Size})
@snindex grow_heap/1
@cnindex grow_heap/1
Increase heap size @var{Size} kilobytes.

@item grow_stack(+@var{Size})
@findex grow_stack/1
@snindex grow_stack/1
@cnindex grow_stack/1
Increase stack size @var{Size} kilobytes.

@end table

@node Undefined Procedures, Messages, Control, Top
@section Handling Undefined Procedures

A predicate in a module is said to be undefined if there are no clauses
defining the predicate, and if the predicate has not been declared to be
dynamic. What YAP does when trying to execute undefined predicates can
be specified in three different ways:
@itemize @bullet
@item By setting an YAP flag, through the @code{yap_flag/2} or
@code{set_prolog_flag/2} built-ins. This solution generalizes the
ISO standard.
@item By using the @code{unknown/2} built-in (this solution is
compatible with previous releases of YAP).
@item By defining clauses for the hook predicate
@code{user:unknown_predicate_handler/3}. This solution is compatible
with SICStus Prolog.
@end itemize

In more detail:
@table @code
@item unknown(-@var{O},+@var{N})
@findex unknown/2
@saindex unknown/2
@cnindex unknown/2
Specifies an handler to be called is a program tries to call an
undefined static procedure @var{P}.

The arity of @var{N} may be zero or one. If the arity is @code{0}, the
new action must be one of @code{fail}, @code{warning}, or
@code{error}. If the arity is @code{1}, @var{P} is an user-defined
handler and at run-time, the argument to the handler @var{P} will be
unified with the undefined goal. Note that @var{N} must be defined prior
to calling @code{unknown/2}, and that the single argument to @var{N} must
be unbound.

In YAP, the default action is to @code{fail} (note that in the ISO
Prolog standard the default action is @code{error}).

After defining @code{undefined/1} by:
@example
undefined(A) :- format('Undefined predicate: ~w~n',[A]), fail.
@end example
@noindent
and executing the goal:
@example
unknown(U,undefined(X)).
@end example
@noindent
a call to a predicate for which no clauses were defined will result in
the output of a message of the form:
@example
Undefined predicate: user:xyz(A1,A2)
@end example
@noindent
followed by the failure of that call.

@item yap_flag(unknown,+@var{SPEC})
Alternatively, one can use @code{yap_flag/2},
@code{current_prolog_flag/2}, or @code{set_prolog_flag/2}, to set this
functionality. In this case, the first argument for the built-ins should
be @code{unknown}, and the second argument should be either
@code{error}, @code{warning}, @code{fail}, or a goal.

@item user:unknown_predicate_handler(+G,+M,?NG)
@findex unknown_predicate_handler/3
@syindex unknown_predicate_handler/3
@cnindex unknown_predicate_handler/3
The user may also define clauses for
@code{user:unknown_predicate_handler/3} hook predicate. This
user-defined procedure is called before any system processing for the
undefined procedure, with the first argument @var{G} set to the current
goal, and the second @var{M} set to the current module. The predicate
@var{G} will be called from within the user module.

If @code{user:unknown_predicate_handler/3} succeeds, the system will
execute @var{NG}. If  @code{user:unknown_predicate_handler/3} fails, the
system will execute default action as specified by @code{unknown/2}.

@item exception(+@var{Exception}, +@var{Context}, -@var{Action})
@findex exception/3
@syindex exception/3
@cnindex exception/3
 Dynamic predicate, normally not defined. Called by the Prolog system on run-time exceptions that can be repaired `just-in-time'. The values for @var{Exception} are described below. See also @code{catch/3} and @code{throw/1}.
If this hook predicate succeeds it must instantiate the @var{Action} argument to the atom @code{fail} to make the operation fail silently, @code{retry} to tell Prolog to retry the operation or @code{error} to make the system generate an exception. The action @code{retry} only makes sense if this hook modified the environment such that the operation can now succeed without error.

@table @code
@item undefined_predicate
@var{Context} is instantiated to a predicate-indicator (@var{Module:Name/Arity}). If the predicate fails Prolog will generate an existence_error exception. The hook is intended to implement alternatives to the SWI built-in autoloader, such as autoloading code from a database. Do not use this hook to suppress existence errors on predicates. See also @code{unknown}.
@item undefined_global_variable
@var{Context} is instantiated to the name of the missing global variable. The hook must call @code{nb_setval/2} or @code{b_setval/2} before returning with the action retry.
@end table

@end table

@node Messages, Testing Terms, Undefined Procedures, Top
@section Message Handling

The interaction between YAP and the user relies on YAP's ability to
portray messages. These messages range from prompts to error
information. All message processing is performed through the builtin
@code{print_message/2}, in two steps:

@itemize @bullet
@item The message is processed into a list of commands 
@item The commands in the list are sent to the @code{format/3} builtin
in sequence.
@end itemize 

The first argument to @code{print_message/2} specifies the importance of
the message. The options are:

@table @code
@item error
error handling
@item warning
compilation and run-time warnings,
@item informational
generic informational messages
@item help 
help messages (not currently implemented in YAP)
@item query
query 	used in query processing (not currently implemented in YAP)
@item  silent
messages that do not produce output but that can be intercepted by hooks.
@end table

The next table shows the main predicates and hooks associated to message
handling in YAP:
@table @code
@item print_message(+@var{Kind}, @var{Term})
@findex print_message/2
@syindex print_message/2
@cnindex print_message/2
The predicate print_message/2 is used to print messages, notably from
exceptions in a human-readable format. @var{Kind} is one of
@code{informational}, @code{banner}, @code{warning}, @code{error},
@code{help} or @code{silent}. A human-readable message is printed to
the stream @code{user_error}.

@c \index{silent}\index{quiet}%
If the Prolog flag @code{verbose} is @code{silent}, messages with
@var{Kind} @code{informational}, or @code{banner} are treated as
silent.@c  See \cmdlineoption{-q}.

This predicate first translates the @var{Term} into a list of `message
lines' (see @code{print_message_lines/3} for details).  Next it will
call the hook @code{message_hook/3} to allow the user intercepting the
message.  If @code{message_hook/3} fails it will print the message unless
@var{Kind} is silent.

@c The print_message/2 predicate and its rules are in the file
@c \file{<plhome>/boot/messages.pl}, which may be inspected for more
@c information on the error messages and related error terms. 
If you need to report errors from your own predicates, we advise you to
stick to the existing error terms if you can; but should you need to
invent new ones, you can define corresponding error messages by
asserting clauses for @code{prolog:message/2}. You will need to declare
the predicate as multifile.

@c See also message_to_string/2.

@item print_message_lines(+@var{Stream}, +@var{Prefix}, +@var{Lines})
@findex print_message_lines/3
@syindex print_message_lines/3
@cnindex print_message_lines/3
Print a message (see @code{print_message/2}) that has been translated to
a list of message elements.  The elements of this list are:

@table @code
    @item @code{<Format>}-@code{<Args>}
        Where @var{Format} is an atom and @var{Args} is a list
	of format argument.  Handed to @code{format/3}.
    @item @code{flush}
	If this appears as the last element, @var{Stream} is flushed
	(see @code{flush_output/1}) and no final newline is generated.
    @item @code{at_same_line}
        If this appears as first element, no prefix is printed for
	the first line and the line-position is not forced to 0
	(see @code{format/1}, @code{~N}).
    @item @code{<Format>}
        Handed to @code{format/3} as @code{format(Stream, Format, [])}.
    @item nl
        A new line is started and if the message is not complete
	the @var{Prefix} is printed too.
@end table

@item user:message_hook(+@var{Term}, +@var{Kind}, +@var{Lines})
@findex message_hook/3
@syindex message_hook/3
@cnindex message_hook/3
Hook predicate that may be define in the module @code{user} to intercept
messages from @code{print_message/2}. @var{Term} and @var{Kind} are the
same as passed to @code{print_message/2}. @var{Lines} is a list of
format statements as described with @code{print_message_lines/3}.

This predicate should be defined dynamic and multifile to allow other
modules defining clauses for it too.

@item message_to_string(+@var{Term}, -@var{String})
@findex message_to_string/2
@snindex message_to_string/2
@cnindex message_to_string/2 
Translates a message-term into a string object. Primarily intended for SWI-Prolog emulation.
@end table

@node Testing Terms, Predicates on Atoms, Messages, Top
@section Predicates on terms

@table @code

@item var(@var{T}) [ISO]
@findex var/1
@syindex var/1
@cyindex var/1
Succeeds if @var{T} is currently a free variable, otherwise fails. 

@item atom(@var{T}) [ISO]
@findex atom/1
@syindex atom/1
@cyindex atom/1
Succeeds if and only if @var{T} is currently instantiated to an  atom.

@item atomic(T) [ISO]
@findex atomic/1
@syindex atomic/1
@cyindex atomic/1
Checks whether @var{T} is an atomic symbol (atom or number).

@item compound(@var{T}) [ISO]
@findex compound/1
@syindex compound/1
@cnindex compound/1
Checks whether @var{T} is a compound term.

@item db_reference(@var{T})
@findex db_reference/1C
@syindex db_reference/1
@cyindex db_reference/1
Checks whether @var{T} is a database reference.

@item float(@var{T}) [ISO]
@findex float/1
@syindex float/1
@cnindex float/1
Checks whether @var{T} is a floating point number.

@item rational(@var{T}) [ISO]
@findex rational/1
@syindex rational/1
@cyindex rational/1
Checks whether @code{T} is a rational number.

@item integer(@var{T}) [ISO]
@findex integer/1
@syindex integer/1
@cyindex integer/1
Succeeds if and only if @var{T} is currently instantiated to an  integer.

@item nonvar(@var{T}) [ISO]
@findex nonvar/1
@syindex nonvar/1
@cyindex nonvar/1
The opposite of @code{var(@var{T})}.

@item number(@var{T}) [ISO]
@findex number/1
@syindex number/1
@cyindex number/1
Checks whether @code{T} is an integer, rational or a float.

@item primitive(@var{T})
@findex primitive/1
@syindex primitive/1
@cyindex primitive/1
Checks whether @var{T} is an atomic term or a database reference.

@item simple(@var{T})
@findex simple/1
@syindex simple/1
@cnindex simple/1
Checks whether @var{T} is unbound, an atom, or a number.

@item callable(@var{T})
@findex callable/1
@syindex callable/1
@cnindex callable/1
Checks whether @var{T} is a callable term, that is, an atom or a
compound term.

@item numbervars(@var{T},+@var{N1},-@var{Nn})
@findex numbervars/3
@syindex numbervars/3
@cnindex numbervars/3
Instantiates each variable in term @var{T} to a term of the form:
@code{'$VAR'(@var{I})}, with @var{I} increasing from @var{N1} to @var{Nn}.

@item ground(@var{T})
@findex ground/1
@syindex ground/1
@cnindex ground/1
Succeeds if there are no free variables in the term @var{T}.

@item arg(+@var{N},+@var{T},@var{A}) [ISO]
@findex arg/3
@syindex arg/3
@cnindex arg/3
Succeeds if the argument @var{N} of the term @var{T} unifies with
@var{A}. The arguments are numbered from 1 to the arity of the term.

The current version will generate an error if @var{T} or @var{N} are
unbound, if @var{T} is not a compound term, of if @var{N} is not a positive
integer. Note that previous versions of YAP would fail silently
under these errors.

@item functor(@var{T},@var{F},@var{N}) [ISO]
@findex functor/3
@syindex functor/3
@cyindex functor/3
The top functor of term @var{T} is named @var{F} and has  arity @var{N}.

When @var{T} is not instantiated, @var{F} and @var{N} must be. If
@var{N} is 0, @var{F} must be an atomic symbol, which will be unified
with @var{T}. If @var{N} is not 0, then @var{F} must be an atom and
@var{T} becomes instantiated to the most general term having functor
@var{F} and arity @var{N}. If @var{T} is instantiated to a term then
@var{F} and @var{N} are respectively unified with its top functor name
and arity.

In the current version of YAP the arity @var{N} must be an
integer. Previous versions allowed evaluable expressions, as long as the
expression would evaluate to an integer. This feature is not available
in the ISO Prolog standard.

@item @var{T} =.. @var{L} [ISO]
@findex =../2
@syindex =../2
@cyindex =../2
The list @var{L} is built with the functor and arguments of the term
@var{T}. If @var{T} is instantiated to a variable, then @var{L} must be
instantiated either to a list whose head is an atom, or to a list
consisting of just a number.

@item @var{X} = @var{Y} [ISO]
@findex =/2
@syindex =/2
@cnindex =/2
Tries to unify terms @var{X} and @var{Y}.

@item @var{X} \= @var{Y} [ISO]
@findex \=/2
@snindex \=/2
@cnindex \=/2
Succeeds if terms @var{X} and @var{Y} are not unifiable.

@item unify_with_occurs_check(?T1,?T2) [ISO]
@findex unify_with_occurs_check/2
@syindex unify_with_occurs_check/2
@cnindex unify_with_occurs_check/2
Obtain the most general unifier of terms @var{T1} and @var{T2}, if there
is one.

This predicate implements the full unification algorithm. An example:n
@example
unify_with_occurs_check(a(X,b,Z),a(X,A,f(B)).
@end example
@noindent
will succeed with the bindings @code{A = b} and @code{Z = f(B)}. On the
other hand:
@example
unify_with_occurs_check(a(X,b,Z),a(X,A,f(Z)).
@end example
@noindent
would fail, because @code{Z} is not unifiable with @code{f(Z)}. Note that
@code{(=)/2} would succeed for the previous examples, giving the following
bindings @code{A = b} and @code{Z = f(Z)}.


@item copy_term(?@var{TI},-@var{TF}) [ISO]
@findex copy_term/2
@syindex copy_term/2
@cnindex copy_term/2
Term @var{TF} is a variant of the original term @var{TI}, such that for
each variable @var{V} in the term @var{TI} there is a new variable @var{V'}
in term @var{TF}. Notice that:

@itemize @bullet
@item suspended goals and attributes for attributed variables in
  @var{TI} are also duplicated;
@item ground terms are shared between the new and the old term.
@end itemize

If you do not want any sharing to occur please use
@code{duplicate_term/2}.

@item duplicate_term(?@var{TI},-@var{TF})
@findex duplicate_term/2
@syindex duplicate_term/2
@cnindex duplicate_term/2
Term @var{TF} is a variant of the original term @var{TI}, such that
for each variable @var{V} in the term @var{TI} there is a new variable
@var{V'} in term @var{TF}, and the two terms do not share any
structure. All suspended goals and attributes for attributed variables
in @var{TI} are also duplicated.

Also refer to @code{copy_term/2}.

@item is_list(+@var{List})
@findex is_list/1
@syindex is_list/1
@cnindex is_list/1
True when @var{List} is a proper list. That is, @var{List}
is bound to the empty list (nil) or a term with functor '.' and arity 2.

@item ?@var{Term1} =@= ?@var{Term2}
@findex  =@=/2
@syindex =@=/2
@cnindex =@=/2

Same as @code{variant/2}, succeeds if @var{Term1} and @var{Term2} are variant terms.


@item subsumes_term(?@var{Subsumer}, ?@var{Subsumed})
@findex  subsumes_term/2
@syindex subsumes_term/2
@cnindex subsumes_term/2

Succeed if @var{Submuser} subsumes @var{Subsuned} but does not bind any
variable in @var{Subsumer}.

@item acyclic_term(?@var{Term})
@findex cyclic_term/1
@syindex cyclic_term/1
@cnindex cyclic_term/1
Succeed if the argument @var{Term} is an acyclic term.


@end table

@node Predicates on Atoms, Predicates on Characters, Testing Terms, Top
@section Predicates on Atoms

The following predicates are used to manipulate atoms:

@table @code
@item name(@var{A},@var{L})
@findex name/2
@syindex name/2
@cyindex name/2
The predicate holds when at least one of the arguments is ground
(otherwise, an error message will be displayed). The argument @var{A} will
be unified with an atomic symbol and @var{L} with the list of the ASCII
codes for the characters of the external representation of @var{A}.

@example
 name(yap,L).
@end example
@noindent
will return:
@example
 L = [121,97,112].
@end example
@noindent
and
@example
 name(3,L).
@end example
@noindent
will return:
@example
 L = [51].
@end example

@item atom_chars(?@var{A},?@var{L}) [ISO]
@findex atom_chars/2
@saindex atom_chars/2
@cnindex atom_chars/2
The predicate holds when at least one of the arguments is ground
(otherwise, an error message will be displayed). The argument @var{A} must
be unifiable with an atom, and the argument @var{L} with the list of the
ASCII codes for the characters of the external representation of @var{A}.

The ISO-Prolog standard dictates that @code{atom_chars/2} should unify
the second argument with a list of one-char atoms, and not the character
codes. For compatibility with previous versions of YAP, and
with other Prolog implementations, YAP unifies the second
argument with the character codes, as in @code{atom_codes/2}. Use the
@code{set_prolog_flag(to_chars_mode,iso)} to obtain ISO standard
compatibility.

@item atom_codes(?@var{A},?@var{L}) [ISO]
@findex atom_codes/2
@syindex atom_codes/2
@cnindex atom_codes/2
The predicate holds when at least one of the arguments is ground
(otherwise, an error message will be displayed). The argument @var{A} will
be unified with an atom and @var{L} with the list of the ASCII
codes for the characters of the external representation of @var{A}.

@item atom_concat(+@var{As},?@var{A})
@findex atom_concat/2
@syindex atom_concat/2
@cnindex atom_concat/2
The predicate holds when the first argument is a list of atoms, and the
second unifies with the atom obtained by concatenating all the atoms in
the first list.

@item atomic_concat(+@var{As},?@var{A})
@findex atomic_concat/2
@snindex atomic_concat/2
@cnindex atomic_concat/2
The predicate holds when the first argument is a list of atomic terms, and
the second unifies with the atom obtained by concatenating all the
atomic terms in the first list. The first argument thus may contain
atoms or numbers.

@item atomic_list_concat(+@var{As},?@var{A})
@findex atomic_list_concat/2
@snindex atomic_list_concat/2
@cnindex atomic_list_concat/2
The predicate holds when the first argument is a list of atomic terms, and
the second unifies with the atom obtained by concatenating all the
atomic terms in the first list. The first argument thus may contain
atoms or numbers.

@item atomic_list_concat(?@var{As},+@var{Separator},?@var{A})
@findex atomic_list_concat/3
@snindex atomic_list_concat/3
@cnindex atomic_list_concat/3
Creates an atom just like @code{atomic_list_concat/2}, but inserts
@var{Separator} between each pair of atoms. For example:

@example
?- atomic_list_concat([gnu, gnat], ', ', A).

A = 'gnu, gnat'
@end example

YAP emulates the SWI-Prolog version of this predicate that can also be
used to split atoms by instantiating @var{Separator} and @var{Atom} as
shown below.

@example
?- atomic_list_concat(L, -, 'gnu-gnat').

L = [gnu, gnat]
@end example

@item atom_length(+@var{A},?@var{I}) [ISO]
@findex atom_length/2
@snindex atom_length/2
@cnindex atom_length/2
The predicate holds when the first argument is an atom, and the second
unifies with the number of characters forming that atom.

@item atom_concat(?@var{A1},?@var{A2},?@var{A12}) [ISO]
@findex atom_concat/3
@snindex atom_concat/3
@cnindex atom_concat/3
The predicate holds when the third argument unifies with an atom, and
the first and second unify with atoms such that their representations
concatenated are the representation for @var{A12}.

If @var{A1} and @var{A2} are unbound, the built-in will find all the atoms
that concatenated give @var{A12}.

@item number_chars(?@var{I},?@var{L})
@findex number_chars/2
@saindex number_chars/2
@cnindex number_chars/2

The predicate holds when at least one of the arguments is ground
(otherwise, an error message will be displayed). The argument @var{I} must
be unifiable with a number, and the argument @var{L} with the list of the
ASCII codes for the characters of the external representation of @var{I}.

The ISO-Prolog standard dictates that @code{number_chars/2} should unify
the second argument with a list of one-char atoms, and not the character
codes. For compatibility with previous versions of YAP, and
with other Prolog implementations, YAP unifies the second
argument with the character codes, as in @code{number_codes/2}. Use the
@code{set_prolog_flag(to_chars_mode,iso)} to obtain ISO standard
compatibility.

@item number_codes(?@var{A},?@var{L}) [ISO]
@findex number_codes/2
@syindex number_codes/2
@cnindex number_codes/2
The predicate holds when at least one of the arguments is ground
(otherwise, an error message will be displayed). The argument @var{A}
will be unified with a number and @var{L} with the list of the ASCII
codes for the characters of the external representation of @var{A}.

@item atom_number(?@var{Atom},?@var{Number})
@findex atom_number/2
@syindex atom_number/2
@cnindex atom_number/2
The predicate holds when at least one of the arguments is ground
(otherwise, an error message will be displayed). If the argument
@var{Atom} is an atom, @var{Number} must be the number corresponding
to the characters in @var{Atom}, otherwise the characters in
@var{Atom} must encode a number @var{Number}.

@item number_atom(?@var{I},?@var{L})
@findex number_atom/2
@snindex number_atom/2
@cnindex number_atom/2

The predicate holds when at least one of the arguments is ground
(otherwise, an error message will be displayed). The argument @var{I} must
be unifiable with a number, and the argument @var{L} must be unifiable
with an atom representing the number.

@item sub_atom(+@var{A},?@var{Bef}, ?@var{Size}, ?@var{After}, ?@var{At_out}) [ISO]
@findex sub_atom/5
@snindex sub_atom/5
@cnindex sub_atom/5
True when @var{A} and @var{At_out} are atoms such that the name of
@var{At_out} has size @var{Size} and is a sub-string of the name of
@var{A}, such that @var{Bef} is the number of characters before and
@var{After} the number of characters afterwards.

Note that @var{A} must always be known, but @var{At_out} can be unbound when
calling this built-in. If all the arguments for @code{sub_atom/5} but @var{A}
are unbound, the built-in will backtrack through all possible
sub-strings of @var{A}.

@end table

@node Predicates on Characters, Comparing Terms, Predicates on Atoms, Top
@section Predicates on Characters

The following predicates are used to manipulate characters:

@table @code
@item char_code(?@var{A},?@var{I}) [ISO]
@findex char_code/2
@syindex char_code/2
@cnindex char_code/2
The built-in succeeds with @var{A} bound to character represented as an
atom, and @var{I} bound to the character code represented as an
integer. At least, one of either @var{A} or @var{I} must be bound before
the call.

@item char_type(?@var{Char}, ?@var{Type})
@findex char_type/2
@snindex char_type/2
@cnindex char_type/2
    Tests or generates alternative @var{Types} or @var{Chars}. The
    character-types are inspired by the standard @code{C}
    @code{<ctype.h>} primitives.

@table @code
@item    alnum
        @var{Char} is a letter (upper- or lowercase) or digit.

@item    alpha
        @var{Char} is a letter (upper- or lowercase).

@item    csym
        @var{Char} is a letter (upper- or lowercase), digit or the underscore (_). These are valid C- and Prolog symbol characters.

@item    csymf
        @var{Char} is a letter (upper- or lowercase) or the underscore (_). These are valid first characters for C- and Prolog symbols

@item    ascii
        @var{Char} is a 7-bits ASCII character (0..127).

@item    white
        @var{Char} is a space or tab. E.i. white space inside a line.

@item    cntrl
        @var{Char} is an ASCII control-character (0..31).

@item    digit
        @var{Char} is a digit.

@item    digit(@var{Weigth})
        @var{Char} is a digit with value
        @var{Weigth}. I.e. @code{char_type(X, digit(6))} yields @code{X =
        '6'}. Useful for parsing numbers.

@item    xdigit(@var{Weigth})
        @var{Char} is a hexa-decimal digit with value @var{Weigth}. I.e. char_type(a, xdigit(X) yields X = '10'. Useful for parsing numbers.

@item    graph
        @var{Char} produces a visible mark on a page when printed. Note that the space is not included!

@item    lower
        @var{Char} is a lower-case letter.

@item    lower(Upper)
        @var{Char} is a lower-case version of @var{Upper}. Only true if
        @var{Char} is lowercase and @var{Upper} uppercase.

@item    to_lower(Upper)
        @var{Char} is a lower-case version of Upper. For non-letters, or letter without case, @var{Char} and Lower are the same. See also upcase_atom/2 and downcase_atom/2.

@item    upper
        @var{Char} is an upper-case letter.

@item    upper(Lower)
        @var{Char} is an upper-case version of Lower. Only true if @var{Char} is uppercase and Lower lowercase.

@item    to_upper(Lower)
        @var{Char} is an upper-case version of Lower. For non-letters, or letter without case, @var{Char} and Lower are the same. See also upcase_atom/2 and downcase_atom/2.

@item    punct
        @var{Char} is a punctuation character. This is a graph character that is not a letter or digit.

@item    space
        @var{Char} is some form of layout character (tab, vertical-tab, newline, etc.).

@item    end_of_file
        @var{Char} is -1.

@item    end_of_line
        @var{Char} ends a line (ASCII: 10..13).

@item    newline
        @var{Char} is a the newline character (10).

@item    period
        @var{Char} counts as the end of a sentence (.,!,?).

@item    quote
        @var{Char} is a quote-character (", ', `).

@item    paren(Close)
        @var{Char} is an open-parenthesis and Close is the corresponding close-parenthesis. 
@end table

@item code_type(?@var{Code}, ?@var{Type})
@findex code_type/2
@snindex code_type/2
@cnindex code_type/2
    As @code{char_type/2}, but uses character-codes rather than
    one-character atoms. Please note that both predicates are as
    flexible as possible. They handle either representation if the
    argument is instantiated and only will instantiate with an integer
    code or one-character atom depending of the version used. See also
    the prolog-flag @code{double_quotes} and the built-in predicates 
    @code{atom_chars/2} and @code{atom_codes/2}.

@end table

@node Comparing Terms, Arithmetic, Predicates on Characters, Top
@section Comparing Terms

The following predicates are used to compare and order terms, using the
standard ordering:

@itemize @bullet
@item
variables come before numbers, numbers come before atoms which in turn
come before compound terms, i.e.: variables @@< numbers @@< atoms @@<
compound terms.
@item
Variables are roughly ordered by "age" (the "oldest" variable is put
first);
@item
Floating point numbers are sorted in increasing order;
@item
Rational numbers are sorted in increasing order;
@item
Integers are sorted in increasing order;
@item
Atoms are sorted in lexicographic order;
@item
Compound terms are ordered first by arity of the main functor, then by
the name of the main functor, and finally by their arguments in
left-to-right order.
@end itemize

@table @code

@item compare(@var{C},@var{X},@var{Y})
@findex compare/3
@syindex compare/3
@cyindex compare/3
As a result of comparing @var{X} and @var{Y}, @var{C} may take one of
the following values:

@itemize @bullet
@item
@code{=} if @var{X} and @var{Y} are identical;
@item
@code{<} if @var{X} precedes @var{Y} in the defined order;
@item
@code{>} if @var{Y} precedes @var{X} in the defined order;
@end itemize

@item @var{X} == @var{Y} [ISO]
@findex ==/2
@syindex ==/2
@cyindex ==/2
Succeeds if terms @var{X} and @var{Y} are strictly identical. The
difference between this predicate and @code{=/2} is that, if one of the
arguments is a free variable, it only succeeds when they have already
been unified.

@example
?- X == Y.
@end example
@noindent
fails, but,
@example
?- X = Y, X == Y.
@end example
@noindent
succeeds.
@example
?- X == 2.
@end example
@noindent
fails, but,
@example
?- X = 2, X == 2.
@end example
@noindent
succeeds.


@item @var{X} \== @var{Y} [ISO]
@findex \==/2
@syindex \==/2
@cyindex \==/2
Terms @var{X} and @var{Y} are not strictly identical.

@item @var{X} @@< @var{Y} [ISO]
@findex @@</2
@syindex @@</2
@cyindex @@</2
Term @var{X} precedes term @var{Y} in the standard order.

@item @var{X} @@=< @var{Y} [ISO]
@findex @@=</2
@syindex @@</2
@cyindex @@</2
Term @var{X} does not follow term @var{Y} in the standard order.

@item @var{X} @@> @var{Y} [ISO]
@findex @@>/2
@syindex @@>/2
@cyindex @@>/2
Term @var{X} follows term @var{Y} in the standard order.

@item @var{X} @@>= @var{Y} [ISO]
@findex @@>=/2
@syindex @@>=/2
@cyindex @@>=/2
Term @var{X} does not precede term @var{Y} in the standard order.

@item sort(+@var{L},-@var{S})
@findex sort/2
@syindex sort/2
@cyindex sort/2
Unifies @var{S} with the list obtained by sorting @var{L} and  merging
identical (in the sense of @code{==}) elements.

@item keysort(+@var{L},@var{S})
@findex keysort/2
@syindex keysort/2
@cyindex keysort/2
Assuming L is a list of the form @code{@var{Key}-@var{Value}},
@code{keysort(+@var{L},@var{S})} unifies @var{S} with the list obtained
from @var{L}, by sorting its elements according to the value of
@var{Key}.
@example
?- keysort([3-a,1-b,2-c,1-a,1-b],S).
@end example
@noindent
would return:
@example
S = [1-b,1-a,1-b,2-c,3-a]
@end example

@item predsort(+@var{Pred}, +@var{List}, -@var{Sorted})
@findex predsort/3
@snindex predsort/3
@cnindex predsort/3
Sorts similar to sort/2, but determines the order of two terms by
calling @var{Pred}(-@var{Delta}, +@var{E1}, +@var{E2}) . This call must
unify @var{Delta} with one of @code{<}, @code{>} or @code{=}. If
built-in predicate compare/3 is used, the result is the same as
sort/2.

@item length(?@var{L},?@var{S})
@findex length/2
@syindex length/2
@cyindex length/2
Unify the well-defined list @var{L} with its length. The procedure can
be used to find the length of a pre-defined list, or to build a list
of length @var{S}.

@end table

@node Arithmetic, I/O, Comparing Terms, Top
@section Arithmetic

YAP now supposets several different numeric types:

@table @code
@item integers
      When YAP is built using the GNU multiple precision arithmetic
      library (GMP), integer arithmetic is unbounded, which means that
      the size of integers is limited by available memory only. Without
      GMP, SWI-Prolog integers have the same size as an address. The
      type of integer support can be detected using the Prolog flags
      bounded, min_integer and max_integer. As the use of GMP is
      default, most of the following descriptions assume unbounded
      integer arithmetic.

      Internally, SWI-Prolog has three integer representations. Small
      integers (defined by the Prolog flag max_tagged_integer) are
      encoded directly. Larger integers are represented as cell values
      on the global stack. Integers that do not fit in 64-bit are
      represented as serialised GNU MPZ structures on the global stack.

@item number
      Rational numbers (Q) are quotients of two integers. Rational
      arithmetic is only provided if GMP is used (see above). Rational
      numbers that are returned from is/2 are canonical, which means M
      is positive and N and M have no common divisors. Rational numbers
      are introduced in the computation using the rational/1,
      rationalize/1 or the rdiv/2 (rational division) function. 

@item float
      Floating point numbers are represented using the C-type double. On most today platforms these are 64-bit IEEE floating point numbers.

@end table

Arithmetic functions that require integer arguments accept, in addition
to integers, rational numbers with denominator `1' and floating point
numbers that can be accurately converted to integers. If the required
argument is a float the argument is converted to float. Note that
conversion of integers to floating point numbers may raise an overflow
exception. In all other cases, arguments are converted to the same type
using the order integer to rational number to floating point number.


Arithmetic expressions in YAP may use the following operators or
@i{evaluable predicates}:

@table @code

@item +@var{X}
The value of @var{X} itself.

@item -@var{X} [ISO]
Symmetric value.

@item @var{X}+@var{Y} [ISO]
Sum.

@item @var{X}-@var{Y} [ISO]
Difference.

@item @var{X}*@var{Y} [ISO]
Product.

@item @var{X}/@var{Y} [ISO]
Quotient.

@item @var{X}//@var{Y} [ISO]
Integer quotient.

@item @var{X} mod @var{Y} [ISO]
Integer remainder.

@item @var{X} rem @var{Y} [ISO]
Integer remainder, the same as @code{mod}.

@item @var{X} div @var{Y} [ISO]
Integer division, as if defined by @code{(@var{X} - @var{X} mod @var{Y}) // @var{Y}}.

@item exp(@var{X}) [ISO]
Natural exponential.

@item log(@var{X}) [ISO]
Natural logarithm.

@item log10(@var{X})
Decimal logarithm.

@item sqrt(@var{X}) [ISO]
Square root.

@item sin(@var{X}) [ISO]
Sine.

@item cos(@var{X}) [ISO]
Cosine.

@item tan(@var{X})
Tangent.

@item asin(@var{X})
Arc sine.

@item acos(@var{X})
Arc cosine.

@item atan(@var{X}) [ISO]
Arc tangent.

@item atan2(@var{X},@var{Y})
Four-quadrant arc tangent.

@item sinh(@var{X})
Hyperbolic sine.

@item cosh(@var{X})
Hyperbolic cosine.

@item tanh(@var{X})
Hyperbolic tangent.

@item asinh(@var{X})
Hyperbolic arc sine.

@item acosh(@var{X})
Hyperbolic arc cosine.

@item atanh(@var{X})
Hyperbolic arc tangent.

@item lgamma(@var{X})
Logarithm of gamma function.

@item erf(@var{X})
Gaussian error function.

@item erfc(@var{X})
Complementary gaussian error function.

@item random(@var{X}) [ISO]
An integer random number between 0 and @var{X}.

In @code{iso} language mode the argument must be a floating
point-number, the result is an integer and it the float is equidistant
it is rounded up, that is, to the least integer greater than @var{X}.

@item integer(@var{X})
If @var{X} evaluates to a float, the integer between the value of @var{X}
and 0 closest to the value of @var{X}, else if @var{X} evaluates to an
integer, the value of @var{X}.

@item float(@var{X}) [ISO]
If @var{X} evaluates to an integer, the corresponding float, else the float
itself.

@item float_fractional_part(@var{X}) [ISO]
The fractional part of the floating point number @var{X}, or @code{0.0}
if @var{X} is an integer. In the @code{iso} language mode,
@var{X} must be an integer.

@item float_integer_part(@var{X}) [ISO]
The float giving the integer part of the floating point number @var{X},
or @var{X} if @var{X} is an integer. In the @code{iso} language mode,
@var{X} must be an integer.

@item abs(@var{X}) [ISO]
The absolute value of @var{X}.

@item ceiling(@var{X}) [ISO]
The integer that is the smallest integral value not smaller than @var{X}.

In @code{iso} language mode the argument must be a floating
point-number and the result is an integer.

@item floor(@var{X}) [ISO]
The integer that is the greatest integral value not greater than @var{X}.

In @code{iso} language mode the argument must be a floating
point-number and the result is an integer.

@item round(@var{X}) [ISO]
The nearest integral value to @var{X}. If @var{X} is
equidistant to two integers, it will be rounded to the closest even
integral value.

In @code{iso} language mode the argument must be a floating
point-number, the result is an integer and it the float is equidistant
it is rounded up, that is, to the least integer greater than @var{X}.

@item sign(@var{X}) [ISO]
Return 1 if the @var{X} evaluates to a positive integer, 0 it if
evaluates to 0, and -1 if it evaluates to a negative integer. If @var{X}
evaluates to a floating-point number return 1.0 for a positive @var{X},
0.0 for 0.0, and -1.0 otherwise.

@item truncate(@var{X}) [ISO]
The integral value between @var{X} and 0 closest to
@var{X}.

@item rational(@var{X})
Convert the expression @var{X} to a rational number or integer. The
function returns the input on integers and rational numbers. For
floating point numbers, the returned rational number exactly represents
the float. As floats cannot exactly represent all decimal numbers the
results may be surprising. In the examples below, doubles can represent
@code{0.25} and the result is as expected, in contrast to the result of
@code{rational(0.1)}. The function @code{rationalize/1} gives a more
intuitive result.

@example
?- A is rational(0.25).

A is 1 rdiv 4
?- A is rational(0.1).
A = 3602879701896397 rdiv 36028797018963968
@end example

@item rationalize(@var{X})
Convert the Expr to a rational number or integer. The function is
similar to @code{rational/1}, but the result is only accurate within the
rounding error of floating point numbers, generally producing a much
smaller denominator. 

@example
?- A is rationalize(0.25).

A = 1 rdiv 4
?- A is rationalize(0.1).

A = 1 rdiv 10
@end example


@item max(@var{X},@var{Y})
The greater value of @var{X} and @var{Y}.

@item min(@var{X},@var{Y})
The lesser value of @var{X} and @var{Y}.

@item @var{X} ^ @var{Y}
@var{X} raised to the power of @var{Y}, (from the C-Prolog syntax).

@item exp(@var{X},@var{Y})
@var{X} raised to the power of @var{Y}, (from the Quintus Prolog syntax).

@item @var{X} ** @var{Y} [ISO]
@var{X} raised to the power of @var{Y}  (from ISO).

@item @var{X} /\ @var{Y} [ISO]
Integer bitwise conjunction.

@item @var{X} \/ @var{Y} [ISO]
Integer bitwise disjunction.

@item @var{X} # @var{Y}
@item @var{X} >< @var{Y}
@item xor(@var{X} , @var{Y})
Integer bitwise exclusive disjunction.

@item @var{X} << @var{Y}
Integer bitwise left logical shift of @var{X} by @var{Y} places.

@item @var{X} >> @var{Y} [ISO]
Integer bitwise right logical shift of @var{X} by @var{Y} places.

@item \ @var{X} [ISO]
Integer bitwise negation.

@item gcd(@var{X},@var{Y})
The greatest common divisor of the two integers @var{X} and @var{Y}.

@item msb(@var{X})
The most significant bit of the non-negative integer @var{X}.

@item lsb(@var{X})
The least significant bit of the non-negative integer @var{X}.

@item popcount(@var{X})
The number of bits set to @code{1} in the binary representation of the
non-negative integer @var{X}.

@item [@var{X}]
Evaluates to @var{X} for expression @var{X}. Useful because character
strings in Prolog are lists of character codes.

@example
X is Y*10+C-"0"
@end example
@noindent
is the same as
@example
X is Y*10+C-[48].
@end example
@noindent
which would be evaluated as:
@example
X is Y*10+C-48.
@end example

@end table

Besides numbers and the arithmetic operators described above, certain
atoms have a special meaning when present in arithmetic expressions:

@table @code
@item pi
The value of @emph{pi}, the ratio of a circle's circumference to its
diameter.

@item e
The base of the natural logarithms.

@item epsilon
The difference between the float @code{1.0} and the first larger floating point
number.

@item inf
Infinity according to the IEEE Floating-Point standard. Note that
evaluating this term will generate a domain error in the @code{iso}
language mode.

@item nan
Not-a-number according to the IEEE Floating-Point standard. Note that
evaluating this term will generate a domain error in the @code{iso}
language mode.

@item cputime
CPU time in seconds, since YAP was invoked.

@item heapused
Heap space used, in bytes.

@item local
Local stack in use, in bytes.

@item global
Global stack in use, in bytes.

@item random
A "random" floating point number between 0 and 1.

@end table

The primitive YAP predicates involving arithmetic expressions are:

@table @code

@item @var{X} is +@var{Y} [2]
@findex is/2
@syindex is/2
@caindex is/2
This predicate succeeds iff the result of evaluating the expression
@var{Y} unifies with @var{X}. This is the predicate normally used to
perform evaluation of arithmetic expressions:

@example
X is 2+3*4
@end example
@noindent
succeeds with @code{X = 14}.

@item +@var{X} < +@var{Y} [ISO]
@findex </2
@syindex </2
@cyindex </2
The value of the expression @var{X} is less than the value of expression
@var{Y}.

@item +@var{X} =< +@var{Y} [ISO]
@findex =</2
@syindex =</2
@cyindex =</2
The value of the expression @var{X} is less than or equal to the value
of expression @var{Y}.


@item +@var{X} > +@var{Y} [ISO]
@findex >/2
@syindex >/2
@cyindex >/2
The value of the expression @var{X} is greater than the value of
expression @var{Y}.

@item +@var{X} >= +@var{Y} [ISO]
@findex >=/2
@syindex >=/2
@cyindex >=/2
The value of the expression @var{X} is greater than or equal to the
value of expression @var{Y}.

@item +@var{X} =:= +@var{Y} [ISO]
@findex =:=/2
@syindex =:=/2
@cyindex =:=/2
The value of the expression @var{X} is equal to the value of expression
@var{Y}.

@item +@var{X} =\= +@var{Y} [ISO]
@findex =\=/2
@syindex =\=/2
@cyindex =\=/2
The value of the expression @var{X} is different from the value of
expression @var{Y}.

@item srandom(+@var{X})
@findex srandom/1
@snindex srandom/1
@cnindex srandom/1
Use the argument @var{X} as a new seed for YAP's random number
generator. The argument should be an integer, but floats are acceptable.
@end table

@noindent
@strong{Notes:}

@itemize @bullet
@item 
Since YAP4, YAP @emph{does not} convert automatically between integers
and floats.
@item
arguments to trigonometric functions are expressed in radians.
@item
if a (non-instantiated) variable occurs in an arithmetic expression YAP
will generate an exception. If no error handler is available, execution
will be thrown back to the top-level.
@end itemize


The following predicates provide counting:

@table @code

@item between(+@var{Low}, +@var{High}, ?@var{Value})
@findex between/3
@syindex between/3
@cnindex between/3

    @var{Low} and @var{High} are integers, @var{High} >=@var{Low}. If
    @var{Value} is an integer, @var{Low} =<@var{Value}
    =<@var{High}. When @var{Value} is a variable it is successively
    bound to all integers between @var{Low} and @var{High}. If
    @var{High} is inf or infinite @code{between/3} is true iff
    @var{Value} >= @var{Low}, a feature that is particularly interesting
    for generating integers from a certain value.

@item succ(?@var{Int1}, ?@var{Int2})
@findex succ/3
@syindex succ/3
@cnindex succ/3

    True if @var{Int2} = @var{Int1} + 1 and @var{Int1} >= 0. At least
    one of the arguments must be instantiated to a natural number. This
    predicate raises the domain-error not_less_than_zero if called with
    a negative integer. E.g. @code{succ(X, 0)} fails silently and succ(X, -1)
    raises a domain-error. The behaviour to deal with natural numbers
    only was defined by Richard O'Keefe to support the common
    count-down-to-zero in a natural way. 

@item plus(?@var{Int1}, ?@var{Int2}, ?@var{Int3})
@findex plus/3
@syindex plus/3
@cnindex plus/3
    True if @var{Int3} = @var{Int1} + @var{Int2}. At least two of the
    three arguments must be instantiated to integers.
@end table

@node I/O, Database, Arithmetic, Top
@section I/O Predicates

Some of the I/O predicates described below will in certain conditions
provide error messages and abort only if the file_errors flag is set.
If this flag is cleared the same predicates will just fail. Details on
setting and clearing this flag are given under 7.7.

@menu

Subnodes of Input/Output
* Streams and Files:: Handling Streams and Files
* C-Prolog File Handling:: C-Prolog Compatible File Handling
* I/O of Terms:: Input/Output of terms
* I/O of Characters:: Input/Output of Characters
* I/O for Streams:: Input/Output using Streams
* C-Prolog to Terminal:: C-Prolog compatible Character I/O to terminal
* I/O Control:: Controlling your Input/Output
* Sockets:: Using Sockets from YAP

@end menu

@node Streams and Files, C-Prolog File Handling, , I/O
@subsection Handling Streams and Files

@table @code

@item open(+@var{F},+@var{M},-@var{S}) [ISO]
@findex open/3
@syindex open/3
@cnindex open/3
Opens the file with name @var{F} in mode @var{M} ('read', 'write' or
'append'), returning @var{S} unified with the stream name.

At most, there are 17 streams  opened at the same time. Each stream is
either an input or an output stream but not both. There are always 3
open streams:  @code{user_input} for reading, @code{user_output} for writing
and @code{user_error} for writing. If there is no  ambiguity, the atoms
@code{user_input} and @code{user_output} may be referred to as  @code{user}.

The @code{file_errors} flag controls whether errors are reported when in
mode 'read' or 'append' the file @var{F} does not exist or is not
readable, and whether in mode 'write' or 'append' the file is not
writable.

@item open(+@var{F},+@var{M},-@var{S},+@var{Opts}) [ISO]
@findex open/4
@saindex open/4
@cnindex open/4
Opens the file with name @var{F} in mode @var{M} ('read',  'write' or
'append'), returning @var{S} unified with the stream name, and following
these options:

@table @code

@item type(+@var{T})
Specify whether the stream is a @code{text} stream (default), or a
@code{binary} stream.

@item reposition(+@var{Bool})
Specify whether it is possible to reposition the stream (@code{true}), or
not (@code{false}). By default, YAP enables repositioning for all
files, except terminal files and sockets.

@item eof_action(+@var{Action})
Specify the action to take if attempting to input characters from a
stream where we have previously found an @code{end_of_file}. The possible
actions are @code{error}, that raises an error, @code{reset}, that tries to
reset the stream and is used for @code{tty} type files, and @code{eof_code},
which generates a new @code{end_of_file} (default for non-tty files).

@item alias(+@var{Name})
Specify an alias to the stream. The alias @t{Name} must be an atom. The
alias can be used instead of the stream descriptor for every operation
concerning the stream.

The operation will fail and give an error if the alias name is already
in use. YAP allows several aliases for the same file, but only
one is returned by @code{stream_property/2}

@item bom(+@var{Bool})
If present and @code{true}, a BOM (@emph{Byte Order Mark}) was
detected while opening the file for reading or a BOM was written while
opening the stream. See @ref{BOM} for details.

@item encoding(+@var{Encoding})
Set the encoding used for text.  See @ref{Encoding} for an overview of
wide character and encoding issues.

@item representation_errors(+@var{Mode})
Change the behaviour when writing characters to the stream that cannot
be represented by the encoding.  The behaviour is one of @code{error}
(throw and I/O error exception), @code{prolog} (write @code{\u...\}
escape code or @code{xml} (write @code{&#...;} XML character entity).
The initial mode is @code{prolog} for the user streams and
@code{error} for all other streams. See also @ref{Encoding}.

@item expand_filename(+@var{Mode})
If @var{Mode} is @code{true} then do filename expansion, then ask Prolog
to do file name expansion before actually trying to opening the file:
this includes processing @code{~} characters and processing @code{$}
environment variables at the beginning of the file. Otherwise, just try
to open the file using the given name.

The default behavior is given by the Prolog flag
@code{open_expands_filename}.

@end table

@item close(+@var{S}) [ISO]
@findex close/1
@syindex close/1
@cyindex close/1
Closes the stream @var{S}. If @var{S} does not stand for a stream
currently opened an error is reported. The streams @code{user_input},
@code{user_output}, and @code{user_error} can never be closed.

@c By default, give a file name, @code{close/1} will also try to close a
@c corresponding open stream. This feature is not available in ISO or
@c SICStus languages mode and is deprecated.

@item close(+@var{S},+@var{O}) [ISO]
@findex close/2
@saindex close/2
@cnindex close/2
Closes the stream @var{S}, following options @var{O}. 

The only valid options are @code{force(true)} and @code{force(false)}.
YAP currently ignores these options.

@item time_file(+@var{File},-@var{Time})
@findex time_file/2
@snindex time_file/2
@cnindex time_file/2
Unify the last modification time of @var{File} with
@var{Time}. @var{Time} is a floating point number expressing the seconds
elapsed since Jan 1, 1970.

@item absolute_file_name(+@var{Name},+@var{Options}, -@var{FullPath})
@item absolute_file_name(+@var{Name}, -@var{FullPath},+@var{Options})
@findex absolute_file_name/3
@syindex absolute_file_name/3
@cnindex absolute_file_name/3

Converts the given file specification into an absolute path.
@var{Option} is a list of options to guide the conversion:

@table @code
    @item extensions(+@var{ListOfExtensions})
List of file-extensions to try.  Default is @samp{''}.  For each
extension, @code{absolute_file_name/3} will first add the extension and then
verify the conditions imposed by the other options.  If the condition
fails, the next extension of the list is tried.  Extensions may be
specified both as @code{.ext} or plain @code{ext}.

    @item relative_to(+@var{FileOrDir})
Resolve the path relative to the given directory or directory the
holding the given file.  Without this option, paths are resolved
relative to the working directory (see @code{working_directory/2}) or,
if @var{Spec} is atomic and @code{absolute_file_name/[2,3]} is executed
in a directive, it uses the current source-file as reference.

    @item access(+@var{Mode})
Imposes the condition access_file(@var{File}, @var{Mode}).  @var{Mode}
is on of @code{read}, @code{write}, @code{append}, @code{exist} or
@code{none} (default). 
See also @code{access_file/2}.

    @item file_type(+@var{Type})
Defines extensions. Current mapping: @code{txt} implies @code{['']},
@code{prolog} implies @code{['.pl', '']}, @code{executable} implies
@code{['.so', '']}, @code{qlf} implies @code{['.qlf', '']} and
@code{directory} implies @code{['']}.  The file-type @code{source}
is an alias for @code{prolog} for compatibility to SICStus Prolog.
See also @code{prolog_file_type/2}.

    @item file_errors(@code{fail}/@code{error})
If @code{error} (default), throw and @code{existence_error} exception
if the file cannot be found.  If @code{fail}, stay silent.

    @item solutions(@code{first}/@code{all})
If @code{first} (default), the predicates leaves no choice-point.
Otherwise a choice-point will be left and backtracking may yield
more solutions.

@c     @item expand(@code{true}/@code{false})
@c If @code{true} (default is @code{false}) and @var{Spec} is atomic,
@c call @code{expand_file_name/2} followed by @code{member/2} on @var{Spec} before
@c proceeding.  This is a SWI-Prolog extension.
@end table

@c The Prolog flag @code{verbose_file_search} can be set to @code{true}
@c to help debugging Prolog's search for files.

Compatibility considerations to common argument-order in ISO as well
as SICStus @code{absolute_file_name/3} forced us to be flexible here.
If the last argument is a list and the 2nd not, the arguments are
swapped, making the call @code{absolute_file_name}(+@var{Spec}, -@var{Path},
+@var{Options}) valid as well.

@item absolute_file_name(+@var{Name},-@var{FullPath})
@findex absolute_file_name/2
@syindex absolute_file_name/2
@cnindex absolute_file_name/2
Give the path a full path @var{FullPath} YAP would use to consult a file
named @var{Name}.  Unify @var{FullPath} with @code{user} if the file
name is @code{user}.

@item file_base_name(+@var{Name},-@var{FileName})
@findex file_base_name/2
@snindex file_base_name/2
@cnindex file_base_name/2
Give the path a full path @var{FullPath} extract the @var{FileName}.

@item file_name_extension(?@var{Base},?@var{Extension}, ?@var{Name})
@findex file_name_extension/3
@snindex file_name_extension/3
@cnindex file_name_extension/3

This predicate is used to add, remove or test filename extensions. The
main reason for its introduction is to deal with different filename
properties in a portable manner. If the file system is
case-insensitive, testing for an extension will be done
case-insensitive too. @var{Extension} may be specified with or
without a leading dot (.). If an @var{Extension} is generated, it
will not have a leading dot.

@item current_stream(@var{F},@var{M},@var{S})
@findex current_stream/3
@syindex current_stream/3
@cnindex current_stream/3
Defines the relation: The stream @var{S} is opened on the file @var{F}
in mode @var{M}. It might be used to obtain all open streams (by
backtracking) or to access the stream for a file @var{F} in mode
@var{M}, or to find properties for a stream @var{S}.

@item is_stream(@var{S})
@findex is_stream/1
@snindex is_stream/1
@cnindex is_stream/1
Succeeds if @var{S} is a currently open stream.

@item flush_output [ISO]
@findex flush_output/0
@syindex flush_output/0
@cnindex flush_output/0
Send out all data in the output buffer of the current output stream.

@item flush_output(+@var{S}) [ISO]
@findex flush_output/1
@syindex flush_output/1
@cnindex flush_output/1
Send all data in the output buffer for stream @var{S}.

@item set_input(+@var{S})
@findex set_input/1
@syindex set_input/1
@cnindex set_input/1
Set stream @var{S} as the current input stream. Predicates like @code{read/1}
and @code{get/1} will start using stream @var{S}.

@item set_output(+@var{S})
@findex set_output/1
@syindex set_output/1
@cnindex set_output/1
Set stream @var{S} as the current output stream. Predicates like
@code{write/1} and @code{put/1} will start using stream @var{S}.

@item stream_select(+@var{STREAMS},+@var{TIMEOUT},-@var{READSTREAMS})
@findex stream_select/3
@syindex stream_select/3
@cnindex stream_select/3
Given a list of open @var{STREAMS} opened in read mode and a @var{TIMEOUT}
return a list of streams who are now available for reading. 

If the @var{TIMEOUT} is instantiated to @code{off},
@code{stream_select/3} will wait indefinitely for a stream to become
open. Otherwise the timeout must be of the form @code{SECS:USECS} where
@code{SECS} is an integer gives the number of seconds to wait for a timeout
and @code{USECS} adds the number of micro-seconds.

This built-in is only defined if the system call @code{select} is
available in the system.

@item current_input(-@var{S}) [ISO]
@findex current_input/1
@syindex current_input/1
@cnindex current_input/1
Unify @var{S} with the current input stream.

@item current_output(-@var{S}) [ISO]
@findex current_output/1
@syindex current_output/1
@cnindex current_output/1
Unify @var{S} with the current output stream.

@item at_end_of_stream [ISO]
@findex at_end_of_stream/0
@syindex at_end_of_stream/0
@cnindex at_end_of_stream/0
Succeed if the current stream has stream position end-of-stream or
past-end-of-stream.

@item at_end_of_stream(+@var{S}) [ISO]
@findex at_end_of_stream/1
@syindex at_end_of_stream/1
@cnindex at_end_of_stream/1
Succeed if the stream @var{S} has stream position end-of-stream or
past-end-of-stream. Note that @var{S} must be a readable stream.

@item set_stream_position(+@var{S}, +@var{POS}) [ISO]
@findex set_stream_position/2
@syindex set_stream_position/2
@cnindex set_stream_position/2
Given a stream position @var{POS} for a stream @var{S}, set the current
stream position for @var{S} to be @var{POS}.

@item stream_property(?@var{Stream},?@var{Prop}) [ISO]
@findex stream_property/2
@snindex stream_property/2
@cnindex stream_property/2

Obtain the properties for the open streams. If the first argument is
unbound, the procedure will backtrack through all open
streams. Otherwise, the first argument must be a stream term (you may
use @code{current_stream} to obtain a current stream given a file name).

The following properties are recognized:

@table @code

@item file_name(@var{P})
An atom giving the file name for the current stream. The file names are
@code{user_input}, @code{user_output}, and @code{user_error} for the
standard streams.

@item mode(@var{P})
The mode used to open the file. It may be one of @code{append},
@code{read}, or @code{write}.

@item input
The stream is readable.

@item output
The stream is writable.

@item alias(@var{A})
ISO-Prolog primitive for stream aliases. @t{YAP} returns one of the
existing aliases for the stream.

@item position(@var{P})
A term describing the position in the stream.

@item end_of_stream(@var{E})
Whether the stream is @code{at} the end of stream, or it has found the
end of stream and is @code{past}, or whether it has @code{not} yet
reached the end of stream.

@item eof_action(@var{A})
The action to take when trying to read after reaching the end of
stream. The action may be one of @code{error}, generate an error,
@code{eof_code}, return character code @code{-1}, or @code{reset} the
stream.

@item reposition(@var{B})
Whether the stream can be repositioned or not, that is, whether it is
seekable.

@item type(@var{T})
Whether the stream is a @code{text} stream or a @code{binary} stream.

@item bom(+@var{Bool})
If present and @code{true}, a BOM (@emph{Byte Order Mark}) was
detected while opening the file for reading or a BOM was written while
opening the stream. See @ref{BOM} for details.

@item encoding(+@var{Encoding})
Query the encoding used for text.  See @ref{Encoding} for an
overview of wide character and encoding issues in YAP.

@item representation_errors(+@var{Mode})
Behaviour when writing characters to the stream that cannot be
represented by the encoding.  The behaviour is one of @code{error}
(throw and I/O error exception), @code{prolog} (write @code{\u...\}
escape code or @code{xml} (write @code{&#...;} XML character entity).
The initial mode is @code{prolog} for the user streams and
@code{error} for all other streams. See also @ref{Encoding} and
@code{open/4}.

@end table

@item current_line_number(-@var{LineNumber})
@findex current_line_number/1
@saindex current_line_number/1
@cnindex current_line_number/1
Unify @var{LineNumber} with the line number for the current stream.

@item current_line_number(+@var{Stream},-@var{LineNumber})
@findex current_line_number/2
@saindex current_line_number/2
@cnindex current_line_number/2
Unify @var{LineNumber} with the line number for the @var{Stream}. 

@item line_count(+@var{Stream},-@var{LineNumber})
@findex line_count/2
@syindex line_count/2
@cnindex line_count/2
Unify @var{LineNumber} with the line number for the @var{Stream}.

@item character_count(+@var{Stream},-@var{CharacterCount})
@findex character_count/2
@syindex character_count/2
@cnindex character_count/2
Unify @var{CharacterCount} with the number of characters written to or
read to @var{Stream}.

@item line_position(+@var{Stream},-@var{LinePosition})
@findex line_position/2
@syindex line_position/2
@cnindex line_position/2
Unify @var{LinePosition} with the position on current text stream
@var{Stream}.

@item stream_position(+@var{Stream},-@var{StreamPosition})
@findex stream_position/2
@syindex stream_position/2
@cnindex stream_position/2
Unify @var{StreamPosition} with the packaged information of position on
current stream @var{Stream}. Use @code{stream_position_data/3} to
retrieve information on charater or line count.

@item stream_position_data(+@var{Field},+@var{StreamPsition},-@var{Info})
@findex stream_position_data/3
@syindex stream_position_data/3
@cnindex stream_position_data/3
Given the packaged stream position term @var{StreamPosition}, unify
@var{Info} with @var{Field} @code{line_count}, @code{byte_count}, or
@code{char_count}.

@end table

@node C-Prolog File Handling, I/O of Terms, Streams and Files, I/O
@subsection Handling Streams and Files

@table @code

@item tell(+@var{S})
@findex tell/1
@syindex tell/1
@cyindex tell/1
If @var{S} is a currently opened stream for output, it becomes the
current output stream. If @var{S} is an atom it is taken to be a
filename.  If there is no output stream currently associated with it,
then it is opened for output, and the new output stream created becomes
the current output stream. If it is not possible to open the file, an
error occurs.  If there is a single opened output stream currently
associated with the file, then it becomes the current output stream; if
there are more than one in that condition, one of them is chosen.

Whenever @var{S} is a stream not currently opened for output, an error
may be reported, depending on the state of the file_errors flag. The
predicate just fails, if @var{S} is neither a stream nor an atom.

@item telling(-@var{S})
@findex telling/1
@syindex telling/1
@cyindex telling/1
The current output stream is unified with @var{S}.

@item told
@findex told/0
@syindex told/0
@cyindex told/0
Closes the current output stream, and the user's terminal becomes again
the current output stream. It is important to remember to close streams
after having finished using them, as the maximum number of
simultaneously opened streams is 17.

@item see(+@var{S})
@findex see/1
@syindex see/1
@cyindex see/1
If @var{S} is a currently opened input stream then it is assumed to be
the current input stream. If @var{S} is an atom it is taken as a
filename. If there is no input stream currently associated with it, then
it is opened for input, and the new input stream thus created becomes
the current input stream. If it is not possible to open the file, an
error occurs.  If there is a single opened input stream currently
associated with the file, it becomes the current input stream; if there
are more than one in that condition, then one of them is chosen.

When @var{S} is a stream not currently opened for input, an error may be
reported, depending on the state of the @code{file_errors} flag. If
@var{S} is neither a stream nor an atom the predicates just fails.

@item seeing(-@var{S})
@findex seeing/1
@syindex seeing/1
@cyindex seeing/1
The current input stream is unified with @var{S}.

@item seen
@findex seen/0
@syindex seen/0
@cyindex seen/0
Closes the current input stream (see 6.7.).

@end table

@node I/O of Terms, I/O of Characters, C-Prolog File Handling, I/O
@subsection Handling Input/Output of Terms

@table @code

@item read(-@var{T}) [ISO]
@findex read/1
@syindex read/1
@cyindex read/1
Reads the next term from the current input stream, and unifies it with
@var{T}. The term must be followed by a dot ('.') and any blank-character
as previously defined. The syntax of the term must match the current
declarations for operators (see op). If the end-of-stream is reached, 
@var{T} is unified with the atom @code{end_of_file}. Further reads from of 
the same stream may cause an error failure (see @code{open/3}).

@item read_term(-@var{T},+@var{Options}) [ISO]
@findex read_term/2
@saindex read_term/2
@cnindex read_term/2
Reads term @var{T} from the current input stream with execution
controlled by the following options:

@table @code

@item  term_position(-@var{Position})
@findex term_position/1 (read_term/2 option)
Unify @var{Position} with a term describing the position of the stream
at the start of parse. Use @code{stream_position_data/3} to obtain extra
information.

@item  singletons(-@var{Names})
@findex singletons/1 (read_term/2 option)
Unify @var{Names} with a list of the form @var{Name=Var}, where
@var{Name} is the name of a non-anonymous singleton variable in the
original term, and @code{Var} is the variable's representation in
YAP.

@item  syntax_errors(+@var{Val})
@findex syntax_errors/1 (read_term/2 option)
Control action to be taken after syntax errors. See @code{yap_flag/2}
for detailed information.

@item  variable_names(-@var{Names})
@findex variable_names/1 (read_term/2 option)
Unify @var{Names} with a list of the form @var{Name=Var}, where @var{Name} is
the name of a non-anonymous variable in the original term, and @var{Var}
is the variable's representation in YAP.

@item  variables(-@var{Names})
@findex variables/1 (read_term/2 option)
Unify @var{Names} with a list of the variables in term @var{T}.

@end table

@item char_conversion(+@var{IN},+@var{OUT}) [ISO]
@findex char_conversion/2
@syindex char_conversion/2
@cnindex char_conversion/2
While reading terms convert unquoted occurrences of the character
@var{IN} to the character @var{OUT}. Both @var{IN} and @var{OUT} must be
bound to single characters atoms.

Character conversion only works if the flag @code{char_conversion} is
on. This is default in the @code{iso} and @code{sicstus} language
modes. As an example, character conversion can be used for instance to
convert characters from the ISO-LATIN-1 character set to ASCII.

If @var{IN} is the same character as @var{OUT}, @code{char_conversion/2}
will remove this conversion from the table.

@item current_char_conversion(?@var{IN},?@var{OUT}) [ISO]
@findex current_char_conversion/2
@syindex current_char_conversion/2
@cnindex current_char_conversion/2
If @var{IN} is unbound give all current character
translations. Otherwise, give the translation for @var{IN}, if one
exists.

@item write(@var{T}) [ISO]
@findex write/1
@syindex write/1
@cyindex write/1
The term @var{T} is written to the current output stream according to
the operator declarations in force.

@item writeln(@var{T}) [ISO]
@findex writeln/1
@snindex writeln/1
@cnindex writeln/1
Same as @code{write/1} followed by @code{nl/0}.

@item display(+@var{T})
@findex display/1
@syindex display/1
@cyindex display/1
Displays term @var{T} on the current output stream. All Prolog terms are
written in standard parenthesized prefix notation.

@item write_canonical(+@var{T}) [ISO]
@findex display/1
@syindex display/1
@cnindex display/1
Displays term @var{T} on the current output stream. Atoms are quoted
when necessary, and operators are ignored, that is, the term is written
in standard parenthesized prefix notation.

@item write_term(+@var{T}, +@var{Opts}) [ISO]
@findex write_term/2
@syindex write_term/2
@cnindex write_term/2
Displays term @var{T} on the current output stream, according to the
following options:

@table @code
@item quoted(+@var{Bool})
If @code{true}, quote atoms if this would be necessary for the atom to
be recognized as an atom by YAP's parser. The default value is
@code{false}.

@item ignore_ops(+@var{Bool})
If @code{true}, ignore operator declarations when writing the term. The
default value is @code{false}.

@item numbervars(+@var{Bool})
If @code{true}, output terms of the form
@code{'$VAR'(N)}, where @var{N} is an integer, as a sequence of capital
letters. The default value is @code{false}.

@item portrayed(+@var{Bool})
If @code{true}, use @t{portray/1} to portray bound terms. The default
value is @code{false}.

@item portray(+@var{Bool})
If @code{true}, use @t{portray/1} to portray bound terms. The default
value is @code{false}.

@item max_depth(+@var{Depth})
If @code{Depth} is a positive integer, use @t{Depth} as
the maximum depth to portray a term. The default is @code{0}, that is,
unlimited depth.

@item priority(+@var{Piority})
If @code{Priority} is a positive integer smaller than @code{1200}, 
give the context priority. The default is @code{1200}.

@item cycles(+@var{Bool})
Do not loop in rational trees (default).
@end table

@item writeq(@var{T}) [ISO]
@findex writeq/1
@syindex writeq/1
@cyindex writeq/1
 Writes the term @var{T}, quoting names to make the result acceptable to
the predicate 'read' whenever necessary.

@item print(@var{T})
@findex print/1
@syindex print/1
@cyindex print/1
Prints the term @var{T} to the current output stream using @code{write/1}
unless T is bound and a call to the user-defined  predicate
@code{portray/1} succeeds. To do pretty  printing of terms the user should
define suitable clauses for @code{portray/1} and use @code{print/1}.

@item format(+@var{T},+@var{L})
@findex format/2
@saindex format/2
@cnindex format/2
Print formatted output to the current output stream. The arguments in
list @var{L} are output according to the string or atom @var{T}.

A control sequence is introduced by a @code{w}. The following control
sequences are available in YAP:

@table @code

@item '~~'
Print a single tilde.

@item '~a'
The next argument must be an atom, that will be printed as if by @code{write}.

@item '~Nc'
The next argument must be an integer, that will be printed as a
character code. The number @var{N} is the number of times to print the
character (default 1).

@item '~Ne'
@itemx '~NE'
@itemx '~Nf'
@itemx '~Ng'
@itemx '~NG'
The next argument must be a floating point number. The float @var{F}, the number
@var{N} and the control code @code{c} will be passed to @code{printf} as:

@example
    printf("%s.Nc", F)
    @end example

As an example:

@example
?- format("~8e, ~8E, ~8f, ~8g, ~8G~w",
          [3.14,3.14,3.14,3.14,3.14,3.14]).
3.140000e+00, 3.140000E+00, 3.140000, 3.14, 3.143.14
@end example

@item '~Nd'
The next argument must be an integer, and @var{N} is the number of digits
after the decimal point. If @var{N} is @code{0} no decimal points will be
printed. The default is @var{N = 0}.

@example
?- format("~2d, ~d",[15000, 15000]).
150.00, 15000
@end example

@item '~ND'
Identical to @code{'~Nd'}, except that commas are used to separate groups
of three digits.

@example
?- format("~2D, ~D",[150000, 150000]).
1,500.00, 150,000
@end example

@item '~i'
Ignore the next argument in the list of arguments:

@example
?- format('The ~i met the boregrove',[mimsy]).
The  met the boregrove
@end example

@item '~k'
Print the next argument with @code{write_canonical}:

@example
?- format("Good night ~k",a+[1,2]).
Good night +(a,[1,2])
@end example

@item '~Nn'
Print @var{N} newlines (where @var{N} defaults to 1).

@item '~NN'
Print @var{N} newlines if at the beginning of the line (where @var{N}
defaults to 1).

@item '~Nr'
The next argument must be an integer, and @var{N} is interpreted as a
radix, such that @code{2 <= N <= 36} (the default is 8).

@example
?- format("~2r, 0x~16r, ~r",
          [150000, 150000, 150000]).
100100100111110000, 0x249f0, 444760
@end example

@noindent
Note that the letters @code{a-z} denote digits larger than 9.

@item '~NR'
Similar to '~NR'. The next argument must be an integer, and @var{N} is
interpreted as a radix, such that @code{2 <= N <= 36} (the default is 8).

@example
?- format("~2r, 0x~16r, ~r",
          [150000, 150000, 150000]).
100100100111110000, 0x249F0, 444760
@end example

@noindent
The only difference is that letters @code{A-Z} denote digits larger than 9.

@item '~p'
Print the next argument with @code{print/1}:

@example
?- format("Good night ~p",a+[1,2]).
Good night a+[1,2]
@end example

@item '~q'
Print the next argument with @code{writeq/1}:

@example
?- format("Good night ~q",'Hello'+[1,2]).
Good night 'Hello'+[1,2]
@end example

@item '~Ns'
The next argument must be a list of character codes. The system then
outputs their representation as a string, where @var{N} is the maximum
number of characters for the string (@var{N} defaults to the length of the
string).

@example
?- format("The ~s are ~4s",["woods","lovely"]).
The woods are love
@end example

@item '~w'
Print the next argument with @code{write/1}:

@example
?- format("Good night ~w",'Hello'+[1,2]).
Good night Hello+[1,2]
@end example

@end table
The number of arguments, @code{N}, may be given as an integer, or it
may be given as an extra argument. The next example shows a small
procedure to write a variable number of @code{a} characters:

@example
write_many_as(N) :-
        format("~*c",[N,0'a]).
@end example

The @code{format/2} built-in also allows for formatted output.  One can
specify column boundaries and fill the intermediate space by a padding
character: 

@table @code
@item '~N|'
Set a column boundary at position @var{N}, where @var{N} defaults to the
current position.

@item '~N+'
Set a column boundary at @var{N} characters past the current position, where
@var{N} defaults to @code{8}.


@item '~Nt'
Set padding for a column, where @var{N} is the fill code (default is
@key{SPC}).

@end table

The next example shows how to align columns and padding. We first show
left-alignment:

@example

@code{
   ?- format("~n*Hello~16+*~n",[]).
*Hello          *
}
@end example

Note that we reserve 16 characters for the column.

The following example shows how to do right-alignment:


@example
@code{
   ?- format("*~tHello~16+*~n",[]).
*          Hello*
}

@end example


The @code{~t} escape sequence forces filling before @code{Hello}. 

We next show how to do centering:

@example
@code{
   ?- format("*~tHello~t~16+*~n",[]).
*     Hello     *
}
@end example


The two @code{~t} escape sequence force filling both before and after
@code{Hello}. Space is then evenly divided between the right and the
left sides.


@item format(+@var{T})
@findex format/1
@saindex format/1
@cnindex format/1
Print formatted output to the current output stream.


@item format(+@var{S},+@var{T},+@var{L})
@findex format/3
@saindex format/3
@cnindex format/3
Print formatted output to stream @var{S}.

@item with_output_to(+@var{Ouput},:@var{Goal})
@findex with_output_to/2
@saindex with_output_to/2
@cnindex with_output_to/2
Run @var{Goal} as @code{once/1}, while characters written to the current
output are sent to @var{Output}. The predicate is SWI-Prolog
specific.

Applications should generally avoid creating atoms by breaking and
concatenating other atoms as the creation of large numbers of
intermediate atoms generally leads to poor performance, even more so in
multi-threaded applications. This predicate supports creating
difference-lists from character data efficiently. The example below
defines the DCG rule @code{term/3} to insert a term in the output:

@example
 term(Term, In, Tail) :-
        with_output_to(codes(In, Tail), write(Term)).

?- phrase(term(hello), X).

X = [104, 101, 108, 108, 111]
@end example

@table @code
@item A Stream handle or alias
    Temporary switch current output to the given stream. Redirection using with_output_to/2 guarantees the original output is restored, also if Goal fails or raises an exception. See also call_cleanup/2. 
@item atom(-@var{Atom})
    Create an atom from the emitted characters. Please note the remark above. 
@item string(-@var{String})
    Create a string-object (not supported in YAP). 
@item codes(-@var{Codes})
    Create a list of character codes from the emitted characters, similar to atom_codes/2. 
@item codes(-@var{Codes}, -@var{Tail})
    Create a list of character codes as a difference-list. 
@item chars(-@var{Chars})
    Create a list of one-character-atoms codes from the emitted characters, similar to atom_chars/2. 
@item chars(-@var{Chars}, -@var{Tail})
    Create a list of one-character-atoms as a difference-list. 
@end table

@end table

@node I/O of Characters, I/O for Streams, I/O of Terms, I/O
@subsection Handling Input/Output of Characters

@table @code

@item put(+@var{N})
@findex put/1
@syindex put/1
@cyindex put/1
Outputs to the current output stream the character whose ASCII code is
@var{N}. The character @var{N} must be a legal ASCII character code, an
expression yielding such a code, or a list in which case only the first
element is used.

@item put_byte(+@var{N}) [ISO]
@findex put_byte/1
@snindex put_byte/1
@cnindex put_byte/1
Outputs to the current output stream the character whose code is
@var{N}. The current output stream must be a binary stream.

@item put_char(+@var{N}) [ISO]
@findex put_char/1
@snindex put_char/1
@cnindex put_char/1
Outputs to the current output stream the character who is used to build
the representation of atom @code{A}. The current output stream must be a
text stream.

@item put_code(+@var{N}) [ISO]
@findex put_code/1
@snindex put_code/1
@cnindex put_code/1
Outputs to the current output stream the character whose ASCII code is
@var{N}. The current output stream must be a text stream. The character
@var{N} must be a legal ASCII character code, an expression yielding such
a code, or a list in which case only the first element is used.

@item get(-@var{C})
@findex get/1
@syindex get/1
@cyindex get/1
The next non-blank character from the current input stream is unified
with @var{C}. Blank characters are the ones whose ASCII codes are not
greater than 32. If there are no more non-blank characters in the
stream, @var{C} is unified with -1. If @code{end_of_stream} has already
been reached in the previous reading, this call will give an error message.

@item get0(-@var{C})
@findex get0/1
@syindex get0/1
@cyindex get0/1
The next character from the current input stream is consumed, and then
unified with @var{C}. There are no restrictions on the possible
values of the ASCII code for the character, but the character will be
internally converted by YAP.

@item get_byte(-@var{C}) [ISO]
@findex get_byte/1
@snindex get_byte/1
@cnindex get_byte/1
If @var{C} is unbound, or is a character code, and the current stream is a
binary stream, read the next byte from the current stream and unify its
code with @var{C}.

@item get_char(-@var{C}) [ISO]
@findex get_char/1
@snindex get_char/1
@cnindex get_char/1
If @var{C} is unbound, or is an atom representation of a character, and
the current stream is a text stream, read the next character from the
current stream and unify its atom representation with @var{C}.

@item get_code(-@var{C}) [ISO]
@findex get_code/1
@snindex get_code/1
@cnindex get_code/1
If @var{C} is unbound, or is the code for a character, and
the current stream is a text stream, read the next character from the
current stream and unify its code with @var{C}.

@item peek_byte(-@var{C}) [ISO]
@findex peek_byte/1
@snindex peek_byte/1
@cnindex peek_byte/1
If @var{C} is unbound, or is a character code, and the current stream is a
binary stream, read the next byte from the current stream and unify its
code with @var{C}, while leaving the current stream position unaltered.

@item peek_char(-@var{C}) [ISO]
@findex peek_char/1
@syindex peek_char/1
@cnindex peek_char/1
If @var{C} is unbound, or is an atom representation of a character, and
the current stream is a text stream, read the next character from the
current stream and unify its atom representation with @var{C}, while
leaving the current stream position unaltered.

@item peek_code(-@var{C}) [ISO]
@findex peek_code/1
@snindex peek_code/1
@cnindex peek_code/1
If @var{C} is unbound, or is the code for a character, and
the current stream is a text stream, read the next character from the
current stream and unify its code with @var{C}, while
leaving the current stream position unaltered.

@item skip(+@var{N})
@findex skip/1
@syindex skip/1
@cyindex skip/1
Skips input characters until the next occurrence of the character with
ASCII code @var{N}. The argument to this predicate can take the same forms
as those for @code{put} (see 6.11).

@item tab(+@var{N})
@findex tab/1
@syindex tab/1
@cyindex tab/1
Outputs @var{N} spaces to the current output stream.

@item nl [ISO]
@findex nl/0
@syindex nl/0
@cyindex nl/0
Outputs a new line to the current output stream.

@end table

@node I/O for Streams, C-Prolog to Terminal, I/O of Characters, I/O
@subsection Input/Output Predicates applied to Streams

@table @code

@item read(+@var{S},-@var{T}) [ISO]
@findex read/2
@syindex read/2
@cnindex read/2
Reads term @var{T} from the stream @var{S} instead of from the current input
stream.

@item read_term(+@var{S},-@var{T},+@var{Options}) [ISO]
@findex read_term/3
@saindex read_term/3
@cnindex read_term/3
Reads term @var{T} from stream @var{S} with execution controlled by the
same options as @code{read_term/2}.

@item write(+@var{S},@var{T}) [ISO]
@findex write/2
@syindex write/2
@cnindex write/2
Writes term @var{T} to stream @var{S} instead of to the current output
stream.

@item write_canonical(+@var{S},+@var{T}) [ISO]
@findex write_canonical/2
@syindex write_canonical/2
@cnindex write_canonical/2
Displays term @var{T} on the stream @var{S}. Atoms are quoted when
necessary, and operators are ignored.

@item write_canonical(+@var{T}) [ISO]
@findex write_canonical/1
@syindex write_canonical/1
@cnindex write_canonical/1
Displays term @var{T}. Atoms are quoted when necessary, and operators
are ignored.

@item write_term(+@var{S}, +@var{T}, +@var{Opts}) [ISO]
@findex write_term/3
@syindex write_term/3
@cnindex write_term/3
Displays term @var{T} on the current output stream, according to the same
options used by @code{write_term/3}.

@item writeq(+@var{S},@var{T}) [ISO]
@findex writeq/2
@syindex writeq/2
@cnindex writeq/2
As @code{writeq/1}, but the output is sent to the stream @var{S}.

@item display(+@var{S},@var{T})
@findex display/2
@syindex display/2
@cnindex display/2
Like @code{display/1}, but using stream @var{S} to display the term.

@item print(+@var{S},@var{T})
@findex print/2
@syindex print/2
@cnindex print/2
Prints term @var{T} to the stream @var{S} instead of to the current output
stream.

@item put(+@var{S},+@var{N})
@findex put/2
@syindex put/2
@cnindex put/2
As @code{put(N)}, but to stream @var{S}.

@item put_byte(+@var{S},+@var{N}) [ISO]
@findex put_byte/2
@snindex put_byte/2
@cnindex put_byte/2
As @code{put_byte(N)}, but to binary stream @var{S}.

@item put_char(+@var{S},+@var{A}) [ISO]
@findex put_char/2
@snindex put_char/2
@cnindex put_char/2
As @code{put_char(A)}, but to text stream @var{S}.

@item put_code(+@var{S},+@var{N}) [ISO]
@findex put_code/2
@snindex put_code/2
@cnindex put_code/2
As @code{put_code(N)}, but to text stream @var{S}.

@item get(+@var{S},-@var{C})
@findex get/2
@syindex get/2
@cnindex get/2
The same as @code{get(C)}, but from stream @var{S}.

@item get0(+@var{S},-@var{C})
@findex get0/2
@syindex get0/2
@cnindex get0/2
The same as @code{get0(C)}, but from stream @var{S}.

@item get_byte(+@var{S},-@var{C}) [ISO]
@findex get_byte/2
@snindex get_byte/2
@cnindex get_byte/2
If @var{C} is unbound, or is a character code, and the stream @var{S} is a
binary stream, read the next byte from that stream and unify its
code with @var{C}.

@item get_char(+@var{S},-@var{C}) [ISO]
@findex get_char/2
@snindex get_char/2
@cnindex get_char/2
If @var{C} is unbound, or is an atom representation of a character, and
the stream @var{S} is a text stream, read the next character from that
stream and unify its representation as an atom with @var{C}.

@item get_code(+@var{S},-@var{C}) [ISO]
@findex get_code/2
@snindex get_code/2
@cnindex get_code/2
If @var{C} is unbound, or is a character code, and the stream @var{S} is a
text stream, read the next character from that stream and unify its
code with @var{C}.

@item peek_byte(+@var{S},-@var{C}) [ISO]
@findex peek_byte/2
@snindex peek_byte/2
@cnindex peek_byte/2
If @var{C} is unbound, or is a character code, and @var{S} is a binary
stream, read the next byte from the current stream and unify its code
with @var{C}, while leaving the current stream position unaltered.

@item peek_char(+@var{S},-@var{C}) [ISO]
@findex peek_char/2
@snindex peek_char/2
@cnindex peek_char/2
If @var{C} is unbound, or is an atom representation of a character, and
the stream @var{S} is a text stream, read the next character from that
stream and unify its representation as an atom with @var{C}, while leaving
the current stream position unaltered.

@item peek_code(+@var{S},-@var{C}) [ISO]
@findex peek_code/2
@snindex peek_code/2
@cnindex peek_code/2
If @var{C} is unbound, or is an atom representation of a character, and
the stream @var{S} is a text stream, read the next character from that
stream and unify its representation as an atom with @var{C}, while leaving
the current stream position unaltered.

@item skip(+@var{S},-@var{C})
@findex skip/2
@syindex skip/2
@cnindex skip/2
Like @code{skip/1}, but using stream @var{S} instead of the current
input stream.

@item tab(+@var{S},+@var{N})
@findex tab/2
@syindex tab/2
@cnindex tab/2
The same as @code{tab/1}, but using stream @var{S}.

@item nl(+@var{S})
@findex nl/1
@syindex nl/1
@cnindex nl/1
Outputs a new line to stream @var{S}.

@end table

@node C-Prolog to Terminal, I/O Control, I/O for Streams, I/O
@subsection Compatible C-Prolog predicates for Terminal I/O

@table @code

@item ttyput(+@var{N})
@findex ttyput/1
@syindex ttyput/1
@cnindex ttyput/1
As @code{put(N)} but always to @code{user_output}.

@item ttyget(-@var{C})
@findex ttyget/1
@syindex ttyget/1
@cnindex ttyget/1
The same as @code{get(C)}, but from stream @code{user_input}.

@item ttyget0(-@var{C})
@findex ttyget0/1
@syindex ttyget0/1
@cnindex ttyget0/1
The same as @code{get0(C)}, but from stream @code{user_input}.

@item ttyskip(-@var{C})
@findex ttyskip/1
@syindex ttyskip/1
@cnindex ttyskip/1
Like @code{skip/1}, but always using stream @code{user_input}.
stream.

@item ttytab(+@var{N})
@findex ttytab/1
@syindex ttytab/1
@cnindex ttytab/1
The same as @code{tab/1}, but using stream @code{user_output}.

@item ttynl
@findex ttynl/0
@syindex ttynl/0
@cnindex ttynl/0
Outputs a new line to stream @code{user_output}.

@end table

@node I/O Control, Sockets, C-Prolog to Terminal, I/O
@subsection Controlling Input/Output

@table @code

@item exists(+@var{F})
@findex exists/1
@snindex exists/1
@cyindex exists/1
Checks if file @var{F} exists in the current directory.

@item nofileerrors
@findex nofileerrors/0
@syindex nofileerrors/0
@cyindex nofileerrors/0
Switches off the file_errors flag, so that the predicates @code{see/1},
@code{tell/1}, @code{open/3} and @code{close/1} just fail, instead of producing
an error message and aborting whenever the specified file cannot be
opened or closed.

@item fileerrors
@findex fileerrors/0
@syindex fileerrors/0
@cyindex fileerrors/0
Switches on the file_errors flag so that in certain error conditions
I/O predicates will produce an appropriated message and abort.

@item write_depth(@var{T},@var{L},@var{A})
@findex write_depth/3
@snindex write_depth/3
@cnindex write_depth/3
Unifies @var{T} with the value of the maximum depth of a term to be
written, @var{L} with the maximum length of a list to write, and @var{A}
with the maximum number of arguments of a compound term to write. The
setting will be used by @code{write/1} or @code{write/2}. The default
value for all arguments is 0, meaning unlimited depth and length.

@example
?- write_depth(3,5,5).
yes
?- write(a(b(c(d(e(f(g))))))).
a(b(c(....)))
yes
?- write([1,2,3,4,5,6,7,8]).
[1,2,3,4,5,...]
yes
?- write(a(1,2,3,4,5,6,7,8)).
a(1,2,3,4,5,...)
yes
@end example

@item write_depth(@var{T},@var{L})
@findex write_depth/2
@snindex write_depth/2
Same as @code{write_depth(@var{T},@var{L},_)}. Unifies @var{T} with the
value of the maximum depth of a term to be
written, and @var{L} with the maximum length of a list to write. The
setting will be used by @code{write/1} or @code{write/2}. The default
value for all arguments is 0, meaning unlimited depth and length.

@example
?- write_depth(3,5,5).
yes
?- write(a(b(c(d(e(f(g))))))).
a(b(c(....)))
yes
?- write([1,2,3,4,5,6,7,8]).
[1,2,3,4,5,...]
yes
@end example

@item always_prompt_user
@findex always_prompt_user/0
@snindex always_prompt_user/0
@cnindex always_prompt_user/0
Force the system to prompt the user even if the @code{user_input} stream
is not a terminal. This command is useful if you want to obtain
interactive control from a pipe or a socket.

@end table

@node Sockets, , I/O Control, I/O
@subsection Using Sockets From YAP

YAP includes a SICStus Prolog compatible socket interface. This
is a low level interface that provides direct access to the major socket
system calls. These calls can be used both to open a new connection in
the network or connect to a networked server. Socket connections are
described as read/write streams, and standard I/O built-ins can be used
to write on or read from sockets. The following calls are available:

@table @code

@item socket(+@var{DOMAIN},+@var{TYPE},+@var{PROTOCOL},-@var{SOCKET})
@findex socket/4
@syindex socket/4
@cnindex socket/4
Corresponds to the BSD system call @code{socket}. Create a socket for
domain @var{DOMAIN} of type @var{TYPE} and protocol
@var{PROTOCOL}. Both @var{DOMAIN} and @var{TYPE} should be atoms,
whereas @var{PROTOCOL} must be an integer. The new socket object is
accessible through a descriptor bound to the variable @var{SOCKET}.

The current implementation of YAP only accepts two socket
domains: @code{'AF_INET'} and @code{'AF_UNIX'}. Socket types depend on the
underlying operating system, but at least the following types are
supported: @code{'SOCK_STREAM'} and @code{'SOCK_DGRAM'}.

@item socket(+@var{DOMAIN},-@var{SOCKET})
@findex socket/2
@syindex socket/2
@cnindex socket/2

Call @code{socket/4} with @var{TYPE} bound to @code{'SOCK_STREAM'} and
@var{PROTOCOL} bound to @code{0}.

@item socket_close(+@var{SOCKET})
@findex socket_close/1
@syindex socket_close/1
@cnindex socket_close/1

Close socket @var{SOCKET}. Note that sockets used in
@code{socket_connect} (that is, client sockets) should not be closed with
@code{socket_close}, as they will be automatically closed when the
corresponding stream is closed with @code{close/1} or @code{close/2}.

@item socket_bind(+@var{SOCKET}, ?@var{PORT})
@findex socket_bind/2
@syindex socket_bind/2
@cnindex socket_bind/2

Interface to system call @code{bind}, as used for servers: bind socket
to a port. Port information depends on the domain:
@table @code
@item 'AF_UNIX'(+@var{FILENAME})
@item 'AF_FILE'(+@var{FILENAME})
use file name @var{FILENAME} for UNIX or local sockets.

@item 'AF_INET'(?@var{HOST},?PORT)
If @var{HOST} is bound to an atom, bind to host @var{HOST}, otherwise
if unbound bind to local host (@var{HOST} remains unbound). If port
@var{PORT} is bound to an integer, try to bind to the corresponding
port. If variable @var{PORT} is unbound allow operating systems to
choose a port number, which is unified with @var{PORT}.

@end table

@item socket_connect(+@var{SOCKET}, +@var{PORT}, -@var{STREAM})
@findex socket_connect/3
@syindex socket_connect/3
@cnindex socket_connect/3

Interface to system call @code{connect}, used for clients: connect
socket @var{SOCKET} to @var{PORT}. The connection results in the
read/write stream @var{STREAM}.

Port information depends on the domain:
@table @code
@item 'AF_UNIX'(+@var{FILENAME})
@item 'AF_FILE'(+@var{FILENAME})
connect to socket at file @var{FILENAME}.

@item 'AF_INET'(+@var{HOST},+@var{PORT})
Connect to socket at host @var{HOST} and port @var{PORT}.
@end table

@item socket_listen(+@var{SOCKET}, +@var{LENGTH})
@findex socket_listen/2
@syindex socket_listen/2
@cnindex socket_listen/2
Interface to system call @code{listen}, used for servers to indicate
willingness to wait for connections at socket @var{SOCKET}. The
integer @var{LENGTH} gives the queue limit for incoming connections,
and should be limited to @code{5} for portable applications. The socket
must be of type @code{SOCK_STREAM} or @code{SOCK_SEQPACKET}.

@item socket_accept(+@var{SOCKET}, -@var{STREAM})
@findex socket_accept/2
@syindex socket_accept/2
@cnindex socket_accept/2

@item socket_accept(+@var{SOCKET}, -@var{CLIENT}, -@var{STREAM})
@findex socket_accept/3
@syindex socket_accept/3
@cnindex socket_accept/3
Interface to system call @code{accept}, used for servers to wait for
connections at socket @var{SOCKET}. The stream descriptor @var{STREAM}
represents the resulting connection.  If the socket belongs to the
domain @code{'AF_INET'}, @var{CLIENT} unifies with an atom containing
the IP address for the client in numbers and dots notation.

@item socket_accept(+@var{SOCKET}, -@var{STREAM})
@findex socket_accept/2
@syindex socket_accept/2
@cnindex socket_accept/2
Accept a connection but do not return client information.

@item socket_buffering(+@var{SOCKET}, -@var{MODE}, -@var{OLD}, +@var{NEW})
@findex socket_buffering/4
@syindex socket_buffering/4
@cnindex socket_buffering/4
Set buffering for @var{SOCKET} in @code{read} or @code{write}
@var{MODE}. @var{OLD} is unified with the previous status, and @var{NEW}
receives the new status which may be one of @code{unbuf} or
@code{fullbuf}.

@item socket_select(+@var{SOCKETS}, -@var{NEWSTREAMS}, +@var{TIMEOUT}, +@var{STREAMS}, -@var{READSTREAMS})
@findex socket_select/5
@syindex socket_select/5
@cnindex socket_select/5
Interface to system call @code{select}, used for servers to wait for
connection requests or for data at sockets. The variable
@var{SOCKETS} is a list of form @var{KEY-SOCKET}, where @var{KEY} is
an user-defined identifier and @var{SOCKET} is a socket descriptor. The
variable @var{TIMEOUT} is either @code{off}, indicating execution will
wait until something is available, or of the form @var{SEC-USEC}, where
@var{SEC} and @var{USEC} give the seconds and microseconds before
@code{socket_select/5} returns. The variable @var{SOCKETS} is a list of
form @var{KEY-STREAM}, where @var{KEY} is an user-defined identifier
and @var{STREAM} is a stream descriptor

Execution of @code{socket_select/5} unifies @var{READSTREAMS} from
@var{STREAMS} with readable data, and @var{NEWSTREAMS} with a list of
the form @var{KEY-STREAM}, where @var{KEY} was the key for a socket
with pending data, and @var{STREAM} the stream descriptor resulting
from accepting the connection.  

@item current_host(?@var{HOSTNAME})
Unify @var{HOSTNAME} with an atom representing the fully qualified
hostname for the current host. Also succeeds if @var{HOSTNAME} is bound
to the unqualified hostname.

@item hostname_address(?@var{HOSTNAME},?@var{IP_ADDRESS})
@var{HOSTNAME} is an host name and @var{IP_ADDRESS} its IP
address in number and dots notation.


@end table

@node Database, Sets, I/O, Top
@section Using the Clausal Data Base

Predicates in YAP may be dynamic or static. By default, when
consulting or reconsulting, predicates are assumed to be static:
execution is faster and the code will probably use less space.
Static predicates impose some restrictions: in general there can be no 
addition or removal of  clauses for a procedure if it is being used in the
current execution.

Dynamic predicates allow programmers to change the Clausal Data Base with
the same flexibility as in C-Prolog. With dynamic predicates it is
always possible to add or remove clauses during execution and the
semantics will be the same as for C-Prolog. But the programmer should be
aware of the fact that asserting or retracting are still expensive operations, 
and therefore he should try to avoid them whenever possible.

@table @code

@item dynamic +@var{P}
@findex dynamic/1
@saindex dynamic/1
@cnindex dynamic/1
Declares predicate @var{P} or list of predicates [@var{P1},...,@var{Pn}]
as a dynamic predicate. @var{P} must be written in form:
@var{name/arity}.

@example
:- dynamic god/1.
@end example

@noindent 
a more convenient form can be used:

@example
:- dynamic son/3, father/2, mother/2.
@end example

or, equivalently,

@example
:- dynamic [son/3, father/2, mother/2].
@end example

@noindent
Note:

a predicate is assumed to be dynamic when 
asserted before being defined.

@item dynamic_predicate(+@var{P},+@var{Semantics})
@findex dynamic_predicate/2
@snindex dynamic_predicate/2
@cnindex dynamic_predicate/2
Declares predicate @var{P} or list of predicates [@var{P1},...,@var{Pn}]
as a dynamic predicate following either @code{logical} or
@code{immediate} semantics.

@menu

Subnodes of Database
* Modifying the Database:: Asserting and Retracting
* Looking at the Database:: Finding out what is in the Data Base
* Database References:: Using Data Base References
* Internal Database:: YAP's Internal Database
* BlackBoard:: Storing and Fetching Terms in the BlackBoard

@end menu

@end table

@node Modifying the Database, Looking at the Database, , Database
@subsection Modification of the Data Base

These predicates can be used either for static or for dynamic
predicates:

@table @code

@item assert(+@var{C})
@findex assert/1
@saindex assert/1
@caindex assert/1
 Adds clause @var{C} to the program. If the predicate is undefined,
declare it as dynamic.

 Most Prolog systems only allow asserting clauses for dynamic
predicates. This is also as specified in the ISO standard. YAP allows
asserting clauses for static predicates, as long as the predicate is not
in use and the language flag is @t{cprolog}. Note that this feature is
deprecated, if you want to assert clauses for static procedures you
should use @code{assert_static/1}.

@item asserta(+@var{C}) [ISO]
@findex asserta/1
@saindex asserta/1
@caindex asserta/1
 Adds clause @var{C} to the beginning of the program. If the predicate is
undefined, declare it as dynamic.

@item assertz(+@var{C}) [ISO]
@findex assertz/1
@saindex assertz/1
@caindex assertz/1
 Adds clause @var{C} to the end of the program. If the predicate is
undefined, declare it as dynamic.

 Most Prolog systems only allow asserting clauses for dynamic
predicates. This is also as specified in the ISO standard. YAP allows
asserting clauses for static predicates. The current version of YAP
supports this feature, but this feature is deprecated and support may go
away in future versions.

@item abolish(+@var{PredSpec}) [ISO]
@findex abolish/1
@saindex abolish/1
@caindex abolish/1
   Deletes the predicate given by @var{PredSpec} from the database. If
@var{PredSpec} is an unbound variable, delete all predicates for the
current module. The
specification must include the name and arity, and it may include module
information. Under @t{iso} language mode this built-in will only abolish
dynamic procedures. Under other modes it will abolish any procedures. 

@item abolish(+@var{P},+@var{N})
@findex abolish/2
@saindex abolish/2
@caindex abolish/2
 Deletes the predicate with name @var{P} and arity @var{N}. It will remove
both static and dynamic predicates.

@item assert_static(:@var{C})
@findex assert_static/1
@snindex assert_static/1
@cnindex assert_static/1
Adds clause @var{C} to a static procedure. Asserting a static clause
for a predicate while choice-points for the predicate are available has
undefined results.

@item asserta_static(:@var{C})
@findex asserta_static/1
@snindex asserta_static/1
@cnindex asserta_static/1
 Adds clause @var{C} to the beginning of a static procedure. 

@item assertz_static(:@var{C})
@findex assertz_static/1
@snindex assertz_static/1
@cnindex assertz_static/1
 Adds clause @var{C} to the end of a static procedure.  Asserting a
static clause for a predicate while choice-points for the predicate are
available has undefined results.

@end table

The following predicates can be used for dynamic predicates and for
static predicates, if source mode was on when they were compiled:

@table @code

@item clause(+@var{H},@var{B}) [ISO]
@findex clause/2
@saindex clause/2
@caindex clause/2
  A clause whose head matches @var{H} is searched for in the
program. Its head and body are respectively unified with @var{H} and
@var{B}. If the clause is a unit clause, @var{B} is unified with
@var{true}.

This predicate is applicable to static procedures compiled with
@code{source} active, and to all dynamic procedures.

@item clause(+@var{H},@var{B},-@var{R})
@findex clause/3
@saindex clause/3
@caindex clause/3
The same as @code{clause/2}, plus @var{R} is unified with the
reference to the clause in the database. You can use @code{instance/2}
to access the reference's value. Note that you may not use
@code{erase/1} on the reference on static procedures.

@item nth_clause(+@var{H},@var{I},-@var{R})
@findex nth_clause/3
@saindex nth_clause/3
@caindex nth_clause/3
Find the @var{I}th clause in the predicate defining @var{H}, and give
a reference to the clause. Alternatively, if the reference @var{R} is
given the head @var{H} is unified with a description of the predicate
and @var{I} is bound to its position.

@end table

The following predicates can only be used for dynamic predicates:

@table @code

@item retract(+@var{C}) [ISO]
@findex retract/1
@saindex retract/1
@cnindex retract/1
Erases the first clause in the program that matches @var{C}. This
predicate may also be used for the static predicates that have been
compiled when the source mode was @code{on}. For more information on
@code{source/0} (@pxref{Setting the Compiler}).

@item retractall(+@var{G})
@findex retractall/1
@saindex retractall/1
@cnindex retractall/1
Retract all the clauses whose head matches the goal @var{G}. Goal
@var{G} must be a call to a dynamic predicate.

@end table

@node Looking at the Database, Database References, Modifying the Database, Database
@subsection Looking at the Data Base

@table @code

@item listing
@findex listing/0
@saindex listing/0
@caindex listing/0
Lists in the current output stream all the clauses for which source code
is available (these include all clauses for dynamic predicates and
clauses for static predicates compiled when source mode was @code{on}).

@item listing(+@var{P})
@findex listing/1
@syindex listing/1
@caindex listing/1
Lists predicate @var{P} if its source code is available.

@item portray_clause(+@var{C})
@findex portray_clause/1
@syindex portray_clause/1
@cnindex portray_clause/1
Write clause @var{C} as if written by @code{listing/0}.

@item portray_clause(+@var{S},+@var{C})
@findex portray_clause/2
@syindex portray_clause/2
@cnindex portray_clause/2
Write clause @var{C} on stream @var{S} as if written by @code{listing/0}.

@item current_atom(@var{A})
@findex current_atom/1
@syindex current_atom/1
@cyindex current_atom/1
Checks whether @var{A} is a currently defined atom. It is used to find all
currently defined atoms by backtracking.

@item current_predicate(@var{F}) [ISO]
@findex current_predicate/1
@syindex current_predicate/1
@cyindex current_predicate/1
@var{F} is the predicate indicator for a currently defined user or
library predicate. @var{F} is of the form @var{Na/Ar}, where the atom
@var{Na} is the name of the predicate, and @var{Ar} its arity.

@item current_predicate(@var{A},@var{P})
@findex current_predicate/2
@syindex current_predicate/2
@cnindex current_predicate/2
Defines the relation: @var{P} is a currently defined predicate whose
name is the atom @var{A}.

@item system_predicate(@var{A},@var{P})
@findex system_predicate/2
@syindex system_predicate/2
@cnindex system_predicate/2
Defines the relation:  @var{P} is a built-in predicate whose name
is the atom @var{A}.

@item predicate_property(@var{P},@var{Prop})
@findex predicate_property/2
@saindex predicate_property/2
@cnindex predicate_property/2
For the predicates obeying the specification @var{P} unify @var{Prop}
with a property of @var{P}. These properties may be:
@table @code
@item built_in
true for built-in predicates,
@item dynamic
true if the predicate is dynamic
@item static
true if the predicate is static
@item meta_predicate(@var{M})
true if the predicate has a meta_predicate declaration @var{M}.
@item multifile
true if the predicate was declared to be multifile
@item imported_from(@var{Mod})
true if the predicate was imported from module @var{Mod}.
@item exported
true if the predicate is exported in the current module.
@item public
true if the predicate is public; note that all dynamic predicates are
public.
@item tabled
true if the predicate is tabled; note that only static predicates can
be tabled in YAP.
@item source
true if source for the predicate is available.
@item number_of_clauses(@var{ClauseCount})
Number of clauses in the predicate definition. Always one if external
or built-in.
@end table

@end table

@node Database References, Internal Database, Looking at the Database, Database
@subsection Using Data Base References

Data Base references are a fast way of accessing terms. The predicates
@code{erase/1} and @code{instance/1} also apply to these references and may
sometimes be used instead of @code{retract/1} and @code{clause/2}.

@table @code

@item assert(+@var{C},-@var{R})
@findex assert/2
@saindex assert/2
@caindex assert/2
 The same as @code{assert(C)} (@pxref{Modifying the Database}) but
unifies @var{R} with the  database reference that identifies the new
clause, in a one-to-one way. Note that @code{asserta/2} only works for dynamic
predicates. If the predicate is undefined, it will automatically be
declared dynamic.

@item asserta(+@var{C},-@var{R})
@findex asserta/2
@saindex asserta/2
@caindex asserta/2
 The same as @code{asserta(C)} but unifying @var{R} with
the  database reference that identifies the new clause, in a 
one-to-one way. Note that @code{asserta/2} only works for dynamic
predicates. If the predicate is undefined, it will automatically be
declared dynamic.

@item assertz(+@var{C},-@var{R})
@findex assertz/2
@saindex assertz/2
@caindex assertz/2
 The same as @code{assertz(C)} but unifying @var{R} with
the  database reference that identifies the new clause, in a 
one-to-one way. Note that @code{asserta/2} only works for dynamic
predicates. If the predicate is undefined, it will automatically be
declared dynamic.

@item retract(+@var{C},-@var{R})
@findex retract/2
@saindex retract/2
@caindex retract/2
 Erases from the program the clause @var{C} whose 
database reference is @var{R}. The predicate must be dynamic.


@end table

@node Internal Database, BlackBoard, Database References, Database
@section Internal Data Base
Some programs need global information for, e.g. counting or collecting 
data obtained by backtracking. As a rule, to keep this information, the
internal data base should be used instead of asserting and retracting
clauses (as most novice programmers  do), .
In YAP (as in some other Prolog systems) the internal data base (i.d.b. 
for short) is faster, needs less space and provides a better insulation of 
program and data than using asserted/retracted clauses.
The i.d.b. is implemented as a set of terms, accessed by keys that 
unlikely what happens in (non-Prolog) data bases are not part of the 
term. Under each key a list of terms is kept. References are provided so that 
terms can be identified: each term in the i.d.b. has a unique reference 
(references are also available for clauses of dynamic predicates).

@table @code

@item recorda(+@var{K},@var{T},-@var{R})
@findex recorda/3
@saindex recorda/3
@cyindex recorda/3
Makes term @var{T} the first record under key @var{K} and  unifies @var{R}
with its reference.

@item recordz(+@var{K},@var{T},-@var{R})
@findex recordz/3
@saindex recordz/3
@cyindex recordz/3
Makes term @var{T} the last record under key @var{K} and unifies @var{R}
with its reference.

@item recorda_at(+@var{R0},@var{T},-@var{R})
@findex recorda_at/3
@snindex recorda_at/3
@cnindex recorda_at/3
Makes term @var{T} the record preceding record with reference
@var{R0}, and unifies @var{R} with its reference.

@item recordz_at(+@var{R0},@var{T},-@var{R})
@findex recordz_at/3
@snindex recordz_at/3
@cnindex recordz_at/3
Makes term @var{T} the record following record with reference
@var{R0}, and unifies @var{R} with its reference.

@item recordaifnot(+@var{K},@var{T},-@var{R})
@findex recordaifnot/3
@saindex recordaifnot/3
@cnindex recordaifnot/3
If a term equal to @var{T} up to variable renaming is stored under key
@var{K} fail. Otherwise, make term @var{T} the first record under key
@var{K} and unify @var{R} with its reference.

@item recordzifnot(+@var{K},@var{T},-@var{R})
@findex recorda/3
@snindex recorda/3
@cnindex recorda/3
If a term equal to @var{T} up to variable renaming is stored under key
@var{K} fail. Otherwise, make term @var{T} the first record under key
@var{K} and unify @var{R} with its reference.

@item recorded(+@var{K},@var{T},@var{R})
@findex recorded/3
@saindex recorded/3
@cyindex recorded/3
Searches in the internal database under the key @var{K}, a term that
unifies with @var{T} and whose reference matches @var{R}. This
built-in may be used in one of two ways:
@itemize @bullet
@item @var{K} may be given, in this case the built-in will return all
elements of the internal data-base that match the key.
@item @var{R} may be given, if so returning the key and element that
match the reference.
@end itemize

@item erase(+@var{R})
@findex erase/1
@saindex erase/1
@cyindex erase/1
The term referred to by @var{R} is erased from the internal database. If
reference @var{R} does not exist in the database, @code{erase} just fails.

@item erased(+@var{R})
@findex erased/1
@saindex erased/1
@cyindex erased/1
Succeeds if the object whose database reference is @var{R} has been
erased.

@item instance(+@var{R},-@var{T})
@findex instance/2
@saindex instance/2
@cyindex instance/2
If @var{R} refers to a clause or a recorded term, @var{T} is unified
with its most general instance. If @var{R} refers to an unit clause
@var{C}, then @var{T} is unified with @code{@var{C} :- true}. When
@var{R} is not a reference to an existing clause or to a recorded term,
this goal fails.

@item eraseall(+@var{K})
@findex eraseall/1
@snindex eraseall/1
@cnindex eraseall/1
All terms belonging to the key @code{K} are erased from the internal
database. The predicate always succeeds.

@item current_key(?@var{A},?@var{K})
@findex current_key/2
@syindex current_key/2
@cnindex current_key/2
Defines the relation: @var{K} is a currently defined database key whose
name is the atom @var{A}. It can be used to generate all the keys for
the internal data-base.

@item nth_instance(?@var{Key},?@var{Index},?@var{R})
@findex nth_instance/3
@saindex nth_instance/3
@cnindex nth_instance/3
Fetches the @var{Index}nth entry in the internal database under the key
@var{Key}. Entries are numbered from one. If the key @var{Key} or the
@var{Index} are bound, a reference is unified with @var{R}. Otherwise,
the reference @var{R} must be given, and YAP will find
the matching key and index.


@item nth_instance(?@var{Key},?@var{Index},@var{T},?@var{R})
@findex nth_instance/4
@saindex nth_instance/4
@cnindex nth_instance/4
Fetches the @var{Index}nth entry in the internal database under the key
@var{Key}. Entries are numbered from one. If the key @var{Key} or the
@var{Index} are bound, a reference is unified with @var{R}. Otherwise,
the reference @var{R} must be given, and YAP will find
the matching key and index.

@item key_statistics(+@var{K},-@var{Entries},-@var{Size},-@var{IndexSize})
@findex key_statistics/4
@snindex key_statistics/4
@cnindex key_statistics/4
Returns several statistics for a key @var{K}. Currently, it says how
many entries we have for that key, @var{Entries}, what is the
total size spent on entries, @var{Size}, and what is the amount of
space spent in indices.

@item key_statistics(+@var{K},-@var{Entries},-@var{TotalSize})
@findex key_statistics/3
@snindex key_statistics/3
@cnindex key_statistics/3
Returns several statistics for a key @var{K}. Currently, it says how
many entries we have for that key, @var{Entries}, what is the
total size spent on this key.

@item get_value(+@var{A},-@var{V})
@findex get_value/2
@snindex get_value/2
@cnindex get_value/2
In YAP, atoms can be associated with constants. If one such
association exists for atom @var{A}, unify the second argument with the
constant. Otherwise, unify @var{V} with @code{[]}.

This predicate is YAP specific.

@item set_value(+@var{A},+@var{C})
@findex set_value/2 
@snindex set_value/2 
@cnindex set_value/2 
Associate atom @var{A} with constant @var{C}.

The @code{set_value} and @code{get_value} built-ins give a fast alternative to
the internal data-base. This is a simple form of implementing a global
counter.
@example
       read_and_increment_counter(Value) :-
                get_value(counter, Value),
                Value1 is Value+1,
                set_value(counter, Value1).
@end example
@noindent
This predicate is YAP specific.

@item recordzifnot(+@var{K},@var{T},-@var{R})
@findex recordzifnot/3
@snindex recordzifnot/3
@cnindex recordzifnot/3
If a variant of @var{T} is stored under key @var{K} fail. Otherwise, make
term @var{T} the last record under key @var{K} and unify @var{R} with its
reference.

This predicate is YAP specific.

@item recordaifnot(+@var{K},@var{T},-@var{R})
@findex recordaifnot/3
@snindex recordaifnot/3
@cnindex recordaifnot/3
If a variant of @var{T} is stored under key @var{K} fail. Otherwise, make
term @var{T} the first record under key @var{K} and unify @var{R} with its
reference.

This predicate is YAP specific.

@end table

There is a strong analogy between the i.d.b. and the way dynamic 
predicates are stored. In fact, the main i.d.b. predicates might be 
implemented using dynamic predicates:

@example
recorda(X,T,R) :- asserta(idb(X,T),R).
recordz(X,T,R) :- assertz(idb(X,T),R).
recorded(X,T,R) :- clause(idb(X,T),R).
@end example
@noindent
 We can take advantage of this, the other way around, as it is quite 
easy to write a simple Prolog interpreter, using the i.d.b.:

@example
asserta(G) :- recorda(interpreter,G,_).
assertz(G) :- recordz(interpreter,G,_).
retract(G) :- recorded(interpreter,G,R), !, erase(R).
call(V) :- var(V), !, fail.
call((H :- B)) :- !, recorded(interpreter,(H :- B),_), call(B).
call(G) :- recorded(interpreter,G,_).
@end example
@noindent
In YAP, much attention has been given to the implementation of the 
i.d.b., especially to the problem of accelerating the access to terms kept in 
a large list under the same key. Besides using the key, YAP uses an internal 
lookup function, transparent to the user, to find only the terms that might 
unify. For instance, in a data base containing the terms

@example
b
b(a)
c(d)
e(g)
b(X)
e(h)
@end example

@noindent
stored under the key k/1, when executing the query 

@example
:- recorded(k(_),c(_),R).
@end example

@noindent
@code{recorded} would proceed directly to the third term, spending almost the 
time as if @code{a(X)} or @code{b(X)} was being searched.
 The lookup function uses the functor of the term, and its first three
arguments (when they exist). So, @code{recorded(k(_),e(h),_)} would go
directly to the last term, while @code{recorded(k(_),e(_),_)} would find
first the fourth term, and then, after backtracking, the last one.

 This mechanism may be useful to implement a sort of hierarchy, where 
the functors of the terms (and eventually the first arguments) work as 
secondary keys.

 In the YAP's i.d.b. an optimized representation is used for 
terms without free variables. This results in a faster retrieval of terms 
and better space usage. Whenever possible, avoid variables in terms in terms stored in the  i.d.b.


@node BlackBoard, , Internal Database, Database
@section The Blackboard

YAP implements a blackboard in the style of the SICStus Prolog
blackboard. The blackboard uses the same underlying mechanism as the
internal data-base but has several important differences:
@itemize @bullet
@item It is module aware, in contrast to the internal data-base.
@item Keys can only be atoms or integers, and not compound terms.
@item A single term can be stored per key.
@item An atomic update operation is provided; this is useful for
parallelism.
@end itemize


@table @code
@item bb_put(+@var{Key},?@var{Term})
@findex bb_put/2
@syindex bb_put/2
@cnindex bb_put/2
Store term table @var{Term} in the blackboard under key @var{Key}. If a
previous term was stored under key @var{Key} it is simply forgotten.

@item bb_get(+@var{Key},?@var{Term})
@findex bb_get/2
@syindex bb_get/2
@cnindex bb_get/2
Unify @var{Term} with a term stored in the blackboard under key
@var{Key}, or fail silently if no such term exists.

@item bb_delete(+@var{Key},?@var{Term})
@findex bb_delete/2
@syindex bb_delete/2
@cnindex bb_delete/2
Delete any term stored in the blackboard under key @var{Key} and unify
it with @var{Term}. Fail silently if no such term exists.

@item bb_update(+@var{Key},?@var{Term},?@var{New})
@findex bb_update/3
@syindex bb_update/3
@cnindex bb_update/3
Atomically  unify a term stored in the blackboard under key @var{Key}
with @var{Term}, and if the unification succeeds replace it by
@var{New}. Fail silently if no such term exists or if unification fails.

@end table

@node Sets, Grammars, Database, Top
@section Collecting Solutions to a Goal

When there are several solutions to a goal, if the user wants to collect all
the solutions he may be led to use the data base, because backtracking will
forget previous solutions.

YAP allows the programmer to choose from several system
predicates instead of writing his own routines.  @code{findall/3} gives you
the fastest, but crudest solution. The other built-in predicates
post-process the result of the query in several different ways:

@table @code

@item findall(@var{T},+@var{G},-@var{L}) [ISO]
@findex findall/3
@syindex findall/3
@cyindex findall/3
Unifies @var{L} with a list that contains all the instantiations of the
term @var{T} satisfying the goal @var{G}.

With the following program:
@example
a(2,1).
a(1,1).
a(2,2).
@end example
@noindent
the answer to the query
@example
findall(X,a(X,Y),L).
@end example
@noindent
would be:
@example
X = _32
Y = _33
L = [2,1,2];
no
@end example

@item findall(@var{T},+@var{G},+@var{L},-@var{L0})
@findex findall/4
@syindex findall/4
@cnindex findall/4
Similar to @code{findall/3}, but appends all answers to list @var{L0}.

@item all(@var{T},+@var{G},-@var{L})
@findex all/3
@snindex all/3
@cnindex all/3
Similar to @code{findall(@var{T},@var{G},@var{L})} but eliminate
repeated elements. Thus, assuming the same clauses as in the above
example, the reply to the query

@example
all(X,a(X,Y),L).
@end example
@noindent
would be:

@example
X = _32
Y = _33
L = [2,1];
no
@end example

Note that @code{all/3} will fail if no answers are found.

@item bagof(@var{T},+@var{G},-@var{L}) [ISO]
@findex bagof/3
@saindex bagof/3
@cyindex bagof/3
For each set of possible instances of the free variables occurring in
@var{G} but not in @var{T}, generates the list @var{L} of the instances of
@var{T} satisfying @var{G}. Again, assuming the same clauses as in the
examples above, the reply to the query

@example
bagof(X,a(X,Y),L).

would be:
X = _32
Y = 1
L = [2,1];
X = _32
Y = 2
L = [2];
no
@end example

@item setof(@var{X},+@var{P},-@var{B}) [ISO]
@findex setof/3
@saindex setof/3
@cyindex setof/3
Similar to @code{bagof(@var{T},@var{G},@var{L})} but sorting list
@var{L} and keeping only one copy of each element.  Again, assuming the
same clauses as in the examples above, the reply to the query
@example
setof(X,a(X,Y),L).
@end example
@noindent
would be:
@example
X = _32
Y = 1
L = [1,2];
X = _32
Y = 2
L = [2];
no
@end example

@end table

@node Grammars, OS, Sets, Top
@section Grammar Rules

Grammar rules in Prolog are both a convenient way to express definite
clause grammars and  an extension of the well known context-free grammars.

A grammar rule is of the form:

@example
@i{ head --> body }
@end example
@noindent
where both @i{head} and @i{body} are sequences of one or more items
linked by the standard conjunction operator ','.

@emph{Items can be:}

@itemize @bullet
@item
a @emph{non-terminal} symbol may be either a complex term or an atom.
@item
a @emph{terminal} symbol may be any Prolog symbol. Terminals are
written as Prolog lists.
@item
an @emph{empty body} is written as the empty list '[ ]'.
@item
@emph{extra conditions} may be inserted as Prolog procedure calls, by being
written inside curly brackets '@{' and '@}'.
@item
the left side of a rule consists of a nonterminal and an optional list
of terminals.
@item
alternatives may be stated in the right-hand side of the rule by using
the disjunction operator ';'.
@item
the @emph{cut} and @emph{conditional} symbol ('->') may be inserted in the 
right hand side of a grammar rule
@end itemize

Grammar related built-in predicates:

@table @code

@item @var{CurrentModule}:expand_term(@var{T},-@var{X})
@item user:expand_term(@var{T},-@var{X})
@findex expand_term/2
@syindex expand_term/2
@cyindex expand_term/2
@findex term_expansion/2
@syindex term_expansion/2
@cyindex term_expansion/2
This predicate is used by YAP for preprocessing each top level
term read when consulting a file and before asserting or executing it.
It rewrites a term @var{T} to a term @var{X} according to the following
rules: first try @code{term_expansion/2}  in the current module, and then try to use the user defined predicate
@code{user:term_expansion/2}. If this call fails then the translating process
for DCG rules is applied, together with the arithmetic optimizer
whenever the compilation of arithmetic expressions is in progress.

@item @var{CurrentModule}:goal_expansion(+@var{G},+@var{M},-@var{NG})
@item user:goal_expansion(+@var{G},+@var{M},-@var{NG})
@findex goal_expansion/3
@snindex goal_expansion/3
@cnindex goal_expansion/3
YAP now supports @code{goal_expansion/3}. This is an user-defined
procedure that is called after term expansion when compiling or
asserting goals for each sub-goal in a clause. The first argument is
bound to the goal and the second to the module under which the goal
@var{G} will execute. If @code{goal_expansion/3} succeeds the new
sub-goal @var{NG} will replace @var{G} and will be processed in the same
way. If @code{goal_expansion/3} fails the system will use the default
rules.

@item phrase(+@var{P},@var{L},@var{R})
@findex phrase/3
@syindex phrase/3
@cnindex phrase/3
This predicate succeeds when the difference list @code{@var{L}-@var{R}}
is a phrase of type @var{P}.

@item phrase(+@var{P},@var{L})
@findex phrase/2
@syindex phrase/2
@cnindex phrase/2
This predicate succeeds when @var{L} is a phrase of type @var{P}. The
same as @code{phrase(P,L,[])}.

Both this predicate and the previous are used as a convenient way to
start execution of grammar rules.

@item 'C'(@var{S1},@var{T},@var{S2})
@findex C/3
@syindex C/3
@cnindex C/3
This predicate is used by the grammar rules compiler and is defined as
@code{'C'([H|T],H,T)}.

@end table

@node OS, Term Modification, Grammars, Top
@section Access to Operating System Functionality

The following built-in predicates allow access to underlying
Operating System functionality: 

@table @code

@item cd(+@var{D})
@findex cd/1
@snindex cd/1
@cnindex cd/1
Changes the current directory (on UNIX environments).

@item environ(+@var{E},-@var{S})
@findex environ/2
@syindex environ/2
@cnindex environ/2
@comment This backtrackable predicate unifies the first argument with an
@comment environment variable @var{E}, and the second with its value @var{S}. It
@comment can used to detect all environment variables.
    Given an environment variable @var{E} this predicate unifies the second argument @var{S} with its value.

@item getcwd(-@var{D})
@findex getcwd/1
@snindex getcwd/1
@cnindex getcwd/1
Unify the current directory, represented as an atom, with the argument
@var{D}.

@item putenv(+@var{E},+@var{S})
@findex putenv/2
@snindex putenv/2
@cnindex putenv/2
Set environment variable @var{E} to the value @var{S}. If the
environment variable @var{E} does not exist, create a new one. Both the
environment variable and the value must be atoms.

@item rename(+@var{F},+@var{G})
@findex rename/2
@snindex rename/2
@cyindex rename/2
Renames file @var{F} to @var{G}.

@item sh
@findex sh/0
@snindex sh/0
@cyindex sh/0
Creates a new shell interaction.

@item system(+@var{S})
@findex system/1
@snindex system/1
@cyindex system/1
Passes command @var{S} to the Bourne shell (on UNIX environments) or the
current command interpreter in WIN32 environments.

@item unix(+@var{S})
@findex unix/1
@snindex unix/1
@cnindex unix/1
Access to Unix-like functionality:
@table @code
@item argv/1
Return a list of arguments to the program. These are the arguments that
follow a @code{--}, as in the usual Unix convention.
@item cd/0
Change to home directory.
@item cd/1
Change to given directory. Acceptable directory names are strings or
atoms.
@item environ/2
If the first argument is an atom, unify the second argument with the
value of the corresponding environment variable.
@item getcwd/1
Unify the first argument with an atom representing the current directory.
@item putenv/2
Set environment variable @var{E} to the value @var{S}. If the
environment variable @var{E} does not exist, create a new one. Both the
environment variable and the value must be atoms.
@item shell/1
Execute command under current shell. Acceptable commands are strings or
atoms.
@item system/1
Execute command with @code{/bin/sh}. Acceptable commands are strings or
atoms.
@item shell/0
Execute a new shell.
@end table

@item alarm(+@var{Seconds},+@var{Callable},+@var{OldAlarm})
@findex alarm/3
@snindex alarm/3
@cnindex alarm/3
Arranges for YAP to be interrupted in @var{Seconds} seconds, or in
[@var{Seconds}|@var{MicroSeconds}]. When interrupted, YAP will execute
@var{Callable} and then return to the previous execution. If
@var{Seconds} is @code{0}, no new alarm is scheduled. In any event,
any previously set alarm is canceled.

The variable @var{OldAlarm} unifies with the number of seconds remaining
until any previously scheduled alarm was due to be delivered, or with
@code{0} if there was no previously scheduled alarm.

Note that execution of @var{Callable} will wait if YAP is
executing built-in predicates, such as Input/Output operations.

The next example shows how @var{alarm/3} can be used to implement a
simple clock:

@example
loop :- loop.

ticker :- write('.'), flush_output,
          get_value(tick, yes),
          alarm(1,ticker,_).

:- set_value(tick, yes), alarm(1,ticker,_), loop.
@end example

The clock, @code{ticker}, writes a dot and then checks the flag
@code{tick} to see whether it can continue ticking. If so, it calls
itself again. Note that there is no guarantee that the each dot
corresponds a second: for instance, if the YAP is waiting for
user input, @code{ticker} will wait until the user types the entry in.

The next example shows how @code{alarm/3} can be used to guarantee that
a certain procedure does not take longer than a certain amount of time:

@example
loop :- loop.

:-   catch((alarm(10, throw(ball), _),loop),
        ball,
        format('Quota exhausted.~n',[])).
@end example
In this case after @code{10} seconds our @code{loop} is interrupted,
@code{ball} is thrown,  and the handler writes @code{Quota exhausted}.
Execution then continues from the handler.

Note that in this case @code{loop/0} always executes until the alarm is
sent. Often, the code you are executing succeeds or fails before the
alarm is actually delivered. In this case, you probably want to disable
the alarm when you leave the procedure. The next procedure does exactly so:
@example
once_with_alarm(Time,Goal,DoOnAlarm) :-
   catch(execute_once_with_alarm(Time, Goal), alarm, DoOnAlarm).

execute_once_with_alarm(Time, Goal) :-
        alarm(Time, alarm, _),
        ( call(Goal) -> alarm(0, alarm, _) ; alarm(0, alarm, _), fail).
@end example

The procedure @code{once_with_alarm/3} has three arguments:
the @var{Time} to wait before the alarm is
sent; the @var{Goal} to execute; and the goal @var{DoOnAlarm} to execute
if the alarm is sent. It uses @code{catch/3} to handle the case the
@code{alarm} is sent. Then it starts the alarm, calls the goal
@var{Goal}, and disables the alarm on success or failure.

@item on_signal(+@var{Signal},?@var{OldAction},+@var{Callable})
@findex on_signal/3
@snindex on_signal/3
@cnindex on_signal/3
Set the interrupt handler for soft interrupt @var{Signal} to be
@var{Callable}. @var{OldAction} is unified with the previous handler.

Only a subset of the software interrupts (signals) can have their
handlers manipulated through @code{on_signal/3}.
Their POSIX names, YAP names and default behavior is given below.
The "YAP name" of the signal is the atom that is associated with
each signal, and should be used as the first argument to
@code{on_signal/3}. It is chosen so that it matches the signal's POSIX
name.

@code{on_signal/3} succeeds, unless when called with an invalid
signal name or one that is not supported on this platform. No checks
are made on the handler provided by the user.

@table @code
@item sig_up (Hangup)
  SIGHUP in Unix/Linux; Reconsult the initialization files
  ~/.yaprc, ~/.prologrc and ~/prolog.ini.
@item sig_usr1 and sig_usr2 (User signals)
  SIGUSR1 and SIGUSR2 in Unix/Linux; Print a message and halt.
@end table

A special case is made, where if @var{Callable} is bound to
@code{default}, then the default handler is restored for that signal.

A call in the form @code{on_signal(@var{S},@var{H},@var{H})} can be used
to retrieve a signal's current handler without changing it.

It must be noted that although a signal can be received at all times,
the handler is not executed while YAP is waiting for a query at the
prompt. The signal will be, however, registered and dealt with as soon
as the user makes a query.

Please also note, that neither POSIX Operating Systems nor YAP guarantee
that the order of delivery and handling is going to correspond with the
order of dispatch.

@end table

@node Term Modification, Global Variables, OS, Top
@section Term Modification

@cindex updating terms
It is sometimes useful to change the value of instantiated
variables. Although, this is against the spirit of logic programming, it
is sometimes useful. As in other Prolog systems, YAP has
several primitives that allow updating Prolog terms. Note that these
primitives are also backtrackable.

The @code{setarg/3} primitive allows updating any argument of a Prolog
compound terms. The @code{mutable} family of predicates provides
@emph{mutable variables}. They should be used instead of @code{setarg/3},
as they allow the encapsulation of accesses to updatable
variables. Their implementation can also be more efficient for long
deterministic computations.

@table @code
@item setarg(+@var{I},+@var{S},?@var{T})
@findex setarg/3n
@snindex setarg/3n
@cnindex setarg/3n
Set the value of the @var{I}th argument of term @var{S} to term @var{T}. 

@cindex mutable variables
@item create_mutable(+@var{D},-@var{M})
@findex create_mutable/2
@syindex create_mutable/2
@cnindex create_mutable/2
Create new mutable variable @var{M} with initial value @var{D}.

@item get_mutable(?@var{D},+@var{M})
@findex get_mutable/2
@syindex get_mutable/2
@cnindex get_mutable/2
Unify the current value of mutable term @var{M} with term @var{D}.

@item is_mutable(?@var{D})
@findex is_mutable/1
@syindex is_mutable/1
@cnindex is_mutable/1
Holds if @var{D} is a mutable term.

@item get_mutable(?@var{D},+@var{M})
@findex get_mutable/2
@syindex get_mutable/2
@cnindex get_mutable/2
Unify the current value of mutable term @var{M} with term @var{D}.

@item update_mutable(+@var{D},+@var{M})
@findex update_mutable/2
@syindex update_mutable/2
@cnindex update_mutable/2
Set the current value of mutable term @var{M} to term @var{D}.
@end table

@node Global Variables, Profiling, Term Modification, Top
@section Global Variables

@cindex global variables

Global variables are associations between names (atoms) and
terms. They differ in various ways from storing information using
@code{assert/1} or @code{recorda/3}.

@itemize @bullet
@item The value lives on the Prolog (global) stack. This implies that
lookup time is independent from the size of the term. This is
particularly interesting for large data structures such as parsed XML
documents or the CHR global constraint store. 

@item They support both global assignment using @code{nb_setval/2} and
backtrackable assignment using @code{b_setval/2}.

@item Only one value (which can be an arbitrary complex Prolog term)
can be associated to a variable at a time. 

@item Their value cannot be shared among threads. Each thread has its own
namespace and values for global variables.
@end itemize

Currently global variables are scoped globally. We may consider module
scoping in future versions.   Both @code{b_setval/2} and
@code{nb_setval/2} implicitly create a variable if the referenced name
does not already refer to a variable.

Global variables may be initialised from directives to make them
available during the program lifetime, but some considerations are
necessary for saved-states and threads. Saved-states to not store
global variables, which implies they have to be declared with
@code{initialization/1} to recreate them after loading the saved
state. Each thread has its own set of global variables, starting with
an empty set. Using @code{thread_initialization/1} to define a global
variable it will be defined, restored after reloading a saved state
and created in all threads that are created after the
registration. Finally, global variables can be initialised using the
exception hook called @code{exception/3}. The latter technique is used
by CHR.

@table @code
@item b_setval(+@var{Name}, +@var{Value}) 
@findex b_setval/2
@snindex b_setval/2
@cnindex b_setval/2
Associate the term @var{Value} with the atom @var{Name} or replaces
the currently associated value with @var{Value}. If @var{Name} does
not refer to an existing global variable a variable with initial value
[] is created (the empty list). On backtracking the assignment is
reversed. 

@item b_getval(+@var{Name}, -@var{Value}) 
@findex b_getval/2
@snindex b_getval/2
@cnindex b_getval/2
Get the value associated with the global variable @var{Name} and unify
it with @var{Value}. Note that this unification may further
instantiate the value of the global variable. If this is undesirable
the normal precautions (double negation or @code{copy_term/2}) must be
taken. The @code{b_getval/2} predicate generates errors if @var{Name} is not
an atom or the requested variable does not exist. 

Notice that for compatibility with other systems @var{Name} @emph{must} be already associated with a term: otherwise the system will generate an error.

@item nb_setval(+@var{Name}, +@var{Value}) 
@findex nb_setval/2
@snindex nb_setval/2
@cnindex nb_setval/2
Associates a copy of @var{Value} created with @code{duplicate_term/2} with
the atom @var{Name}. Note that this can be used to set an initial
value other than @code{[]} prior to backtrackable assignment.

@item nb_getval(+@var{Name}, -@var{Value}) 
@findex nb_getval/2
@snindex nb_getval/2
@cnindex nb_getval/2
The @code{nb_getval/2} predicate is a synonym for @code{b_getval/2},
introduced for compatibility and symmetry. As most scenarios will use
a particular global variable either using non-backtrackable or
backtrackable assignment, using @code{nb_getval/2} can be used to
document that the variable is used non-backtrackable.

@item nb_linkval(+@var{Name}, +@var{Value}) 
@findex nb_linkval/2
@snindex nb_linkval/2
@cnindex nb_linkval/2
Associates the term @var{Value} with the atom @var{Name} without
copying it. This is a fast special-purpose variation of @code{nb_setval/2}
intended for expert users only because the semantics on backtracking
to a point before creating the link are poorly defined for compound
terms. The principal term is always left untouched, but backtracking
behaviour on arguments is undone if the original assignment was
trailed and left alone otherwise, which implies that the history that
created the term affects the behaviour on backtracking. Please
consider the following example:

@example
demo_nb_linkval :-
        T = nice(N),
        (   N = world,
            nb_linkval(myvar, T),
            fail
        ;   nb_getval(myvar, V),
            writeln(V)
        ).
@end example

@item nb_set_shared_val(+@var{Name}, +@var{Value}) 
@findex nb_set_shared_val/2
@snindex nb_set_shared_val/2
@cnindex nb_set_shared_val/2
Associates the term @var{Value} with the atom @var{Name}, but sharing
non-backtrackable terms. This may be useful if you want to rewrite a
global variable so that the new copy will survive backtracking, but
you want to share structure with the previous term.

The next example shows the differences between the three built-ins:
@example
?- nb_setval(a,a(_)),nb_getval(a,A),nb_setval(b,t(C,A)),nb_getval(b,B).
A = a(_A),
B = t(_B,a(_C)) ? 

?- nb_setval(a,a(_)),nb_getval(a,A),nb_set_shared_val(b,t(C,A)),nb_getval(b,B).

?- nb_setval(a,a(_)),nb_getval(a,A),nb_linkval(b,t(C,A)),nb_getval(b,B).
A = a(_A),
B = t(C,a(_A)) ?
@end example

@item nb_setarg(+@{Arg], +@var{Term}, +@var{Value})
@findex nb_setarg/3
@snindex nb_setarg/3
@cnindex nb_setarg/3

Assigns the @var{Arg}-th argument of the compound term @var{Term} with
the given @var{Value} as setarg/3, but on backtracking the assignment
is not reversed. If @var{Term} is not atomic, it is duplicated using
duplicate_term/2. This predicate uses the same technique as
@code{nb_setval/2}. We therefore refer to the description of
@code{nb_setval/2} for details on non-backtrackable assignment of
terms. This predicate is compatible to GNU-Prolog
@code{setarg(A,T,V,false)}, removing the type-restriction on
@var{Value}. See also @code{nb_linkarg/3}. Below is an example for
counting the number of solutions of a goal. Note that this
implementation is thread-safe, reentrant and capable of handling
exceptions. Realising these features with a traditional implementation
based on assert/retract or flag/3 is much more complicated.

@example
    succeeds_n_times(Goal, Times) :-
            Counter = counter(0),
            (   Goal,
                arg(1, Counter, N0),
                N is N0 + 1,
                nb_setarg(1, Counter, N),
                fail
            ;   arg(1, Counter, Times)
            ).
@end example

@item nb_set_shared_arg(+@var{Arg}, +@var{Term}, +@var{Value}) 
@findex nb_set_shared_arg/3
@snindex nb_set_shared_arg/3
@cnindex nb_set_shared_arg/3

As @code{nb_setarg/3}, but like @code{nb_linkval/2} it does not
duplicate the global sub-terms in @var{Value}. Use with extreme care
and consult the documentation of @code{nb_linkval/2} before use.

@item nb_linkarg(+@var{Arg}, +@var{Term}, +@var{Value}) 
@findex nb_linkarg/3
@snindex nb_lnkarg/3
@cnindex nb_linkarg/3

As @code{nb_setarg/3}, but like @code{nb_linkval/2} it does not
duplicate @var{Value}. Use with extreme care and consult the
documentation of @code{nb_linkval/2} before use.


@item nb_current(?@var{Name}, ?@var{Value}) 
@findex nb_current/2
@snindex nb_current/2
@cnindex nb_current/2
Enumerate all defined variables with their value. The order of
enumeration is undefined. 

@item nb_delete(+@var{Name}) 
@findex nb_delete/2
@snindex nb_delete/2
@cnindex nb_delete/2
Delete the named global variable. 
@end table

Global variables have been introduced by various Prolog
implementations recently. We follow the implementation of them in
SWI-Prolog, itself based on hProlog by Bart Demoen.

GNU-Prolog provides a rich set of global variables, including
arrays. Arrays can be implemented easily in YAP and SWI-Prolog using
@code{functor/3} and @code{setarg/3} due to the unrestricted arity of
compound terms.


@node Profiling, Call Counting, Global Variables, Top
@section Profiling Prolog Programs

@cindex profiling

YAP includes two profilers. The count profiler keeps information on the
number of times a predicate was called. This information can be used to
detect what are the most commonly called predicates in the program.  The
count profiler can be compiled by setting YAP's flag @code{profiling}
to @code{on}. The time-profiler is a @code{gprof} profiler, and counts
how many ticks are being spent on specific predicates, or on other
system functions such as internal data-base accesses or garbage collects.

The YAP profiling sub-system is currently under
development. Functionality for this sub-system will increase with newer
implementation.

@subsection The Count Profiler

@strong{Notes:}

The count profiler works by incrementing counters at procedure entry or
backtracking. It provides exact information:

@itemize @bullet
@item Profiling works for both static and dynamic predicates.
@item Currently only information on entries and retries to a predicate
are maintained. This may change in the future.
@item As an example, the following user-level program gives a list of
the most often called procedures in a program. The procedure
@code{list_profile} shows all procedures, irrespective of module, and
the procedure @code{list_profile/1} shows the procedures being used in
a specific module.
@example
list_profile :-
        % get number of calls for each profiled procedure
        setof(D-[M:P|D1],(current_module(M),profile_data(M:P,calls,D),profile_data(M:P,retries,D1)),LP),
        % output so that the most often called
        % predicates will come last:
        write_profile_data(LP).

list_profile(Module) :-
        % get number of calls for each profiled procedure
        setof(D-[Module:P|D1],(profile_data(Module:P,calls,D),profile_data(Module:P,retries,D1)),LP),
        % output so that the most often called
        % predicates will come last:
        write_profile_data(LP).

write_profile_data([]).
write_profile_data([D-[M:P|R]|SLP]) :-
        % swap the two calls if you want the most often
        %  called predicates first.
        format('~a:~w: ~32+~t~d~12+~t~d~12+~n', [M,P,D,R]),
        write_profile_data(SLP).
@end example
@end itemize

These are  the current predicates to access and clear profiling data:

@table @code
@item profile_data(?@var{Na/Ar}, ?@var{Parameter}, -@var{Data})
@findex profile_data/3
@snindex profile_data/3
@cnindex profile_data/3
Give current profile data on @var{Parameter} for a predicate described
by the predicate indicator @var{Na/Ar}. If any of @var{Na/Ar} or
@var{Parameter} are unbound, backtrack through all profiled predicates
or stored parameters. Current parameters are:

@table @code
@item calls
Number of times a procedure was called.

@item retries
 Number of times a call to the procedure was backtracked to and retried.
@end table

@item profile_reset
@findex profiled_reset/0
@snindex profiled_reset/0
@cnindex profiled_reset/0
Reset all profiling information.

@end table

@subsection Tick Profiler
The tick profiler works by interrupting the Prolog code every so often
and checking at each point the code was. The profiler must be able to
retrace the state of the abstract machine at every moment. The major
advantage of this approach is that it gives the actual amount of time
being spent per procedure, or whether garbage collection dominates
execution time. The major drawback is that tracking down the state of
the abstract machine may take significant time, and in the worst case
may slow down the whole execution.

The following procedures are available:

@table @code
@item profinit
@findex profinit/0
@snindex profinit/0
@cnindex profinit/0
Initialise the data-structures for the profiler. Unnecessary for
dynamic profiler.

@item profon
@findex profon/0
@snindex profon/0
@cnindex profon/0
Start profiling.

@item profoff
@findex profoff/0
@snindex profoff/0
@cnindex profoff/0
Stop profiling.

@item showprofres
@findex showprofres/0
@snindex showprofres/0
@cnindex showprofres/0
Show profiling info.

@item showprofres(@var{N})
@findex showprofres/1
@snindex showprofres/1
@cnindex showprofres/1
Show profiling info for the top-most @var{N} predicates.

@end table


@node Call Counting, Arrays, Profiling, Top
@section Counting Calls

@cindex Counting Calls
Predicates compiled with YAP's flag @code{call_counting} set to
@code{on} update counters on the numbers of calls and of
retries. Counters are actually decreasing counters, so that they can be
used as timers.  Three counters are available:
@itemize @bullet
@item @code{calls}: number of predicate calls since execution started or since
system was reset; 
@item @code{retries}: number of retries for predicates called since
execution started or since counters were reset;
@item @code{calls_and_retries}: count both on predicate calls and
retries.
@end itemize
These counters can be used to find out how many calls a certain
goal takes to execute. They can also be used as timers.

The code for the call counters piggybacks on the profiling
code. Therefore, activating the call counters also activates the profiling
counters.

These are  the predicates that access and manipulate the call counters:

@table @code
@item call_count_data(-@var{Calls}, -@var{Retries}, -@var{CallsAndRetries})
@findex call_count_data/3
@snindex call_count_data/3
@cnindex call_count_data/3
Give current call count data. The first argument gives the current value
for the @var{Calls} counter, next the @var{Retries} counter, and last
the @var{CallsAndRetries} counter.

@item call_count_reset
@findex call_count_data/0
@snindex call_count_data/0
@cnindex call_count_data/0
Reset call count counters. All timers are also reset.

@item call_count(?@var{CallsMax}, ?@var{RetriesMax}, ?@var{CallsAndRetriesMax})
@findex call_count_data/3
@snindex call_count_data/3
@cnindex call_count_data/3
Set call count counter as timers. YAP will generate an exception
if one of the instantiated call counters decreases to 0. YAP will ignore
unbound arguments:
@itemize @bullet
@item @var{CallsMax}: throw the exception @code{call_counter} when the
counter @code{calls} reaches 0;
@item @var{RetriesMax}: throw the exception @code{retry_counter} when the
counter @code{retries} reaches 0;
@item @var{CallsAndRetriesMax}: throw the exception
@code{call_and_retry_counter} when the counter @code{calls_and_retries}
reaches 0.
@end itemize
@end table

Next, we show a simple example of how to use call counters:
@example
   ?- yap_flag(call_counting,on), [-user]. l :- l. end_of_file. yap_flag(call_counting,off).

yes

yes
   ?- catch((call_count(10000,_,_),l),call_counter,format("limit_exceeded.~n",[])).

limit_exceeded.

yes
@end example
Notice that we first compile the looping predicate @code{l/0} with
@code{call_counting} @code{on}. Next, we @code{catch/3} to handle an
exception when @code{l/0} performs more than 10000 reductions.


@node Arrays, Preds, Call Counting , Top
@section Arrays

The YAP system includes experimental support for arrays. The
support is enabled with the option @code{YAP_ARRAYS}.

There are two very distinct forms of arrays in YAP. The
@emph{dynamic arrays} are a different way to access compound terms
created during the execution. Like any other terms, any bindings to
these terms and eventually the terms themselves will be destroyed during
backtracking. Our goal in supporting dynamic arrays is twofold. First,
they provide an alternative to the standard @code{arg/3}
built-in. Second, because dynamic arrays may have name that are globally
visible, a dynamic array can be visible from any point in the
program. In more detail, the clause
@example
g(X) :- array_element(a,2,X).
@end example
will succeed as long as the programmer has used the built-in @t{array/2}
to create an array term with at least 3 elements in the current
environment, and the array was associated with the name @code{a}.  The
element @code{X} is a Prolog term, so one can bind it and any such
bindings will be undone when backtracking. Note that dynamic arrays do
not have a type: each element may be any Prolog term.

The @emph{static arrays} are an extension of the database. They provide
a compact way for manipulating data-structures formed by characters,
integers, or floats imperatively. They can also be used to provide
two-way communication between YAP and external programs through
shared memory.

In order to efficiently manage space elements in a static array must
have a type. Currently, elements of static arrays in YAP should
have one of the following predefined types:

@itemize @bullet
@item  @code{byte}: an 8-bit signed character.
@item  @code{unsigned_byte}: an 8-bit unsigned character.
@item  @code{int}: Prolog integers. Size would be the natural size for
the machine's architecture.
@item  @code{float}: Prolog floating point number. Size would be equivalent
to a double in @code{C}.
@item  @code{atom}: a Prolog atom.
@item  @code{dbref}: an internal database reference.
@item  @code{term}: a generic Prolog term. Note that this will term will
not be stored in the array itself, but instead will be stored in the
Prolog internal database.
@end itemize

Arrays may be @emph{named} or @emph{anonymous}. Most arrays will be
@emph{named}, that is associated with an atom that will be used to find
the array. Anonymous arrays do not have a name, and they are only of
interest if the @code{TERM_EXTENSIONS} compilation flag is enabled. In
this case, the unification and parser are extended to replace
occurrences of Prolog terms of the form @code{X[I]} by run-time calls to
@code{array_element/3}, so that one can use array references instead of
extra calls to @code{arg/3}. As an example:
@example
g(X,Y,Z,I,J) :- X[I] is Y[J]+Z[I].
@end example
should give the same results as:
@example
G(X,Y,Z,I,J) :-
        array_element(X,I,E1),
        array_element(Y,J,E2),  
        array_element(Z,I,E3),  
        E1 is E2+E3.
@end example

Note that the only limitation on array size are the stack size for
dynamic arrays; and, the heap size for static (not memory mapped)
arrays. Memory mapped arrays are limited by available space in the file
system and in the virtual memory space.

The following predicates manipulate arrays:

@table @code

@item array(+@var{Name}, +@var{Size})
@findex array/2
@snindex array/2
@cnindex array/2
Creates a new dynamic array. The @var{Size} must evaluate to an
integer. The @var{Name} may be either an atom (named array) or an
unbound variable (anonymous array).

Dynamic arrays work as standard compound terms, hence space for the
array is recovered automatically on backtracking.

@item static_array(+@var{Name}, +@var{Size}, +@var{Type})
@findex static_array/3
@snindex static_array/3
@cnindex static_array/3
Create a new static array with name @var{Name}. Note that the @var{Name}
must be an atom (named array). The @var{Size} must evaluate to an
integer.  The @var{Type} must be bound to one of types mentioned
previously.

@item reset_static_array(+@var{Name})
@findex reset_static_array/1
@snindex reset_static_array/1
@cnindex reset_static_array/1
Reset static array with name @var{Name} to its initial value.

@item static_array_location(+@var{Name}, -@var{Ptr})
@findex static_array_location/4
@snindex static_array_location/4
@cnindex static_array_location/4
Give the location for  a static array with name
@var{Name}.

@item static_array_properties(?@var{Name}, ?@var{Size}, ?@var{Type})
@findex static_array_properties/3
@snindex static_array_properties/3
@cnindex static_array_properties/3
Show the properties size and type of a static array with name
@var{Name}. Can also be used to enumerate all current
static arrays. 

This built-in will silently fail if the there is no static array with
that name.

@item static_array_to_term(?@var{Name}, ?@var{Term})
@findex static_array_to_term/3
@snindex static_array_to_term/3
@cnindex static_array_to_term/3
Convert a static array with name
@var{Name} to a compound term of name @var{Name}.

This built-in will silently fail if the there is no static array with
that name.

@item mmapped_array(+@var{Name}, +@var{Size}, +@var{Type}, +@var{File})
@findex static_array/3
@snindex static_array/3
@cnindex static_array/3
Similar to @code{static_array/3}, but the array is memory mapped to file
@var{File}. This means that the array is initialized from the file, and
that any changes to the array will also be stored in the file. 

This built-in is only available in operating systems that support the
system call @code{mmap}. Moreover, mmapped arrays do not store generic
terms (type @code{term}).

@item close_static_array(+@var{Name})
@findex close_static_array/1
@snindex close_static_array/1
@cnindex close_static_array/1
Close an existing static array of name @var{Name}. The @var{Name} must
be an atom (named array). Space for the array will be recovered and
further accesses to the array will return an error. 

@item resize_static_array(+@var{Name}, -@var{OldSize}, +@var{NewSize})
@findex resize_static_array/3
@snindex resize_static_array/3
@cnindex resize_static_array/3
Expand or reduce a static array, The @var{Size} must evaluate to an
integer. The @var{Name} must be an atom (named array). The @var{Type}
must be bound to one of @code{int}, @code{dbref}, @code{float} or
@code{atom}.

Note that if the array is a mmapped array the size of the mmapped file
will be actually adjusted to correspond to the size of the array.

@item array_element(+@var{Name}, +@var{Index}, ?@var{Element})
@findex array_element/3
@snindex array_element/3
@cnindex array_element/3
Unify @var{Element} with @var{Name}[@var{Index}]. It works for both
static and dynamic arrays, but it is read-only for static arrays, while
it can be used to unify with an element of a dynamic array.

@item update_array(+@var{Name}, +@var{Index}, ?@var{Value}) 
@findex update_array/3
@snindex update_array/3
@cnindex update_array/3
Attribute value @var{Value} to @var{Name}[@var{Index}]. Type
restrictions must be respected for static arrays. This operation is
available for dynamic arrays if @code{MULTI_ASSIGNMENT_VARIABLES} is
enabled (true by default). Backtracking undoes @var{update_array/3} for
dynamic arrays, but not for static arrays.

Note that @code{update_array/3} actually uses @code{setarg/3} to update
elements of dynamic arrays, and @code{setarg/3} spends an extra cell for
every update. For intensive operations we suggest it may be less
expensive to unify each element of the array with a mutable terms and
to use the operations on mutable terms.

@item add_to_array_element(+@var{Name}, +@var{Index}, , +@var{Number}, ?@var{NewValue}) 
@findex add_to_array_element/4
@snindex add_to_array_element/4
@cnindex add_to_array_element/4
Add @var{Number} @var{Name}[@var{Index}] and unify @var{NewValue} with
the incremented value. Observe that @var{Name}[@var{Index}] must be an
number. If @var{Name} is a static array the type of the array must be
@code{int} or @code{float}. If the type of the array is @code{int} you
only may add integers, if it is @code{float} you may add integers or
floats. If @var{Name} corresponds to a dynamic array the array element
must have been previously bound to a number and @code{Number} can be
any kind of number.

The @code{add_to_array_element/3} built-in actually uses
@code{setarg/3} to update elements of dynamic arrays. For intensive
operations we suggest it may be less expensive to unify each element
of the array with a mutable terms and to use the operations on mutable
terms.

@end table

@node Preds, Misc, Arrays, Top
@section Predicate Information

Built-ins that return information on the current predicates and modules:

@table @code
@c ......... begin of 'module' documentation .........
@item current_module(@var{M})
@findex current_module/1
@syindex current_module/1
@cnindex current_module/1
Succeeds if @var{M} are defined modules. A module is defined as soon as some
predicate defined in the module is loaded, as soon as a goal in the
module is called, or as soon as it becomes the current type-in module.

@item current_module(@var{M},@var{F})
@findex current_module/2
@syindex current_module/2
@cnindex current_module/2
Succeeds if @var{M} are current modules associated to the file @var{F}.

@c .......... end of 'module' documentation ..........
@end table

@node Misc, , Preds, Top
@section Miscellaneous

@table @code

@item statistics/0
@findex statistics/0
@saindex statistics/0
@cyindex statistics/0
Send to the current user error stream general information on space used and time
spent by the system.
@example
?- statistics.
memory (total)        4784124 bytes
   program space      3055616 bytes:    1392224 in use,      1663392 free
                                                             2228132  max
   stack space        1531904 bytes:        464 in use,      1531440 free
     global stack:                           96 in use,       616684  max
      local stack:                          368 in use,       546208  max
   trail stack         196604 bytes:          8 in use,       196596 free

       0.010 sec. for 5 code, 2 stack, and 1 trail space overflows
       0.130 sec. for 3 garbage collections which collected 421000 bytes
       0.000 sec. for 0 atom garbage collections which collected 0 bytes
       0.880 sec. runtime
       1.020 sec. cputime
      25.055 sec. elapsed time

@end example
The example shows how much memory the system spends. Memory is divided
into Program Space, Stack Space and Trail. In the example we have 3MB
allocated for program spaces, with less than half being actually
used. YAP also shows the maximum amount of heap space having been used
which was over 2MB.

The stack space is divided into two stacks which grow against each
other. We are in the top level so very little stack is being used. On
the other hand, the system did use a lot of global and local stack
during the previous execution (we refer the reader to a WAM tutorial in
order to understand what are the global and local stacks).

YAP also shows information on how many memory overflows and garbage
collections the system executed, and statistics on total execution
time. Cputime includes all running time, runtime excludes garbage
collection and stack overflow time.

@item statistics(?@var{Param},-@var{Info})
@findex statistics/2
@saindex statistics/2
@cnindex statistics/2
Gives statistical information on the system parameter given by first
argument:

@table @code

@item atoms
@findex atoms (statistics/2 option)
@code{[@var{NumberOfAtoms},@var{SpaceUsedBy Atoms}]}
@* 
This gives the total number of atoms @code{NumberOfAtoms} and how much
space they require in bytes, @var{SpaceUsedBy Atoms}.

@item cputime
@findex cputime (statistics/2 option)
@code{[@var{Time since Boot},@var{Time From Last Call to Cputime}]}
@* 
This gives the total cputime in milliseconds spent executing Prolog code,
garbage collection and stack shifts time included.

@item dynamic_code
@findex dynamic_code (statistics/2 option)
@code{[@var{Clause Size},@var{Index Size},@var{Tree Index
Size},@var{Choice Point Instructions
Size},@var{Expansion Nodes Size},@var{Index Switch Size}]}
@*
Size of static code in YAP in bytes: @var{Clause Size}, the number of
bytes allocated for clauses, plus
@var{Index Size}, the number of bytes spent in the indexing code. The
indexing code is divided into main tree, @var{Tree Index
Size}, tables that implement choice-point manipulation, @var{Choice Point Instructions
Size}, tables that cache clauses for future expansion of the index
tree, @var{Expansion Nodes Size}, and 
tables such as hash tables that select according to value,  @var{Index Switch Size}.

@item garbage_collection
@findex garbage_collection (statistics/2 option)
@code{[@var{Number of GCs},@var{Total Global Recovered},@var{Total Time
Spent}]}
@*
Number of garbage collections, amount of space recovered in kbytes, and
total time spent doing garbage collection in milliseconds. More detailed
information is available using @code{yap_flag(gc_trace,verbose)}.

@item global_stack
@findex global_stack (statistics/2 option)
@code{[@var{Global Stack Used},@var{Execution Stack Free}]}
@*
Space in kbytes currently used in the global stack, and space available for
expansion by the local and global stacks.

@item local_stack
@findex local_stack (statistics/2 option)
@code{[@var{Local Stack Used},@var{Execution Stack Free}]}
@*
Space in kbytes currently used in the local stack, and space available for
expansion by the local and global stacks.

@item heap
@findex heap (statistics/2 option)
@code{[@var{Heap Used},@var{Heap Free}]}
@*
Total space in kbytes not recoverable
in backtracking. It includes the program code, internal data base, and,
atom symbol table.

@item program
@findex program (statistics/2 option)
@code{[@var{Program Space Used},@var{Program Space Free}]}
@*
Equivalent to @code{heap}.

@item runtime
@findex runtime (statistics/2 option)
@code{[@var{Time since Boot},@var{Time From Last Call to Runtime}]}
@* 
This gives the total cputime in milliseconds spent executing Prolog
code, not including garbage collections and stack shifts. Note that
until YAP4.1.2 the @code{runtime} statistics would return time spent on
garbage collection and stack shifting.

@item stack_shifts
@findex stack_shifts (statistics/2 option)
@code{[@var{Number of Heap Shifts},@var{Number of Stack
Shifts},@var{Number of Trail Shifts}]}
@*
Number of times YAP had to
expand the heap, the stacks, or the trail. More detailed information is
available using @code{yap_flag(gc_trace,verbose)}.

@item static_code
@findex static_code (statistics/2 option)
@code{[@var{Clause Size},@var{Index Size},@var{Tree Index
Size},@var{Expansion Nodes Size},@var{Index Switch Size}]}
@*
Size of static code in YAP in bytes: @var{Clause Size}, the number of
bytes allocated for clauses, plus
@var{Index Size}, the number of bytes spent in the indexing code. The
indexing code is divided into a main tree, @var{Tree Index
Size}, table that cache clauses for future expansion of the index
tree, @var{Expansion Nodes Size}, and and 
tables such as hash tables that select according to value,  @var{Index Switch Size}.

@item trail
@findex trail (statistics/2 option)
@code{[@var{Trail Used},@var{Trail Free}]}
@*
Space in kbytes currently being used and still available for the trail.

@item walltime
@findex walltime (statistics/2 option)
@code{[@var{Time since Boot},@var{Time From Last Call to Walltime}]}
@* 
This gives the clock time in milliseconds since starting Prolog.

@end table

@item time(:@var{Goal})
@findex time/1
@snindex time/1
@cnindex time/1
Prints the CPU time and the wall time for the execution of @var{Goal}.
Possible choice-points of @var{Goal} are removed. Based on the SWI-Prolog 
definition (minus reporting the number of inferences, which YAP currently
does not support).

@item yap_flag(?@var{Param},?@var{Value})
@findex yap_flag/2
@snindex yap_flag/2
@cnindex yap_flag/2
Set or read system properties for @var{Param}:

@table @code

@item argv
@findex argv (yap_flag/2 option)
@* Read-only flag. It unifies with a list of atoms that gives the
arguments to YAP after @code{--}.

@item agc_margin
@findex agc_margin (yap_flag/2 option)
An integer: if this amount of atoms has been created since the last
atom-garbage collection, perform atom garbage collection at the first
opportunity. Initial value is 10,000. May be changed. A value of 0
(zero) disables atom garbage collection.

@item bounded [ISO]
@findex bounded (yap_flag/2 option)
@*
Read-only flag telling whether integers are bounded. The value depends
on whether YAP uses the GMP library or not.

@item  profiling
@findex call_counting (yap_flag/2 option)
@*
If @code{off} (default) do not compile call counting information for
procedures. If @code{on} compile predicates so that they calls and
retries to the predicate may be counted. Profiling data can be read through the
@code{call_count_data/3} built-in.

@item char_conversion [ISO]
@findex  char_conversion (yap_flag/2 option)
@*
Writable flag telling whether a character conversion table is used when
reading terms. The default value for this flag is @code{off} except in
@code{sicstus} and @code{iso} language modes, where it is @code{on}.

@item character_escapes [ISO]
@findex  character_escapes (yap_flag/2 option)
@* Writable flag telling whether a character escapes are enables,
@code{on}, or disabled, @code{off}. The default value for this flag is
@code{on}.

@c You can also use @code{cprolog} mode, which corresponds to @code{off},
@c @code{iso} mode, which corresponds to @code{on}, and @code{sicstus}
@c mode, which corresponds to the mode traditionally used in SICStus
@c Prolog. In this mode back-quoted escape sequences should not close with
@c a backquote and unrecognized escape codes do not result in error.

@item debug [ISO]
@findex debug (yap_flag/2 option)
@*
If @var{Value} is unbound, tell whether debugging is @code{on} or
@code{off}. If @var{Value} is bound to @code{on} enable debugging, and if
it is bound to @code{off} disable debugging.

+@item debugger_print_options
@findex debugger_print_options (yap_flag/2 option)
@* 
If bound, set the argument to the @code{write_term/3} options the
debugger uses to write terms. If unbound, show the current options.

@item dialect
@findex dialect (yap_flag/2 option)
@*
Read-only flag that always returns @code{yap}.

@item discontiguous_warnings
@findex discontiguous_warnings (yap_flag/2 option)
@*
If @var{Value} is unbound, tell whether warnings for discontiguous
predicates are @code{on} or
@code{off}. If @var{Value} is bound to @code{on} enable these warnings,
and if it is bound to @code{off} disable them. The default for YAP is
@code{off}, unless we are in @code{sicstus} or @code{iso} mode.

@item  dollar_as_lower_case
@findex dollar_as_lower_case (yap_flag/2 option)
@*
If @code{off} (default)  consider the character '$' a control character, if
@code{on} consider '$' a lower case character.

@item double_quotes [ISO]
@findex double_quotes (yap_flag/2 option)
@*
If @var{Value} is unbound, tell whether a double quoted list of characters
token is converted to a list of atoms, @code{chars}, to a list of integers,
@code{codes}, or to a single atom, @code{atom}. If @var{Value} is bound, set to
the corresponding behavior. The default value is @code{codes}.

@item executable
@findex executable(yap_flag/2 option)
@* Read-only flag. It unifies with an atom that gives the
original program path.

@item  fast
@findex fast (yap_flag/2 option)
@*
If @code{on} allow fast machine code, if @code{off} (default) disable it. Only
available in experimental implementations.

@item  fileerrors
@findex fileerrors (yap_flag/2 option)
@*
If @code{on} @code{fileerrors} is @code{on}, if @code{off} (default)
@code{fileerrors} is disabled.

@item float_format
@findex float_format (yap_flag/2 option)
@* C-library @code{printf()} format specification used by @code{write/1} and
friends to determine how floating point numbers are printed. The
default is @code{%.15g}. The specified value is passed to @code{printf()}
without further checking. For example, if you want less digits
printed, @code{%g} will print all floats using 6 digits instead of the
default 15.

@item  gc
@findex gc (yap_flag/2 option)
@*
If @code{on} allow garbage collection (default), if @code{off} disable it.

@item  gc_margin
@findex gc_margin (yap_flag/2 option)
@*
Set or show the minimum free stack before starting garbage
collection. The default depends on total stack size. 

@item  gc_trace
@findex gc_trace (yap_flag/2 option)
@* If @code{off} (default) do not show information on garbage collection
and stack shifts, if @code{on} inform when a garbage collection or stack
shift happened, if @code{verbose} give detailed information on garbage
collection and stack shifts. Last, if @code{very_verbose} give detailed
information on data-structures found during the garbage collection
process, namely, on choice-points.

@item  generate_debugging_info
@findex generate_debugging_info (yap_flag/2 option)
@* If @code{true} (default) generate debugging information for
procedures, including source mode. If @code{false} predicates no
information is generated, although debugging is still possible, and
source mode is disabled.

@item  host_type
@findex host_type (yap_flag/2 option)
@* Return @code{configure} system information, including the machine-id
for which YAP was compiled and Operating System information. 

@item  index
@findex index (yap_flag/2 option)
@*
If @code{on} allow indexing (default), if @code{off} disable it.

@item  informational_messages
@findex informational_messages (yap_flag/2 option)
@*
If @code{on} allow printing of informational messages, such as the ones
that are printed when consulting. If @code{off} disable printing
these messages. It is @code{on} by default except if YAP is booted with
the @code{-L} flag.

@item integer_rounding_function [ISO]
@findex integer_rounding_function (yap_flag/2 option)
@*
Read-only flag telling the rounding function used for integers. Takes the value
@code{down} for the current version of YAP.

@item language
@findex language (yap_flag/2 option)
@* 
Choose whether YAP is closer to C-Prolog, @code{cprolog}, iso-prolog,
@code{iso} or SICStus Prolog, @code{sicstus}. The current default is
@code{cprolog}. This flag affects update semantics, leashing mode,
style checking, handling calls to undefined procedures, how directives
are interpreted, when to use dynamic, character escapes, and how files
are consulted.

@item max_arity [ISO]
@findex max_arity (yap_flag/2 option)
@*
Read-only flag telling the maximum arity of a functor. Takes the value
@code{unbounded} for the current version of YAP.

@item max_integer [ISO]
@findex max_integer (yap_flag/2 option)
@*
Read-only flag telling the maximum integer in the
implementation. Depends on machine and Operating System
architecture, and on whether YAP uses the @code{GMP} multi-precision
library. If @code{bounded} is false, requests for @code{max_integer}
will fail.

@item max_tagged_integer 
@findex max_tagged_integer (yap_flag/2 option)
@*
Read-only flag telling the maximum integer we can store as a single
word. Depends on machine and Operating System
architecture. It can be used to find the word size of the current machine.

@item min_integer [ISO]
@findex min_integer (yap_flag/2 option)
@* Read-only flag telling the minimum integer in the
implementation. Depends on machine and Operating System architecture,
and on whether YAP uses the @code{GMP} multi-precision library. If
@code{bounded} is false, requests for @code{min_integer} will fail.

@item min_tagged_integer 
@findex max_tagged_integer (yap_flag/2 option)
@*
Read-only flag telling the minimum integer we can store as a single
word. Depends on machine and Operating System
architecture.

@item n_of_integer_keys_in_bb
@findex n_of_integer_keys_in_bb (yap_flag/2 option)
@*
Read or set the size of the hash table that is used for looking up the
blackboard when the key is an integer.

@item n_of_integer_keys_in_db
@findex n_of_integer_keys_in_db (yap_flag/2 option)
@*
Read or set the size of the hash table that is used for looking up the
internal data-base when the key is an integer.

@item open_expands_filename
@findex open_expands_filename (yap_flag/2 option)
@*
If @code{true} the @code{open/3} builtin performs filename-expansion
before opening a file (SICStus Prolog like). If @code{false} it does not
(SWI-Prolog like).

@item open_shared_object
@findex open_shared_object (yap_flag/2 option)
@*
If true, @code{open_shared_object/2} and friends are implemented,
providing access to shared libraries (@code{.so} files) or to dynamic link
libraries (@code{.DLL} files).

@item  profiling
@findex profiling (yap_flag/2 option)
@*
If @code{off} (default) do not compile profiling information for
procedures. If @code{on} compile predicates so that they will output
profiling information. Profiling data can be read through the
@code{profile_data/3} built-in.

@item  prompt_alternatives_on(atom, changeable)
@findex  prompt_alternatives_on  (yap_flag/2 option)
SWI-Compatible opttion, determines prompting for alternatives in the Prolog toplevel. Default is @t{groundness}, YAP prompts for alternatives if and only if the query contains variables. The alternative, default in SWI-Prolog is @t{determinism} which implies the system prompts for alternatives if the goal succeeded while leaving choicepoints.


@item redefine_warnings
@findex discontiguous_warnings (yap_flag/2 option)
@*
If @var{Value} is unbound, tell whether warnings for procedures defined
in several different files are @code{on} or
@code{off}. If @var{Value} is bound to @code{on} enable these warnings,
and if it is bound to @code{off} disable them. The default for YAP is
@code{off}, unless we are in @code{sicstus} or @code{iso} mode.

@item shared_object_search_path
@findex shared_object_search_path (yap_flag/2 option)
Name of the environment variable used by the system to search for shared
objects.

@item single_var_warnings
@findex single_var_warnings (yap_flag/2 option)
@*
If @var{Value} is unbound, tell whether warnings for singleton variables
are @code{on} or @code{off}. If @var{Value} is bound to @code{on} enable
these warnings, and if it is bound to @code{off} disable them. The
default for YAP is @code{off}, unless we are in @code{sicstus} or
@code{iso} mode.

@item strict_iso
@findex strict_iso (yap_flag/2 option)
@*
 If @var{Value} is unbound, tell whether strict ISO compatibility mode
is @code{on} or @code{off}. If @var{Value} is bound to @code{on} set
language mode to @code{iso} and enable strict mode. If @var{Value} is
bound to @code{off} disable strict mode, and keep the current language
mode. The default for YAP is @code{off}.

Under strict ISO Prolog mode all calls to non-ISO built-ins generate an
error. Compilation of clauses that would call non-ISO built-ins will
also generate errors. Pre-processing for grammar rules is also
disabled. Module expansion is still performed.

Arguably, ISO Prolog does not provide all the functionality required
from a modern Prolog system. Moreover, because most Prolog
implementations do not fully implement the standard and because the
standard itself gives the implementor latitude in a few important
questions, such as the unification algorithm and maximum size for
numbers there is no guarantee that programs compliant with this mode
will work the same way in every Prolog and in every platform. We thus
believe this mode is mostly useful when investigating how a program
depends on a Prolog's platform specific features.

@item  stack_dump_on_error
@findex stack_dump_on_error (yap_flag/2 option)
@*
If @code{on} show a stack dump when YAP finds an error. The default is
@code{off}.

@item  syntax_errors
@findex syntax_errors (yap_flag/2 option)
@*
Control action to be taken after syntax errors while executing @code{read/1},
@code{read/2}, or @code{read_term/3}:
@table @code

@item dec10
@*
Report the syntax error and retry reading the term.
 
@item fail
@*
Report the syntax error and fail (default).

@item error
@*
Report the syntax error and generate an error.

@item quiet
@*
Just fail
@end table

@item system_options
@findex system_options (yap_flag/2 option)
@* This read only flag tells which options were used to compile
YAP. Currently it informs whether the system supports @code{big_numbers},
@code{coroutining}, @code{depth_limit}, @code{low_level_tracer},
@code{or-parallelism}, @code{rational_trees}, @code{readline}, @code{tabling},
@code{threads}, or the @code{wam_profiler}.

@item tabling_mode
@* Sets or reads the tabling mode for all tabled predicates. Please
@pxref{Tabling} for the list of options.

@item to_chars_mode
@findex to_chars_modes (yap_flag/2 option)
@* Define whether YAP should follow @code{quintus}-like
semantics for the @code{atom_chars/1} or @code{number_chars/1} built-in,
or whether it should follow the ISO standard (@code{iso} option).

+@item toplevel_hook
@findex toplevel_hook (yap_flag/2 option)
@*
+If bound, set the argument to a goal to be executed before entering the
top-level. If unbound show the current goal or @code{true} if none is
presented. Only the first solution is considered and the goal is not
backtracked into.

+@item toplevel_print_options
@findex toplevel_print_options (yap_flag/2 option)
@*
+If bound, set the argument to the @code{write_term/3} options used to write
terms from the top-level. If unbound, show the current options.

@item typein_module
@findex typein_module (yap_flag/2 option)
@*
If bound, set the current working or type-in module to the argument,
which must be an atom. If unbound, unify the argument with the current
working module.

@item  unix
@findex unix (yap_flag/2 option)
@* Read-only Boolean flag that unifies with @code{true} if YAP is
running on an Unix system.  Defined if the C-compiler used to compile
this version of YAP either defines @code{__unix__} or @code{unix}.

@item unknown [ISO]
@findex unknown (yap_flag/2 option)
@*
Corresponds to calling the @code{unknown/2} built-in.

@item update_semantics
@findex update_semantics (yap_flag/2 option)
@*
Define whether YAP should follow @code{immediate} update
semantics, as in C-Prolog (default), @code{logical} update semantics,
as in Quintus Prolog, SICStus Prolog, or in the ISO standard. There is
also an intermediate mode, @code{logical_assert}, where dynamic
procedures follow logical semantics but the internal data base still
follows immediate semantics.

@item user_error
@findex user_error (yap_flag/2 option)
@*
If the second argument is bound to a stream, set @code{user_error} to
this stream. If the second argument is unbound, unify the argument with
the current @code{user_error} stream.

By default, the @code{user_error} stream is set to a stream
corresponding to the Unix @code{stderr} stream.

The next example shows how to use this flag:
@example
   ?- open( '/dev/null', append, Error,
           [alias(mauri_tripa)] ).

Error = '$stream'(3) ? ;

no
   ?- set_prolog_flag(user_error, mauri_tripa).

close(mauri_tripa).

yes
   ?- 
@end example
We execute three commands. First, we open a stream in write mode and
give it an alias, in this case @code{mauri_tripa}. Next, we set
@code{user_error} to the stream via the alias. Note that after we did so
prompts from the system were redirected to the stream
@code{mauri_tripa}. Last, we close the stream. At this point, YAP
automatically redirects the @code{user_error} alias to the original
@code{stderr}.

@item user_flags
@findex user_flags  (yap_flag/2 option)
@*
Define the behaviour of @code{set_prolog_flag/2} if the flag is not known. Values are @code{silent}, @code{warning} and @code{error}. The first two create the flag on-the-fly, with @code{warning} printing a message. The value @code{error} is consistent with ISO: it raises an existence error and does not create the flag. See also @code{create_prolog_flag/3}. The default is@code{error}, and developers are encouraged to use @code{create_prolog_flag/3} to create flags for their library.

@item user_input
@findex user_input (yap_flag/2 option)
@*
If the second argument is bound to a stream, set @code{user_input} to
this stream. If the second argument is unbound, unify the argument with
the current @code{user_input} stream.

By default, the @code{user_input} stream is set to a stream
corresponding to the Unix @code{stdin} stream.

@item user_output
@findex user_output (yap_flag/2 option)
@*
If the second argument is bound to a stream, set @code{user_output} to
this stream. If the second argument is unbound, unify the argument with
the current @code{user_output} stream.

By default, the @code{user_output} stream is set to a stream
corresponding to the Unix @code{stdout} stream.

@item  verbose
@findex verbose (yap_flag/2 option)
@*
If @code{normal} allow printing of informational and banner messages,
such as the ones that are printed when consulting. If @code{silent}
disable printing these messages. It is @code{normal} by default except if
YAP is booted with the @code{-q} or @code{-L} flag.

@item  verbose_load
@findex verbose_load (yap_flag/2 option)
@* If @code{true} allow printing of informational messages when
consulting files. If @code{false} disable printing these messages. It
is @code{normal} by default except if YAP is booted with the @code{-L}
flag.

@item version
@findex version (yap_flag/2 option)
@* Read-only flag that returns an atom with the current version of
YAP.

@item version_data
@findex version_data (yap_flag/2 option)
@* Read-only flag that reads a term of the form
@code{yap}(@var{Major},@var{Minor},@var{Patch},@var{Undefined}), where
@var{Major} is the major version, @var{Minor} is the minor version,
and @var{Patch} is the patch number.

@item  windows
@findex windoes (yap_flag/2 option)
@* 
Read-only boolean flag that unifies with tr @code{true} if YAP is
running on an Windows machine.

@item write_strings
@findex  write_strings (yap_flag/2 option)
@* Writable flag telling whether the system should write lists of
integers that are writable character codes using the list notation. It
is @code{on} if enables or @code{off} if disabled. The default value for
this flag is @code{off}.

@item max_workers
@findex  max_workers (yap_flag/2 option)
@* Read-only flag telling the maximum number of parallel processes.

@item max_threads
@findex  max_threads (yap_flag/2 option)
@* Read-only flag telling the maximum number of Prolog threads that can 
be created.

@end table

@item current_prolog_flag(?@var{Flag},-@var{Value}) [ISO]
@findex current_prolog_flag/2
@snindex current_prolog_flag/2
@cnindex current_prolog_flag/2

Obtain the value for a YAP Prolog flag. Equivalent to calling
@code{yap_flag/2} with the second argument unbound, and unifying the
returned second argument with @var{Value}.

@item prolog_flag(?@var{Flag},-@var{OldValue},+@var{NewValue})
@findex prolog_flag/3
@syindex prolog_flag/3
@cnindex prolog_flag/3

Obtain the value for a YAP Prolog flag and then set it to a new
value. Equivalent to first calling @code{current_prolog_flag/2} with the
second argument @var{OldValue} unbound and then calling
@code{set_prolog_flag/2} with the third argument @var{NewValue}.

@item set_prolog_flag(+@var{Flag},+@var{Value}) [ISO]
@findex set_prolog_flag/2
@snindex set_prolog_flag/2
@cnindex set_prolog_flag/2

Set the value for YAP Prolog flag @code{Flag}. Equivalent to
calling @code{yap_flag/2} with both arguments bound.


@item create_prolog_flag(+@var{Flag},+@var{Value},+@var{Options})
@findex create_prolog_flag/2
@snindex create_prolog_flag/2
@cnindex create_prolog_flag/2

Create a new YAP Prolog flag. @var{Options} include @code{type(+Type)} and @code{access(+Access)} with @var{Access}
one of @code{read_only} or @code{read_write} and @var{Type} one of @code{boolean}, @code{integer}, @code{float}, @code{atom}
and @code{term} (that is, no type).

@item op(+@var{P},+@var{T},+@var{A}) [ISO]
@findex op/3
@syindex op/3
@cyindex op/3
Defines the operator @var{A} or the list of operators @var{A} with type
@var{T} (which must be one of @code{xfx}, @code{xfy},@code{yfx},
@code{xf}, @code{yf}, @code{fx} or @code{fy}) and precedence @var{P}
(see appendix iv for a list of predefined operators).

Note that if there is a preexisting operator with the same name and
type, this operator will be discarded. Also, @code{','} may not be defined
as an operator, and it is not allowed to have the same for an infix and
a postfix operator.

@item current_op(@var{P},@var{T},@var{F}) [ISO]
@findex current_op/3
@syindex current_op/3
@cnindex current_op/3
Defines the relation: @var{P} is a currently defined  operator of type
@var{T} and precedence @var{P}.

@item prompt(-@var{A},+@var{B})
@findex prompt/2
@syindex prompt/2
@cyindex prompt/2
Changes YAP input prompt from @var{A} to @var{B}.

@item initialization
@findex initialization/0
@syindex initialization/0
@cnindex initialization/0
Execute the goals defined by initialization/1. Only the first answer is
considered.

@item prolog_initialization(@var{G})
@findex prolog_initialization/1
@saindex prolog_initialization/1
@cnindex prolog_initialization/1
Add a goal to be executed on system initialization. This is compatible
with SICStus Prolog's @code{initialization/1}.

@item version
@findex version/0
@saindex version/0
@cnindex version/0
Write YAP's boot message. 

@item version(-@var{Message})
@findex version/1
@syindex version/1
@cnindex version/1
Add a message to be written when yap boots or after aborting. It is not
possible to remove messages.

@item prolog_load_context(?@var{Key}, ?@var{Value})
@findex prolog_load_context/2
@syindex prolog_load_context/2
@cnindex prolog_load_context/2
Obtain information on what is going on in the compilation process. The
following keys are available:

@table @code

@item directory
@findex directory (prolog_load_context/2 option)
@* 
Full name for the directory where YAP is currently consulting the
file.

@item file
@findex file (prolog_load_context/2 option)
@*
Full name for the file currently being consulted. Notice that included
filed are ignored.

@item module
@findex module (prolog_load_context/2 option)
@*
Current source module.

@item source
@findex file (prolog_load_context/2 option)
@*
Full name for the file currently being read in, which may be consulted,
reconsulted, or included.

@item stream
@findex file (prolog_load_context/2 option)
@*
Stream currently being read in.

@item term_position
@findex file (prolog_load_context/2 option)
@*
Stream position at the stream currently being read in.
@end table

@item source_location(?@var{FileName}, ?@var{Line})
@findex source_location/2
@syindex source_location/2
@cnindex source_location/2
SWI-compatible predicate. If the last term has been read from a physical file (i.e., not from the file user or a string), unify File with an absolute path to the file and Line with the line-number in the file. Please use @code{prolog_load_context/2}.

@item source_file(?@var{File})
@findex source_file/1
@syindex source_file/1
@cnindex source_file/1
SWI-compatible predicate. True if @var{File} is a loaded Prolog source file.

@item source_file(?@var{ModuleAndPred},?@var{File})
@findex source_file/2
@syindex source_file/2
@cnindex source_file/2
SWI-compatible predicate. True if the predicate specified by @var{ModuleAndPred} was loaded from file @var{File}, where @var{File} is an absolute path name (see @code{absolute_file_name/2}).



@end table

@node Library, SWI-Prolog, Built-ins, Top

@chapter Library Predicates

Library files reside in the library_directory path (set by the
@code{LIBDIR} variable in the Makefile for YAP). Currently,
most files in the library are from the Edinburgh Prolog library. 

@menu
 
Library, Extensions, Built-ins, Top
* Aggregate :: SWI and SICStus compatible aggregate library
* Apply:: SWI-Compatible Apply library.
* Association Lists:: Binary Tree Implementation of Association Lists.
* AVL Trees:: Predicates to add and lookup balanced binary  trees.
* Cleanup:: Call With registered Cleanup Calls
* DGraphs:: Directed Graphs Implemented With Red-Black Trees
* Heaps:: Labelled binary tree where the key of each node is less
    than or equal to the keys of its children.
* LAM:: LAM MPI
* Lambda:: Ulrich Neumerkel's Lambda Library
* Lists:: List Manipulation
* LineUtilities:: Line Manipulation Utilities
* MapList:: SWI-Compatible Apply library.
* matrix:: Matrix Objects
* MATLAB:: Matlab Interface
* Non-Backtrackable Data Structures:: Queues, Heaps, and Beams.
* Ordered Sets:: Ordered Set Manipulation
* Pseudo Random:: Pseudo Random Numbers
* Queues:: Queue Manipulation
* Random:: Random Numbers
* Read Utilities:: SWI inspired utilities for fast stream scanning.
* Red-Black Trees:: Predicates to add, lookup and delete in red-black binary  trees.
* RegExp:: Regular Expression Manipulation
* shlib:: SWI Prolog shlib library
* Splay Trees:: Splay Trees
* String I/O:: Writing To and Reading From Strings
* System:: System Utilities
* Terms:: Utilities on Terms
* Timeout:: Call With Timeout
* Trees:: Updatable Binary Trees
* Tries:: Trie Data Structure
* UGraphs:: Unweighted Graphs
* UnDGraphs:: Undirected Graphs Using DGraphs


@end menu

 
@node Aggregate, Apply, , Library
@section Aggregate
@cindex aggregate
This is the SWI-Prolog library based on  the Quintus and SICStus 4
library. Notice that  @code{forall/2}
is a SWI-Prolog built-in and @code{term_variables/3} is a SWI-Prolog with a
different definition.  @c To be done - Analysing the aggregation template
@c and compiling a predicate for the list aggregation can be done at
@c compile time.  - aggregate_all/3 can be rewritten to run in constant
@c space using non-backtrackable assignment on a term.

This library provides aggregating operators over the solutions of a
predicate. The operations are a generalisation of the @code{bagof/3},
@code{setof/3} and @code{findall/3} built-in predicates. The defined
aggregation operations are counting, computing the sum, minimum,
maximum, a bag of solutions and a set of solutions. We first give a
simple example, computing the country with the smallest area:

@example
smallest_country(Name, Area) :-
        aggregate(min(A, N), country(N, A), min(Area, Name)).
@end example

There are four aggregation predicates, distinguished on two properties.

@table @code

@item aggregate vs. aggregate_all
    The aggregate predicates use setof/3 (aggregate/4) or bagof/3
    (aggregate/3), dealing with existential qualified variables
    (@var{Var}/\@var{Goal}) and providing multiple solutions for the
    remaining free variables in @var{Goal}. The aggregate_all/3
    predicate uses findall/3, implicitly qualifying all free variables
    and providing exactly one solution, while aggregate_all/4 uses
    sort/2 over solutions and Distinguish (see below) generated using
    findall/3. 
@item The @var{Distinguish} argument
    The versions with 4 arguments provide a @var{Distinguish} argument
    that allow for keeping duplicate bindings of a variable in the
    result. For example, if we wish to compute the total population of
    all countries we do not want to lose results because two countries
    have the same population. Therefore we use:

@example
        aggregate(sum(P), Name, country(Name, P), Total)
@end example

@end table

All aggregation predicates support the following operator below in
@var{Template}. In addition, they allow for an arbitrary named compound
term where each of the arguments is a term from the list below. I.e. the
term @code{r(min(X), max(X))} computes both the minimum and maximum
binding for @var{X}.

@table @code

@item count
    Count number of solutions. Same as @code{sum(1)}. 
@item sum(@var{Expr})
    Sum of @var{Expr} for all solutions. 
@item min(@var{Expr})
    Minimum of @var{Expr} for all solutions. 
@item min(@var{Expr}, @var{Witness})
    A term min(@var{Min}, @var{Witness}), where @var{Min} is the minimal version of @var{Expr}
    over all Solution and @var{Witness} is any other template applied to
    Solution that produced @var{Min}. If multiple solutions provide the same
    minimum, @var{Witness} corresponds to the first solution. 
@item max(@var{Expr})
    Maximum of @var{Expr} for all solutions. 
@item max(@var{Expr}, @var{Witness})
    As min(@var{Expr}, @var{Witness}), but producing the maximum result. 
@item set(@var{X})
    An ordered set with all solutions for @var{X}. 
@item bag(@var{X})
    A list of all solutions for @var{X}. 
@end table

The predicates are:
@table @code

@item [nondet]aggregate(+@var{Template}, :@var{Goal}, -@var{Result})
@findex aggregate/3
@syindex aggregate/3
@cnindex aggregate/3
    Aggregate bindings in @var{Goal} according to @var{Template}. The
    aggregate/3 version performs bagof/3 on @var{Goal}.
@item [nondet]aggregate(+@var{Template}, +@var{Discriminator}, :@var{Goal}, -@var{Result})
@findex aggregate/4
@syindex aggregate/4
@cnindex aggregate/4
    Aggregate bindings in @var{Goal} according to @var{Template}. The
    aggregate/3 version performs setof/3 on @var{Goal}.
@item [semidet]aggregate_all(+@var{Template}, :@var{Goal}, -@var{Result})
@findex aggregate_all/3
@syindex aggregate_all/3
@cnindex aggregate_all/3
    Aggregate bindings in @var{Goal} according to @var{Template}. The
    aggregate_all/3 version performs findall/3 on @var{Goal}.
@item [semidet]aggregate_all(+@var{Template}, +@var{Discriminator}, :@var{Goal}, -@var{Result})
@findex aggregate_all/4
@syindex aggregate_all/4
@cnindex aggregate_all/4
    Aggregate bindings in @var{Goal} according to @var{Template}. The
    aggregate_all/3 version performs findall/3 followed by sort/2 on
    @var{Goal}.
@item foreach(:Generator, :@var{Goal})
@findex foreach/2
@syindex foreach/2
@cnindex foreach/2
    True if the conjunction of instances of @var{Goal} using the
    bindings from Generator is true. Unlike forall/2, which runs a
    failure-driven loop that proves @var{Goal} for each solution of
    Generator, foreach creates a conjunction. Each member of the
    conjunction is a copy of @var{Goal}, where the variables it shares
    with Generator are filled with the values from the corresponding
    solution.

    The implementation executes forall/2 if @var{Goal} does not contain
    any variables that are not shared with Generator.

    Here is an example:
@example
    ?- foreach(between(1,4,X), dif(X,Y)), Y = 5.
    Y = 5
    ?- foreach(between(1,4,X), dif(X,Y)), Y = 3.
    No
@end example

    Notice that @var{Goal} is copied repeatetly, which may cause
    problems if attributed variables are involved.

@item [det]free_variables(:Generator, +@var{Template}, +VarList0, -VarList)
@findex free_variables/4
@syindex free_variables/4
@cnindex free_variables/4
    In order to handle variables properly, we have to find all the universally quantified variables in the Generator. All variables as yet unbound are universally quantified, unless

@enumerate
@item they occur in the template
@item they are bound by X/\P, setof, or bagof
@end enumerate

    @code{free_variables(Generator, Template, OldList, NewList)} finds this set, using OldList as an accumulator.
@end table

The original author of this code was Richard O'Keefe. Jan Wielemaker
    made some SWI-Prolog enhancements, sponsored by SecuritEase,
    http://www.securitease.com. The code is public domain (from DEC10 library).
    @c To be done
    @c     - Distinguish between control-structures and data terms.
    @c     - Exploit our built-in term_variables/2 at some places? 



@node Apply, Association Lists, Aggregate, Library
@section Apply Macros
@cindex apply

This library provides a SWI-compatible set of utilities for applying a
predicate to all elements of a list. The library just forwards
definitions from the @code{maplist} library.



@node Association Lists, AVL Trees, Apply, Library
@section Association Lists
@cindex association list

The following association list manipulation predicates are available
once included with the @code{use_module(library(assoc))} command. The
original library used Richard O'Keefe's implementation, on top of
unbalanced binary trees. The current code utilises code from the
red-black trees library and emulates the SICStus Prolog interface.

@table @code
@item assoc_to_list(+@var{Assoc},?@var{List})
@findex assoc_to_list/2
@syindex assoc_to_list/2
@cnindex assoc_to_list/2
Given an association list @var{Assoc} unify @var{List} with a list of
the form @var{Key-Val}, where the elements @var{Key} are in ascending
order.

@item del_assoc(+@var{Key}, +@var{Assoc}, ?@var{Val}, ?@var{NewAssoc})
@findex del_assoc/4
@syindex del_assoc/4
@cnindex del_assoc/4
Succeeds if @var{NewAssoc} is an association list, obtained by removing
the element with @var{Key} and @var{Val} from the list @var{Assoc}.

@item del_max_assoc(+@var{Assoc}, ?@var{Key}, ?@var{Val}, ?@var{NewAssoc})
@findex del_max_assoc/4
@syindex del_max_assoc/4
@cnindex del_max_assoc/4
Succeeds if @var{NewAssoc} is an association list, obtained by removing
the largest element of the list, with @var{Key} and @var{Val} from the
list @var{Assoc}.

@item del_min_assoc(+@var{Assoc}, ?@var{Key}, ?@var{Val}, ?@var{NewAssoc})
@findex del_min_assoc/4
@syindex del_min_assoc/4
@cnindex del_min_assoc/4
Succeeds if @var{NewAssoc} is an association list, obtained by removing
the smallest element of the list, with @var{Key} and @var{Val}
from the list @var{Assoc}.

@item empty_assoc(+@var{Assoc})
@findex empty_assoc/1
@syindex empty_assoc/1
@cnindex empty_assoc/1
Succeeds if association list @var{Assoc} is empty.

@item gen_assoc(+@var{Assoc},?@var{Key},?@var{Value})
@findex gen_assoc/3
@syindex gen_assoc/3
@cnindex gen_assoc/3
Given the association list @var{Assoc}, unify @var{Key} and @var{Value}
with two associated elements. It can be used to enumerate all elements
in the association list.

@item get_assoc(+@var{Key},+@var{Assoc},?@var{Value})
@findex get_next_assoc/4
@syindex get_next_assoc/4
@cnindex get_next_assoc/4
If @var{Key} is one of the elements in the association list @var{Assoc},
return the associated value.

@item get_assoc(+@var{Key},+@var{Assoc},?@var{Value},+@var{NAssoc},?@var{NValue})
@findex get_assoc/5
@syindex get_assoc/5
@cnindex get_assoc/5
If @var{Key} is one of the elements in the association list @var{Assoc},
return the associated value @var{Value} and a new association list
@var{NAssoc} where @var{Key} is associated with @var{NValue}.

@item get_prev_assoc(+@var{Key},+@var{Assoc},?@var{Next},?@var{Value})
@findex get_prev_assoc/4
@syindex get_prev_assoc/4
@cnindex get_prev_assoc/4
If @var{Key} is one of the elements in the association list @var{Assoc},
return the previous key, @var{Next}, and its value, @var{Value}.

@item get_next_assoc(+@var{Key},+@var{Assoc},?@var{Next},?@var{Value})
@findex get_assoc/3
@syindex get_assoc/3
@cnindex get_assoc/3
If @var{Key} is one of the elements in the association list @var{Assoc},
return the next key, @var{Next}, and its value, @var{Value}.

@item is_assoc(+@var{Assoc})
@findex is_assoc/1
@syindex is_assoc/1
@cnindex is_assoc/1
Succeeds if @var{Assoc} is an association list, that is, if it is a
red-black tree.

@item list_to_assoc(+@var{List},?@var{Assoc})
@findex list_to_assoc/2
@syindex list_to_assoc/2
@cnindex list_to_assoc/2
Given a list @var{List} such that each element of @var{List} is of the
form @var{Key-Val}, and all the @var{Keys} are unique, @var{Assoc} is
the corresponding association list.

@item map_assoc(+@var{Pred},+@var{Assoc})
@findex map_assoc/2
@syindex map_assoc/2
@cnindex map_assoc/2
Succeeds if the unary predicate name @var{Pred}(@var{Val}) holds for every
element in the association list.

@item map_assoc(+@var{Pred},+@var{Assoc},?@var{New})
@findex map_assoc/3
@syindex map_assoc/3
@cnindex map_assoc/3
Given the binary predicate name @var{Pred} and the association list
@var{Assoc}, @var{New} in an association list with keys in @var{Assoc},
and such that if @var{Key-Val} is in @var{Assoc}, and @var{Key-Ans} is in
@var{New}, then @var{Pred}(@var{Val},@var{Ans}) holds.

@item max_assoc(+@var{Assoc},-@var{Key},?@var{Value})
@findex max_assoc/3
@syindex max_assoc/3
@cnindex max_assoc/3
Given the association list
@var{Assoc}, @var{Key} in the largest key in the list, and @var{Value}
the associated value.

@item min_assoc(+@var{Assoc},-@var{Key},?@var{Value})
@findex min_assoc/3
@syindex min_assoc/3
@cnindex min_assoc/3
Given the association list
@var{Assoc}, @var{Key} in the smallest key in the list, and @var{Value}
the associated value.

@item ord_list_to_assoc(+@var{List},?@var{Assoc})
@findex ord_list_to_assoc/2
@syindex ord_list_to_assoc/2
@cnindex ord_list_to_assoc/2
Given an ordered list @var{List} such that each element of @var{List} is
of the form @var{Key-Val}, and all the @var{Keys} are unique, @var{Assoc} is
the corresponding association list.

@item put_assoc(+@var{Key},+@var{Assoc},+@var{Val},+@var{New})
@findex put_assoc/4
@syindex put_assoc/4
@cnindex put_assoc/4
The association list @var{New} includes and element of association
@var{key} with @var{Val}, and all elements of @var{Assoc} that did not
have key @var{Key}.

@end table

@node AVL Trees, Heaps, Association Lists, Library
@section AVL Trees
@cindex AVL trees

AVL trees are balanced search binary trees. They are named after their
inventors, Adelson-Velskii and Landis, and they were the first
dynamically balanced trees to be proposed. The YAP AVL tree manipulation
predicates library uses code originally written by Martin van Emdem and
published in the Logic Programming Newsletter, Autumn 1981.  A bug in
this code was fixed by Philip Vasey, in the Logic Programming
Newsletter, Summer 1982. The library currently only includes routines to
insert and lookup elements in the tree. Please try red-black trees if
you need deletion.

@table @code
@item avl_new(+@var{T})
@findex avl_new/1
@snindex avl_new/1
@cnindex avl_new/1
Create a new tree.

@item avl_insert(+@var{Key},?@var{Value},+@var{T0},-@var{TF})
@findex avl_insert/4
@snindex avl_insert/4
@cnindex avl_insert/4
Add an element with key @var{Key} and @var{Value} to the AVL tree
@var{T0} creating a new AVL tree @var{TF}. Duplicated elements are
allowed.

@item avl_lookup(+@var{Key},-@var{Value},+@var{T})
@findex avl_lookup/3
@snindex avl_lookup/3
@cnindex avl_lookup/3
Lookup an element with key @var{Key} in the AVL tree
@var{T}, returning the value @var{Value}.

@end table

@node Heaps, Lists, AVL Trees, Library
@section Heaps
@cindex heap

A heap is a labelled binary tree where the key of each node is less than
or equal to the keys of its sons.  The point of a heap is that we can
keep on adding new elements to the heap and we can keep on taking out
the minimum element.  If there are N elements total, the total time is
O(NlgN).  If you know all the elements in advance, you are better off
doing a merge-sort, but this file is for when you want to do say a
best-first search, and have no idea when you start how many elements
there will be, let alone what they are.

The following heap manipulation routines are available once included
with the @code{use_module(library(heaps))} command. 

@table @code

@item add_to_heap(+@var{Heap},+@var{key},+@var{Datum},-@var{NewHeap})
@findex add_to_heap/4
@syindex        add_to_heap/4
@cnindex        add_to_heap/4
Inserts the new @var{Key-Datum} pair into the heap. The insertion is not
stable, that is, if you insert several pairs with the same @var{Key} it
is not defined which of them will come out first, and it is possible for
any of them to come out first depending on the  history of the heap.

@item empty_heap(?@var{Heap})
@findex empty_heap/1
@syindex        empty_heap/1
@cnindex        empty_heap/1
Succeeds if @var{Heap} is an empty heap.

@item get_from_heap(+@var{Heap},-@var{key},-@var{Datum},-@var{Heap})
@findex get_from_heap/4
@syindex        get_from_heap/4
@cnindex        get_from_heap/4
Returns the @var{Key-Datum} pair in @var{OldHeap} with the smallest
@var{Key}, and also a @var{Heap} which is the @var{OldHeap} with that
pair deleted.

@item heap_size(+@var{Heap}, -@var{Size})
@findex heap_size/2
@syindex        heap_size/2
@cnindex        heap_size/2
Reports the number of elements currently in the heap.

@item heap_to_list(+@var{Heap}, -@var{List})
@findex heap_to_list/2
@syindex        heap_to_list/2
@cnindex        heap_to_list/2
Returns the current set of @var{Key-Datum} pairs in the @var{Heap} as a
@var{List}, sorted into ascending order of @var{Keys}.

@item list_to_heap(+@var{List}, -@var{Heap})
@findex list_to_heap/2
@syindex        list_to_heap/2
@cnindex        list_to_heap/2
Takes a list of @var{Key-Datum} pairs (such as keysort could be used to sort)
and forms them into a heap.

@item min_of_heap(+@var{Heap},  -@var{Key},  -@var{Datum})
@findex min_of_heap/3
@syindex min_of_heap/3
@cnindex min_of_heap/3
Returns the Key-Datum pair at the top of the heap (which is of course
the pair with the smallest Key), but does not remove it from the heap.

@item min_of_heap(+@var{Heap},  -@var{Key1},  -@var{Datum1},
-@var{Key2},  -@var{Datum2})
@findex min_of_heap/5
@syindex min_of_heap/5
@cnindex min_of_heap/5
Returns the smallest (Key1) and second smallest (Key2) pairs in the
heap, without deleting them.
@end table

@node Lists, LineUtilities, Heaps, Library
@section List Manipulation
@cindex list manipulation

The following list manipulation routines are available once included
with the @code{use_module(library(lists))} command. 

@table @code

@item append(?@var{Prefix},?@var{Suffix},?@var{Combined})
@findex append/3
@syindex append/3
@cnindex append/3
True when all three arguments are lists, and the members of
@var{Combined} are the members of @var{Prefix} followed by the members of @var{Suffix}.
It may be used to form @var{Combined} from a given @var{Prefix}, @var{Suffix} or to take
a given @var{Combined} apart.

@item append(?@var{Lists},?@var{Combined})
@findex append/2
@syindex append/2
@cnindex append/2
Holds if the lists of @var{Lists} can be concatenated as a
@var{Combined} list.

@item delete(+@var{List}, ?@var{Element}, ?@var{Residue})
@findex delete/3
@syindex delete/3
@cnindex delete/3
True when @var{List} is a list, in which @var{Element} may or may not
occur, and @var{Residue} is a copy of @var{List} with all elements
identical to @var{Element} deleted.

@item flatten(+@var{List}, ?@var{FlattenedList})
@findex flatten/2
@syindex flatten/2
@cnindex flatten/2
Flatten a list of lists @var{List} into a single list
@var{FlattenedList}.

@example
?- flatten([[1],[2,3],[4,[5,6],7,8]],L).

L = [1,2,3,4,5,6,7,8] ? ;

no
@end example

@item last(+@var{List},?@var{Last})
@findex last/2
@syindex last/2
@cnindex last/2
True when @var{List} is a list and @var{Last} is identical to its last element.

@item list_concat(+@var{Lists},?@var{List})
@findex list_concat/2
@snindex list_concat/2
@cnindex list_concat/2
True when @var{Lists} is a list of lists and @var{List} is the
concatenation of @var{Lists}.

@item member(?@var{Element}, ?@var{Set})
@findex member/2
@syindex member/2
@cnindex member/2
True when @var{Set} is a list, and @var{Element} occurs in it.  It may be used
to test for an element or to enumerate all the elements by backtracking.

@item memberchk(+@var{Element}, +@var{Set})
@findex memberchk/2
@syindex memberchk/2
@cnindex memberchk/2
As @code{member/2}, but may only be used to test whether a known
@var{Element} occurs in a known Set.  In return for this limited use, it
is more efficient when it is applicable.

@item nth0(?@var{N}, ?@var{List}, ?@var{Elem})
@findex nth0/3
@syindex nth0/3
@cnindex nth0/3
True when @var{Elem} is the Nth member of @var{List},
counting the first as element 0.  (That is, throw away the first
N elements and unify @var{Elem} with the next.)  It can only be used to
select a particular element given the list and index.  For that
task it is more efficient than @code{member/2}

@item nth1(?@var{N}, ?@var{List}, ?@var{Elem})
@findex nth1/3
@syindex nth1/3
@cnindex nth1/3
The same as @code{nth0/3}, except that it counts from
1, that is @code{nth(1, [H|_], H)}.

@item nth(?@var{N}, ?@var{List}, ?@var{Elem})
@findex nth/3
@syindex nth/3
@cnindex nth/3
The same as @code{nth1/3}.

@item nth0(?@var{N}, ?@var{List}, ?@var{Elem}, ?@var{Rest})
@findex nth0/4
@syindex nth0/4
@cnindex nth0/4
Unifies @var{Elem} with the Nth element of @var{List},
counting from 0, and @var{Rest} with the other elements.  It can be used
to select the Nth element of @var{List} (yielding @var{Elem} and @var{Rest}), or to
insert @var{Elem} before the Nth (counting from 1) element of @var{Rest}, when
it yields @var{List}, e.g. @code{nth0(2, List, c, [a,b,d,e])} unifies List with
@code{[a,b,c,d,e]}.  @code{nth/4} is the same except that it counts from 1.  @code{nth0/4}
can be used to insert @var{Elem} after the Nth element of @var{Rest}.

@item nth1(?@var{N}, ?@var{List}, ?@var{Elem}, ?@var{Rest})
@findex nth1/4
@syindex nth1/4
@cnindex nth1/4
Unifies @var{Elem} with the Nth element of @var{List}, counting from 1,
and @var{Rest} with the other elements.  It can be used to select the
Nth element of @var{List} (yielding @var{Elem} and @var{Rest}), or to
insert @var{Elem} before the Nth (counting from 1) element of
@var{Rest}, when it yields @var{List}, e.g. @code{nth(3, List, c,
[a,b,d,e])} unifies List with @code{[a,b,c,d,e]}.  @code{nth/4}
can be used to insert @var{Elem} after the Nth element of @var{Rest}.

@item nth(?@var{N}, ?@var{List}, ?@var{Elem}, ?@var{Rest})
@findex nth/4
@syindex nth/4
@cnindex nth/4
Same as @code{nth1/4}.

@item permutation(+@var{List},?@var{Perm})
@findex permutation/2
@syindex permutation/2
@cnindex permutation/2
True when @var{List} and @var{Perm} are permutations of each other.

@item remove_duplicates(+@var{List}, ?@var{Pruned})
@findex remove_duplicates/2
@syindex remove_duplicates/2
@cnindex remove_duplicates/2
Removes duplicated elements from @var{List}.  Beware: if the @var{List} has
non-ground elements, the result may surprise you.

@item reverse(+@var{List}, ?@var{Reversed})
@findex reverse/2
@syindex reverse/2
@cnindex reverse/2
True when @var{List} and @var{Reversed} are lists with the same elements
but in opposite orders. 
 
@item same_length(?@var{List1}, ?@var{List2})
@findex same_length/2
@syindex same_length/2
@cnindex same_length/2
True when @var{List1} and @var{List2} are both lists and have the same number
of elements.  No relation between the values of their elements is
implied.
Modes @code{same_length(-,+)} and @code{same_length(+,-)} generate either list given
the other; mode @code{same_length(-,-)} generates two lists of the same length,
in which case the arguments will be bound to lists of length 0, 1, 2, ...

@item select(?@var{Element}, ?@var{List}, ?@var{Residue})
@findex select/3
@syindex select/3
@cnindex select/3
True when @var{Set} is a list, @var{Element} occurs in @var{List}, and
@var{Residue} is everything in @var{List} except @var{Element} (things
stay in the same order).
 
@item selectchk(?@var{Element}, ?@var{List}, ?@var{Residue})
@findex selectchk/3
@snindex selectchk/3
@cnindex selectchk/3
Semi-deterministic selection from a list. Steadfast: defines as

@example
selectchk(Elem, List, Residue) :-
        select(Elem, List, Rest0), !,
        Rest = Rest0.
@end example

 
@item sublist(?@var{Sublist}, ?@var{List})
@findex sublist/2
@syindex sublist/2
@cnindex sublist/2
True when both @code{append(_,Sublist,S)} and @code{append(S,_,List)} hold.
 
@item suffix(?@var{Suffix}, ?@var{List})
@findex suffix/2
@syindex suffix/2
@cnindex suffix/2
Holds when @code{append(_,Suffix,List)} holds.

@item sum_list(?@var{Numbers}, ?@var{Total})
@findex sum_list/2
@syindex sum_list/2
@cnindex sum_list/2
True when @var{Numbers} is a list of numbers, and @var{Total} is their sum.

@item sum_list(?@var{Numbers}, +@var{SoFar}, ?@var{Total})
@findex sum_list/3
@syindex sum_list/3
@cnindex sum_list/3
True when @var{Numbers} is a list of numbers, and @var{Total} is the sum of their total plus @var{SoFar}.

@item sumlist(?@var{Numbers}, ?@var{Total})
@findex sumlist/2
@syindex sumlist/2
@cnindex sumlist/2
True when @var{Numbers} is a list of integers, and @var{Total} is their
sum. The same as @code{sum_list/2}, please do use @code{sum_list/2}
instead.

@item max_list(?@var{Numbers}, ?@var{Max})
@findex max_list/2
@syindex max_list/2
@cnindex max_list/2
True when @var{Numbers} is a list of numbers, and @var{Max} is the maximum.

@item min_list(?@var{Numbers}, ?@var{Min})
@findex min_list/2
@syindex min_list/2
@cnindex min_list/2
True when @var{Numbers} is a list of numbers, and @var{Min} is the minimum.

@item numlist(+@var{Low}, +@var{High}, +@var{List})
@findex numlist/3
@syindex numlist/3
@cnindex numlist/3
If @var{Low} and @var{High} are integers with @var{Low} =<
@var{High}, unify @var{List} to a list @code{[Low, Low+1, ...High]}. See
also @code{between/3}.

@item intersection(+@var{Set1}, +@var{Set2}, +@var{Set3})
@findex intersection/3
@syindex intersection/3
@cnindex intersection/3
Succeeds if @var{Set3} unifies with the intersection of @var{Set1} and
@var{Set2}. @var{Set1} and @var{Set2} are lists without duplicates. They
need not be ordered.
@end table

@node LineUtilities, MapList, Lists, Library
@section Line Manipulation Utilities
@cindex Line Utilities Library

This package provides a set of useful predicates to manipulate
sequences of characters codes, usually first read in as a line. It is
avalailable by loading the library @code{library(lineutils)}.

@table @code

@item search_for(+@var{Char},+@var{Line})
@findex search_for/2
@snindex search_for/2
@cnindex search_for/2

Search for a character @var{Char} in the list of codes @var{Line}.

@item search_for(+@var{Char},+@var{Line})
@findex search_for/2
@snindex search_for/2
@cnindex search_for/2

Search for a character @var{Char} in the list of codes @var{Line}.

@item search_for(+@var{Char},+@var{Line},-@var{RestOfine})
@findex search_for/2
@snindex search_for/2
@cnindex search_for/2

Search for a character @var{Char} in the list of codes @var{Line},
@var{RestOfLine} has the line to the right.

@item scan_natural(?@var{Nat},+@var{Line},+@var{RestOfLine})
@findex scan_natural/3
@snindex scan_natural/3
@cnindex scan_natural/3

Scan the list of codes @var{Line} for a natural number @var{Nat}, zero
or a positive integer, and unify @var{RestOfLine} with the remainder
of the line.

@item scan_integer(?@var{Int},+@var{Line},+@var{RestOfLine})
@findex scan_integer/3
@snindex scan_integer/3
@cnindex scan_integer/3

Scan the list of codes @var{Line} for an integer @var{Nat}, either a
positive, zero, or negative integer, and unify @var{RestOfLine} with
the remainder of the line.

@item split(+@var{Line},+@var{Separators},-@var{Split})
@findex split/3
@snindex split/3
@cnindex split/3

Unify @var{Words} with a set of strings obtained from @var{Line} by
using the character codes in @var{Separators} as separators. As an
example, consider:
@example
?- split("Hello * I am free"," *",S).

S = ["Hello","I","am","free"] ?

no
@end example

@item split(+@var{Line},-@var{Split})
@findex split/2
@snindex split/2
@cnindex split/2

Unify @var{Words} with a set of strings obtained from @var{Line} by
using the blank characters  as separators.

@item fields(+@var{Line},+@var{Separators},-@var{Split})
@findex fields/3
@snindex fields/3
@cnindex fields/3

Unify @var{Words} with a set of strings obtained from @var{Line} by
using the character codes in @var{Separators} as separators for
fields. If two separators occur in a row, the field is considered
empty. As an example, consider:
@example
?- fields("Hello  I am  free"," *",S).

S = ["Hello","","I","am","","free"] ?
@end example

@item fields(+@var{Line},-@var{Split})
@findex fields/2
@snindex fields/2
@cnindex fields/2

Unify @var{Words} with a set of strings obtained from @var{Line} by
using the blank characters  as field separators.

@item glue(+@var{Words},+@var{Separator},-@var{Line})
@findex glue/3
@snindex glue/3
@cnindex glue/3

Unify @var{Line} with  string obtained by glueing @var{Words} with
the character code @var{Separator}.

@item copy_line(+@var{StreamInput},+@var{StreamOutput})
@findex copy_line/2
@snindex copy_line/2
@cnindex copy_line/2

Copy a line from @var{StreamInput} to @var{StreamOutput}.

@item copy_line(+@var{StreamInput},+@var{StreamOutput})
@findex copy_line/2
@snindex copy_line/2
@cnindex copy_line/2

Copy a line from @var{StreamInput} to @var{StreamOutput}.

@item process(+@var{StreamInp}, +@var{Goal})
@findex process/2
@snindex process/2
@cnindex process/2

For every line @var{LineIn} in stream @var{StreamInp}, call
@code{call(Goal,LineIn)}.

@item filter(+@var{StreamInp}, +@var{StreamOut}, +@var{Goal})
@findex filter/3
@snindex filter/3
@cnindex filter/3

For every line @var{LineIn} in stream @var{StreamInp}, execute
@code{call(Goal,LineIn,LineOut)}, and output @var{LineOut} to
stream @var{StreamOut}.

@item file_filter(+@var{FileIn}, +@var{FileOut}, +@var{Goal})
@findex file_filter/3
@snindex file_filter/3
@cnindex file_filter/3

For every line @var{LineIn} in file @var{FileIn}, execute
@code{call(Goal,LineIn,LineOut)}, and output @var{LineOut} to file
@var{FileOut}.

@item file_filter(+@var{FileIn}, +@var{FileOut}, +@var{Goal},
+@var{FormatCommand},   +@var{Arguments})
@findex file_filter_with_init/5
@snindex file_filter_with_init/5
@cnindex file_filter_with_init/5

Same as @code{file_filter/3}, but before starting the filter execute
@code{format/3} on the output stream, using @var{FormatCommand} and
@var{Arguments}.

@end table



@node MapList, matrix, LineUtilities, Library
@section Maplist
@cindex macros

This library provides a set of utilities for applying a predicate to
all elements of a list or to all sub-terms of a term. They allow to
easily perform the most common do-loop constructs in Prolog. To avoid
performance degradation due to @code{apply/2}, each call creates an
equivalent Prolog program, without meta-calls, which is executed by
the Prolog engine instead. Note that if the equivalent Prolog program
already exists, it will be simply used. The library is based on code
by Joachim Schimpf and on code from SWI-Prolog.

The following routines are available once included with the
@code{use_module(library(apply_macros))} command.

@table @code
@item maplist(+@var{Pred}, ?@var{ListIn}, ?@var{ListOut})
@findex maplist/3
@snindex maplist/3
@cnindex maplist/3
      Creates @var{ListOut} by applying the predicate @var{Pred} to all
elements of @var{ListIn}.

@item maplist(+@var{Pred}, ?@var{ListIn})
@findex maplist/3
@snindex maplist/3
@cnindex maplist/3
      Creates @var{ListOut} by applying the predicate @var{Pred} to all
elements of @var{ListIn}.

@item maplist(+@var{Pred}, ?@var{L1}, ?@var{L2}, ?@var{L3})
@findex maplist/4
@snindex maplist/4
@cnindex maplist/4
      @var{L1},  @var{L2}, and @var{L3} are such that
      @code{call(@var{Pred},@var{A1},@var{A2},@var{A3})} holds for every
      corresponding element in lists @var{L1},  @var{L2}, and @var{L3}.

@item maplist(+@var{Pred}, ?@var{L1}, ?@var{L2}, ?@var{L3}, ?@var{L4})
@findex maplist/5
@snindex maplist/5
@cnindex maplist/5
      @var{L1}, @var{L2}, @var{L3}, and @var{L4} are such that
      @code{call(@var{Pred},@var{A1},@var{A2},@var{A3},@var{A4})} holds
      for every corresponding element in lists @var{L1}, @var{L2}, @var{L3}, and
      @var{L4}.

@item checklist(+@var{Pred}, +@var{List})
@findex checklist/2
@snindex checklist/2
@cnindex checklist/2
      Succeeds if the predicate @var{Pred} succeeds on all elements of @var{List}.

@item selectlist(+@var{Pred}, +@var{ListIn}, ?@var{ListOut})
@findex selectlist/3
@snindex selectlist/3
@cnindex selectlist/3
      Creates @var{ListOut} of all list elements of @var{ListIn} that pass a given test

@item convlist(+@var{Pred}, +@var{ListIn}, ?@var{ListOut})
@findex convlist/3
@snindex convlist/3
@cnindex convlist/3
      A combination of @code{maplist} and @code{selectlist}: creates @var{ListOut} by
applying the predicate @var{Pred} to all list elements on which
@var{Pred} succeeds

@item sumlist(+@var{Pred}, +@var{List}, ?@var{AccIn}, ?@var{AccOut})
@findex sumlist/4
@snindex sumlist/4
@cnindex sumlist/4
      Calls @var{Pred} on all elements of List and collects a result in
@var{Accumulator}.

@item mapargs(+@var{Pred}, ?@var{TermIn}, ?@var{TermOut})
@findex mapargs/3
@snindex mapargs/3
@cnindex mapargs/3
      Creates @var{TermOut} by applying the predicate @var{Pred} to all
      arguments of @var{TermIn}

@item sumargs(+@var{Pred}, +@var{Term}, ?@var{AccIn}, ?@var{AccOut})
@findex sumargs/4
@snindex sumargs/4
@cnindex sumargs/4
  Calls the predicate @var{Pred} on all arguments of @var{Term} and
collects a  result in @var{Accumulator}

@item mapnodes(+@var{Pred}, +@var{TermIn}, ?@var{TermOut}) 
@findex mapnodes/3
@snindex mapnodes/3
@cnindex mapnodes/3
      Creates @var{TermOut} by applying the predicate @var{Pred}
      to all sub-terms of @var{TermIn} (depth-first and left-to-right order)

@item checknodes(+@var{Pred}, +@var{Term}) 
@findex checknodes/3
@snindex checknodes/3
@cnindex checknodes/3
      Succeeds if the predicate @var{Pred} succeeds on all sub-terms of
      @var{Term} (depth-first and left-to-right order)

@item sumnodes(+@var{Pred}, +@var{Term}, ?@var{AccIn}, ?@var{AccOut})
@findex sumnodes/4
@snindex sumnodes/4
@cnindex sumnodes/4
      Calls the predicate @var{Pred} on all sub-terms of @var{Term} and
collect a result in @var{Accumulator} (depth-first and left-to-right
order)

@item include(+@var{Pred}, +@var{ListIn}, ?@var{ListOut})
@findex include/3
@snindex include/3
@cnindex include/3
      Same as @code{selectlist/3}.

@item exclude(+@var{Goal}, +@var{List1}, ?@var{List2})
@findex exclude/3
@snindex exclude/3
@cnindex exclude/3
Filter elements for which @var{Goal} fails. True if @var{List2} contains
      those elements @var{Xi} of @var{List1} for which @code{call(Goal, Xi)} fails.

@item partition(+@var{Pred},  +@var{List1}, ?@var{Included}, ?@var{Excluded})
@findex partition/4
@snindex partition/4
@cnindex partition/4
Filter elements of @var{List} according to @var{Pred}. True if
@var{Included} contains all elements for which @code{call(Pred, X)}
succeeds and @var{Excluded} contains the remaining elements.

@item partition(+@var{Pred},  +@var{List1}, ?@var{Lesser}, ?@var{Equal}, ?@var{Greater})
@findex partition/5
@snindex partition/5
@cnindex partition/5
Filter list according to @var{Pred} in three sets. For each element
@var{Xi} of @var{List}, its destination is determined by
@code{call(Pred, Xi, Place)}, where @var{Place} must be unified to one
of @code{<}, @code{=} or @code{>}. @code{Pred} must be deterministic.

@end table

Examples:

@example
%given
plus(X,Y,Z) :- Z is X + Y.
plus_if_pos(X,Y,Z) :- Y > 0, Z is X + Y.
vars(X, Y, [X|Y]) :- var(X), !.
vars(_, Y, Y).
trans(TermIn, TermOut) :-
  (compound(TermIn) ; atom(TermIn)),
  TermIn =.. [p|Args],
  TermOut =..[q|Args],
  !.
trans(X,X).

%success

maplist(plus(1), [1,2,3,4], [2,3,4,5]).
checklist(var, [X,Y,Z]).
selectlist(<(0), [-1,0,1], [1]).
convlist(plus_if_pos(1), [-1,0,1], [2]).
sumlist(plus, [1,2,3,4], 1, 11).
mapargs(number_atom,s(1,2,3), s('1','2','3')).
sumargs(vars, s(1,X,2,Y), [], [Y,X]).
mapnodes(trans, p(a,p(b,a),c), q(a,q(b,a),c)).
checknodes(\==(T), p(X,p(Y,X),Z)).
sumnodes(vars, [c(X), p(X,Y), q(Y)], [], [Y,Y,X,X]).
% another one
maplist(mapargs(number_atom),[c(1),s(1,2,3)],[c('1'),s('1','2','3')]).
@end example

@node matrix, MATLAB, MapList, Library
@section Matrix Library
@cindex Matrix Library

This package provides a fast implementation of multi-dimensional
matrices of integers and floats. In contrast to dynamic arrays, these
matrices are multi-dimensional and compact. In contrast to static
arrays. these arrays are allocated in the stack. Matrices are available
by loading the library @code{library(matrix)}.

Accessing the matlab dynamic libraries can be complicated. In Linux
machines, to use this interface, you may have to set the environment
variable @t{LD_LIBRARY_PATH}. Next, follows an example using bash in a
64-bit Linux PC:
@example
export LD_LIBRARY_PATH=''$MATLAB_HOME"/sys/os/glnxa64:''$MATLAB_HOME"/bin/glnxa64:''$LD_LIBRARY_PATH"
@end example
where @code{MATLAB_HOME} is the directory where matlab is installed
at. Please replace @code{ax64} for @code{x86} on a 32-bit PC.

Notice that the functionality in this library is only partial. Please
contact the YAP maintainers if you require extra functionality.

@table @code

@item matrix_new(+@var{Type},+@var{Dims},-@var{Matrix})
@findex matrix_new/3
@snindex matrix_new/3
@cnindex matrix_new/3

Create a new matrix @var{Matrix} of type @var{Type}, which may be one of
@code{ints} or @code{floats}, and with a list of dimensions @var{Dims}.
The matrix will be initialised to zeros.

@example
?- matrix_new(ints,[2,3],Matrix).

Matrix = 0
@end example
Notice that currently YAP will always write a matrix as @code{0}.

@item matrix_new(+@var{Type},+@var{Dims},+@var{List},-@var{Matrix})
@findex matrix_new/4
@snindex matrix_new/4
@cnindex matrix_new/4

Create a new matrix @var{Matrix} of type @var{Type}, which may be one of
@code{ints} or @code{floats}, with dimensions @var{Dims}, and
initialised from list @var{List}.

@item matrix_new_set(?@var{Dims},+@var{OldMatrix},+@var{Value},-@var{NewMatrix})
@findex matrix_new_set/4
@snindex matrix_new_set/4
@cnindex matrix_new_set/4

Create a new matrix @var{NewMatrix} of type @var{Type}, with dimensions
@var{Dims}. The elements of @var{NewMatrix} are set to @var{Value}.

@item matrix_dims(+@var{Matrix},-@var{Dims})
@findex matrix_dims/2
@snindex matrix_dims/2
@cnindex matrix_dims/2

Unify @var{Dims} with a list of dimensions for @var{Matrix}.

@item matrix_ndims(+@var{Matrix},-@var{Dims})
@findex matrix_ndims/2
@snindex matrix_ndims/2
@cnindex matrix_ndims/2

Unify @var{NDims} with the number of dimensions for @var{Matrix}.

@item matrix_size(+@var{Matrix},-@var{NElems})
@findex matrix_size/2
@snindex matrix_size/2
@cnindex matrix_size/2

Unify @var{NElems} with the number of elements for @var{Matrix}.

@item matrix_type(+@var{Matrix},-@var{Type})
@findex matrix_type/2
@snindex matrix_type/2
@cnindex matrix_type/2

Unify @var{NElems} with the type of the elements in @var{Matrix}.

@item matrix_to_list(+@var{Matrix},-@var{Elems})
@findex matrix_to_list/2
@snindex matrix_to_list/2
@cnindex matrix_to_list/2

Unify @var{Elems} with the list including all the elements in @var{Matrix}.

@item matrix_get(+@var{Matrix},+@var{Position},-@var{Elem})
@findex matrix_get/3
@snindex matrix_get/3
@cnindex matrix_get/3

Unify @var{Elem} with the element of @var{Matrix} at position
@var{Position}.

@item matrix_set(+@var{Matrix},+@var{Position},+@var{Elem})
@findex matrix_set/3
@snindex matrix_set/3
@cnindex matrix_set/3

Set the element of @var{Matrix} at position
@var{Position} to  @var{Elem}.

@item matrix_set_all(+@var{Matrix},+@var{Elem})
@findex matrix_set_all/2
@snindex matrix_set_all/2
@cnindex matrix_set_all/2

Set all element of @var{Matrix} to @var{Elem}.

@item matrix_add(+@var{Matrix},+@var{Position},+@var{Operand})
@findex matrix_add/3
@snindex matrix_add/3
@cnindex matrix_add/3

Add @var{Operand} to the element of @var{Matrix} at position
@var{Position}.

@item matrix_inc(+@var{Matrix},+@var{Position})
@findex matrix_inc/2
@snindex matrix_inc/2
@cnindex matrix_inc/2

Increment the element of @var{Matrix} at position @var{Position}.

@item matrix_inc(+@var{Matrix},+@var{Position},-@var{Element})
@findex matrix_inc/3
@snindex matrix_inc/3
@cnindex matrix_inc/3

Increment the element of @var{Matrix} at position @var{Position} and
unify with @var{Element}.

@item matrix_dec(+@var{Matrix},+@var{Position})
@findex matrix_dec/2
@snindex matrix_dec/2
@cnindex matrix_dec/2

Decrement the element of @var{Matrix} at position @var{Position}.

@item matrix_dec(+@var{Matrix},+@var{Position},-@var{Element})
@findex matrix_dec/3
@snindex matrix_dec/3
@cnindex matrix_dec/3

Decrement the element of @var{Matrix} at position @var{Position} and
unify with @var{Element}.

@item matrix_arg_to_offset(+@var{Matrix},+@var{Position},-@var{Offset})
@findex matrix_arg_to_offset/3
@snindex matrix_arg_to_offset/3
@cnindex matrix_arg_to_offset/3

Given matrix @var{Matrix} return what is the numerical @var{Offset} of
the element at @var{Position}.

@item matrix_offset_to_arg(+@var{Matrix},-@var{Offset},+@var{Position})
@findex matrix_offset_to_arg/3
@snindex matrix_offset_to_arg/3
@cnindex matrix_offset_to_arg/3

Given a position @var{Position } for matrix @var{Matrix} return the
corresponding numerical @var{Offset} from the beginning of the matrix.

@item matrix_max(+@var{Matrix},+@var{Max})
@findex matrix_max/2
@snindex matrix_max/2
@cnindex matrix_max/2

Unify @var{Max} with the maximum in matrix  @var{Matrix}.

@item matrix_maxarg(+@var{Matrix},+@var{Maxarg})
@findex matrix_maxarg/2
@snindex matrix_maxarg/2
@cnindex matrix_maxarg/2

Unify @var{Max} with the position of the maximum in matrix  @var{Matrix}.

@item matrix_min(+@var{Matrix},+@var{Min})
@findex matrix_min/2
@snindex matrix_min/2
@cnindex matrix_min/2

Unify @var{Min} with the minimum in matrix  @var{Matrix}.

@item matrix_minarg(+@var{Matrix},+@var{Minarg})
@findex matrix_minarg/2
@snindex matrix_minarg/2
@cnindex matrix_minarg/2

Unify @var{Min} with the position of the minimum in matrix  @var{Matrix}.

@item matrix_sum(+@var{Matrix},+@var{Sum})
@findex matrix_sum/2
@snindex matrix_sum/2
@cnindex matrix_sum/2

Unify @var{Sum} with the sum of all elements in matrix  @var{Matrix}.

@c @item matrix_add_to_all(+@var{Matrix},+@var{Element})
@c @findex matrix_add_to_all/2
@c @snindex matrix_add_to_all/2
@c @cnindex matrix_add_to_all/2

@c Add @var{Element} to all elements of matrix  @var{Matrix}.

@item matrix_agg_lines(+@var{Matrix},+@var{Aggregate})
@findex matrix_agg_lines/2
@snindex matrix_agg_lines/2
@cnindex matrix_agg_lines/2

If @var{Matrix} is a n-dimensional matrix, unify @var{Aggregate} with
the n-1 dimensional matrix where each element is obtained by adding all
Matrix elements with same last n-1 index.

@item matrix_agg_cols(+@var{Matrix},+@var{Aggregate})
@findex matrix_agg_cols/2
@snindex matrix_agg_cols/2
@cnindex matrix_agg_cols/2

If @var{Matrix} is a n-dimensional matrix, unify @var{Aggregate} with
the one dimensional matrix where each element is obtained by adding all
Matrix elements with same  first index.

@item matrix_op(+@var{Matrix1},+@var{Matrix2},+@var{Op},-@var{Result})
@findex matrix_op/4
@snindex matrix_op/4
@cnindex matrix_op/4

@var{Result} is the result of applying @var{Op} to matrix @var{Matrix1}
and @var{Matrix2}. Currently, only addition (@code{+}) is supported.

@item matrix_op_to_all(+@var{Matrix1},+@var{Op},+@var{Operand},-@var{Result})
@findex matrix_op/4
@snindex matrix_op/4
@cnindex matrix_op/4

@var{Result} is the result of applying @var{Op} to all elements of
@var{Matrix1}, with @var{Operand} as the second argument. Currently,
only addition (@code{+}), multiplication (@code{+}), and division
(@code{/}) are supported.

@item matrix_op_to_lines(+@var{Matrix1},+@var{Lines},+@var{Op},-@var{Result})
@findex matrix_op_to_lines/4
@snindex matrix_op_to_lines/4
@cnindex matrix_op_to_lines/4

@var{Result} is the result of applying @var{Op} to all elements of
@var{Matrix1}, with the corresponding element in @var{Lines} as the
second argument. Currently, only division (@code{/}) is supported.

@item matrix_op_to_cols(+@var{Matrix1},+@var{Cols},+@var{Op},-@var{Result})
@findex matrix_op_to_cols/4
@snindex matrix_op_to_cols/4
@cnindex matrix_op_to_cols/4

@var{Result} is the result of applying @var{Op} to all elements of
@var{Matrix1}, with the corresponding element in @var{Cols} as the
second argument. Currently, only addition (@code{+}) is
supported. Notice that @var{Cols} will have n-1 dimensions.

@item matrix_shuffle(+@var{Matrix},+@var{NewOrder},-@var{Shuffle})
@findex matrix_shuffle/3
@snindex matrix_shuffle/3
@cnindex matrix_shuffle/3

Shuffle the dimensions of matrix @var{Matrix} according to
@var{NewOrder}. The list @var{NewOrder} must have all the dimensions of
@var{Matrix}, starting from 0.

@item matrix_transpose(+@var{Matrix},-@var{Transpose})
@findex matrix_reorder/3
@snindex matrix_reorder/3
@cnindex matrix_reorder/3

Transpose matrix @var{Matrix} to  @var{Transpose}. Equivalent to:
@example
matrix_transpose(Matrix,Transpose) :-
        matrix_shuffle(Matrix,[1,0],Transpose).
@end example

@item matrix_expand(+@var{Matrix},+@var{NewDimensions},-@var{New})
@findex matrix_expand/3
@snindex matrix_expand/3
@cnindex matrix_expand/3

Expand @var{Matrix} to occupy new dimensions. The elements in
@var{NewDimensions} are either 0, for an existing dimension, or a
positive integer with the size of the new dimension.

@item matrix_select(+@var{Matrix},+@var{Dimension},+@var{Index},-@var{New})
@findex matrix_select/4
@snindex matrix_select/4
@cnindex matrix_select/4

Select from @var{Matrix} the elements who have @var{Index} at
@var{Dimension}.

@item matrix_row(+@var{Matrix},+@var{Column},-@var{NewMatrix})
@findex matrix_row/3
@snindex matrix_row/3
@cnindex matrix_row/3

Select from @var{Matrix} the row matching @var{Column} as new matrix @var{NewMatrix}. @var{Column} must have one less dimension than the original matrix.
@var{Dimension}.

@end table

@node MATLAB, Non-Backtrackable Data Structures, matrix, Library
@section MATLAB Package Interface
@cindex Matlab Interface

The MathWorks MATLAB is a widely used package for array
processing. YAP now includes a straightforward interface to MATLAB. To
actually use it, you need to install YAP calling @code{configure} with
the @code{--with-matlab=DIR} option, and you need to call
@code{use_module(library(lists))} command.

@table @code

@item start_matlab(+@var{Options})
@findex start_matlab/1
@snindex start_matlab/1
@cnindex start_matlab/1
Start a matlab session. The argument @var{Options} may either be the
empty string/atom or the command to call matlab. The command may fail.

@item close_matlab
@findex close_matlab/0
@snindex close_matlab/0
@cnindex close_matlab/0
Stop the current matlab session.

@item matlab_on
@findex matlab_on/0
@snindex matlab_on/0
@cnindex matlab_on/0
Holds if a matlab session is on.

@item matlab_eval_string(+@var{Command})
@findex matlab_eval_string/1
@snindex matlab_eval_string/1
@cnindex matlab_eval_string/1
Holds if matlab evaluated successfully the command @var{Command}.

@item matlab_eval_string(+@var{Command}, -@var{Answer})
@findex matlab_eval_string/2
@snindex matlab_eval_string/2
@cnindex matlab_eval_string/2
MATLAB will evaluate the command @var{Command} and unify @var{Answer}
with a string reporting the result.


@item matlab_cells(+@var{Size}, ?@var{Array})
@findex matlab_cells/2
@snindex matlab_cells/2
@cnindex matlab_cells/2
MATLAB will create an empty vector of cells of size @var{Size}, and if
@var{Array} is bound to an atom, store the array in the matlab
variable with name @var{Array}. Corresponds to the MATLAB command @code{cells}.


@item matlab_cells(+@var{SizeX}, +@var{SizeY}, ?@var{Array})
@findex matlab_cells/3
@snindex matlab_cells/3
@cnindex matlab_cells/3
MATLAB will create an empty array of cells of size @var{SizeX} and
@var{SizeY}, and if @var{Array} is bound to an atom, store the array
in the matlab variable with name @var{Array}.  Corresponds to the
MATLAB command @code{cells}.

@item matlab_initialized_cells(+@var{SizeX}, +@var{SizeY}, +@var{List}, ?@var{Array})
@findex matlab_initialized_cells/4
@snindex matlab_initialized_cells/4
@cnindex matlab_initialized_cells/4
MATLAB will create an array of cells of size @var{SizeX} and
@var{SizeY}, initialized from the list @var{List}, and if @var{Array}
is bound to an atom, store the array in the matlab variable with name
@var{Array}.

@item matlab_matrix(+@var{SizeX}, +@var{SizeY}, +@var{List}, ?@var{Array})
@findex matlab_matrix/4
@snindex matlab_matrix/4
@cnindex matlab_matrix/4
MATLAB will create an array of floats of size @var{SizeX} and @var{SizeY},
initialized from the list @var{List}, and if @var{Array} is bound to
an atom, store the array in the matlab variable with name @var{Array}.

@item matlab_set(+@var{MatVar}, +@var{X}, +@var{Y}, +@var{Value})
@findex matlab_set/4
@snindex matlab_set/4
@cnindex matlab_set/4
Call MATLAB to set element @var{MatVar}(@var{X}, @var{Y}) to
@var{Value}. Notice that this command uses the MATLAB array access
convention.

@item matlab_get_variable(+@var{MatVar}, -@var{List})
@findex matlab_get_variable/2
@snindex matlab_get_variable/2
@cnindex matlab_get_variable/2
Unify MATLAB variable @var{MatVar} with the List @var{List}.

@item matlab_item(+@var{MatVar}, +@var{X}, ?@var{Val})
@findex matlab_item/3
@snindex matlab_item/3
@cnindex matlab_item/3
Read or set MATLAB @var{MatVar}(@var{X}) from/to @var{Val}. Use
@code{C} notation for matrix access (ie, starting from 0).

@item matlab_item(+@var{MatVar}, +@var{X}, +@var{Y}, ?@var{Val})
@findex matlab_item/4
@snindex matlab_item/4
@cnindex matlab_item/4
Read or set MATLAB @var{MatVar}(@var{X},@var{Y}) from/to @var{Val}. Use
@code{C} notation for matrix access (ie, starting from 0).

@item matlab_item1(+@var{MatVar}, +@var{X}, ?@var{Val})
@findex matlab_item/3
@snindex matlab_item/3
@cnindex matlab_item/3
Read or set MATLAB @var{MatVar}(@var{X}) from/to @var{Val}. Use
MATLAB notation for matrix access (ie, starting from 1).

@item matlab_item1(+@var{MatVar}, +@var{X}, +@var{Y}, ?@var{Val})
@findex matlab_item/4
@snindex matlab_item/4
@cnindex matlab_item/4
Read or set MATLAB @var{MatVar}(@var{X},@var{Y}) from/to @var{Val}. Use
MATLAB notation for matrix access (ie, starting from 1).

@item matlab_sequence(+@var{Min}, +@var{Max}, ?@var{Array})
@findex matlab_sequence/3
@snindex matlab_sequence/3
@cnindex matlab_sequence/3
MATLAB will create a sequence going from @var{Min} to @var{Max}, and
if @var{Array} is bound to an atom, store the sequence in the matlab
variable with name @var{Array}.

@item matlab_vector(+@var{Size}, +@var{List}, ?@var{Array})
@findex matlab_vector/4
@snindex matlab_vector/4
@cnindex matlab_vector/4
MATLAB will create a vector of floats of size @var{Size}, initialized
from the list @var{List}, and if @var{Array} is bound to an atom,
store the array in the matlab variable with name @var{Array}.

@item matlab_zeros(+@var{Size}, ?@var{Array})
@findex matlab_zeros/2
@snindex matlab_zeros/2
@cnindex matlab_zeros/2
MATLAB will create a vector of zeros of size @var{Size}, and if
@var{Array} is bound to an atom, store the array in the matlab
variable with name @var{Array}. Corresponds to the MATLAB command
@code{zeros}.

@item matlab_zeros(+@var{SizeX}, +@var{SizeY}, ?@var{Array})
@findex matlab_zeros/3
@snindex matlab_zeros/3
@cnindex matlab_zeros/3
MATLAB will create an array of zeros of size @var{SizeX} and
@var{SizeY}, and if @var{Array} is bound to an atom, store the array
in the matlab variable with name @var{Array}.  Corresponds to the
MATLAB command @code{zeros}.


@item matlab_zeros(+@var{SizeX}, +@var{SizeY}, +@var{SizeZ}, ?@var{Array})
@findex matlab_zeros/4
@snindex matlab_zeros/4
@cnindex matlab_zeros/4
MATLAB will create an array of zeros of size @var{SizeX}, @var{SizeY},
and @var{SizeZ}. If @var{Array} is bound to an atom, store the array
in the matlab variable with name @var{Array}.  Corresponds to the
MATLAB command @code{zeros}.


@item matlab_zeros(+@var{SizeX}, +@var{SizeY}, +@var{SizeZ}, ?@var{Array})
@findex matlab_zeros/4
@snindex matlab_zeros/4
@cnindex matlab_zeros/4
MATLAB will create an array of zeros of size @var{SizeX}, @var{SizeY},
and @var{SizeZ}. If @var{Array} is bound to an atom, store the array
in the matlab variable with name @var{Array}.  Corresponds to the
MATLAB command @code{zeros}.




@end table

@node Non-Backtrackable Data Structures, Ordered Sets, MATLAB, Library
@section Non-Backtrackable Data Structures

The following routines implement well-known data-structures using global
non-backtrackable variables (implemented on the Prolog stack). The
data-structures currently supported are Queues, Heaps, and Beam for Beam
search. They are allowed through @code{library(nb)}. 

@table @code
@item nb_queue(-@var{Queue})
@findex nb_queue/1
@snindex nb_queue/1
@cnindex nb_queue/1
Create a @var{Queue}.

@item nb_queue_close(+@var{Queue}, -@var{Head}, ?@var{Tail})
@findex nb_queue_close/3
@snindex nb_queue_close/3
@cnindex nb_queue_close/3
Unify the queue  @var{Queue} with a difference list
@var{Head}-@var{Tail}. The queue will now be empty and no further
elements can be added.

@item nb_queue_enqueue(+@var{Queue}, +@var{Element})
@findex nb_queue_enqueue/2
@snindex nb_queue_enqueue/2
@cnindex nb_queue_enqueue/2
Add @var{Element} to the front of the queue  @var{Queue}.

@item nb_queue_dequeue(+@var{Queue}, -@var{Element})
@findex nb_queue_dequeue/2
@snindex nb_queue_dequeue/2
@cnindex nb_queue_dequeue/2
Remove @var{Element} from the front of the queue  @var{Queue}. Fail if
the queue is empty.

@item nb_queue_peek(+@var{Queue}, -@var{Element})
@findex nb_queue_peek/2
@snindex nb_queue_peek/2
@cnindex nb_queue_peek/2
@var{Element} is the front of the queue  @var{Queue}. Fail if
the queue is empty.

@item nb_queue_size(+@var{Queue}, -@var{Size})
@findex nb_queue_size/2
@snindex nb_queue_size/2
@cnindex nb_queue_size/2
Unify @var{Size} with the number of elements in the queue  @var{Queue}.

@item nb_queue_empty(+@var{Queue})
@findex nb_queue_empty/1
@snindex nb_queue_empty/1
@cnindex nb_queue_empty/1
Succeeds if  @var{Queue} is empty.

@item nb_heap(+@var{DefaultSize},-@var{Heap})
@findex nb_heap/1
@snindex nb_heap/1
@cnindex nb_heap/1
Create a @var{Heap} with default size @var{DefaultSize}. Note that size
will expand as needed.

@item nb_heap_close(+@var{Heap})
@findex nb_heap_close/1
@snindex nb_heap_close/1
@cnindex nb_heap_close/1
Close the heap @var{Heap}: no further elements can be added.

@item nb_heap_add(+@var{Heap}, +@var{Key}, +@var{Value})
@findex nb_heap_add/3
@snindex nb_heap_add/3
@cnindex nb_heap_add/3
Add @var{Key}-@var{Value} to the heap @var{Heap}. The key is sorted on
@var{Key} only.

@item nb_heap_del(+@var{Heap}, -@var{Key}, -@var{Value})
@findex nb_heap_del/3
@snindex nb_heap_del/3
@cnindex nb_heap_del/3
Remove element @var{Key}-@var{Value} with smallest @var{Value} in heap
@var{Heap}. Fail if the heap is empty.

@item nb_heap_peek(+@var{Heap}, -@var{Key}, -@var{Value}))
@findex nb_heap_peek/3
@snindex nb_heap_peek/3
@cnindex nb_heap_peek/3
@var{Key}-@var{Value} is the element with smallest @var{Key} in the heap
@var{Heap}. Fail if the heap is empty.

@item nb_heap_size(+@var{Heap}, -@var{Size})
@findex nb_heap_size/2
@snindex nb_heap_size/2
@cnindex nb_heap_size/2
Unify @var{Size} with the number of elements in the heap  @var{Heap}.

@item nb_heap_empty(+@var{Heap})
@findex nb_heap_empty/1
@snindex nb_heap_empty/1
@cnindex nb_heap_empty/1
Succeeds if  @var{Heap} is empty.

@item nb_beam(+@var{DefaultSize},-@var{Beam})
@findex nb_beam/1
@snindex nb_beam/1
@cnindex nb_beam/1
Create a @var{Beam} with default size @var{DefaultSize}. Note that size
is fixed throughout.

@item nb_beam_close(+@var{Beam})
@findex nb_beam_close/1
@snindex nb_beam_close/1
@cnindex nb_beam_close/1
Close the beam @var{Beam}: no further elements can be added.

@item nb_beam_add(+@var{Beam}, +@var{Key}, +@var{Value})
@findex nb_beam_add/3
@snindex nb_beam_add/3
@cnindex nb_beam_add/3
Add @var{Key}-@var{Value} to the beam @var{Beam}. The key is sorted on
@var{Key} only.

@item nb_beam_del(+@var{Beam}, -@var{Key}, -@var{Value})
@findex nb_beam_del/3
@snindex nb_beam_del/3
@cnindex nb_beam_del/3
Remove element @var{Key}-@var{Value} with smallest @var{Value} in beam
@var{Beam}. Fail if the beam is empty.

@item nb_beam_peek(+@var{Beam}, -@var{Key}, -@var{Value}))
@findex nb_beam_peek/3
@snindex nb_beam_peek/3
@cnindex nb_beam_peek/3
@var{Key}-@var{Value} is the element with smallest @var{Key} in the beam
@var{Beam}. Fail if the beam is empty.

@item nb_beam_size(+@var{Beam}, -@var{Size})
@findex nb_beam_size/2
@snindex nb_beam_size/2
@cnindex nb_beam_size/2
Unify @var{Size} with the number of elements in the beam  @var{Beam}.

@item nb_beam_empty(+@var{Beam})
@findex nb_beam_empty/1
@snindex nb_beam_empty/1
@cnindex nb_beam_empty/1
Succeeds if  @var{Beam} is empty.

@end table


@node Ordered Sets, Pseudo Random, Non-Backtrackable Data Structures, Library
@section Ordered Sets
@cindex ordered set

The following ordered set manipulation routines are available once
included with the @code{use_module(library(ordsets))} command.  An
ordered set is represented by a list having unique and ordered
elements. Output arguments are guaranteed to be ordered sets, if the
relevant inputs are. This is a slightly patched version of Richard
O'Keefe's original library.

@table @code
@item list_to_ord_set(+@var{List}, ?@var{Set})
@findex list_to_ord_set/2
@syindex list_to_ord_set/2
@cnindex list_to_ord_set/2
Holds when @var{Set} is the ordered representation of the set
represented by the unordered representation @var{List}.

@item merge(+@var{List1}, +@var{List2}, -@var{Merged})
@findex merge/3
@syindex merge/3
@cnindex merge/3
Holds when @var{Merged} is the stable merge of the two given lists.

Notice that @code{merge/3} will not remove duplicates, so merging
ordered sets will not necessarily result in an ordered set. Use
@code{ord_union/3} instead.

@item ord_add_element(+@var{Set1}, +@var{Element}, ?@var{Set2})
@findex ord_add_element/3
@syindex ord_add_element/3
@cnindex ord_add_element/3
Inserting @var{Element} in @var{Set1} returns @var{Set2}.  It should give
exactly the same result as @code{merge(Set1, [Element], Set2)}, but a
bit faster, and certainly more clearly. The same as @code{ord_insert/3}.

@item ord_del_element(+@var{Set1}, +@var{Element}, ?@var{Set2})
@findex ord_del_element/3
@syindex ord_del_element/3
@cnindex ord_del_element/3
Removing @var{Element} from @var{Set1} returns @var{Set2}.

@item ord_disjoint(+@var{Set1}, +@var{Set2})
@findex ord_disjoint/2
@syindex ord_disjoint/2
@cnindex ord_disjoint/2
Holds when the two ordered sets have no element in common.

@item ord_member(+@var{Element}, +@var{Set})
@findex ord_member/2
@syindex ord_member/2
@cnindex ord_member/2
Holds when @var{Element} is a member of @var{Set}.

@item ord_insert(+@var{Set1}, +@var{Element}, ?@var{Set2})
@findex ord_insert/3
@syindex ord_insert/3
@cnindex ord_insert/3
Inserting @var{Element} in @var{Set1} returns @var{Set2}.  It should give
exactly the same result as @code{merge(Set1, [Element], Set2)}, but a
bit faster, and certainly more clearly. The same as @code{ord_add_element/3}.

@item ord_intersect(+@var{Set1}, +@var{Set2})
@findex ord_intersect/2
@syindex ord_intersect/2
@cnindex ord_intersect/2
Holds when the two ordered sets have at least one element in common.

@item ord_intersection(+@var{Set1}, +@var{Set2}, ?@var{Intersection})
@findex ord_intersect/3
@syindex ord_intersect/3
@cnindex ord_intersect/3
Holds when Intersection is the ordered representation of @var{Set1}
and @var{Set2}.

@item ord_intersection(+@var{Set1}, +@var{Set2}, ?@var{Intersection}, ?@var{Diff})
@findex ord_intersect/4
@syindex ord_intersect/4
@cnindex ord_intersect/4
Holds when Intersection is the ordered representation of @var{Set1}
and @var{Set2}. @var{Diff} is the difference between @var{Set2} and @var{Set1}.

@item ord_seteq(+@var{Set1}, +@var{Set2})
@findex ord_seteq/2
@syindex ord_seteq/2
@cnindex ord_seteq/2
Holds when the two arguments represent the same set.

@item ord_setproduct(+@var{Set1}, +@var{Set2}, -@var{Set})
@findex ord_setproduct/3
@syindex ord_setproduct/3
@cnindex ord_setproduct/3
If Set1 and Set2 are ordered sets, Product will be an ordered
set of x1-x2 pairs.

@item ord_subset(+@var{Set1}, +@var{Set2})
@findex ordsubset/2
@syindex ordsubset/2
@cnindex ordsubset/2
Holds when every element of the ordered set @var{Set1} appears in the
ordered set @var{Set2}.

@item ord_subtract(+@var{Set1}, +@var{Set2}, ?@var{Difference})
@findex ord_subtract/3
@syindex ord_subtract/3
@cnindex ord_subtract/3
Holds when @var{Difference} contains all and only the elements of @var{Set1}
which are not also in @var{Set2}.

@item ord_symdiff(+@var{Set1}, +@var{Set2}, ?@var{Difference})
@findex ord_symdiff/3
@syindex ord_symdiff/3
@cnindex ord_symdiff/3
Holds when @var{Difference} is the symmetric difference of @var{Set1}
and @var{Set2}.

@item ord_union(+@var{Sets}, ?@var{Union})
@findex ord_union/2
@syindex ord_union/2
@cnindex ord_union/2
Holds when @var{Union} is the union of the lists @var{Sets}.

@item ord_union(+@var{Set1}, +@var{Set2}, ?@var{Union})
@findex ord_union/3
@syindex ord_union/3
@cnindex ord_union/3
Holds when @var{Union} is the union of @var{Set1} and @var{Set2}.

@item ord_union(+@var{Set1}, +@var{Set2}, ?@var{Union}, ?@var{Diff})
@findex ord_union/4
@syindex ord_union/4
@cnindex ord_union/4
Holds when @var{Union} is the union of @var{Set1} and @var{Set2} and
@var{Diff} is the difference.

@end table

@node Pseudo Random, Queues, Ordered Sets, Library
@section Pseudo Random Number Integer Generator
@cindex pseudo random

The following routines produce random non-negative integers in the range
0 .. 2^(w-1) -1, where w is the word size available for integers, e.g.
32 for Intel machines and 64 for Alpha machines. Note that the numbers
generated by this random number generator are repeatable. This generator
was originally written by Allen Van Gelder and is based on Knuth Vol 2.

@table @code 
@item rannum(-@var{I})
@findex rannum/1
@snindex rannum/1
@cnindex rannum/1
Produces a random non-negative integer @var{I} whose low bits are not
all that random, so it should be scaled to a smaller range in general.
The integer @var{I} is in the range 0 .. 2^(w-1) - 1. You can use:
@example
rannum(X) :- yap_flag(max_integer,MI), rannum(R), X is R/MI.
@end example
to obtain a floating point number uniformly distributed between 0 and 1.

@item ranstart
@findex ranstart/0
@snindex ranstart/0
@cnindex ranstart/0
Initialize the random number generator using a built-in seed. The
@code{ranstart/0} built-in is always called by the system when loading
the package.

@item ranstart(+@var{Seed})
@findex ranstart/1
@snindex ranstart/1
@cnindex ranstart/1
Initialize the random number generator with user-defined @var{Seed}. The
same @var{Seed} always produces the same sequence of numbers.

@item ranunif(+@var{Range},-@var{I})
@findex ranunif/2
@snindex ranunif/2
@cnindex ranunif/2
@code{ranunif/2} produces a uniformly distributed non-negative random
integer @var{I} over a caller-specified range @var{R}.  If range is @var{R},
the result is in 0 .. @var{R}-1.

@end table

@node Queues, Random, Pseudo Random, Library
@section Queues
@cindex queue

The following queue manipulation routines are available once
included with the @code{use_module(library(queues))} command. Queues are
implemented with difference lists.

@table @code

@item make_queue(+@var{Queue})
@findex make_queue/1
@syindex make_queue/1
@cnindex make_queue/1
Creates a new empty queue. It should only be used to create a new queue.

@item join_queue(+@var{Element}, +@var{OldQueue}, -@var{NewQueue})
@findex join_queue/3
@syindex join_queue/3
@cnindex join_queue/3
Adds the new element at the end of the queue.

@item list_join_queue(+@var{List}, +@var{OldQueue}, -@var{NewQueue})
@findex list_join_queue/3
@syindex list_join_queue/3
@cnindex list_join_queue/3
Ads the new elements at the end of the queue.

@item jump_queue(+@var{Element}, +@var{OldQueue}, -@var{NewQueue})
@findex jump_queue/3
@syindex jump_queue/3
@cnindex jump_queue/3
Adds the new element at the front of the list.

@item list_jump_queue(+@var{List}, +@var{OldQueue}, +@var{NewQueue})
@findex list_jump_queue/3
@syindex list_jump_queue/3
@cnindex list_jump_queue/3
Adds all the elements of @var{List} at the front of the queue.

@item head_queue(+@var{Queue}, ?@var{Head})
@findex head_queue/2
@syindex head_queue/2
@cnindex head_queue/2
Unifies Head with the first element of the queue.

@item serve_queue(+@var{OldQueue}, +@var{Head}, -@var{NewQueue})
@findex serve_queue/3
@syindex serve_queue/3
@cnindex serve_queue/3
Removes the first element of the queue for service.

@item empty_queue(+@var{Queue})
@findex empty_queue/1
@syindex empty_queue/1
@cnindex empty_queue/1
Tests whether the queue is empty.

@item length_queue(+@var{Queue}, -@var{Length})
@findex length_queue/2
@syindex length_queue/2
@cnindex length_queue/2
Counts the number of elements currently in the queue.

@item list_to_queue(+@var{List}, -@var{Queue})
@findex list_to_queue/2
@syindex list_to_queue/2
@cnindex list_to_queue/2
Creates a new queue with the same elements as @var{List.}

@item queue_to_list(+@var{Queue}, -@var{List})
@findex queue_to_list/2
@syindex queue_to_list/2
@cnindex queue_to_list/2
Creates a new list with the same elements as @var{Queue}.

@end table


@node Random, Read Utilities, Queues, Library
@section Random Number Generator
@cindex queue

The following random number operations are included with the
@code{use_module(library(random))} command. Since YAP-4.3.19 YAP uses
the O'Keefe public-domain algorithm, based on the "Applied Statistics"
algorithm AS183.

@table @code

@item getrand(-@var{Key})
@findex getrand/1
@syindex getrand/1
@cnindex getrand/1
Unify @var{Key} with a term of the form @code{rand(X,Y,Z)} describing the
current state of the random number generator.

@item random(-@var{Number})
@findex random/1
@syindex random/1
@cnindex random/1
Unify @var{Number} with a floating-point number in the range @code{[0...1)}.

@item random(+@var{LOW}, +@var{HIGH}, -@var{NUMBER})
@findex random/3
@syindex random/3
@cnindex random/3
Unify @var{Number} with a number in the range
@code{[LOW...HIGH)}. If both @var{LOW} and @var{HIGH} are
integers then @var{NUMBER} will also be an integer, otherwise
@var{NUMBER} will be a floating-point number.

@item randseq(+@var{LENGTH}, +@var{MAX}, -@var{Numbers})
@findex randseq/3
@syindex randseq/3
@cnindex randseq/3
Unify @var{Numbers} with a list of @var{LENGTH} unique random integers
in the range @code{[1�...@var{MAX})}.

@item randset(+@var{LENGTH}, +@var{MAX}, -@var{Numbers})
@findex randset/3
@syindex randset/3
@cnindex randset/3
Unify @var{Numbers} with an ordered list of @var{LENGTH} unique random
integers in the range @code{[1�...@var{MAX})}.

@item setrand(+@var{Key})
@findex setrand/1
@syindex setrand/1
@cnindex setrand/1
Use a term of the form @code{rand(X,Y,Z)} to set a new state for the
random number generator. The integer @code{X} must be in the range
@code{[1...30269)}, the integer @code{Y} must be in the range
@code{[1...30307)}, and the integer @code{Z} must be in the range
@code{[1...30323)}.

@end table

@node Read Utilities, Red-Black Trees, Random, Library
@section Read Utilities

The @code{readutil} library contains primitives to read lines, files,
multiple terms, etc.

@table @code
@item read_line_to_codes(+@var{Stream}, -@var{Codes})
@findex read_line_to_codes/2
@snindex read_line_to_codes/2
@cnindex read_line_to_codes/2

Read the next line of input from @var{Stream} and unify the result with
@var{Codes} @emph{after} the line has been read.  A line is ended by a
newline character or end-of-file. Unlike @code{read_line_to_codes/3},
this predicate removes trailing newline character.

On end-of-file the atom @code{end_of_file} is returned.  See also
@code{at_end_of_stream/[0,1]}.

@item read_line_to_codes(+@var{Stream}, -@var{Codes}, ?@var{Tail})
@findex read_line_to_codes/3
@snindex read_line_to_codes/3
@cnindex read_line_to_codes/3
Difference-list version to read an input line to a list of character
codes.  Reading stops at the newline or end-of-file character, but
unlike @code{read_line_to_codes/2}, the newline is retained in the
output.  This predicate is especially useful for reading a block of
lines upto some delimiter.  The following example reads an HTTP header
ended by a blank line:

@example
read_header_data(Stream, Header) :-
	read_line_to_codes(Stream, Header, Tail),
	read_header_data(Header, Stream, Tail).

read_header_data("\r\n", _, _) :- !.
read_header_data("\n", _, _) :- !.
read_header_data("", _, _) :- !.
read_header_data(_, Stream, Tail) :-
	read_line_to_codes(Stream, Tail, NewTail),
	read_header_data(Tail, Stream, NewTail).
@end example

@item read_stream_to_codes(+@var{Stream}, -@var{Codes})
@findex read_stream_to_codes/3
@snindex read_stream_to_codes/3
@cnindex read_stream_to_codes/3
Read all input until end-of-file and unify the result to @var{Codes}.

@item read_stream_to_codes(+@var{Stream}, -@var{Codes}, ?@var{Tail})
@findex read_stream_to_codes/3
@snindex read_stream_to_codes/3
@cnindex read_stream_to_codes/3
Difference-list version of @code{read_stream_to_codes/2}.

@item read_file_to_codes(+@var{Spec}, -@var{Codes}, +@var{Options})
@findex read_file_to_codes/3
@snindex read_file_to_codes/3
@cnindex read_file_to_codes/3
Read a file to a list of character codes. Currently ignores
@var{Options}.

@c  @var{Spec} is a
@c file-specification for absolute_file_name/3.  @var{Codes} is the
@c resulting code-list.  @var{Options} is a list of options for
@c absolute_file_name/3 and open/4.  In addition, the option
@c \term{tail}{Tail} is defined, forming a difference-list.

@item read_file_to_terms(+@var{Spec}, -@var{Terms}, +@var{Options})
@findex read_file_to_terms/3
@snindex read_file_to_terms/3
@cnindex read_file_to_terms/3
Read a file to a list of Prolog terms (see read/1). @c @var{Spec} is a
@c file-specification for absolute_file_name/3.  @var{Terms} is the
@c resulting list of Prolog terms.  @var{Options} is a list of options for
@c absolute_file_name/3 and open/4.  In addition, the option
@c \term{tail}{Tail} is defined, forming a difference-list.
@c \end{description}

@end table



@node Red-Black Trees, RegExp, Read Utilities, Library
@section Red-Black Trees
@cindex Red-Black Trees

Red-Black trees are balanced search binary trees. They are named because
nodes can be classified as either red or black. The code we include is
based on "Introduction to Algorithms", second edition, by Cormen,
Leiserson, Rivest and Stein.  The library includes routines to insert,
lookup and delete elements in the tree.

@table @code
@item rb_new(?@var{T})
@findex rb_new/1
@snindex rb_new/1
@cnindex rb_new/1
Create a new tree.

@item rb_empty(?@var{T})
@findex rb_empty/1
@snindex rb_empty/1
@cnindex rb_empty/1
Succeeds if tree @var{T} is empty.

@item is_rbtree(+@var{T})
@findex is_rbtree/1
@snindex is_rbtree/1
@cnindex is_rbtree/1
Check whether @var{T} is a valid red-black tree.

@item rb_insert(+@var{T0},+@var{Key},?@var{Value},+@var{TF})
@findex rb_insert/4
@snindex rb_insert/4
@cnindex rb_insert/4
Add an element with key @var{Key} and @var{Value} to the tree
@var{T0} creating a new red-black tree @var{TF}. Duplicated elements are not
allowed.

@snindex rb_insert_new/4
@cnindex rb_insert_new/4
Add a new element with key @var{Key} and @var{Value} to the tree
@var{T0} creating a new red-black tree @var{TF}. Fails is an element
with @var{Key} exists in the tree.

@item rb_lookup(+@var{Key},-@var{Value},+@var{T})
@findex rb_lookup/3
@snindex rb_lookup/3
@cnindex rb_lookup/3
Backtrack through all elements with key @var{Key} in the red-black tree
@var{T}, returning for each the value @var{Value}.

@item rb_lookupall(+@var{Key},-@var{Value},+@var{T})
@findex rb_lookupall/3
@snindex rb_lookupall/3
@cnindex rb_lookupall/3
Lookup all elements with key @var{Key} in the red-black tree
@var{T}, returning the value @var{Value}.

@item rb_delete(+@var{T},+@var{Key},-@var{TN})
@findex rb_delete/3
@snindex rb_delete/3
@cnindex rb_delete/3
Delete element with key @var{Key} from the tree @var{T}, returning a new
tree @var{TN}.

@item rb_delete(+@var{T},+@var{Key},-@var{Val},-@var{TN})
@findex rb_delete/4
@snindex rb_delete/4
@cnindex rb_delete/4
Delete element with key @var{Key} from the tree @var{T}, returning the
value @var{Val} associated with the key and a new tree @var{TN}.

@item rb_del_min(+@var{T},-@var{Key},-@var{Val},-@var{TN})
@findex rb_del_min/4
@snindex rb_del_min/4
@cnindex rb_del_min/4
Delete the least element from the tree @var{T}, returning the key
@var{Key}, the value @var{Val} associated with the key and a new tree
@var{TN}.

@item rb_del_max(+@var{T},-@var{Key},-@var{Val},-@var{TN})
@findex rb_del_max/4
@snindex rb_del_max/4
@cnindex rb_del_max/4
Delete the largest element from the tree @var{T}, returning the key
@var{Key}, the value @var{Val} associated with the key and a new tree
@var{TN}.

@item rb_update(+@var{T},+@var{Key},+@var{NewVal},-@var{TN})
@findex rb_update/4
@snindex rb_update/4
@cnindex rb_update/4
Tree @var{TN} is tree @var{T}, but with value for @var{Key} associated
with @var{NewVal}. Fails if it cannot find @var{Key} in @var{T}.

@item rb_apply(+@var{T},+@var{Key},+@var{G},-@var{TN})
@findex rb_apply/4
@snindex rb_apply/4
@cnindex rb_apply/4
If the value associated with key @var{Key} is @var{Val0} in @var{T}, and
if @code{call(G,Val0,ValF)} holds, then @var{TN} differs from
@var{T} only in that @var{Key} is associated with value @var{ValF} in
tree @var{TN}. Fails if it cannot find @var{Key} in @var{T}, or if
@code{call(G,Val0,ValF)} is not satisfiable.

@item rb_visit(+@var{T},-@var{Pairs})
@findex rb_visit/2
@snindex rb_visit/2
@cnindex rb_visit/2
@var{Pairs} is an infix visit of tree @var{T}, where each element of
@var{Pairs} is of the form  @var{K}-@var{Val}.

@item rb_size(+@var{T},-@var{Size})
@findex rb_size/2
@snindex rb_size/2
@cnindex rb_size/2
@var{Size} is the number of elements in @var{T}.

@item rb_keys(+@var{T},+@var{Keys})
@findex rb_keys/2
@snindex rb_keys/2
@cnindex rb_keys/2
@var{Keys} is an infix visit with all keys in tree @var{T}. Keys will be
sorted, but may be duplicate.

@item rb_map(+@var{T},+@var{G},-@var{TN})
@findex rb_map/3
@snindex rb_map/3
@cnindex rb_map/3
For all nodes @var{Key} in the tree @var{T}, if the value associated with
key @var{Key} is @var{Val0} in tree @var{T}, and if
@code{call(G,Val0,ValF)} holds, then the value associated with @var{Key}
in @var{TN} is @var{ValF}. Fails if or if @code{call(G,Val0,ValF)} is not
satisfiable for all @var{Var0}.

@item rb_partial_map(+@var{T},+@var{Keys},+@var{G},-@var{TN})
@findex rb_partial_map/4
@snindex rb_partial_map/4
@cnindex rb_partial_map/4
For all nodes @var{Key} in @var{Keys}, if the value associated with key
@var{Key} is @var{Val0} in tree @var{T}, and if @code{call(G,Val0,ValF)}
holds, then the value associated with @var{Key} in @var{TN} is
@var{ValF}. Fails if or if @code{call(G,Val0,ValF)} is not satisfiable
for all @var{Var0}. Assumes keys are not repeated.

@item rb_clone(+@var{T},+@var{NT},+@var{Nodes})
@findex rb_clone/3
@snindex rb_clone/3
@cnindex rb_clone/3
``Clone'' the red-back tree into a new tree with the same keys as the
original but with all values set to unbound values. Nodes is a list
containing all new nodes as pairs @var{K-V}.

@item rb_min(+@var{T},-@var{Key},-@var{Value})
@findex rb_min/3
@snindex rb_min/3
@cnindex rb_min/3
@var{Key}  is the minimum key in @var{T}, and is associated with @var{Val}.

@item rb_max(+@var{T},-@var{Key},-@var{Value})
@findex rb_max/3
@snindex rb_max/3
@cnindex rb_max/3
@var{Key}  is the maximal key in @var{T}, and is associated with @var{Val}.

@item rb_next(+@var{T}, +@var{Key},-@var{Next},-@var{Value})
@findex rb_next/4
@snindex rb_next/4
@cnindex rb_next/4
@var{Next} is the next element after @var{Key} in @var{T}, and is
associated with @var{Val}.

@item rb_previous(+@var{T}, +@var{Key},-@var{Previous},-@var{Value})
@findex rb_previous/4
@snindex rb_previous/4
@cnindex rb_previous/4
@var{Previous} is the previous element after @var{Key} in @var{T}, and is
associated with @var{Val}.

@item list_to_rbtree(+@var{L}, -@var{T})
@findex list_to_rbtree/2
@snindex list_to_rbtree/2
@cnindex list_to_rbtree/2
@var{T} is the red-black tree corresponding to the mapping in list @var{L}.

@item ord_list_to_rbtree(+@var{L}, -@var{T})
@findex list_to_rbtree/2
@snindex list_to_rbtree/2
@cnindex list_to_rbtree/2
@var{T} is the red-black tree corresponding to the mapping in ordered
list @var{L}.
@end table

@node RegExp, shlib, Red-Black Trees, Library
@section Regular Expressions
@cindex regular expressions

This library includes routines to determine whether a regular expression
matches part or all of a string. The routines can also return which
parts parts of the string matched the expression or subexpressions of
it. This library relies on Henry Spencer's @code{C}-package and is only
available in operating systems that support dynamic loading. The
@code{C}-code has been obtained from the sources of FreeBSD-4.0 and is
protected by copyright from Henry Spencer and from the Regents of the
University of California (see the file library/regex/COPYRIGHT for
further details).

Much of the description of regular expressions below is copied verbatim
from Henry Spencer's manual page.

A regular expression is zero or more branches, separated by ``|''.  It
matches anything that matches one of the branches.

A branch is zero or more pieces, concatenated.  It matches a match for
the first, followed by a match for the second, etc.

A piece is an atom possibly followed by ``*'', ``+'', or ``?''.  An atom
followed by ``*'' matches a sequence of 0 or more matches of the atom.
An atom followed by ``+'' matches a sequence of 1 or more matches of the
atom.  An atom followed by ``?'' matches a match of the atom, or the
null string.

An atom is a regular expression in parentheses (matching a match for the
regular expression), a range (see below), ``.''  (matching any single
character), ``^'' (matching the null string at the beginning of the
input string), ``$'' (matching the null string at the end of the input
string), a ``\'' followed by a single character (matching that
character), or a single character with no other significance (matching
that character).

A range is a sequence of characters enclosed in ``[]''.  It normally
matches any single character from the sequence.  If the sequence begins
with ``^'', it matches any single character not from the rest of the
sequence.  If two characters in the sequence are separated by ``-'',
this is shorthand for the full list of ASCII characters between them
(e.g. ``[0-9]'' matches any decimal digit).  To include a literal ``]''
in the sequence, make it the first character (following a possible
``^'').  To include a literal ``-'', make it the first or last
character.

@table @code

@item regexp(+@var{RegExp},+@var{String},+@var{Opts})
@findex regexp/3
@snindex regexp/3
@cnindex regexp/3

Match regular expression @var{RegExp} to input string @var{String}
according to options @var{Opts}. The options may be:
@itemize @bullet
@item @code{nocase}: Causes upper-case characters  in  @var{String} to
        be treated  as  lower case during the matching process.
@end itemize

@item regexp(+@var{RegExp},+@var{String},+@var{Opts},?@var{SubMatchVars})
@findex regexp/4
@snindex regexp/4
@cnindex regexp/4

Match regular expression @var{RegExp} to input string @var{String}
according to options @var{Opts}. The variable @var{SubMatchVars} should
be originally unbound or a list of unbound variables all will contain a
sequence of matches, that is, the head of @var{SubMatchVars} will
contain the characters in @var{String} that matched the leftmost
parenthesized subexpression within @var{RegExp}, the next head of list
will contain the characters that matched the next parenthesized
subexpression to the right in @var{RegExp}, and so on.

The options may be:
@itemize @bullet
@item @code{nocase}: Causes upper-case characters  in  @var{String} to
        be treated  as  lower case during the matching process.
@item @code{indices}: Changes what  is  stored  in
@var{SubMatchVars}. Instead  of storing the matching characters from
@var{String}, each variable will contain a term of the form @var{IO-IF}
giving the indices in @var{String} of the first and last characters  in
the  matching range of characters.

@end itemize

In general there may be more than one way to match a regular expression
to an input string.  For example,  consider the command
@example
  regexp("(a*)b*","aabaaabb", [], [X,Y])
@end example
Considering only the rules given so far, @var{X} and @var{Y} could end up
with the values @code{"aabb"} and @code{"aa"}, @code{"aaab"} and
@code{"aaa"}, @code{"ab"} and @code{"a"}, or any of several other
combinations.  To resolve this potential ambiguity @code{regexp} chooses among
alternatives using the rule ``first then longest''.  In other words, it
considers the possible matches in order working from left to right
across the input string and the pattern, and it attempts to match longer
pieces of the input string before shorter ones.  More specifically, the
following rules apply in decreasing order of priority:


@enumerate 
@item    If a regular expression could match  two  different parts of an
input string then it will match the one that begins earliest.

@item  If a regular expression contains "|"  operators  then the leftmost matching sub-expression is chosen.

@item In *, +, and ? constructs, longer matches are chosen in preference to shorter ones.

@item In sequences of expression  components  the  components are considered from left to right.
@end enumerate

In the example from above, @code{"(a*)b*"} matches @code{"aab"}: the
@code{"(a*)"} portion of the pattern is matched first and it consumes
the leading @code{"aa"}; then the @code{"b*"} portion of the pattern
consumes the next @code{"b"}.  Or, consider the following example: 
@example
  regexp("(ab|a)(b*)c",  "abc", [], [X,Y,Z])
@end example

After this command @var{X} will be @code{"abc"}, @var{Y} will be
@code{"ab"}, and @var{Z} will be an empty string.  Rule 4 specifies that
@code{"(ab|a)"} gets first shot at the input string and Rule 2 specifies
that the @code{"ab"} sub-expression is checked before the @code{"a"}
sub-expression.  Thus the @code{"b"} has already been claimed before the
@code{"(b*)"} component is checked and @code{(b*)} must match an empty string.

@end table

@node shlib, Splay Trees, RegExp, Library
@section SWI-Prolog's shlib library

@cindex SWI-Compatible foreign file loading
This section discusses the functionality of the (autoload)
@code{library(shlib)}, providing an interface to manage shared
libraries.

One of the files provides a global function @code{install_mylib()} that
initialises the module using calls to @code{PL_register_foreign()}. Here is a
simple example file @code{mylib.c}, which creates a Windows MessageBox:

@example
#include <windows.h>
#include <SWI-Prolog.h>

static foreign_t
pl_say_hello(term_t to)
@{ char *a;

  if ( PL_get_atom_chars(to, &a) )
  @{ MessageBox(NULL, a, "DLL test", MB_OK|MB_TASKMODAL);

    PL_succeed;
  @}

  PL_fail;
@}

install_t
install_mylib()
@{ PL_register_foreign("say_hello", 1, pl_say_hello, 0);
@}
@end example

Now write a file mylib.pl:

@example
:- module(mylib, [ say_hello/1 ]).
:- use_foreign_library(foreign(mylib)).
@end example

The file mylib.pl can be loaded as a normal Prolog file and provides the predicate defined in C.

@table @code
@item [det]load_foreign_library(:@var{FileSpec})
@item [det]load_foreign_library(:@var{FileSpec}, +@var{Entry}:atom)
@findex load_foreign_library/1
@snindex load_foreign_library/1
@cnindex load_foreign_library/1
@findex load_foreign_library/2
@snindex load_foreign_library/2
@cnindex load_foreign_library/2
    Load a shared object or DLL. After loading the @var{Entry} function is
    called without arguments. The default entry function is composed
    from @code{install_}, followed by the file base-name. E.g., the
    load-call below calls the function @code{install_mylib()}. If the platform
    prefixes extern functions with @code{_}, this prefix is added before
    calling.

@example
          ...
          load_foreign_library(foreign(mylib)),
          ...
@end example

    @var{FileSpec} is a specification for
    @code{absolute_file_name/3}. If searching the file fails, the plain
    name is passed to the OS to try the default method of the OS for
    locating foreign objects. The default definition of
    @code{file_search_path/2} searches <prolog home>/lib/Yap.

    See also
        @code{use_foreign_library/1,2} are intended for use in
        directives. 

@item [det]use_foreign_library(+@var{FileSpec})
@item [det]use_foreign_library(+@var{FileSpec}, +@var{Entry}:atom)
@findex use_foreign_library/1
@snindex use_foreign_library/1
@cnindex use_foreign_library/1
@findex use_foreign_library/2
@snindex use_foreign_library/2
@cnindex use_foreign_library/2
    Load and install a foreign library as load_foreign_library/1,2 and
    register the installation using @code{initialization/2} with the option
    now. This is similar to using:

@example
    :- initialization(load_foreign_library(foreign(mylib))).
@end example

    but using the @code{initialization/1} wrapper causes the library to
    be loaded after loading of the file in which it appears is
    completed, while @code{use_foreign_library/1} loads the library
    immediately. I.e. the difference is only relevant if the remainder
    of the file uses functionality of the @code{C}-library. 

@item [det]unload_foreign_library(+@var{FileSpec})
@item [det]unload_foreign_library(+@var{FileSpec}, +@var{Exit}:atom) 
@findex unload_foreign_library/1
@snindex unload_foreign_library/1
@cnindex unload_foreign_library/1
@findex unload_foreign_library/2
@snindex unload_foreign_library/2
@cnindex unload_foreign_library/2

Unload a shared
object or DLL. After calling the @var{Exit} function, the shared object is
removed from the process. The default exit function is composed from
@code{uninstall_}, followed by the file base-name.

@item current_foreign_library(?@var{File}, ?@var{Public}) 
@findex current_foreign_library/2
@snindex current_foreign_library/2
@cnindex current_foreign_library/2

Query currently
loaded shared libraries.  

@c @item reload_foreign_libraries 
@c @findex reload_foreign_libraries/0
@c @snindex reload_foreign_libraries/0
@c @cnindex reload_foreign_libraries/0
@c Reload all foreign
@c libraries loaded (after restore of a state created using
@c @code{qsave_program/2}).
@end table

@node Splay Trees, String I/O, shlib, Library
@section Splay Trees
@cindex splay trees

Splay trees are explained in the paper "Self-adjusting Binary Search
Trees", by D.D. Sleator and R.E. Tarjan, JACM, vol. 32, No.3, July 1985,
p. 668. They are designed to support fast insertions, deletions and
removals in binary search trees without the complexity of traditional
balanced trees. The key idea is to allow the tree to become
unbalanced. To make up for this, whenever we find a node, we move it up
to the top. We use code by Vijay Saraswat originally posted to the Prolog
mailing-list.

@table @code

@item splay_access(-@var{Return},+@var{Key},?@var{Val},+@var{Tree},-@var{NewTree})
@findex splay_access/5
@snindex splay_access/5
@cnindex splay_access/5
If item @var{Key} is in tree @var{Tree}, return its @var{Val} and
unify @var{Return} with @code{true}. Otherwise unify @var{Return} with
@code{null}. The variable @var{NewTree} unifies with the new tree.

@item splay_delete(+@var{Key},?@var{Val},+@var{Tree},-@var{NewTree})
@findex splay_delete/4
@snindex splay_delete/4
@cnindex splay_delete/4
Delete item @var{Key} from tree @var{Tree}, assuming that it is present
already. The variable @var{Val} unifies with a value for key @var{Key},
and the variable @var{NewTree} unifies with the new tree. The predicate
will fail if @var{Key} is not present.

@item splay_init(-@var{NewTree})
@findex splay_init/3
@snindex splay_init/3
@cnindex splay_init/3
Initialize a new splay tree.

@item splay_insert(+@var{Key},?@var{Val},+@var{Tree},-@var{NewTree})
@findex splay_insert/4
@snindex splay_insert/4
@cnindex splay_insert/4
Insert item @var{Key} in tree @var{Tree}, assuming that it is not
there already. The variable @var{Val} unifies with a value for key
@var{Key}, and the variable @var{NewTree} unifies with the new
tree. In our implementation, @var{Key} is not inserted if it is
already there: rather it is unified with the item already in the tree.

@item splay_join(+@var{LeftTree},+@var{RighTree},-@var{NewTree})
@findex splay_join/3
@snindex splay_join/3
@cnindex splay_join/3
Combine trees @var{LeftTree} and @var{RighTree} into a single
tree@var{NewTree} containing all items from both trees. This operation
assumes that all items in @var{LeftTree} are less than all those in
@var{RighTree} and destroys both @var{LeftTree} and @var{RighTree}.

@item splay_split(+@var{Key},?@var{Val},+@var{Tree},-@var{LeftTree},-@var{RightTree})
@findex splay_split/5
@snindex splay_split/5
@cnindex splay_split/5
Construct and return two trees @var{LeftTree} and @var{RightTree},
where @var{LeftTree} contains all items in @var{Tree} less than
@var{Key}, and @var{RightTree} contains all items in @var{Tree}
greater than @var{Key}. This operations destroys @var{Tree}.

@end table

@node String I/O, System, Splay Trees, Library
@section Reading From and Writing To Strings
@cindex string I/O

From Version 4.3.2 onwards YAP implements SICStus Prolog compatible
String I/O. The library allows users to read from and write to a memory
buffer as if it was a file. The memory buffer is built from or converted
to a string of character codes by the routines in library. Therefore, if
one wants to read from a string the string must be fully instantiated
before the library built-in opens the string for reading. These commands
are available through the @code{use_module(library(charsio))} command.

@table @code

@item format_to_chars(+@var{Form}, +@var{Args}, -@var{Result})
@findex format_to_chars/3
@syindex format_to_chars/3
@cnindex format_to_chars/3

Execute the built-in procedure @code{format/2} with form @var{Form} and
arguments @var{Args} outputting the result to the string of character
codes @var{Result}.

@item format_to_chars(+@var{Form}, +@var{Args}, -@var{Result}, -@var{Result0})
@findex format_to_chars/4
@syindex format_to_chars/4
@cnindex format_to_chars/4

Execute the built-in procedure @code{format/2} with form @var{Form} and
arguments @var{Args} outputting the result to the difference list of
character codes @var{Result-Result0}.

@item write_to_chars(+@var{Term}, -@var{Result})
@findex write_to_chars/2
@syindex write_to_chars/2
@cnindex write_to_chars/2

Execute the built-in procedure @code{write/1} with argument @var{Term}
outputting the result to the string of character codes @var{Result}.

@item write_to_chars(+@var{Term}, -@var{Result0}, -@var{Result})
@findex write_to_chars/3
@syindex write_to_chars/3
@cnindex write_to_chars/3

Execute the built-in procedure @code{write/1} with argument @var{Term}
outputting the result to the difference list of character codes
@var{Result-Result0}.

@item atom_to_chars(+@var{Atom}, -@var{Result})
@findex atom_to_chars/2
@syindex atom_to_chars/2
@cnindex atom_to_chars/2

Convert the atom @var{Atom} to the string of character codes
@var{Result}.

@item atom_to_chars(+@var{Atom}, -@var{Result0}, -@var{Result})
@findex atom_to_chars/3
@syindex atom_to_chars/3
@cnindex atom_to_chars/3

Convert the atom @var{Atom} to the difference list of character codes
@var{Result-Result0}.

@item number_to_chars(+@var{Number}, -@var{Result})
@findex number_to_chars/2
@syindex number_to_chars/2
@cnindex number_to_chars/2

Convert the number @var{Number} to the string of character codes
@var{Result}.

@item number_to_chars(+@var{Number}, -@var{Result0}, -@var{Result})
@findex number_to_chars/3
@syindex number_to_chars/3
@cnindex number_to_chars/3

Convert the atom @var{Number} to the difference list of character codes
@var{Result-Result0}.

@item atom_to_term(+@var{Atom}, -@var{Term}, -@var{Bindings})
@findex atom_to_term/3
@syindex atom_to_term/3
@cnindex atom_to_term/3
Use @var{Atom} as input to @code{read_term/2} using the option @code{variable_names} and return the read term in @var{Term} and the variable bindings in @var{Bindings}. @var{Bindings} is a list of @code{Name = Var} couples, thus providing access to the actual variable names. See also @code{read_term/2}. If Atom has no valid syntax, a syntax_error exception is raised.

@item term_to_atom(?@var{Term}, ?@var{Atom})
@findex term_to_atom/2
@syindex term_to_atom/2
@cnindex term_to_atom/2
True if @var{Atom} describes a term that unifies with @var{Term}. When
@var{Atom} is instantiated @var{Atom} is converted and then unified with
@var{Term}. If @var{Atom} has no valid syntax, a syntax_error exception
is raised. Otherwise @var{Term} is ``written'' on @var{Atom} using
@code{write_term/2} with the option quoted(true).

@item read_from_chars(+@var{Chars}, -@var{Term})
@findex read_from_chars/2
@syindex read_from_chars/2
@cnindex read_from_chars/2

Parse the list of character codes @var{Chars} and return the result in
the term @var{Term}. The character codes to be read must terminate with
a dot character such that either (i) the dot character is followed by
blank characters; or (ii) the dot character is the last character in the
string.

@item open_chars_stream(+@var{Chars}, -@var{Stream})
@findex open_chars_stream/2
@syindex open_chars_stream/2
@cnindex open_chars_stream/2

Open the list of character codes @var{Chars} as a stream @var{Stream}.

@item with_output_to_chars(?@var{Goal}, -@var{Chars})
@findex with_output_to_chars/2
@syindex with_output_to_chars/2
@cnindex with_output_to_chars/2

Execute goal @var{Goal} such that its standard output will be sent to a
memory buffer. After successful execution the contents of the memory
buffer will be converted to the list of character codes @var{Chars}.

@item with_output_to_chars(?@var{Goal}, ?@var{Chars0}, -@var{Chars})
@findex with_output_to_chars/3
@syindex with_output_to_chars/3
@cnindex with_output_to_chars/3

Execute goal @var{Goal} such that its standard output will be sent to a
memory buffer. After successful execution the contents of the memory
buffer will be converted to the difference list of character codes
@var{Chars-Chars0}.

@item with_output_to_chars(?@var{Goal}, -@var{Stream}, ?@var{Chars0}, -@var{Chars})
@findex with_output_to_chars/4
@syindex with_output_to_chars/4
@cnindex with_output_to_chars/4

Execute goal @var{Goal} such that its standard output will be sent to a
memory buffer. After successful execution the contents of the memory
buffer will be converted to the difference list of character codes
@var{Chars-Chars0} and @var{Stream} receives the stream corresponding to
the memory buffer.

@end table

The implementation of the character IO operations relies on three YAP
built-ins:
@table @code

@item charsio:open_mem_read_stream(+@var{String}, -@var{Stream})
Store a string in a memory buffer and output a stream that reads from this
memory buffer.

@item charsio:open_mem_write_stream(-@var{Stream})
Create a new memory buffer and output a stream that writes to  it.

@item charsio:peek_mem_write_stream(-@var{Stream}, L0, L)
Convert the memory buffer associated with stream @var{Stream} to the
difference list of character codes @var{L-L0}.

@end table
@noindent
These built-ins are initialized to belong to the module @code{charsio} in
@code{init.yap}. Novel procedures for manipulating strings by explicitly
importing these built-ins.

YAP does not currently support opening a @code{charsio} stream in
@code{append} mode, or seeking in such a stream.

@node System, Terms, String I/O, Library
@section Calling The Operating System from YAP
@cindex Operating System Utilities

YAP now provides a library of system utilities compatible with the
SICStus Prolog system library. This library extends and to some point
replaces the functionality of Operating System access routines. The
library includes Unix/Linux and Win32 @code{C} code. They
are available through the @code{use_module(library(system))} command.

@table @code

@item datime(datime(-@var{Year}, -@var{Month}, -@var{DayOfTheMonth},
-@var{Hour}, -@var{Minute}, -@var{Second})
@findex  datime/1
@syindex datime/1
@cnindex datime/1
The @code{datime/1} procedure returns the current date and time, with
information on @var{Year}, @var{Month}, @var{DayOfTheMonth},
@var{Hour}, @var{Minute}, and @var{Second}. The @var{Hour} is returned
on local time. This function uses the WIN32
@code{GetLocalTime} function or the Unix @code{localtime} function.

@example
   ?- datime(X).

X = datime(2001,5,28,15,29,46) ? 
@end example

@item mktime(datime(+@var{Year}, +@var{Month}, +@var{DayOfTheMonth},
+@var{Hour}, +@var{Minute}, +@var{Second}), -@var{Seconds})
@findex  mktime/2
@snindex mktime/2
@cnindex mktime/2
The @code{mktime/1} procedure returns the number of @var{Seconds}
elapsed since 00:00:00 on January 1, 1970, Coordinated Universal Time
(UTC).  The user provides information on @var{Year}, @var{Month},
@var{DayOfTheMonth}, @var{Hour}, @var{Minute}, and @var{Second}. The
@var{Hour} is given on local time. This function uses the WIN32
@code{GetLocalTime} function or the Unix @code{mktime} function.

@example
   ?- mktime(datime(2001,5,28,15,29,46),X).

X = 991081786 ? ;
@end example

@item delete_file(+@var{File})
@findex  delete_file/1
@syindex delete_file/1
@cnindex delete_file/1
The @code{delete_file/1} procedure removes file @var{File}. If
@var{File} is a directory, remove the directory @emph{and all its
subdirectories}.

@example
   ?- delete_file(x).
@end example

@item delete_file(+@var{File},+@var{Opts})
@findex  delete_file/2
@syindex delete_file/2
@cnindex delete_file/2
The @code{delete_file/2} procedure removes file @var{File} according to
options @var{Opts}. These options are @code{directory} if one should
remove directories, @code{recursive} if one should remove directories
recursively, and @code{ignore} if errors are not to be reported.

This example is equivalent to using the @code{delete_file/1} predicate:
@example
   ?- delete_file(x, [recursive]).
@end example


@item directory_files(+@var{Dir},+@var{List})
@findex  directory_files/2
@syindex directory_files/2
@cnindex directory_files/2
Given a directory @var{Dir},  @code{directory_files/2} procedures a
listing of all files and directories in the directory:
@example
    ?- directory_files('.',L), writeq(L).
['Makefile.~1~','sys.so','Makefile','sys.o',x,..,'.']
@end example
The predicates uses the @code{dirent} family of routines in Unix
environments, and @code{findfirst} in WIN32.

@item file_exists(+@var{File})
@findex  file_exists/1
@syindex file_exists/1
@cnindex file_exists/1
The atom @var{File} corresponds to an existing file.

@item file_exists(+@var{File},+@var{Permissions})
@findex  file_exists/2
@syindex file_exists/2
@cnindex file_exists/2
The atom @var{File} corresponds to an existing file with permissions
compatible with @var{Permissions}. YAP currently only accepts for
permissions to be described as a number. The actual meaning of this
number is Operating System dependent.

@item file_property(+@var{File},?@var{Property})
@findex  file_property/2
@syindex file_property/2
@cnindex file_property/2
The atom @var{File} corresponds to an existing file, and @var{Property}
will be unified with a property of this file. The properties are of the
form @code{type(@var{Type})}, which gives whether the file is a regular
file, a directory, a fifo file, or of unknown type;
@code{size(@var{Size})}, with gives the size for a file, and
@code{mod_time(@var{Time})}, which gives the last time a file was
modified according to some Operating System dependent
timestamp; @code{mode(@var{mode})}, gives the permission flags for the
file, and @code{linkto(@var{FileName})}, gives the file pointed to by a
symbolic link. Properties can be obtained through backtracking:

@example
   ?- file_property('Makefile',P).

P = type(regular) ? ;

P = size(2375) ? ;

P = mod_time(990826911) ? ;

no
@end example

@item make_directory(+@var{Dir})
@findex  make_directory/2
@syindex make_directory/2
@cnindex make_directory/2
Create a directory @var{Dir}. The name of the directory must be an atom.

@item rename_file(+@var{OldFile},+@var{NewFile})
@findex  rename_file/2
@syindex rename_file/2
@cnindex rename_file/2
Create file @var{OldFile} to @var{NewFile}. This predicate uses the
@code{C} built-in function @code{rename}.


@item environ(?@var{EnvVar},+@var{EnvValue})
@findex  environ/2
@syindex environ/2
@cnindex environ/2
Unify environment variable @var{EnvVar} with its value @var{EnvValue},
if there is one. This predicate is backtrackable in Unix systems, but
not currently in Win32 configurations.

@example
   ?- environ('HOME',X).

X = 'C:\\cygwin\\home\\administrator' ?
@end example

@item host_id(-@var{Id})
@findex  host_id/1
@syindex host_id/1
@cnindex host_id/1

Unify @var{Id} with an identifier of the current host. YAP uses the
@code{hostid} function when available, 

@item host_name(-@var{Name})
@findex  host_name/1
@syindex host_name/1
@cnindex host_name/1

Unify @var{Name} with a name for the current host. YAP uses the
@code{hostname} function in Unix systems when available, and the
@code{GetComputerName} function in WIN32 systems. 

@item kill(@var{Id},+@var{SIGNAL})
@findex  kill/2
@syindex kill/2
@cnindex kill/2

Send signal @var{SIGNAL} to process @var{Id}. In Unix this predicate is
a direct interface to @code{kill} so one can send signals to groups of
processes. In WIN32 the predicate is an interface to
@code{TerminateProcess}, so it kills @var{Id} independently of @var{SIGNAL}.

@item mktemp(@var{Spec},-@var{File})
@findex  mktemp/2
@syindex mktemp/2
@cnindex mktemp/2

Direct interface to @code{mktemp}: given a @var{Spec}, that is a file
name with six @var{X} to it, create a file name @var{File}. Use
@code{tmpnam/1} instead.

@item pid(-@var{Id})
@findex  pid/1
@syindex pid/1
@cnindex pid/1

Unify @var{Id} with the process identifier for the current
process. An interface to the @t{getpid} function.

@item tmpnam(-@var{File})
@findex  tmpnam/1
@syindex tmpnam/1
@cnindex tmpnam/1

Interface with @var{tmpnam}: obtain a new, unique file name @var{File}.

@item tmp_file(-@var{File})
@findex  tmp_file/2
@snindex tmp_file/2
@cnindex tmp_file/2

Create a name for a temporary file. @var{Base} is an user provided
identifier for the category of file. The @var{TmpName} is guaranteed to
be unique. If the system halts, it will automatically remove all created
temporary files.


@item exec(+@var{Command},[+@var{InputStream},+@var{OutputStream},+@var{ErrorStream}],-@var{PID})
@findex  exec/3
@syindex exec/3
@cnindex exec/3
Execute command @var{Command} with its streams connected to
@var{InputStream}, @var{OutputStream}, and @var{ErrorStream}. The
process that executes the command is returned as @var{PID}. The
command is executed by the default shell @code{bin/sh -c} in Unix.

The following example demonstrates the use of @code{exec/3} to send a
command and process its output:

@example
exec(ls,[std,pipe(S),null],P),repeat, get0(S,C), (C = -1, close(S) ! ; put(C)).
@end example

The streams may be one of standard stream, @code{std}, null stream,
@code{null}, or @code{pipe(S)}, where @var{S} is a pipe stream. Note
that it is up to the user to close the pipe.

@item working_directory(-@var{CurDir},?@var{NextDir})
@findex  working_directory/2
@syindex working_directory/2
@cnindex working_directory/2    @c 
Fetch the current directory at @var{CurDir}. If @var{NextDir} is bound
to an atom, make its value the current working directory.

@item popen(+@var{Command}, +@var{TYPE}, -@var{Stream})
@findex  popen/3
@syindex popen/3
@cnindex popen/3
Interface to the @t{popen} function. It opens a process by creating a
pipe, forking and invoking @var{Command} on the current shell. Since a
pipe is by definition unidirectional the @var{Type} argument may be
@code{read} or @code{write}, not both. The stream should be closed
using @code{close/1}, there is no need for a special @code{pclose}
command.

The following example demonstrates the use of @code{popen/3} to process
the output of a command, as @code{exec/3} would do:

@example
   ?- popen(ls,read,X),repeat, get0(X,C), (C = -1, ! ; put(C)).

X = 'C:\\cygwin\\home\\administrator' ?
@end example


The WIN32 implementation of @code{popen/3} relies on @code{exec/3}.

@item shell
@findex  shell/0
@syindex shell/0
@cnindex shell/0
Start a new shell and leave YAP in background until the shell
completes. YAP uses the shell given by the environment variable
@code{SHELL}. In WIN32 environment YAP will use @code{COMSPEC} if
@code{SHELL} is undefined.

@item shell(+@var{Command})
@findex  shell/1
@syindex shell/1
@cnindex shell/1
Execute command @var{Command} under a new shell. YAP will be in
background until the command completes. In Unix environments YAP uses
the shell given by the environment variable @code{SHELL} with the option
@code{" -c "}. In WIN32 environment YAP will use @code{COMSPEC} if
@code{SHELL} is undefined, in this case with the option @code{" /c "}.

@item shell(+@var{Command},-@var{Status})
@findex  shell/1
@syindex shell/1
@cnindex shell/1
Execute command @var{Command} under a new shell and unify @var{Status}
with the exit for the command. YAP will be in background until the
command completes. In Unix environments YAP uses the shell given by the
environment variable @code{SHELL} with the option @code{" -c "}. In
WIN32 environment YAP will use @code{COMSPEC} if @code{SHELL} is
undefined, in this case with the option @code{" /c "}.

@item sleep(+@var{Time})
@findex  sleep/1
@syindex sleep/1
@cnindex sleep/1
Block the current thread for @var{Time} seconds. When YAP is compiled 
without multi-threading support, this predicate blocks the YAP process. 
The number of seconds must be a positive number, and it may an integer 
or a float. The Unix implementation uses @code{usleep} if the number of 
seconds is below one, and @code{sleep} if it is over a second. The WIN32 
implementation uses @code{Sleep} for both cases.

@item system
@findex  system/0
@syindex system/0
@cnindex system/0
Start a new default shell and leave YAP in background until the shell
completes. YAP uses @code{/bin/sh} in Unix systems and @code{COMSPEC} in
WIN32.

@item system(+@var{Command},-@var{Res})
@findex  system/2
@syindex system/2
@cnindex system/2
Interface to @code{system}: execute command @var{Command} and unify
@var{Res} with the result.

@item wait(+@var{PID},-@var{Status})
@findex  wait/2
@syindex wait/2
@cnindex wait/2
Wait until process @var{PID} terminates, and return its exits @var{Status}.

@end table


@node Terms, Tries, System, Library
@section Utilities On Terms
@cindex utilities on terms

The next routines provide a set of commonly used utilities to manipulate
terms. Most of these utilities have been implemented in @code{C} for
efficiency. They are available through the
@code{use_module(library(terms))} command.

@table @code

@item cyclic_term(?@var{Term})
@findex cyclic_term/1
@syindex cyclic_term/1
@cnindex cyclic_term/1
Succeed if the argument @var{Term} is a cyclic term.

@item term_hash(+@var{Term}, ?@var{Hash})
@findex  term_hash/2
@syindex term_hash/2
@cnindex term_hash/2

If @var{Term} is ground unify @var{Hash} with a positive integer
calculated from the structure of the term. Otherwise the argument
@var{Hash} is left unbound. The range of the positive integer is from
@code{0} to, but not including, @code{33554432}.

@item term_hash(+@var{Term}, +@var{Depth}, +@var{Range}, ?@var{Hash})
@findex  term_hash/4
@syindex term_hash/4
@cnindex term_hash/4

Unify @var{Hash} with a positive integer calculated from the structure
of the term.  The range of the positive integer is from @code{0} to, but
not including, @var{Range}. If @var{Depth} is @code{-1} the whole term
is considered. Otherwise, the term is considered only up to depth
@code{1}, where the constants and the principal functor have depth
@code{1}, and an argument of a term with depth @var{I} has depth @var{I+1}. 

@item term_variables(?@var{Term}, -@var{Variables})
@findex  term_variables/2
@syindex term_variables/2
@cnindex term_variables/2

Unify @var{Variables} with the list of all variables of term
@var{Term}.  The variables occur in the order of their first
appearance when traversing the term depth-first, left-to-right.

@item variables_within_term(+@var{Variables},?@var{Term}, -@var{OutputVariables})
@findex  variables_within_term/3
@snindex variables_within_term/3 
@cnindex variables_within_term/3  

Unify @var{OutputVariables} with the subset of the variables @var{Variables} that occurs in @var{Term}.

@item new_variables_in_term(+@var{Variables},?@var{Term}, -@var{OutputVariables})
@findex  new_variables_in_term/3
@snindex new_variables_in_term/3 
@cnindex new_variables_in_term/3  

Unify @var{OutputVariables} with all variables occurring in @var{Term} that are not in the list @var{Variables}.

@item variant(?@var{Term1}, ?@var{Term2})
@findex  variant/2
@syindex variant/2
@cnindex variant/2

Succeed if @var{Term1} and @var{Term2} are variant terms.

@item subsumes(?@var{Term1}, ?@var{Term2})
@findex  subsumes/2
@syindex subsumes/2
@cnindex subsumes/2

Succeed if @var{Term1} subsumes @var{Term2}.  Variables in term
@var{Term1} are bound so that the two terms become equal.


@item subsumes_chk(?@var{Term1}, ?@var{Term2})
@findex  subsumes_chk/2
@syindex subsumes_chk/2
@cnindex subsumes_chk/2

Succeed if @var{Term1} subsumes @var{Term2} but does not bind any
variable in @var{Term1}.

@item variable_in_term(?@var{Term},?@var{Var})
@findex variable_in_term/2
@snindex variable_in_term/2
@cnindex variable_in_term/2
Succeed if the second argument @var{Var} is a variable and occurs in
term @var{Term}.

@item unifiable(?@var{Term1}, ?@var{Term2}, -@var{Bindings})
@findex  unifiable/3
@syindex unifiable/3
@cnindex unifiable/3

Succeed if @var{Term1} and @var{Term2} are unifiable with substitution
@var{Bindings}.

@end table

@node Tries, Cleanup, Terms, Library
@section Trie DataStructure
@cindex tries

The next routines provide a set of utilities to create and manipulate
prefix trees of Prolog terms. Tries were originally proposed to
implement tabling in Logic Programming, but can be used for other
purposes. The tries will be stored in the Prolog database and can seen
as alternative to @code{assert} and @code{record} family of
primitives. Most of these utilities have been implemented in @code{C}
for efficiency. They are available through the
@code{use_module(library(tries))} command.

@table @code
@item trie_open(-@var{Id})
@findex trie_open/1
@snindex trie_open/1
@cnindex trie_open/1

Open a new trie with identifier @var{Id}.

@item trie_close(+@var{Id})
@findex trie_close/1
@snindex trie_close/1
@cnindex trie_close/1

Close trie with identifier @var{Id}.

@item trie_close_all
@findex trie_close_all/0
@snindex trie_close_all/0
@cnindex trie_close_all/0

Close all available tries.

@item trie_mode(?@var{Mode})
@findex trie_mode/1
@snindex trie_mode/1
@cnindex trie_mode/1

Unify @var{Mode} with trie operation mode. Allowed values are either
@code{std} (default) or @code{rev}.

@item trie_put_entry(+@var{Trie},+@var{Term},-@var{Ref})
@findex trie_put_entry/3
@snindex trie_put_entry/3
@cnindex trie_put_entry/3

Add term @var{Term} to trie @var{Trie}. The handle @var{Ref} gives
a reference to the term.

@item trie_check_entry(+@var{Trie},+@var{Term},-@var{Ref})
@findex trie_check_entry/3
@snindex trie_check_entry/3
@cnindex trie_check_entry/3

Succeeds if a variant of term @var{Term} is in trie @var{Trie}. An handle
 @var{Ref} gives a reference to the term.

@item trie_get_entry(+@var{Ref},-@var{Term})
@findex trie_get_entry/2
@snindex trie_get_entry/2
@cnindex trie_get_entry/2
Unify @var{Term} with the entry for handle @var{Ref}.

@item trie_remove_entry(+@var{Ref})
@findex trie_remove_entry/1
@snindex trie_remove_entry/1
@cnindex trie_remove_entry/1

Remove entry for handle @var{Ref}.

@item trie_remove_subtree(+@var{Ref})
@findex trie_remove_subtree/1
@snindex trie_remove_subtree/1
@cnindex trie_remove_subtree/1

Remove subtree rooted at handle @var{Ref}.

@item trie_save(+@var{Trie},+@var{FileName})
@findex trie_save/2
@snindex trie_save/2
@cnindex trie_save/2
Dump trie @var{Trie} into file @var{FileName}.


@item trie_load(+@var{Trie},+@var{FileName})
@findex trie_load/2
@snindex trie_load/2
@cnindex trie_load/2
Load trie @var{Trie} from the contents of file @var{FileName}.

@item trie_stats(-@var{Memory},-@var{Tries},-@var{Entries},-@var{Nodes})
@findex trie_stats/4
@snindex trie_stats/4
@cnindex trie_stats/4
Give generic statistics on tries, including the amount of memory,
@var{Memory}, the number of tries, @var{Tries}, the number of entries,
@var{Entries}, and the total number of nodes, @var{Nodes}.

@item trie_max_stats(-@var{Memory},-@var{Tries},-@var{Entries},-@var{Nodes})
@findex trie_max_stats/4
@snindex trie_max_stats/4
@cnindex trie_max_stats/4
Give maximal statistics on tries, including the amount of memory,
@var{Memory}, the number of tries, @var{Tries}, the number of entries,
@var{Entries}, and the total number of nodes, @var{Nodes}.


@item trie_usage(+@var{Trie},-@var{Entries},-@var{Nodes},-@var{VirtualNodes})
@findex trie_usage/4
@snindex trie_usage/4
@cnindex trie_usage/4
Give statistics on trie @var{Trie}, the number of entries,
@var{Entries}, and the total number of nodes, @var{Nodes}, and the
number of @var{VirtualNodes}.

@item trie_print(+@var{Trie})
@findex trie_print/1
@snindex trie_print/1
@cnindex trie_print/1
Print trie @var{Trie} on standard output.




@end table


@node Cleanup, Timeout, Tries, Library
@section Call Cleanup
@cindex cleanup

@t{call_cleanup/1} and @t{call_cleanup/2} allow predicates to register
code for execution after the call is finished. Predicates can be
declared to be @t{fragile} to ensure that @t{call_cleanup} is called
for any Goal which needs it. This library is loaded with the
@code{use_module(library(cleanup))} command.

@table @code
@item :- fragile @var{P},....,@var{Pn}
@findex fragile
@syindex fragile
@cnindex fragile
Declares the predicate @var{P}=@t{[module:]name/arity} as a fragile
predicate, module is optional, default is the current
typein_module. Whenever such a fragile predicate is used in a query
it will be called through call_cleanup/1.
@example
:- fragile foo/1,bar:baz/2.
@end example

@item call_cleanup(:@var{Goal})
@findex call_cleanup/1
@syindex call_cleanup/1
@cnindex call_cleanup/1
Execute goal @var{Goal} within a cleanup-context. Called predicates
might register cleanup Goals which are called right after the end of
the call to @var{Goal}. Cuts and exceptions inside Goal do not prevent the
execution of the cleanup calls. @t{call_cleanup} might be nested.

@item call_cleanup(:@var{Goal}, :@var{CleanUpGoal})
@findex call_cleanup/2
@syindex call_cleanup/2
@cnindex call_cleanup/2
This is similar to @t{call_cleanup/1} with an additional
@var{CleanUpGoal} which gets called after @var{Goal} is finished.

@item setup_call_cleanup(:@var{Setup},:@var{Goal}, :@var{CleanUpGoal})
@findex setup_call_cleanup/3
@snindex setup_call_cleanup/3
@cnindex setup_call_cleanup/3
Calls @code{(Setup, Goal)}. For each sucessful execution of @var{Setup}, calling @var{Goal}, the
cleanup handler @var{Cleanup} is guaranteed to be called exactly once.
This will happen after @var{Goal} completes, either through failure,
deterministic success, commit, or an exception.  @var{Setup} will
contain the goals that need to be protected from asynchronous interrupts
such as the ones received from @code{call_with_time_limit/2} or @code{thread_signal/2}.  In
most uses, @var{Setup} will perform temporary side-effects required by
@var{Goal} that are finally undone by @var{Cleanup}.

Success or failure of @var{Cleanup} is ignored and choice-points it
created are destroyed (as @code{once/1}). If @var{Cleanup} throws an exception,
this is executed as normal.

Typically, this predicate is used to cleanup permanent data storage
required to execute @var{Goal}, close file-descriptors, etc. The example
below provides a non-deterministic search for a term in a file, closing
the stream as needed.

@example
term_in_file(Term, File) :-
	setup_call_cleanup(open(File, read, In),
			   term_in_stream(Term, In),
			   close(In) ).

term_in_stream(Term, In) :-
	repeat,
	read(In, T),
	(   T == end_of_file
	->  !, fail
	;   T = Term
	).
@end example

Note that it is impossible to implement this predicate in Prolog other than
by reading all terms into a list, close the file and call @code{member/2}.
Without @code{setup_call_cleanup/3} there is no way to gain control if the
choice-point left by @code{repeat} is removed by a cut or an exception.

@code{setup_call_cleanup/2} can also be used to test determinism of a goal:

@example
?- setup_call_cleanup(true,(X=1;X=2), Det=yes).

X = 1 ;

X = 2,
Det = yes ;
@end example

This predicate is under consideration for inclusion into the ISO standard.
For compatibility with other Prolog implementations see @code{call_cleanup/2}.

 @item setup_call_catcher_cleanup(:@var{Setup},:@var{Goal}, +@var{Catcher},:@var{CleanUpGoal})
@findex setup_call_catcher_cleanup/4
@snindex setup_call_catcher_cleanup/4
@cnindex setup_call_catcher_cleanup/4
Similar to @code{setup_call_cleanup(@var{Setup}, @var{Goal}, @var{Cleanup})} with
additional information on the reason of calling @var{Cleanup}.  Prior
to calling @var{Cleanup}, @var{Catcher} unifies with the termination
code.  If this unification fails, @var{Cleanup} is
@strong{not} called.


@item on_cleanup(+@var{CleanUpGoal})
@findex on_cleanup/1
@syindex on_cleanup/1
@cnindex on_cleanup/1
Any Predicate might registers a @var{CleanUpGoal}. The
@var{CleanUpGoal} is put onto the current cleanup context. All such
CleanUpGoals are executed in reverse order of their registration when
the surrounding cleanup-context ends. This call will throw an exception
if a predicate tries to register a @var{CleanUpGoal} outside of any
cleanup-context.

@item cleanup_all
@findex cleanup_all/0
@syindex cleanup_all/0
@cnindex cleanup_all/0
Calls all pending CleanUpGoals and resets the cleanup-system to an
initial state. Should only be used as one of the last calls in the
main program.

@end table

There are some private predicates which could be used in special
cases, such as manually setting up cleanup-contexts and registering
CleanUpGoals for other than the current cleanup-context.
Read the Source Luke.


@node Timeout, Trees, Cleanup, Library
@section Calls With Timeout
@cindex timeout

The @t{time_out/3} command relies on the @t{alarm/3} built-in to
implement a call with a maximum time of execution. The command is
available with the @code{use_module(library(timeout))} command.

@table @code


@item time_out(+@var{Goal}, +@var{Timeout}, -@var{Result})
@findex time_out/3
@syindex time_out/3
@cnindex time_out/3
Execute goal @var{Goal} with time limited @var{Timeout}, where
@var{Timeout} is measured in milliseconds. If the goal succeeds, unify
@var{Result} with success. If the timer expires before the goal
terminates, unify @var{Result} with @t{time_out}.

This command is implemented by activating an alarm at procedure
entry. If the timer expires before the goal completes, the alarm will
throw an exception @var{timeout}.

One should note that @code{time_out/3} is not reentrant, that is, a goal
called from @code{time_out} should never itself call
@code{time_out/3}. Moreover, @code{time_out/3} will deactivate any previous
alarms set by @code{alarm/3} and vice-versa, hence only one of these
calls should be used in a program.

Last, even though the timer is set in milliseconds, the current
implementation relies on @t{alarm/3}, and therefore can only offer
precision on the scale of seconds.

@end table

@node Trees, UGraphs, Timeout, Library
@section Updatable Binary Trees
@cindex updatable tree

The following queue manipulation routines are available once
included with the @code{use_module(library(trees))} command.

@table @code

@item get_label(+@var{Index}, +@var{Tree}, ?@var{Label})
@findex get_label/3
@syindex get_label/3
@cnindex get_label/3
Treats the tree as an array of @var{N} elements and returns the
@var{Index}-th.

@item list_to_tree(+@var{List}, -@var{Tree})
@findex list_to_tree/2
@syindex list_to_tree/2
@cnindex list_to_tree/2
Takes a given @var{List} of @var{N} elements and constructs a binary
@var{Tree}.

@item map_tree(+@var{Pred}, +@var{OldTree}, -@var{NewTree})
@findex map_tree/3
@syindex map_tree/3
@cnindex map_tree/3
Holds when @var{OldTree} and @var{NewTree} are binary trees of the same shape
and @code{Pred(Old,New)} is true for corresponding elements of the two trees.

@item put_label(+@var{Index}, +@var{OldTree}, +@var{Label}, -@var{NewTree})
@findex put_label/4
@syindex put_label/4
@cnindex put_label/4
constructs a new tree the same shape as the old which moreover has the
same elements except that the @var{Index}-th one is @var{Label}.

@item tree_size(+@var{Tree}, -@var{Size})
@findex tree_size/2
@syindex tree_size/2
@cnindex tree_size/2
Calculates the number of elements in the @var{Tree}.

@item tree_to_list(+@var{Tree}, -@var{List})
@findex tree_to_list/2
@syindex tree_to_list/2
@cnindex tree_to_list/2
Is the converse operation to list_to_tree.

@end table

@node UGraphs, DGraphs, Trees, Library
@section Unweighted Graphs
@cindex unweighted graphs

The following graph manipulation routines are based in code originally
written by Richard O'Keefe. The code was then extended to be compatible
with the SICStus Prolog ugraphs library. The routines assume directed
graphs, undirected graphs may be implemented by using two edges. Graphs
are represented in one of two ways:

@itemize @bullet
@item The P-representation of a graph is a list of (from-to) vertex
pairs, where the pairs can be in any old order.  This form is
convenient for input/output.
 
@item The S-representation of a graph is a list of (vertex-neighbors)
pairs, where the pairs are in standard order (as produced by keysort)
and the neighbors of each vertex are also in standard order (as
produced by sort).  This form is convenient for many calculations.
@end itemize

These built-ins are available once included with the
@code{use_module(library(ugraphs))} command.

@table @code

@item vertices_edges_to_ugraph(+@var{Vertices}, +@var{Edges}, -@var{Graph})
@findex  vertices_edges_to_ugraph/3
@syindex vertices_edges_to_ugraph/3
@cnindex vertices_edges_to_ugraph/3
Given a graph with a set of vertices @var{Vertices} and a set of edges
@var{Edges}, @var{Graph} must unify with the corresponding
s-representation. Note that the vertices without edges will appear in
@var{Vertices} but not in @var{Edges}. Moreover, it is sufficient for a
vertex to appear in @var{Edges}.
@example
?- vertices_edges_to_ugraph([],[1-3,2-4,4-5,1-5],L).

L = [1-[3,5],2-[4],3-[],4-[5],5-[]] ? 

@end example
In this case all edges are defined implicitly. The next example shows
three unconnected edges:
@example 
?- vertices_edges_to_ugraph([6,7,8],[1-3,2-4,4-5,1-5],L).

L = [1-[3,5],2-[4],3-[],4-[5],5-[],6-[],7-[],8-[]] ? 

@end example

@item vertices(+@var{Graph}, -@var{Vertices})
@findex  vertices/2
@syindex vertices/2
@cnindex vertices/2
Unify @var{Vertices} with all vertices appearing in graph
@var{Graph}. In the next example:
@example
?- vertices([1-[3,5],2-[4],3-[],4-[5],5-[]], V).

L = [1,2,3,4,5]
@end example

@item edges(+@var{Graph}, -@var{Edges})
@findex  edges/2
@syindex edges/2
@cnindex edges/2
Unify @var{Edges} with all edges appearing in graph
@var{Graph}. In the next example:
@example
?- vertices([1-[3,5],2-[4],3-[],4-[5],5-[]], V).

L = [1,2,3,4,5]
@end example

@item add_vertices(+@var{Graph}, +@var{Vertices}, -@var{NewGraph})
@findex  add_vertices/3
@syindex add_vertices/3
@cnindex add_vertices/3
Unify @var{NewGraph} with a new graph obtained by adding the list of
vertices @var{Vertices} to the graph @var{Graph}. In the next example:
@example
?- add_vertices([1-[3,5],2-[4],3-[],4-[5],
                 5-[],6-[],7-[],8-[]],
                [0,2,9,10,11],
                   NG).

NG = [0-[],1-[3,5],2-[4],3-[],4-[5],5-[],
      6-[],7-[],8-[],9-[],10-[],11-[]]
@end example

@item del_vertices(+@var{Graph}, +@var{Vertices}, -@var{NewGraph})
@findex  del_vertices/3
@syindex del_vertices/3
@cnindex del_vertices/3
Unify @var{NewGraph} with a new graph obtained by deleting the list of
vertices @var{Vertices} and all the edges that start from or go to a
vertex in @var{Vertices} to the graph @var{Graph}. In the next example:
@example
?- del_vertices([2,1],[1-[3,5],2-[4],3-[],
                 4-[5],5-[],6-[],7-[2,6],8-[]],NL).

NL = [3-[],4-[5],5-[],6-[],7-[6],8-[]]
@end example

@item add_edges(+@var{Graph}, +@var{Edges}, -@var{NewGraph})
@findex  add_edges/3
@syindex add_edges/3
@cnindex add_edges/3
Unify @var{NewGraph} with a new graph obtained by adding the list of
edges @var{Edges} to the graph @var{Graph}. In the next example:
@example
?- add_edges([1-[3,5],2-[4],3-[],4-[5],5-[],6-[],
              7-[],8-[]],[1-6,2-3,3-2,5-7,3-2,4-5],NL).

NL = [1-[3,5,6],2-[3,4],3-[2],4-[5],5-[7],6-[],7-[],8-[]]
@end example

@item del_edges(+@var{Graph}, +@var{Edges}, -@var{NewGraph})
@findex  del_edges/3
@syindex del_edges/3
@cnindex del_edges/3
Unify @var{NewGraph} with a new graph obtained by removing the list of
edges @var{Edges} from the graph @var{Graph}. Notice that no vertices
are deleted. In the next example:
@example
?- del_edges([1-[3,5],2-[4],3-[],4-[5],5-[],
              6-[],7-[],8-[]],
             [1-6,2-3,3-2,5-7,3-2,4-5,1-3],NL).

NL = [1-[5],2-[4],3-[],4-[],5-[],6-[],7-[],8-[]]
@end example

@item transpose(+@var{Graph}, -@var{NewGraph})
@findex  transpose/3
@syindex transpose/3
@cnindex transpose/3
Unify @var{NewGraph} with a new graph obtained from @var{Graph} by
replacing all edges of the form @var{V1-V2} by edges of the form
@var{V2-V1}. The cost is @code{O(|V|^2)}. In the next example:
@example
?- transpose([1-[3,5],2-[4],3-[],
              4-[5],5-[],6-[],7-[],8-[]], NL).

NL = [1-[],2-[],3-[1],4-[2],5-[1,4],6-[],7-[],8-[]]
@end example
Notice that an undirected graph is its own transpose.

@item neighbors(+@var{Vertex}, +@var{Graph}, -@var{Vertices})
@findex  neighbors/3
@syindex neighbors/3
@cnindex neighbors/3
Unify @var{Vertices} with the list of neighbors of vertex @var{Vertex}
in @var{Graph}. If the vertice is not in the graph fail. In the next
example:
@example
?- neighbors(4,[1-[3,5],2-[4],3-[],
                4-[1,2,7,5],5-[],6-[],7-[],8-[]],
             NL).

NL = [1,2,7,5]
@end example

@item neighbours(+@var{Vertex}, +@var{Graph}, -@var{Vertices})
@findex  neighbours/3
@syindex neighbours/3
@cnindex neighbours/3
Unify @var{Vertices} with the list of neighbours of vertex @var{Vertex}
in @var{Graph}. In the next example:
@example
?- neighbours(4,[1-[3,5],2-[4],3-[],
                 4-[1,2,7,5],5-[],6-[],7-[],8-[]], NL).

NL = [1,2,7,5]
@end example

@item complement(+@var{Graph}, -@var{NewGraph})
@findex  complement/2
@syindex complement/2
@cnindex complement/2
Unify @var{NewGraph} with the graph complementary to @var{Graph}.
 In the next example:
@example
?- complement([1-[3,5],2-[4],3-[],
               4-[1,2,7,5],5-[],6-[],7-[],8-[]], NL).

NL = [1-[2,4,6,7,8],2-[1,3,5,6,7,8],3-[1,2,4,5,6,7,8],
      4-[3,5,6,8],5-[1,2,3,4,6,7,8],6-[1,2,3,4,5,7,8],
      7-[1,2,3,4,5,6,8],8-[1,2,3,4,5,6,7]]
@end example

@item compose(+@var{LeftGraph}, +@var{RightGraph}, -@var{NewGraph})
@findex  compose/3
@syindex compose/3
@cnindex compose/3
Compose the graphs @var{LeftGraph} and @var{RightGraph} to form @var{NewGraph}.
 In the next example:
@example
?- compose([1-[2],2-[3]],[2-[4],3-[1,2,4]],L).

L = [1-[4],2-[1,2,4],3-[]]
@end example

@item top_sort(+@var{Graph}, -@var{Sort})
@findex  top_sort/2
@syindex top_sort/2
@cnindex top_sort/2
Generate the set of nodes @var{Sort} as a topological sorting of graph
@var{Graph}, if one is possible.
 In the next example we show how topological sorting works for a linear graph:
@example
?- top_sort([_138-[_219],_219-[_139], _139-[]],L).

L = [_138,_219,_139]
@end example

@item top_sort(+@var{Graph}, -@var{Sort0}, -@var{Sort})
@findex  top_sort/3
@syindex top_sort/3
@cnindex top_sort/3
Generate the difference list @var{Sort}-@var{Sort0} as a topological
sorting of graph @var{Graph}, if one is possible.

@item transitive_closure(+@var{Graph}, +@var{Closure})
@findex  transitive_closure/2
@syindex transitive_closure/2
@cnindex transitive_closure/2
Generate the graph @var{Closure} as the transitive closure of graph
@var{Graph}.
 In the next example:
@example
?- transitive_closure([1-[2,3],2-[4,5],4-[6]],L).

L = [1-[2,3,4,5,6],2-[4,5,6],4-[6]]
@end example

@item reachable(+@var{Node}, +@var{Graph}, -@var{Vertices})
@findex  reachable/3
@syindex reachable/3
@cnindex reachable/3
Unify @var{Vertices} with the set of all vertices in graph
@var{Graph} that are reachable from @var{Node}. In the next example:
@example
?- reachable(1,[1-[3,5],2-[4],3-[],4-[5],5-[]],V).

V = [1,3,5]
@end example

@end table

@node DGraphs, UnDGraphs, UGraphs, Library
@section Directed Graphs
@cindex Efficient Directed Graphs

The following graph manipulation routines use the red-black tree library
to try to avoid linear-time scans of the graph for all graph
operations. Graphs are represented as a red-black tree, where the key is
the vertex, and the associated value is a list of vertices reachable
from that vertex through an edge (ie, a list of edges). 

@table @code

@item dgraph_new(+@var{Graph})
@findex  dgraph_new/1
@snindex dgraph_new/1
@cnindex dgraph_new/1
Create a new directed graph. This operation must be performed before
trying to use the graph.

@item dgraph_vertices(+@var{Graph}, -@var{Vertices})
@findex  dgraph_vertices/2
@snindex dgraph_vertices/2
@cnindex dgraph_vertices/2
Unify @var{Vertices} with all vertices appearing in graph
@var{Graph}.

@item dgraph_edge(+@var{N1}, +@var{N2}, +@var{Graph})
@findex  dgraph_edge/2
@snindex dgraph_edge/2
@cnindex dgraph_edge/2
Edge @var{N1}-@var{N2} is an edge in directed graph @var{Graph}.

@item dgraph_edges(+@var{Graph}, -@var{Edges})
@findex  dgraph_edges/2
@snindex dgraph_edges/2
@cnindex dgraph_edges/2
Unify @var{Edges} with all edges appearing in graph
@var{Graph}.

@item dgraph_add_vertices(+@var{Graph}, +@var{Vertex}, -@var{NewGraph})
@findex  dgraph_add_vertex/3
@snindex dgraph_add_vertex/3
@cnindex dgraph_add_vertex/3
Unify @var{NewGraph} with a new graph obtained by adding
vertex @var{Vertex} to the graph @var{Graph}.

@item dgraph_add_vertices(+@var{Graph}, +@var{Vertices}, -@var{NewGraph})
@findex  dgraph_add_vertices/3
@snindex dgraph_add_vertices/3
@cnindex dgraph_add_vertices/3
Unify @var{NewGraph} with a new graph obtained by adding the list of
vertices @var{Vertices} to the graph @var{Graph}.

@item dgraph_del_vertex(+@var{Graph}, +@var{Vertex}, -@var{NewGraph})
@findex  dgraph_del_vertex/3
@syindex dgraph_del_vertex/3
@cnindex dgraph_del_vertex/3
Unify @var{NewGraph} with a new graph obtained by deleting vertex
@var{Vertex} and all the edges that start from or go to @var{Vertex} to
the graph @var{Graph}.

@item dgraph_del_vertices(+@var{Graph}, +@var{Vertices}, -@var{NewGraph})
@findex  dgraph_del_vertices/3
@syindex dgraph_del_vertices/3
@cnindex dgraph_del_vertices/3
Unify @var{NewGraph} with a new graph obtained by deleting the list of
vertices @var{Vertices} and all the edges that start from or go to a
vertex in @var{Vertices} to the graph @var{Graph}.

@item dgraph_add_edge(+@var{Graph}, +@var{N1}, +@var{N2}, -@var{NewGraph})
@findex  dgraph_add_edge/4
@snindex dgraph_add_edge/4
@cnindex dgraph_add_edge/4
Unify @var{NewGraph} with a new graph obtained by adding the edge
@var{N1}-@var{N2} to the graph @var{Graph}.

@item dgraph_add_edges(+@var{Graph}, +@var{Edges}, -@var{NewGraph})
@findex  dgraph_add_edges/3
@snindex dgraph_add_edges/3
@cnindex dgraph_add_edges/3
Unify @var{NewGraph} with a new graph obtained by adding the list of
edges @var{Edges} to the graph @var{Graph}.

@item dgraph_del_edge(+@var{Graph}, +@var{N1}, +@var{N2}, -@var{NewGraph})
@findex  dgraph_del_edge/4
@snindex dgraph_del_edge/4
@cnindex dgraph_del_edge/4
Succeeds if @var{NewGraph} unifies with a new graph obtained by
removing the edge @var{N1}-@var{N2} from the graph @var{Graph}. Notice
that no vertices are deleted.

@item dgraph_del_edges(+@var{Graph}, +@var{Edges}, -@var{NewGraph})
@findex  dgraph_del_edges/3
@snindex dgraph_del_edges/3
@cnindex dgraph_del_edges/3
Unify @var{NewGraph} with a new graph obtained by removing the list of
edges @var{Edges} from the graph @var{Graph}. Notice that no vertices
are deleted.

@item dgraph_to_ugraph(+@var{Graph}, -@var{UGraph})
@findex  dgraph_to_ugraph/2
@snindex dgraph_to_ugraph/2
@cnindex dgraph_to_ugraph/2
Unify @var{UGraph} with the representation used by the @var{ugraphs}
unweighted graphs library, that is, a list of the form
@var{V-Neighbors}, where @var{V} is a node and @var{Neighbors} the nodes
children.

@item ugraph_to_dgraph( +@var{UGraph}, -@var{Graph})
@findex  ugraph_to_dgraph/2
@snindex ugraph_to_dgraph/2
@cnindex ugraph_to_dgraph/2
Unify @var{Graph} with the directed graph obtain from @var{UGraph},
represented in the form used in the @var{ugraphs} unweighted graphs
library.

@item dgraph_neighbors(+@var{Vertex}, +@var{Graph}, -@var{Vertices})
@findex  dgraph_neighbors/3
@snindex dgraph_neighbors/3
@cnindex dgraph_neighbors/3
Unify @var{Vertices} with the list of neighbors of vertex @var{Vertex}
in @var{Graph}. If the vertice is not in the graph fail.

@item dgraph_neighbours(+@var{Vertex}, +@var{Graph}, -@var{Vertices})
@findex  dgraph_neighbours/3
@snindex dgraph_neighbours/3
@cnindex dgraph_neighbours/3
Unify @var{Vertices} with the list of neighbours of vertex @var{Vertex}
in @var{Graph}.

@item dgraph_complement(+@var{Graph}, -@var{NewGraph})
@findex  dgraph_complement/2
@snindex dgraph_complement/2
@cnindex dgraph_complement/2
Unify @var{NewGraph} with the graph complementary to @var{Graph}.

@item dgraph_transpose(+@var{Graph}, -@var{Transpose})
@findex  dgraph_transpose/2
@snindex dgraph_transpose/2
@cnindex dgraph_transpose/2
Unify @var{NewGraph} with a new graph obtained from @var{Graph} by
replacing all edges of the form @var{V1-V2} by edges of the form
@var{V2-V1}. 

@item dgraph_compose(+@var{Graph1}, +@var{Graph2}, -@var{ComposedGraph})
@findex  dgraph_compose/3
@snindex dgraph_compose/3
@cnindex dgraph_compose/3
Unify @var{ComposedGraph} with a new graph obtained by composing
@var{Graph1} and @var{Graph2}, ie, @var{ComposedGraph} has an edge
@var{V1-V2} iff there is a @var{V} such that @var{V1-V} in @var{Graph1}
and @var{V-V2} in @var{Graph2}.

@item dgraph_transitive_closure(+@var{Graph}, -@var{Closure})
@findex  dgraph_transitive_closure/2
@snindex dgraph_transitive_closure/2
@cnindex dgraph_transitive_closure/2
Unify @var{Closure} with the transitive closure of graph @var{Graph}.

@item dgraph_symmetric_closure(+@var{Graph}, -@var{Closure})
@findex  dgraph_symmetric_closure/2
@snindex dgraph_symmetric_closure/2
@cnindex dgraph_symmetric_closure/2
Unify @var{Closure} with the symmetric closure of graph @var{Graph},
that is,  if @var{Closure} contains an edge @var{U-V} it must also
contain the edge @var{V-U}.

@item dgraph_top_sort(+@var{Graph}, -@var{Vertices})
@findex  dgraph_top_sort/2
@snindex dgraph_top_sort/2
@cnindex dgraph_top_sort/2
Unify @var{Vertices} with the topological sort of graph @var{Graph}.

@item dgraph_top_sort(+@var{Graph}, -@var{Vertices}, ?@var{Vertices0})
@findex  dgraph_top_sort/3
@snindex dgraph_top_sort/3
@cnindex dgraph_top_sort/3
Unify the difference list @var{Vertices}-@var{Vertices0} with the
topological sort of graph @var{Graph}.

@item dgraph_min_path(+@var{V1}, +@var{V1}, +@var{Graph}, -@var{Path}, ?@var{Costt})
@findex  dgraph_min_path/5
@snindex dgraph_min_path/5
@cnindex dgraph_min_path/5
Unify the list @var{Path} with the minimal cost path between nodes
@var{N1} and @var{N2} in graph @var{Graph}. Path @var{Path} has cost
@var{Cost}.

@item dgraph_max_path(+@var{V1}, +@var{V1}, +@var{Graph}, -@var{Path}, ?@var{Costt})
@findex  dgraph_max_path/5
@snindex dgraph_max_path/5
@cnindex dgraph_max_path/5
Unify the list @var{Path} with the maximal cost path between nodes
@var{N1} and @var{N2} in graph @var{Graph}. Path @var{Path} has cost
@var{Cost}.

@item dgraph_min_paths(+@var{V1}, +@var{Graph}, -@var{Paths})
@findex  dgraph_min_paths/3
@snindex dgraph_min_paths/3
@cnindex dgraph_min_paths/3
Unify the list @var{Paths} with the minimal cost paths from node
@var{N1} to the nodes in graph @var{Graph}.

@item dgraph_isomorphic(+@var{Vs}, +@var{NewVs}, +@var{G0}, -@var{GF})
@findex  dgraph_isomorphic/4
@snindex dgraph_isomorphic/4
@cnindex dgraph_isomorphic/4
Unify the list @var{GF} with the graph isomorphic to @var{G0} where 
vertices in @var{Vs} map to vertices in @var{NewVs}.

@item dgraph_path(+@var{Vertex}, +@var{Graph}, ?@var{Path})
@findex  dgraph_path/3
@snindex dgraph_path/3
@cnindex dgraph_path/3
The path @var{Path} is a path starting at vertex @var{Vertex} in graph
@var{Graph}.

@item dgraph_reachable(+@var{Vertex}, +@var{Graph}, ?@var{Edges})
@findex  dgraph_path/3
@snindex dgraph_path/3
@cnindex dgraph_path/3
The path @var{Path} is a path starting at vertex @var{Vertex} in graph
@var{Graph}.

@end table

@node UnDGraphs, Lambda , DGraphs, Library
@section Undirected Graphs
@cindex undirected graphs

The following graph manipulation routines use the red-black tree graph
library to implement undirected graphs. Mostly, this is done by having
two directed edges per undirected edge.

@table @code

@item undgraph_new(+@var{Graph})
@findex  undgraph_new/1
@snindex undgraph_new/1
@cnindex undgraph_new/1
Create a new directed graph. This operation must be performed before
trying to use the graph.

@item undgraph_vertices(+@var{Graph}, -@var{Vertices})
@findex  undgraph_vertices/2
@snindex undgraph_vertices/2
@cnindex undgraph_vertices/2
Unify @var{Vertices} with all vertices appearing in graph
@var{Graph}.

@item undgraph_edge(+@var{N1}, +@var{N2}, +@var{Graph})
@findex  undgraph_edge/2
@snindex undgraph_edge/2
@cnindex undgraph_edge/2
Edge @var{N1}-@var{N2} is an edge in undirected graph @var{Graph}.

@item undgraph_edges(+@var{Graph}, -@var{Edges})
@findex  undgraph_edges/2
@snindex undgraph_edges/2
@cnindex undgraph_edges/2
Unify @var{Edges} with all edges appearing in graph
@var{Graph}.

@item undgraph_add_vertices(+@var{Graph}, +@var{Vertices}, -@var{NewGraph})
@findex  undgraph_add_vertices/3
@snindex undgraph_add_vertices/3
@cnindex undgraph_add_vertices/3
Unify @var{NewGraph} with a new graph obtained by adding the list of
vertices @var{Vertices} to the graph @var{Graph}.

@item undgraph_del_vertices(+@var{Graph}, +@var{Vertices}, -@var{NewGraph})
@findex  undgraph_del_vertices/3
@syindex undgraph_del_vertices/3
@cnindex undgraph_del_vertices/3
Unify @var{NewGraph} with a new graph obtained by deleting the list of
vertices @var{Vertices} and all the edges that start from or go to a
vertex in @var{Vertices} to the graph @var{Graph}.

@item undgraph_add_edges(+@var{Graph}, +@var{Edges}, -@var{NewGraph})
@findex  undgraph_add_edges/3
@snindex undgraph_add_edges/3
@cnindex undgraph_add_edges/3
Unify @var{NewGraph} with a new graph obtained by adding the list of
edges @var{Edges} to the graph @var{Graph}.

@item undgraph_del_edges(+@var{Graph}, +@var{Edges}, -@var{NewGraph})
@findex  undgraph_del_edges/3
@snindex undgraph_del_edges/3
@cnindex undgraph_del_edges/3
Unify @var{NewGraph} with a new graph obtained by removing the list of
edges @var{Edges} from the graph @var{Graph}. Notice that no vertices
are deleted.

@item undgraph_neighbors(+@var{Vertex}, +@var{Graph}, -@var{Vertices})
@findex  undgraph_neighbors/3
@snindex undgraph_neighbors/3
@cnindex undgraph_neighbors/3
Unify @var{Vertices} with the list of neighbors of vertex @var{Vertex}
in @var{Graph}. If the vertice is not in the graph fail.

@item undgraph_neighbours(+@var{Vertex}, +@var{Graph}, -@var{Vertices})
@findex  undgraph_neighbours/3
@snindex undgraph_neighbours/3
@cnindex undgraph_neighbours/3
Unify @var{Vertices} with the list of neighbours of vertex @var{Vertex}
in @var{Graph}.

@item undgraph_complement(+@var{Graph}, -@var{NewGraph})
@findex  undgraph_complement/2
@snindex undgraph_complement/2
@cnindex undgraph_complement/2
Unify @var{NewGraph} with the graph complementary to @var{Graph}.

@item dgraph_to_undgraph( +@var{DGraph}, -@var{UndGraph})
@findex  dgraph_to_undgraph/2
@snindex dgraph_to_undgraph/2
@cnindex dgraph_to_undgraph/2
Unify @var{UndGraph} with the undirected graph obtained from the
directed graph @var{DGraph}.

@end table

@node Lambda, LAM, UnDGraphs, Library
@section Lambda Expressions
@cindex Lambda Expressions


This library, designed and implemented by Ulrich Neumerkel, provides
lambda expressions to simplify higher order programming based on @code{call/N}.

Lambda expressions are represented by ordinary Prolog terms.  There are
two kinds of lambda expressions:

@example
    Free+\X1^X2^ ..^XN^Goal

         \X1^X2^ ..^XN^Goal
@end example

The second is a shorthand for@code{ t+\X1^X2^..^XN^Goal}, where @code{Xi} are the parameters.

@var{Goal} is a goal or continuation (Syntax note: @var{Operators} within @var{Goal}
require parentheses due to the low precedence of the @code{^} operator).

Free contains variables that are valid outside the scope of the lambda
expression. They are thus free variables within.

All other variables of @var{Goal} are considered local variables. They must
not appear outside the lambda expression. This restriction is
currently not checked. Violations may lead to unexpected bindings.

In the following example the parentheses around @code{X>3} are necessary.

@example
?- use_module(library(lambda)).
?- use_module(library(apply)).

?- maplist(\X^(X>3),[4,5,9]).
true.
@end example

In the following @var{X} is a variable that is shared by both instances
of the lambda expression. The second query illustrates the cooperation
of continuations and lambdas. The lambda expression is in this case a
continuation expecting a further argument.

@example
?- Xs = [A,B], maplist(X+\Y^dif(X,Y), Xs).
Xs = [A, B],
dif(X, A),
dif(X, B).

?- Xs = [A,B], maplist(X+\dif(X), Xs).
Xs = [A, B],
dif(X, A),
dif(X, B).

@end example

The following queries are all equivalent. To see this, use
the fact @code{f(x,y)}.

@example
?- call(f,A1,A2).
?- call(\X^f(X),A1,A2).
?- call(\X^Y^f(X,Y), A1,A2).
?- call(\X^(X+\Y^f(X,Y)), A1,A2).
?- call(call(f, A1),A2).
?- call(f(A1),A2).
?- f(A1,A2).
A1 = x,
A2 = y.
@end example

Further discussions
at @url{http://www.complang.tuwien.ac.at/ulrich/Prolog-inedit/ISO-Hiord}.


@node LAM, , Lambda, Library
@section LAM
@cindex lam

This library provides a set of utilities for interfacing with LAM MPI.
The following routines are available once included with the
@code{use_module(library(lam_mpi))} command. The yap should be
invoked using the LAM mpiexec or mpirun commands (see LAM manual for
more details).

@table @code
@item mpi_init
@findex mpi_init/0
@snindex mpi_init/0
@cnindex mpi_init/0
      Sets up the mpi environment. This predicate should be called before any other MPI predicate.

@item mpi_finalize
@findex mpi_finalize/0
@snindex mpi_finalize/0
@cnindex mpi_finalize/0
      Terminates the MPI execution environment. Every process must call this predicate before  exiting.

@item mpi_comm_size(-@var{Size})
@findex mpi_comm_size/1
@snindex mpi_comm_size/1
@cnindex mpi_comm_size/1
      Unifies @var{Size} with the number of processes in the MPI environment.


@item mpi_comm_rank(-@var{Rank})
@findex mpi_comm_rank/1
@snindex mpi_comm_rank/1
@cnindex mpi_comm_rank/1
      Unifies @var{Rank} with the rank of the current process in the MPI environment.

@item mpi_version(-@var{Major},-@var{Minor})
@findex mpi_version/2
@snindex mpi_version/2
@cnindex mpi_version/2
      Unifies @var{Major} and @var{Minor} with, respectively, the major and minor version of the MPI.


@item mpi_send(+@var{Data},+@var{Dest},+@var{Tag})
@findex mpi_send/3
@snindex mpi_send/3
@cnindex mpi_send/3

Blocking communication predicate. The message in @var{Data}, with tag
@var{Tag}, is sent immediately to the processor with rank @var{Dest}.
The predicate succeeds after the message being sent.



@item mpi_isend(+@var{Data},+@var{Dest},+@var{Tag},-@var{Handle})
@findex mpi_isend/4
@snindex mpi_isend/4
@cnindex mpi_isend/4

Non blocking communication predicate. The message in @var{Data}, with
tag @var{Tag}, is sent whenever possible to the processor with rank
@var{Dest}. An @var{Handle} to the message is returned to be used to
check for the status of the message, using the @code{mpi_wait} or
@code{mpi_test} predicates. Until @code{mpi_wait} is called, the
memory allocated for the buffer containing the message is not
released.

@item mpi_recv(?@var{Source},?@var{Tag},-@var{Data})
@findex mpi_recv/3
@snindex mpi_recv/3
@cnindex mpi_recv/3

Blocking communication predicate. The predicate blocks until a message
is received from processor with rank @var{Source} and tag @var{Tag}.
The message is placed in @var{Data}.

@item mpi_irecv(?@var{Source},?@var{Tag},-@var{Handle})
@findex mpi_irecv/3
@snindex mpi_irecv/3
@cnindex mpi_irecv/3

Non-blocking communication predicate. The predicate returns an
@var{Handle} for a message that will be received from processor with
rank @var{Source} and tag @var{Tag}. Note that the predicate succeeds
immediately, even if no message has been received. The predicate
@code{mpi_wait_recv} should be used to obtain the data associated to
the handle.

@item mpi_wait_recv(?@var{Handle},-@var{Status},-@var{Data})
@findex mpi_wait_recv/3
@snindex mpi_wait_recv/3
@cnindex mpi_wait_recv/3

Completes a non-blocking receive operation. The predicate blocks until
a message associated with handle @var{Hanlde} is buffered. The
predicate succeeds unifying @var{Status} with the status of the
message and @var{Data} with the message itself. 

@item mpi_test_recv(?@var{Handle},-@var{Status},-@var{Data})
@findex mpi_test_recv/3
@snindex mpi_test_recv/3
@cnindex mpi_test_recv/3

Provides information regarding a handle. If the message associated
with handle @var{Hanlde} is buffered then the predicate succeeds
unifying @var{Status} with the status of the message and @var{Data}
with the message itself. Otherwise, the predicate fails.


@item mpi_wait(?@var{Handle},-@var{Status})
@findex mpi_wait/2
@snindex mpi_wait/2
@cnindex mpi_wait/2

Completes a non-blocking operation. If the operation was a
@code{mpi_send}, the predicate blocks until the message is buffered
or sent by the runtime system. At this point the send buffer is
released. If the operation was a @code{mpi_recv}, it waits until the
message is copied to the receive buffer. @var{Status} is unified with
the status of the message.

@item mpi_test(?@var{Handle},-@var{Status})
@findex mpi_test/2
@snindex mpi_test/2
@cnindex mpi_test/2

Provides information regarding the handle @var{Handle}, ie., if a
communication operation has been completed.  If the operation
associate with @var{Hanlde} has been completed the predicate succeeds
with the completion status in @var{Status}, otherwise it fails.

@item mpi_barrier
@findex mpi_barrier/0
@snindex mpi_barrier/0
@cnindex mpi_barrier/0

Collective communication predicate.  Performs a barrier
synchronization among all processes. Note that a collective
communication means that all processes call the same predicate. To be
able to use a regular @code{mpi_recv} to receive the messages, one
should use @code{mpi_bcast2}.


@item mpi_bcast2(+@var{Root}, +@var{Data})
@findex mpi_bcast/2
@snindex mpi_bcast/2
@cnindex mpi_bcast/2

Broadcasts the message @var{Data} from the process with rank @var{Root}
to all other processes.

@item mpi_bcast3(+@var{Root}, +@var{Data}, +@var{Tag})
@findex mpi_bcast/3
@snindex mpi_bcast/3
@cnindex mpi_bcast/3

Broadcasts the message @var{Data} with tag @var{Tag} from the process with rank @var{Root}
to all other processes.

@item mpi_ibcast(+@var{Root}, +@var{Data}, +@var{Tag})
@findex mpi_bcast/3
@snindex mpi_bcast/3
@cnindex mpi_bcast/3

Non-blocking operation. Broadcasts the message @var{Data} with tag @var{Tag}
from the process with rank @var{Root} to all other processes.

@item mpi_gc
@findex mpi_gc/0
@snindex mpi_gc/0
@cnindex mpi_gc/0

Attempts to perform garbage collection with all the open handles
associated with send and non-blocking broadcasts. For each handle it
tests it and the message has been delivered the handle and the buffer
are released.

@end table


@node SWI-Prolog, SWI-Prolog Global Variables, Library, Top
@cindex SWI-Prolog

@menu SWI-Prolog Emulation
Subnodes of SWI-Prolog
* Invoking Predicates on all Members of a List :: maplist and friends
* Forall :: forall built-in
@end menu

@include swi.tex

@node Extensions,Debugging,SWI-Prolog Global Variables,Top 
@chapter Extensions to Prolog

YAP includes several extensions that are not enabled by
default, but that can be used to extend the functionality of the
system. These options can be set at compilation time by enabling the
related compilation flag, as explained in the @code{Makefile}

@menu
Extensions to Traditional Prolog

* Rational Trees:: Working with Rational Trees
* Co-routining:: Changing the Execution of Goals
* Attributed Variables:: Using attributed Variables
* CLPR:: The CLP(R) System
* Logtalk:: The Logtalk Object-Oriented system
* MYDDAS:: The MYDDAS Database Interface package
* Threads:: Thread Library
* Parallelism:: Running in Or-Parallel
* Tabling:: Storing Intermediate Solutions of programs 
* Low Level Profiling:: Profiling Abstract Machine Instructions
* Low Level Tracing:: Tracing at Abstract Machine Level
@end menu

@node Rational Trees, Co-routining, , Extensions
@section Rational Trees

Prolog unification is not a complete implementation. For efficiency
considerations, Prolog systems do not perform occur checks while
unifying terms. As an example, @code{X = a(X)} will not fail but instead
will create an infinite term of the form @code{a(a(a(a(a(...)))))}, or
@emph{rational tree}.

Rational trees are now supported by default in YAP. In previous
versions, this was not the default and these terms could easily lead
to infinite computation. For example, @code{X = a(X), X = X} would
enter an infinite loop.

The @code{RATIONAL_TREES} flag improves support for these
terms. Internal primitives are now aware that these terms can exist, and
will not enter infinite loops. Hence, the previous unification will
succeed. Another example, @code{X = a(X), ground(X)} will succeed
instead of looping. Other affected built-ins include the term comparison
primitives, @code{numbervars/3}, @code{copy_term/2}, and the internal
data base routines. The support does not extend to Input/Output routines
or to @code{assert/1} YAP does not allow directly reading
rational trees, and you need to use @code{write_depth/2} to avoid
entering an infinite cycle when trying to write an infinite term.

@node Co-routining, Attributed Variables, Rational Trees, Extensions
@section Co-routining

Prolog uses a simple left-to-right flow of control. It is sometimes
convenient to change this control so that goals will only be executed
when conditions are fulfilled. This may result in a more "data-driven"
execution, or may be necessary to correctly implement extensions such as
negation by default.

The @code{COROUTINING} flag enables this option. Note that the support for
coroutining  will in general slow down execution.

The following declaration is supported:

@table @code
@item block/1
The argument to @code{block/1} is a condition on a goal or a conjunction
of conditions, with each element separated by commas. Each condition is
of the form @code{predname(@var{C1},...,@var{CN})}, where @var{N} is the
arity of the goal, and each @var{CI} is of the form @code{-}, if the
argument must suspend until the variable is bound, or @code{?}, otherwise.

@item wait/1
The argument to @code{wait/1} is a predicate descriptor or a conjunction
of these predicates. These predicates will suspend until their first
argument is bound.
@end table

The following primitives are supported:

@table @code
@item dif(@var{X},@var{Y})
@findex dif/2
@syindex dif/2
@cnindex dif/2
Succeed if the two arguments do not unify. A call to @code{dif/2} will
suspend if unification may still succeed or fail, and will fail if they
always unify.

@item freeze(?@var{X},:@var{G})
@findex freeze/2
@syindex freeze/2
@cnindex freeze/2
Delay execution of goal @var{G} until the variable @var{X} is bound.

@item frozen(@var{X},@var{G})
@findex frozen/2
@syindex frozen/2
@cnindex frozen/2
Unify @var{G} with a conjunction of goals suspended on variable @var{X},
or @code{true} if no goal has suspended.

@item when(+@var{C},:@var{G})
@findex when/2
@syindex when/2
@cnindex when/2
Delay execution of goal @var{G} until the conditions @var{C} are
satisfied. The conditions are of the following form:

@table @code
@item @var{C1},@var{C2}
Delay until both conditions @var{C1} and @var{C2} are satisfied.
@item @var{C1};@var{C2}
Delay until either condition @var{C1} or condition @var{C2} is satisfied.
@item ?=(@var{V1},@var{C2})
Delay until terms @var{V1} and @var{V1} have been unified.
@item nonvar(@var{V})
Delay until variable @var{V} is bound.
@item ground(@var{V})
Delay until variable @var{V} is ground.
@end table

Note that @code{when/2} will fail if the conditions fail.

@item call_residue(:@var{G},@var{L})
@findex call_residue/2
@syindex call_residue/2
@cnindex call_residue/2

Call goal @var{G}. If subgoals of @var{G} are still blocked, return
a list containing these goals and the variables they are blocked in. The
goals are then considered as unblocked. The next example shows a case
where @code{dif/2} suspends twice, once outside @code{call_residue/2},
and the other inside:

@example
?- dif(X,Y),
       call_residue((dif(X,Y),(X = f(Z) ; Y = f(Z))), L).

X = f(Z),
L = [[Y]-dif(f(Z),Y)],
dif(f(Z),Y) ? ;

Y = f(Z),
L = [[X]-dif(X,f(Z))],
dif(X,f(Z)) ? ;

no
@end example
The system only reports one invocation of @code{dif/2} as having
suspended. 

@item call_residue_vars(:@var{G},@var{L})
@findex call_residue_vars/2
@syindex call_residue_vars/2
@cnindex call_residue_vars/2

Call goal @var{G} and unify @var{L} with a list of all constrained variables created @emph{during} execution of @var{G}:

@example
  ?- dif(X,Z), call_residue_vars(dif(X,Y),L).
dif(X,Z), call_residue_vars(dif(X,Y),L).
L = [Y],
dif(X,Z),
dif(X,Y) ? ;

no
@end example

@end table

@node Attributed Variables, CLPR, Co-routining, Extensions
@chapter Attributed Variables
@cindex attributed variables

@menu
* New Style Attribute Declarations:: New Style code
* Old Style Attribute Declarations:: Old Style code (deprecated)
@end menu

YAP supports attributed variables, originally developed at OFAI by
Christian Holzbaur. Attributes are a means of declaring that an
arbitrary term is a property for a variable. These properties can be
updated during forward execution. Moreover, the unification algorithm is
aware of attributed variables and will call user defined handlers when
trying to unify these variables.

Attributed variables provide an elegant abstraction over which one can
extend Prolog systems. Their main application so far has been in
implementing constraint handlers, such as Holzbaur's CLPQR, Fruewirth
and Holzbaur's CHR, and CLP(BN). 

Different Prolog systems implement attributed variables in different
ways. Traditionally, YAP has used the interface designed by SICStus
Prolog. This interface is still
available in the @t{atts} library, but from YAP-6.0.3 we recommend using
the hProlog, SWI style interface. The main reason to do so is that 
most packages included in YAP that use attributed variables, such as CHR, CLP(FD), and CLP(QR),
rely on the SWI-Prolog interface.


@node New Style Attribute Declarations, Old Style Attribute Declarations, , Attributed Variables
@section hProlog and SWI-Prolog style Attribute Declarations

The following documentation is taken from the SWI-Prolog manual.

Binding an attributed variable schedules a goal to be executed at the
first possible opportunity. In the current implementation the hooks are
executed immediately after a successful unification of the clause-head
or successful completion of a foreign language (built-in) predicate. Each
attribute is associated to a module and the hook @code{attr_unify_hook/2} is
executed in this module.  The example below realises a very simple and
incomplete finite domain reasoner.

@example
:- module(domain,
	  [ domain/2			% Var, ?Domain
	  ]).
:- use_module(library(ordsets)).

domain(X, Dom) :-
	var(Dom), !,
	get_attr(X, domain, Dom).
domain(X, List) :-
	list_to_ord_set(List, Domain),
	put_attr(Y, domain, Domain),
	X = Y.

%	An attributed variable with attribute value Domain has been
%	assigned the value Y

attr_unify_hook(Domain, Y) :-
	(   get_attr(Y, domain, Dom2)
	->  ord_intersection(Domain, Dom2, NewDomain),
	    (   NewDomain == []
	    ->	fail
	    ;	NewDomain = [Value]
	    ->	Y = Value
	    ;	put_attr(Y, domain, NewDomain)
	    )
	;   var(Y)
	->  put_attr( Y, domain, Domain )
	;   ord_memberchk(Y, Domain)
	).

%	Translate attributes from this module to residual goals

attribute_goals(X) -->
	@{ get_attr(X, domain, List) @},
	[domain(X, List)].
@end example


Before explaining the code we give some example queries:

@multitable @columnfractions .70 .30
            @item @code{?- domain(X, [a,b]), X = c}
@tab @code{fail}
@item @code{domain(X, [a,b]), domain(X, [a,c]).}
           @tab @code{X=a}
    @item @code{domain(X, [a,b,c]), domain(X, [a,c]).}
     @tab @code{domain(X, [a,c]).}
    @end multitable

The predicate @code{domain/2} fetches (first clause) or assigns
(second clause) the variable a @emph{domain}, a set of values it can
be unified with.  In the second clause first associates the domain
with a fresh variable and then unifies X to this variable to deal
with the possibility that X already has a domain. The
predicate @code{attr_unify_hook/2} is a hook called after a variable with
a domain is assigned a value.  In the simple case where the variable
is bound to a concrete value we simply check whether this value is in
the domain. Otherwise we take the intersection of the domains and either
fail if the intersection is empty (first example), simply assign the
value if there is only one value in the intersection (second example) or
assign the intersection as the new domain of the variable (third
example). The nonterminal @code{attribute_goals/3} is used to translate
remaining attributes to user-readable goals that, when executed, reinstate
these attributes.

@table @code

@item attvar(?@var{Term})
@findex attvar/1
@snindex attvar/1
@cnindex attvar/1

Succeeds if @code{Term} is an attributed variable. Note that @code{var/1} also
succeeds on attributed variables.  Attributed variables are created with
@code{put_attr/3}.

@item put_attr(+@var{Var},+@var{Module},+@var{Value})
@findex put_attr/3
@snindex put_attr/3
@cnindex put_attr/3

If @var{Var} is a variable or attributed variable, set the value for the
attribute named @var{Module} to @var{Value}. If an attribute with this
name is already associated with @var{Var}, the old value is replaced.
Backtracking will restore the old value (i.e., an attribute is a mutable
term. See also @code{setarg/3}). This predicate raises a representation error if
@var{Var} is not a variable and a type error if @var{Module} is not an atom.

@item get_attr(+@var{Var},+@var{Module},-@var{Value})
@findex get_attr/3
@snindex get_attr/3
@cnindex get_attr/3

Request the current @var{value} for the attribute named @var{Module}.  If
@var{Var} is not an attributed variable or the named attribute is not
associated to @var{Var} this predicate fails silently.  If @var{Module}
is not an atom, a type error is raised.

@item del_attr(+@var{Var},+@var{Module})
@findex del_attr/2
@snindex del_attr/2
@cnindex del_attr/2

Delete the named attribute.  If @var{Var} loses its last attribute it
is transformed back into a traditional Prolog variable.  If @var{Module}
is not an atom, a type error is raised. In all other cases this
predicate succeeds regardless whether or not the named attribute is
present.

@item attr_unify_hook(+@var{AttValue},+@var{VarValue})
@findex attr_unify_hook/2
@snindex attr_unify_hook/2
@cnindex attr_unify_hook/2

Hook that must be defined in the module an attributed variable refers
to. Is is called @emph{after} the attributed variable has been
unified with a non-var term, possibly another attributed variable.
@var{AttValue} is the attribute that was associated to the variable
in this module and @var{VarValue} is the new value of the variable.
Normally this predicate fails to veto binding the variable to
@var{VarValue}, forcing backtracking to undo the binding.  If
@var{VarValue} is another attributed variable the hook often combines
the two attribute and associates the combined attribute with
@var{VarValue} using @code{put_attr/3}.

@item attr_portray_hook(+@var{AttValue},+@var{Var})
@findex attr_portray_hook/2
@snindex attr_portray_hook/2
@cnindex attr_portray_hook/2

Called by @code{write_term/2} and friends for each attribute if the option
@code{attributes(portray)} is in effect.  If the hook succeeds the
attribute is considered printed.  Otherwise  @code{Module = ...} is
printed to indicate the existence of a variable.

@item attribute_goals(+@var{Var},-@var{Gs},+@var{GsRest})
@findex attribute_goals/2
@snindex attribute_goals/2
@cnindex attribute_goals/2

This nonterminal, if it is defined in a module, is used by @var{copy_term/3}
to project attributes of that module to residual goals. It is also
used by the toplevel to obtain residual goals after executing a query.
@end table

Normal user code should deal with @code{put_attr/3}, @code{get_attr/3} and @code{del_attr/2}.
The routines in this section fetch or set the entire attribute list of a
variables. Use of these predicates is anticipated to be restricted to
printing and other special purpose operations.

@table @code

@item get_attrs(+@var{Var},-@var{Attributes})
@findex get_attrs/2
@snindex get_attrs/2
@cnindex get_attrs/2

Get all attributes of @var{Var}. @var{Attributes} is a term of the form
@code{att(@var{Module}, @var{Value}, @var{MoreAttributes})}, where @var{MoreAttributes} is
@code{[]} for the last attribute.

@item put_attrs(+@var{Var},+@var{Attributes})
@findex put_attrs/2
@snindex put_attrs/2
@cnindex put_attrs/2
Set all attributes of @var{Var}.  See @code{get_attrs/2} for a description of
@var{Attributes}.

@item del_attrs(+@var{Var})
@findex del_attrs/1
@snindex del_attrs/1
@cnindex del_attrs/1
If @var{Var} is an attributed variable, delete @emph{all} its
attributes.  In all other cases, this predicate succeeds without
side-effects.

@item term_attvars(+@var{Term},-@var{AttVars})
@findex term_attvars/2
@snindex term_attvars/2
@cnindex term_attvars/2
@var{AttVars} is a list of all attributed variables in @var{Term} and
its attributes. I.e., @code{term_attvars/2} works recursively through
attributes.  This predicate is Cycle-safe.

@item copy_term(?@var{TI},-@var{TF},-@var{Goals}) 
@findex copy_term/3
@syindex copy_term/3
@cnindex copy_term/3
Term @var{TF} is a variant of the original term @var{TI}, such that for
each variable @var{V} in the term @var{TI} there is a new variable @var{V'}
in term @var{TF} without any attributes attached.  Attributed
variables are thus converted to standard variables.  @var{Goals} is
unified with a list that represents the attributes.  The goal
@code{maplist(call,@var{Goals})} can be called to recreate the
attributes.

Before the actual copying, @code{copy_term/3} calls
@code{attribute_goals/1} in the module where the attribute is
defined.

@item copy_term_nat(?@var{TI},-@var{TF}) 
@findex copy_term_nat/2
@syindex copy_term_nat/2
@cnindex copy_term_nat/2
As @code{copy_term/2}.  Attributes however, are @emph{not} copied but replaced
by fresh variables.
@end table

@node Old Style Attribute Declarations, , New Style Attribute Declarations, Attributed Variables
@section SICStus Prolog style Attribute Declarations

@menu
* Attribute Declarations:: Declaring New Attributes
* Attribute Manipulation:: Setting and Reading Attributes
* Attributed Unification:: Tuning the Unification Algorithm
* Displaying Attributes:: Displaying Attributes in User-Readable Form
* Projecting Attributes:: Obtaining the Attributes of Interest
* Attribute Examples:: Two Simple Examples of how to use Attributes.
@end menu

Old style attribute declarations are activated through loading the library @t{atts} . The command
@example
| ?- use_module(library(atts)).
@end example
enables this form of use of attributed variables. The package provides the
following functionality:
@itemize @bullet
@item Each attribute must be declared first. Attributes are described by a functor
and are declared per module. Each Prolog module declares its own sets of
attributes. Different modules may have different functors with the same
module.
@item The built-in @code{put_atts/2} adds or deletes attributes to a
variable. The variable may be unbound or may be an attributed
variable. In the latter case, YAP discards previous values for the
attributes.
@item The built-in @code{get_atts/2} can be used to check the values of
an attribute associated with a variable.
@item The unification algorithm calls the user-defined predicate
@t{verify_attributes/3} before trying to bind an attributed
variable. Unification will resume after this call.
@item The user-defined predicate
@t{attribute_goal/2} converts from an attribute to a goal.
@item The user-defined predicate
@t{project_attributes/2} is used from a set of variables into a set of
constraints or goals. One application of @t{project_attributes/2} is in
the top-level, where it is used to output the set of
floundered constraints at the end of a query.
@end itemize

@node Attribute Declarations, Attribute Manipulation, , Old Style Attribute Declarations
@subsection Attribute Declarations

Attributes are compound terms associated with a variable. Each attribute
has a @emph{name} which is @emph{private} to the module in which the
attribute was defined. Variables may have at most one attribute with a
name. Attribute names are defined with the following declaration:

@cindex attribute declaration
@cindex declaration, attribute
@findex attribute/1 (declaration)

@example
:- attribute @var{AttributeSpec}, ..., @var{AttributeSpec}.
@end example

@noindent
where each @var{AttributeSpec} has the form (@var{Name}/@var{Arity}).
One single such declaration is allowed per module @var{Module}.

Although the YAP module system is predicate based, attributes are local
to modules. This is implemented by rewriting all calls to the
built-ins that manipulate attributes so that attribute names are
preprocessed depending on the module.  The @code{user:goal_expansion/3}
mechanism is used for this purpose.


@node Attribute Manipulation, Attributed Unification, Attribute Declarations, Old Style Attribute Declarations
@subsection Attribute Manipulation


The  attribute manipulation predicates always work as follows:
@enumerate
@item The first argument is the unbound variable associated with
attributes,
@item The second argument is a list of attributes. Each attribute will
be a Prolog term or a constant, prefixed with the @t{+} and @t{-} unary
operators. The prefix @t{+} may be dropped for convenience.
@end enumerate

The following three procedures are available to the user. Notice that
these built-ins are rewritten by the system into internal built-ins, and
that the rewriting process @emph{depends} on the module on which the
built-ins have been invoked.

@table @code
@item @var{Module}:get_atts(@var{-Var},@var{?ListOfAttributes})
@findex get_atts/2
@syindex get_atts/2
@cnindex get_atts/2
Unify the list @var{?ListOfAttributes} with the attributes for the unbound
variable @var{Var}. Each member of the list must be a bound term of the
form @code{+(@var{Attribute})}, @code{-(@var{Attribute})} (the @t{kbd}
prefix may be dropped). The meaning of @t{+} and @t{-} is:
@item +(@var{Attribute})
Unifies @var{Attribute} with a corresponding attribute associated with
@var{Var}, fails otherwise.

@item -(@var{Attribute})
Succeeds if a corresponding attribute is not associated with
@var{Var}. The arguments of @var{Attribute} are ignored.

@item @var{Module}:put_atts(@var{-Var},@var{?ListOfAttributes})
@findex put_atts/2
@syindex put_atts/2
@cnindex put_atts/2
Associate with or remove attributes from a variable @var{Var}. The
attributes are given in @var{?ListOfAttributes}, and the action depends
on how they are prefixed:
@item +(@var{Attribute})
Associate @var{Var} with @var{Attribute}. A previous value for the
attribute is simply replace (like with @code{set_mutable/2}).

@item -(@var{Attribute})
Remove the attribute with the same name. If no such attribute existed,
simply succeed.
@end table

@node Attributed Unification, Displaying Attributes, Attribute Manipulation, Old Style Attribute Declarations
@subsection Attributed Unification

The user-predicate predicate @code{verify_attributes/3} is called when
attempting to unify an attributed variable which might have attributes
in some @var{Module}.

@table @code
@item @var{Module}:verify_attributes(@var{-Var}, @var{+Value}, @var{-Goals})
@findex verify_attributes/3
@syindex verify_attributes/3
@cnindex verify_attributes/3

The predicate is called when trying to unify the attributed variable
@var{Var} with the Prolog term @var{Value}. Note that @var{Value} may be
itself an attributed variable, or may contain attributed variables.  The
goal @t{verify_attributes/3} is actually called before @var{Var} is
unified with @var{Value}.

It is up to the user to define which actions may be performed by
@t{verify_attributes/3} but the procedure is expected to return in
@var{Goals} a list of goals to be called @emph{after} @var{Var} is
unified with @var{Value}. If @t{verify_attributes/3} fails, the
unification will fail.

Notice that the @t{verify_attributes/3} may be called even if @var{Var}
has no attributes in module @t{Module}. In this case the routine should
simply succeed with @var{Goals} unified with the empty list.

@item attvar(@var{-Var})
@findex attvar/1
@snindex attvar/1
@cnindex attvar/1
Succeed if @var{Var} is an attributed variable.
@end table



@node Displaying Attributes, Projecting Attributes,Attributed Unification, Old Style Attribute Declarations 
@subsection Displaying Attributes

Attributes are usually presented as goals. The following routines are
used by built-in predicates such as @code{call_residue/2} and by the
Prolog top-level to display attributes:

@table @code
@item @var{Module}:attribute_goal(@var{-Var}, @var{-Goal})
@findex attribute_goal/2
@syindex attribute_goal/2
@cnindex attribute_goal/2
User-defined procedure, called to convert the attributes in @var{Var} to
a @var{Goal}. Should fail when no interpretation is available.

@item @var{Module}:project_attributes(@var{-QueryVars}, @var{+AttrVars})
@findex project_attributes/2
@syindex project_attributes/2
@cnindex project_attributes/2
User-defined procedure, called to project the attributes in the query,
@var{AttrVars}, given that the set of variables in the query is
@var{QueryVars}.

@end table

@node Projecting Attributes, Attribute Examples, Displaying Attributes, Old Style Attribute Declarations
@subsection Projecting Attributes

Constraint solvers must be able to project a set of constraints to a set
of variables. This is useful when displaying the solution to a goal, but
may also be used to manipulate computations. The user-defined
@code{project_attributes/2} is responsible for implementing this
projection.


@table @code
@item @var{Module}:project_attributes(@var{+QueryVars}, @var{+AttrVars})
@findex project_attributes/2
@syindex project_attributes/2
@cnindex project_attributes/2
Given a list of variables @var{QueryVars} and list of attributed
variables @var{AttrVars}, project all attributes in @var{AttrVars} to
@var{QueryVars}. Although projection is constraint system dependent,
typically this will involve expressing all constraints in terms of
@var{QueryVars} and considering all remaining variables as existentially
quantified.
@end table

Projection interacts with @code{attribute_goal/2} at the Prolog top
level. When the query succeeds, the system first calls
@code{project_attributes/2}. The system then calls
@code{attribute_goal/2} to get a user-level representation of the
constraints. Typically, @code{attribute_goal/2} will convert from the
original constraints into a set of new constraints on the projection,
and these constraints are the ones that will have an
@code{attribute_goal/2} handler.

@node Attribute Examples, ,Projecting Attributes, Old Style Attribute Declarations
@subsection Attribute Examples

The following two examples example is taken from the SICStus Prolog manual. It
sketches the implementation of a simple finite domain ``solver''.  Note
that an industrial strength solver would have to provide a wider range
of functionality and that it quite likely would utilize a more efficient
representation for the domains proper.  The module exports a single
predicate @code{domain(@var{-Var},@var{?Domain})} which associates
@var{Domain} (a list of terms) with @var{Var}.  A variable can be
queried for its domain by leaving @var{Domain} unbound.

We do not present here a definition for @code{project_attributes/2}.
Projecting finite domain constraints happens to be difficult.


@example
:- module(domain, [domain/2]).

:- use_module(library(atts)).
:- use_module(library(ordsets), [
        ord_intersection/3,
        ord_intersect/2,
        list_to_ord_set/2
   ]).

:- attribute dom/1.

verify_attributes(Var, Other, Goals) :-
        get_atts(Var, dom(Da)), !,          % are we involved?
        (   var(Other) ->                   % must be attributed then
            (   get_atts(Other, dom(Db)) -> %   has a domain?
                ord_intersection(Da, Db, Dc),
                Dc = [El|Els],              % at least one element
                (   Els = [] ->             % exactly one element
                    Goals = [Other=El]      % implied binding
                ;   Goals = [],
                    put_atts(Other, dom(Dc))% rescue intersection
                )
            ;   Goals = [],
                put_atts(Other, dom(Da))    % rescue the domain
            )
        ;   Goals = [],
            ord_intersect([Other], Da)      % value in domain?
        ).
verify_attributes(_, _, []).                % unification triggered
                                            % because of attributes
                                            % in other modules

attribute_goal(Var, domain(Var,Dom)) :-     % interpretation as goal
        get_atts(Var, dom(Dom)).

domain(X, Dom) :-
        var(Dom), !,
        get_atts(X, dom(Dom)).
domain(X, List) :-
        list_to_ord_set(List, Set),
        Set = [El|Els],                     % at least one element
        (   Els = [] ->                     % exactly one element
            X = El                          % implied binding
        ;   put_atts(Fresh, dom(Set)),
            X = Fresh                       % may call
                                            % verify_attributes/3
        ).
@end example

Note that the ``implied binding'' @code{Other=El} was deferred until after
the completion of @code{verify_attribute/3}.  Otherwise, there might be a
danger of recursively invoking @code{verify_attribute/3}, which might bind
@code{Var}, which is not allowed inside the scope of @code{verify_attribute/3}.
Deferring unifications into the third argument of @code{verify_attribute/3}
effectively serializes the calls to @code{verify_attribute/3}.

Assuming that the code resides in the file @file{domain.yap}, we
can use it via:

@example
| ?- use_module(domain).
@end example

Let's test it:

@example
| ?- domain(X,[5,6,7,1]), domain(Y,[3,4,5,6]), domain(Z,[1,6,7,8]).

domain(X,[1,5,6,7]),
domain(Y,[3,4,5,6]),
domain(Z,[1,6,7,8]) ? 

yes
| ?- domain(X,[5,6,7,1]), domain(Y,[3,4,5,6]), domain(Z,[1,6,7,8]), 
     X=Y.

Y = X,
domain(X,[5,6]),
domain(Z,[1,6,7,8]) ? 

yes
| ?- domain(X,[5,6,7,1]), domain(Y,[3,4,5,6]), domain(Z,[1,6,7,8]),
     X=Y, Y=Z.

X = 6,
Y = 6,
Z = 6
@end example

To demonstrate the use of the @var{Goals} argument of
@code{verify_attributes/3}, we give an implementation of
@code{freeze/2}.  We have to name it @code{myfreeze/2} in order to
avoid a name clash with the built-in predicate of the same name.

@example
:- module(myfreeze, [myfreeze/2]).

:- use_module(library(atts)).

:- attribute frozen/1.

verify_attributes(Var, Other, Goals) :-
        get_atts(Var, frozen(Fa)), !,       % are we involved?
        (   var(Other) ->                   % must be attributed then
            (   get_atts(Other, frozen(Fb)) % has a pending goal?
            ->  put_atts(Other, frozen((Fa,Fb))) % rescue conjunction
            ;   put_atts(Other, frozen(Fa)) % rescue the pending goal
            ),
            Goals = []
        ;   Goals = [Fa]
        ).
verify_attributes(_, _, []).

attribute_goal(Var, Goal) :-                % interpretation as goal
        get_atts(Var, frozen(Goal)).

myfreeze(X, Goal) :-
        put_atts(Fresh, frozen(Goal)),
        Fresh = X.
@end example

Assuming that this code lives in file @file{myfreeze.yap},
we would use it via:

@example
| ?- use_module(myfreeze).
| ?- myfreeze(X,print(bound(x,X))), X=2.

bound(x,2)                      % side effect
X = 2                           % bindings
@end example

The two solvers even work together:

@example
| ?- myfreeze(X,print(bound(x,X))), domain(X,[1,2,3]),
     domain(Y,[2,10]), X=Y.

bound(x,2)                      % side effect
X = 2,                          % bindings
Y = 2
@end example

The two example solvers interact via bindings to shared attributed
variables only.  More complicated interactions are likely to be found
in more sophisticated solvers.  The corresponding
@code{verify_attributes/3} predicates would typically refer to the
attributes from other known solvers/modules via the module prefix in
@code{@var{Module}:get_atts/2}.

@node CLPR, CHR, Attributed Variables, Extensions
@cindex CLPQ
@cindex CLPR

@menu
* CLPR Solver Predicates::
* CLPR Syntax::
* CLPR Unification::
* CLPR Non-linear Constraints::               
@end menu

@include clpr.tex

@node CHR, Logtalk, CLPR, Top

@menu
* CHR Introduction::            
* CHR Syntax and Semantics::
* CHR in YAP Programs::
* CHR Debugging::               
* CHR Examples::       
* CHR Compatibility::     
* CHR Guidelines::  
@end menu

@include chr.tex

@node Logtalk, MYDDAS, CHR, Extensions
@chapter Logtalk
@cindex Logtalk

The Logtalk object-oriented extension is available after running its 
standalone installer by using the @code{yaplgt} command in POSIX 
systems or by using the @code{Logtalk - YAP} shortcut in the Logtalk 
program group in the Start Menu on Windows systems. For more information 
please see the URL @url{http://logtalk.org/}.

@node MYDDAS, Threads, Logtalk, Extensions
@chapter MYDDAS
@cindex MYDDAS

The MYDDAS database project was developed within a FCT project aiming at
the development of a highly efficient deductive database system, based
on the coupling of the MySQL relational database system with the Yap
Prolog system. MYDDAS was later expanded to support the ODBC interface.

@menu
Subnodes of MYDDAS
* Requirements and Installation Guide:: 
* MYDDAS Architecture:: 
* Loading MYDDAS:: 
* Connecting to and disconnecting from a Database Server:: 
* Accessing a Relation:: 
* View Level Interface :: 
* Accessing Tables in Data Sources Using SQL:: 
* Insertion of Rows:: 
* Types of Attributes:: 
* Number of Fields:: 
* Describing a Relation:: 
* Enumerating Relations:: 
* The MYDDAS MySQL Top Level:: 
* Other MYDDAS Properties:: 
@end menu

@node Requirements and Installation Guide, MYDDAS Architecture, , MYDDAS
@section Requirements and Installation Guide

Next, we describe how to usen of the YAP with the MYDDAS System.  The
use of this system is entirely depend of the MySQL development libraries
or the ODBC development libraries. At least one of the this development
libraries must be installed on the computer system, otherwise MYDDAS
will not compile. The MySQL development libraries from MySQL 3.23 an
above are know to work. We recommend the usage of MySQL versusODBC,
but it is possible to have both options installed

At the same time, without any problem. The MYDDAS system automatically
controls the two options. Currently, MYDDAS is know to compile without
problems in Linux. The usage of this system on Windows has not been
tested yet.  MYDDAS must be enabled at configure time. This can be done
with the following options: 

@table @code

@item --enable-myddas
 This option will detect which development libraries are installed on the computer system, MySQL, ODBC or both, and will compile the Yap system with the support for which libraries it detects;
@item  --enable-myddas-stats
This option is only available in MySQL. It includes code to get
statistics from the MYDDAS system;
@item  --enable-top-level
This option is only available in MySQL.  It enables the option to interact with the MySQL server in
two different ways. As if we were on the MySQL Client Shell, and as if
we were using Datalog. 
@end table

@node MYDDAS Architecture, Loading MYDDAS, Requirements and Installation Guide, MYDDAS
@section MYDDAS Architecture

The system includes four main blocks that are put together through the
MYDDAS interface: the Yap Prolog compiler, the MySQL database system, an
ODBC layer and a Prolog to SQL compiler. Current effort is put on the
MySQL interface rather than on the ODBC interface. If you want to use
the full power of the MYDDAS interface we recommend you to use a MySQL
database. Other databases, such as Oracle, PostGres or Microsoft SQL
Server, can be interfaced through the ODBC layer, but with limited
performance and features support.  

The main structure of the MYDDAS interface is simple. Prolog queries
involving database goals are translated to SQL using the Prolog to SQL
compiler; then the SQL expression is sent to the database system, which
returns the set of tuples satisfying the query; and finally those tuples
are made available to the Prolog engine as terms. For recursive queries
involving database goals, the YapTab tabling engine provides the
necessary support for an efficient evaluation of such queries.

An important aspect of the MYDDAS interface is that for the programmer
the use of predicates which are defined in database relations is
completely transparent. An example of this transparent support is the
Prolog cut operator, which has exactly the same behaviour from
predicates defined in the Prolog program source code, or from predicates
defined in database as relations.

@node Loading MYDDAS, Connecting to and disconnecting from a Database Server, MYDDAS Architecture, MYDDAS 
@section Loading MYDDAS

Begin by starting YAP and loading the library
@code{use_module(library(myddas))}.  This library already includes the
Prolog to SQL Compiler described in [2] and [1]. In MYDDAS this compiler
has been extended to support further constructs which allow a more
efficient SQL translation.  

@node Connecting to and disconnecting from a Database Server, Accessing a Relation, Loading MYDDAS, MYDDAS
@section Connecting to and disconnecting from a Database Server


@table @code
@item db open(+,+,+,+,+). 
@findex db_open/5
@snindex db_open/5
@cnindex db_open/5

@item db open(+,+,+,+). 
@findex db_open/4
@snindex db_open/4
@cnindex db_open/4

@item db close(+). 
@findex db_close/1
@snindex db_close/1
@cnindex db_close/1

@item db_close.

@end table 

  Assuming the MySQL server is running and we have an account, we can
login to MySQL by invoking @code{db_open/5} as one of the following:
@example
?- db_open(mysql,Connection,Host/Database,User,Password). 
?- db_open(mysql,Connection,Host/Database/Port,User,Password).
?- db_open(mysql,Connection,Host/Database/UnixSocket,User,Password). 
?- db_open(mysql,Connection,Host/Database/Port/UnixSocket,User,Password).

@end example
If the login is successful, there will be a response of @code{yes}. For
instance:
 @example
?- db_open(mysql,con1,localhost/guest_db,guest,'').
@end example
uses the MySQL native interface, selected by the first argument, to open
a connection identified by the @code{con1} atom, to an instance of a
MySQL server running on host @code{localhost}, using database guest @code{db}
and user @code{guest} with empty @code{password}.  To disconnect from the @code{con1}
connection we use: 
@example
?- db_close(con1).
@end example
 Alternatively, we can use @code{db_open/4} and @code{db_close/0,} without an argument
to identify the connection. In this case the default connection is used,
with atom @code{myddas}. Thus using 
@example
?- db_open(mysql,localhost/guest_db,guest,''). 
?- db_close.  
@end example
or
@example
?- db_open(mysql,myddas,localhost/guest_db,guest,''). 
?- db_close(myddas). 
@end example
is exactly the same.

MYDDAS also supports ODBC. To connect to a database using an ODBC driver
you must have configured on your system a ODBC DSN. If so, the @code{db_open/4}
and @code{db_open/5} have the following mode:
@example
 ?- db_open(odbc,Connection,ODBC_DSN,User,Password). 
 ?- db_open(odbc,ODBC_DSN,User,Password).
@end example

For instance, if you do @code{db_open(odbc,odbc_dsn,guest,'')}. it will connect
to a database, through ODBC, using the definitions on the @code{odbc_dsn} DSN
configured on the system. The user will be the user @code{guest} with no
password.

@node Accessing a Relation, View Level Interface , Connecting to and disconnecting from a Database Server, MYDDAS 
@section Accessing a Relation

@table @code
@item db_import(+Conn,+RelationName,+PredName). 
@findex db_import/3
@snindex db_import/3
@cnindex db_import/3

@item db_import(+RelationName,+PredName).  
@findex db_import/2
@snindex db_import/2
@cnindex db_import/2
@end table

Assuming you have access permission for the relation you wish to import,
you can use @code{db_import/3} or @code{db_import/2} as:
@example
?- db_import(Conn,RelationName,PredName).
?- db_import(RelationName,PredName).
@end example
 where @var{RelationName}, is the name of
relation we wish to access, @var{PredName} is the name of the predicate we
wish to use to access the relation from YAP. @var{Conn}, is the connection
identifier, which again can be dropped so that the default myddas connection
is used. For instance, if we want to access the relation phonebook,
using the predicate @code{phonebook/3} we write: 
@example
?- db_import(con1,phonebook,phonebook). 
yes
?- phonebook(Letter,Name,Number).
Letter = 'D',
Name = 'John Doe',
Number = 123456789 ? 
yes
@end example
Backtracking can then be used to retrieve the next row
of the relation phonebook.  Records with particular field values may be
selected in the same way as in Prolog. (In particular, no mode
specification for database predicates is required). For instance: 
@example
?- phonebook(Letter,'John Doe',Letter). 
Letter = 'D', 
Number = 123456789 ?
yes
@end example
generates the query @example
SELECT A.Letter , 'John Doe' , A.Number 
FROM 'phonebook' A 
WHERE A.Name = 'John Doe';
@end example

@node View Level Interface, Accessing Tables in Data Sources Using SQL, Accessing a Relation, MYDDAS
@section View Level Interface

@table @code
@item db view(+,+,+).
@findex db_view/3
@snindex db_view/3
@cnindex db_view/3

@item db view(+,+).
@findex db_view/2
@snindex db_view/2
@cnindex db_view/2
@end table
If we import a database relation, such as an edge relation representing the edges of a directed graph, through
@example
?- db_import('Edge',edge). 
yes
@end example
and we then write a query to retrieve all the direct cycles in the
graph, such as
@example
?- edge(A,B), edge(B,A). 
A = 10, 
B = 20 ?
@end example
this is clearly inefficient [3], because of relation-level
access. Relation-level access means that a separate SQL query will be
generated for every goal in the body of the clause. For the second
@code{edge/2} goal, a SQL query is generated using the variable bindings that
result from the first @code{edge/2} goal execution. If the second
@code{edge/2} goal
fails, or if alternative solutions are demanded, backtracking access the
next tuple for the first @code{edge/2} goal and another SQL query will be
generated for the second @code{edge/2} goal. The generation of this large
number of queries and the communication overhead with the database
system for each of them, makes the relation-level approach inefficient.
To solve this problem the view level interface can be used for the
definition of rules whose bodies includes only imported database
predicates.  One can use the view level interface through the predicates
@code{db_view/3} and @code{db_view/2}:
@example
?- db_view(Conn,PredName(Arg_1,...,Arg_n),DbGoal).  
?- db_view(PredName(Arg_1,...,Arg_n),DbGoal).
@end example
 All arguments are standard Prolog terms. @var{Arg1} through @var{Argn}
define the attributes to be retrieved from the database, while
@var{DbGoal} defines the selection restrictions and join
conditions. @var{Conn} is the connection identifier, which again can be
dropped. Calling predicate @code{PredName/n} will retrieve database
tuples using a single SQL query generated for the @var{DbGoal}.  We next show
an example of a view definition for the direct cycles discussed
above. Assuming the declaration: 
@example
?- db_import('Edge',edge). 
yes
@end example
we
write:@example
?- db_view(direct_cycle(A,B),(edge(A,B), edge(B,A))). 
yes 
?- direct_cycle(A,B)). 
A = 10, 
B = 20 ?  
@end example
This call generates the SQL
statement: @example
SELECT A.attr1 , A.attr2
FROM Edge A , Edge B 
WHERE B.attr1 = A.attr2 AND B.attr2 = A.attr1;
@end example

Backtracking, as in relational level interface, can be used to retrieve the next row of the view.
The view interface also supports aggregate function predicates such as
@code{sum}, @code{avg}, @code{count}, @code{min} and @code{max}. For
instance:
@example
?- db_view(count(X),(X is count(B, B^edge(10,B)))).
@end example
generates the query :
@example
SELECT COUNT(A.attr2) 
FROM Edge A WHERE A.attr1 = 10;
@end example

To know how to use db @code{view/3}, please refer to Draxler's Prolog to
SQL Compiler Manual. 

@node Accessing Tables in Data Sources Using SQL, Insertion of Rows, View Level Interface , MYDDAS 
@section Accessing Tables in Data Sources Using SQL 

@table @code
@item db_sql(+,+,?). 
@findex db_sql/3
@snindex db_sql/3
@cnindex db_sql/3

@item db_sql(+,?).
@findex db_sql/2
@snindex db_sql/2
@cnindex db_sql/2
@end table

It is also possible to explicitly send a SQL query to the database server using
@example
?- db_sql(Conn,SQL,List).
?- db_sql(SQL,List).
@end example
where @var{SQL} is an arbitrary SQL expression, and @var{List} is a list
holding the first tuple of result set returned by the server. The result
set can also be navigated through backtracking.

Example:
@example
?- db_sql('SELECT * FROM phonebook',LA).
LA = ['D','John Doe',123456789] ?
@end example

@node  Insertion of Rows, Types of Attributes, Accessing Tables in Data Sources Using SQL, MYDDAS 
@section Insertion of Rows 

@table @code
@item db_assert(+,+). 
@findex db_assert/2
@snindex db_assert/2
@cnindex db_assert/2

@item db_assert(+).
@findex db_assert/1
@snindex db_assert/1
@cnindex db_assert/1

@end table

Assuming you have imported the related base table using
 @code{db_import/2} or @code{db_import/3}, you can insert to that table
 by using @code{db_assert/2} predicate any given fact.
@example
?- db_assert(Conn,Fact).
?- db_assert(Fact).
@end example
The second argument must be declared with all of its arguments bound to
constants. For example assuming @code{helloWorld} is imported through
@code{db_import/2}:
@example
?- db_import('Hello World',helloWorld).
yes
?- db_assert(helloWorld('A' ,'Ana',31)). 
yes
@end example
This, would generate the following query 
@example
INSERT INTO helloWorld
VALUES ('A','Ana',3)
@end example
which would insert into the helloWorld, the following row:
@code{A,Ana,31}. If we want to insert @code{NULL}  values into the
relation, we call @code{db_assert/2} with a uninstantiated variable in
the data base imported predicate. For example, the following query on
the YAP-prolog system:

@example
?- db_assert(helloWorld('A',NULL,31)).
yes
@end example

Would insert the row: @code{A,null value,31} into the relation
@code{Hello World}, assuming that the second row allows null values.

@table @code
@item db insert(+,+,+). 
@findex db_insert/3
@snindex db_insert/3
@cnindex db_insert/3

@item db insert(+,+).  
@findex db_insert/2
@snindex db_insert/2
@cnindex db_insert/2
@end table
 
This predicate would create a new database predicate, which will insert
any given tuple into the database.
@example
?- db_insert(Conn,RelationName,PredName).
?- db_insert(RelationName,PredName).
@end example
This would create a new predicate with name @var{PredName}, that will
insert tuples into the relation @var{RelationName}. is the connection
identifier. For example, if we wanted to insert the new tuple
@code{('A',null,31)} into the relation @code{Hello World}, we do: 
@example
?- db_insert('Hello World',helloWorldInsert). 
yes
?- helloWorldInsert('A',NULL,31).
yes
@end example

@node  Types of Attributes, Number of Fields, Insertion of Rows, MYDDAS 
@section Types of Attributes


@table @code
@item db_get_attributes_types(+,+,?).
@findex db_get_attributes_types/3
@snindex db_get_attributes_types/3
@cnindex db_get_attributes_types/3

@item db_get_attributes_types(+,?). 
@findex db_get_attributes_types/2
@snindex db_get_attributes_types/2
@cnindex db_get_attributes_types/2

@end table
 
The prototype for this predicate is the following: 
@example
?- db_get_attributes_types(Conn,RelationName,ListOfFields).
?- db_get_attributes_types(RelationName,ListOfFields). 
@end example

You can use the
predicate @code{db_get_attributes types/2} or @code{db_get_attributes_types/3}, to
know what are the names and attributes types of the fields of a given
relation. For example: 
@example
?- db_get_attributes_types(myddas,'Hello World',LA).
LA = ['Number',integer,'Name',string,'Letter',string] ? 
yes
@end example
where @t{Hello World} is the name of the relation and @t{myddas} is the
connection identifier. 

@node  Number of Fields, Describing a Relation, Types of Attributes, MYDDAS 
@section Number of Fields

@table @code
@item db_number_of_fields(+,?).
@findex db_number_of_fields/2
@snindex db_number_of_fields/2
@cnindex db_number_of_fields/2

@item db_number_of_fields(+,+,?).
@findex db_number_of_fields/3
@snindex db_number_of_fields/3
@cnindex db_number_of_fields/3
@end table
 
The prototype for this
predicate is the following:
@example
 ?- db_number_of_fields(Conn,RelationName,Arity).
 ?- db_number_of_fields(RelationName,Arity).  
@end example
You can use the predicate @code{db_number_of_fields/2} or
@code{db_number_of_fields/3} to know what is the arity of a given
relation. Example: 
@example
?- db_number_of_fields(myddas,'Hello World',Arity). 
Arity = 3 ? 
yes 
@end example
where @code{Hello World} is the name of the
relation and @code{myddas} is the connection identifier.

@node Describing a Relation, Enumerating Relations, Number of Fields, MYDDAS
@section Describing a Relation

@table @code
@item db_datalog_describe(+,+).
@findex db_datalog_describe/2
@snindex db_datalog_describe/2
@cnindex db_datalog_describe/2

@item db_datalog_describe(+).
@findex db_datalog_describe/1
@snindex db_datalog_describe/1
@cnindex db_datalog_describe/1
@end table
 

The db @code{datalog_describe/2} predicate does not really returns any
value. It simply prints to the screen the result of the MySQL describe
command, the same way as @code{DESCRIBE} in the MySQL prompt would.
@example
?- db_datalog_describe(myddas,'Hello World'). 
+----------+----------+------+-----+---------+-------+ 
|   Field  |  Type    | Null | Key | Default | Extra |
+----------+----------+------+-----+---------+-------+
+  Number  | int(11)  | YES  |     |   NULL  |       |
+  Name    | char(10) | YES  |     |   NULL  |       |
+  Letter  | char(1)  | YES  |     |   NULL  |       |
+----------+----------+------+-----+---------+-------+
yes
@end example

@table @code
@item db_describe(+,+).
@findex db_describe/2
@snindex db_describe/2
@cnindex db_describe/2

@item db_describe(+).
@findex db_describe/1
@snindex db_describe/1
@cnindex db_describe/1

@end table
 
The @code{db_describe/3} predicate does the same action as
@code{db_datalog_describe/2} predicate but with one major
difference. The results are returned by backtracking. For example, the
last query:
@example
 ?- db_describe(myddas,'Hello World',Term). 
Term = tableInfo('Number',int(11),'YES','',null(0),'') ? ;
Term = tableInfo('Name',char(10),'YES','',null(1),'' ? ; 
Term = tableInfo('Letter',char(1),'YES','',null(2),'') ? ;
no
@end example

@node Enumerating Relations, The MYDDAS MySQL Top Level, Describing a Relation, MYDDAS
@section Enumeration Relations

@table @code
@item db_datalog_show_tables(+).
@item db_datalog_show_tables
@end table
 

If we need to know what relations exists in a given MySQL Schema, we can use
the @code{db_datalog_show_tables/1} predicate. As @t{db_datalog_describe/2},
it does not returns any value, but instead prints to the screen the result of the
@code{SHOW TABLES} command, the same way as it would be in the MySQL prompt. 
@example
?- db_datalog_show_tables(myddas).
+-----------------+
| Tables_in_guest |
+-----------------+
|   Hello World   |
+-----------------+ 
yes
@end example

@table @code
@item db_show_tables(+, ?).
@findex db_show_tables/2
@snindex db_show_tables/2
@cnindex db_show_tables/2

@item db_show_tables(?)
@findex db_show_tables/1
@snindex db_show_tables/1
@cnindex db_show_tables/1

@end table
 
The @code{db_show_tables/2} predicate does the same action as
@code{db_show_tables/1} predicate but with one major difference. The
results are returned by backtracking. For example, given the last query:
@example
?- db_show_tables(myddas,Table).
Table = table('Hello World') ? ;
no
@end example

@node The MYDDAS MySQL Top Level, Other MYDDAS Properties, Enumerating Relations, MYDDAS
@section The MYDDAS MySQL Top Level


@table @code
@item db_top_level(+,+,+,+,+). 
@findex db_top_level/5
@snindex db_top_level/5
@cnindex db_top_level/5

@item db_top_level(+,+,+,+).
@findex db_top_level/4
@snindex db_top_level/4
@cnindex db_top_level/4

@end table
 
Through MYDDAS is also possible to access the MySQL Database Server, in
the same wthe mysql client. In this mode, is possible to query the
SQL server by just using the standard SQL language. This mode is exactly the same as 
different from the standard mysql client. We can use this
mode, by invoking the db top level/5. as one of the following:
@example
?- db_top_level(mysql,Connection,Host/Database,User,Password). 
?- db_top_level(mysql,Connection,Host/Database/Port,User,Password). 
?- db_top_level(mysql,Connection,Host/Database/UnixSocket,User,Password). 
?- db_top_level(mysql,Connection,Host/Database/Port/UnixSocket,User,Password).
@end example

Usage is similar as the one described for the @code{db_open/5} predicate
discussed above. If the login is successful, automatically the prompt of
the mysql client will be used.  For example:
@example
 ?- db_top_level(mysql,con1,localhost/guest_db,guest,''). 
@end  example
opens a
connection identified by the @code{con1} atom, to an instance of a MySQL server
running on host @code{localhost}, using database guest @code{db} and user @code{guest} with
empty password. After this is possible to use MYDDAS as the mysql
client.
@example
  ?- db_top_level(mysql,con1,localhost/guest_db,guest,''). 
Reading table information for completion of table and column names
You can turn off this feature to get a quicker startup with -A

Welcome to the MySQL monitor.
Commands end with ; or \g.

Your MySQL connection id is 4468 to server version: 4.0.20
Type 'help;' or '\h' for help.
Type '\c' to clear the buffer. 
mysql> exit
Bye 
yes
?- 
@end  example

@node Other MYDDAS Properties, , The MYDDAS MySQL Top Level , MYDDAS 
@section Other MYDDAS Properties 


@table @code
@item db_verbose(+). 
@item db_top_level(+,+,+,+).
@end table
 
When we ask a question to YAP, using a predicate asserted by
@code{db_import/3}, or by @code{db_view/3}, this will generate a SQL
@code{QUERY}. If we want to see that query, we must to this at a given
point in our session on YAP.
@example
?- db_verbose(1).
yes 
?- 
@end example
If we want to
disable this feature, we must call the @code{db_verbose/1} predicate with the value 0.

@table @code
@item db_module(?). 
@findex db_module/1
@snindex db_module/1
@cnindex db_module/1

@end table

When we create a new database predicate, by using @code{db_import/3},
@code{db_view/3} or @code{db_insert/3}, that predicate will be asserted
by default on the @code{user} module. If we want to change this value, we can
use the @code{db_module/1} predicate to do so.
@example
?- db_module(lists).
yes
?-
@end example
By executing this predicate, all of the predicates asserted by the
predicates enumerated earlier will created in the lists module.
If we want to put back the value on default, we can manually put the
value user. Example: 
@example
?- db_module(user).
yes
?-
@end example

We can also see in what module the predicates are being asserted by doing:
@example
?- db_module(X). 
X=user
yes
 ?-
@end example

@table @code
@item db_my_result_set(?).
@findex db_my_result_set/1
@snindex db_my_result_set/1
@cnindex db_my_result_set/1

@end table


The MySQL C API permits two modes for transferring the data generated by
a query to the client, in our case YAP. The first mode, and the default
mode used by the MYDDAS-MySQL, is to store the result. This mode copies all the
information generated to the client side.@example
?- db_my_result_set(X).
X=store_result
yes
@end example


The other mode that we can use is use result. This one uses the result
set created directly from the server. If we want to use this mode, he
simply do
@example
 ?- db_my_result_set(use_result). 
yes
@end example
After this command, all
of the database predicates will use use result by default. We can change
this by doing again @code{db_my_result_set(store_result)}.  

@table @code
@item db_my_sql_mode(+Conn,?SQL_Mode).
@findex db_my_sql_mode/2
@snindex db_my_sql_mode/2
@cnindex db_my_sql_mode/2

@item db_my_sql_mode(?SQL_Mode).
@findex db_my_sql_mode/1
@snindex db_my_sql_mode/1
@cnindex db_my_sql_mode/1

@end table

The MySQL server allows the user to change the SQL mode. This can be
very useful for debugging proposes. For example, if we want MySQL server
not to ignore the INSERT statement warnings and instead of taking
action, report an error, we could use the following SQL mode.
@example
  ?-db_my_sql_mode(traditional). yes
@end example
You can see the available SQL Modes at the MySQL homepage at
@url{http://www.mysql.org}.

@node Threads, Parallelism, MYDDAS, Extensions
@chapter Threads

YAP implements a SWI-Prolog compatible multithreading
library. Like in SWI-Prolog, Prolog threads have their own stacks and
only share the Prolog @emph{heap}: predicates, records, flags and other
global non-backtrackable data.  The package is based on the POSIX thread
standard (Butenhof:1997:PPT) used on most popular systems except
for MS-Windows.

@comment On Windows it uses the
@comment \url[pthread-win32]{http://sources.redhat.com/pthreads-win32/} emulation
@comment of POSIX threads mixed with the Windows native API for smoother and
@comment faster operation.

@menu
Subnodes of Threads
* Creating and Destroying Prolog Threads::               
* Monitoring Threads::            
* Thread Communication::   
* Thread Synchronisation::                 

Subnodes of Thread Communication
* Message Queues::
* Signalling Threads::            
* Threads and Dynamic Predicates::   
@end menu

@node Creating and Destroying Prolog Threads, Monitoring Threads, ,Threads
@section Creating and Destroying Prolog Threads

@table @code

@item thread_create(:@var{Goal}, -@var{Id}, +@var{Options})
@findex thread_create/3
@snindex thread_create/3
@cnindex thread_create/3

Create a new Prolog thread (and underlying C-thread) and start it
by executing @var{Goal}.  If the thread is created successfully, the
thread-identifier of the created thread is unified to @var{Id}.
@var{Options} is a list of options.  Currently defined options are:

@table @code
    @item stack
Set the limit in K-Bytes to which the Prolog stacks of
this thread may grow.  If omitted, the limit of the calling thread is
used.  See also the  commandline @code{-S} option.

    @item trail
Set the limit in K-Bytes to which the trail stack of this thread may
grow.  If omitted, the limit of the calling thread is used. See also the
commandline option @code{-T}.

    @item alias
Associate an alias-name with the thread.  This named may be used to
refer to the thread and remains valid until the thread is joined
(see @code{thread_join/2}).

    @item at_exit
Define an exit hook for the thread.  This hook is called when the thread
terminates, no matter its exit status.

    @item detached
If @code{false} (default), the thread can be waited for using
@code{thread_join/2}. @code{thread_join/2} must be called on this thread
to reclaim the all resources associated to the thread. If @code{true},
the system will reclaim all associated resources automatically after the
thread finishes. Please note that thread identifiers are freed for reuse
after a detached thread finishes or a normal thread has been joined.
See also @code{thread_join/2} and @code{thread_detach/1}.
@end table

The @var{Goal} argument is @emph{copied} to the new Prolog engine.
This implies further instantiation of this term in either thread does
not have consequences for the other thread: Prolog threads do not share
data from their stacks.

@item thread_create(:@var{Goal}, -@var{Id})
@findex thread_create/2
@snindex thread_create/2
@cnindex thread_create/2

Create a new Prolog thread using default options. See @code{thread_create/3}.

@item thread_create(:@var{Goal})
@findex thread_create/1
@snindex thread_create/1
@cnindex thread_create/1

Create a new Prolog detached thread using default options. See @code{thread_create/3}.

@item thread_self(-@var{Id})
@findex thread_self/1
@snindex thread_self/1
@cnindex thread_self/1
Get the Prolog thread identifier of the running thread.  If the thread
has an alias, the alias-name is returned.

@item thread_join(+@var{Id}, -@var{Status})
@findex thread_join/2
@snindex thread_join/2
@cnindex thread_join/2
Wait for the termination of thread with given @var{Id}.  Then unify the
result-status of the thread with @var{Status}.  After this call,
@var{Id} becomes invalid and all resources associated with the thread
are reclaimed.  Note that threads with the attribute @code{detached}
@code{true} cannot be joined.  See also @code{current_thread/2}.

A thread that has been completed without @code{thread_join/2} being
called on it is partly reclaimed: the Prolog stacks are released and the
C-thread is destroyed. A small data-structure representing the
exit-status of the thread is retained until @code{thread_join/2} is called on
the thread.  Defined values for @var{Status} are:

@table @code
    @item true
The goal has been proven successfully.

    @item false
The goal has failed.

    @item exception(@var{Term})
 The thread is terminated on an
exception.  See @code{print_message/2} to turn system exceptions into
readable messages.

    @item exited(@var{Term})
The thread is terminated on @code{thread_exit/1} using the argument @var{Term}.
@end table


@item thread_detach(+@var{Id})
@findex thread_detach/1
@snindex thread_detach/1
@cnindex thread_detach/1
Switch thread into detached-state (see @code{detached} option at
@code{thread_create/3} at runtime.  @var{Id} is the identifier of the thread
placed in detached state.

One of the possible applications is to simplify debugging. Threads that
are created as @code{detached} leave no traces if they crash. For
not-detached threads the status can be inspected using
@code{current_thread/2}.  Threads nobody is waiting for may be created
normally and detach themselves just before completion.  This way they
leave no traces on normal completion and their reason for failure can be
inspected.

@item thread_yield
@findex thread_yield/0
@snindex thread_yield/0
@cnindex thread_yield/0
Voluntarily relinquish the processor.

@item thread_exit(+@var{Term})
@findex thread_exit/1
@snindex thread_exit/1
@cnindex thread_exit/1
Terminates the thread immediately, leaving @code{exited(@var{Term})} as
result-state for @code{thread_join/2}.  If the thread has the attribute
@code{detached} @code{true} it terminates, but its exit status cannot be
retrieved using @code{thread_join/2} making the value of @var{Term}
irrelevant.  The Prolog stacks and C-thread are reclaimed.

@item thread_at_exit(:@var{Term})
@findex thread_at_exit/1
@snindex thread_at_exit/1
@cnindex thread_at_exit/1
Run @var{Goal} just before releasing the thread resources. This is to
be compared to @code{at_halt/1}, but only for the current
thread. These hooks are ran regardless of why the execution of the
thread has been completed. As these hooks are run, the return-code is
already available through @code{thread_property/2} using the result of
@code{thread_self/1} as thread-identifier. If you want to guarantee the 
execution of an exit hook no matter how the thread terminates (the thread 
can be aborted before reaching the @code{thread_at_exit/1} call), consider
using instead the @code{at_exit/1} option of @code{thread_create/3}. 

@item thread_setconcurrency(+@var{Old}, -@var{New})
@findex thread_setconcurrency/2
@snindex thread_setconcurrency/2
@cnindex thread_setconcurrency/2
Determine the concurrency of the process, which is defined as the
maximum number of concurrently active threads. `Active' here means
they are using CPU time. This option is provided if the
thread-implementation provides
@code{pthread_setconcurrency()}. Solaris is a typical example of this
family. On other systems this predicate unifies @var{Old} to 0 (zero)
and succeeds silently.

@item thread_sleep(+@var{Time})
@findex thread_sleep/1
@snindex thread_sleep/1
@cnindex thread_sleep/1
Make current thread sleep for @var{Time} seconds. @var{Time} may be an
integer or a floating point number. When time is zero or a negative value 
the call succeeds and returns immediately. This call should not be used if
alarms are also being used.
@end table


@node Monitoring Threads, Thread Communication,Creating and Destroying Prolog Threads,Threads
@section Monitoring Threads

Normal multi-threaded applications should not need these the predicates
from this section because almost any usage of these predicates is
unsafe. For example checking the existence of a thread before signalling
it is of no use as it may vanish between the two calls. Catching
exceptions using @code{catch/3} is the only safe way to deal with
thread-existence errors.

These predicates are provided for diagnosis and monitoring tasks.


@table @code
@item thread_property(?@var{Id}, ?@var{Property})
@findex thread_property/2
@snindex thread_property/2
@cnindex thread_property/2
Enumerates the properties of the specified thread.
Calling @code{thread_property/2} does not influence any thread.  See also
@code{thread_join/2}.  For threads that have an alias-name, this name can
be used in @var{Id} instead of the numerical thread identifier.
@var{Property} is one of:

@table @code
    @item status(@var{Status})
The thread status of a thread (see below).

    @item alias(@var{Alias})
The thread alias, if it exists.

    @item at_exit(@var{AtExit})
The thread exit hook, if defined (not available if the thread is already terminated).

    @item detached(@var{Boolean})
The detached state of the thread.

    @item stack(@var{Size})
The thread stack data-area size.

    @item trail(@var{Size})
The thread trail data-area size.

    @item system(@var{Size})
The thread system data-area size.
@end table

@item current_thread(+@var{Id}, -@var{Status})
@findex current_thread/2
@snindex current_thread/2
@cnindex current_thread/2
Enumerates identifiers and status of all currently known threads.
Calling @code{current_thread/2} does not influence any thread.  See also
@code{thread_join/2}.  For threads that have an alias-name, this name is
returned in @var{Id} instead of the numerical thread identifier.
@var{Status} is one of:

@table @code
    @item running
The thread is running.  This is the initial status of a thread.  Please
note that threads waiting for something are considered running too.

    @item false
The @var{Goal} of the thread has been completed and failed.

    @item true
The @var{Goal} of the thread has been completed and succeeded.

    @item exited(@var{Term})
The @var{Goal} of the thread has been terminated using @code{thread_exit/1}
with @var{Term} as argument.  If the underlying native thread has
exited (using pthread_exit()) @var{Term} is unbound.

    @item exception(@var{Term})
The @var{Goal} of the thread has been terminated due to an uncaught
exception (see @code{throw/1} and @code{catch/3}).
@end table

@item thread_statistics(+@var{Id}, +@var{Key}, -@var{Value})
@findex thread_statistics/3
@snindex thread_statistics/3
@cnindex thread_statistics/3
Obtains statistical information on thread @var{Id} as @code{statistics/2}
does in single-threaded applications.  This call returns all keys
of @code{statistics/2}, although only information statistics about the
stacks and CPU time yield different values for each thread.

@item mutex_statistics
@findex mutex_statistics/0
@snindex mutex_statistics/0
@cnindex mutex_statistics/0
Print usage statistics on internal mutexes and mutexes associated
with dynamic predicates.  For each mutex two numbers are printed:
the number of times the mutex was acquired and the number of
collisions: the number times the calling thread has to
wait for the mutex.  The collision-count is not available on
Windows as this would break portability to Windows-95/98/ME or
significantly harm performance.  Generally collision count is
close to zero on single-CPU hardware.

@item threads
@findex threads/0
@snindex threads/0
@cnindex threads/0
Prints a table of current threads and their status.
@end table


@node Thread Communication, Thread Synchronisation, Monitoring Threads, Threads
@section Thread communication

@menu
Subnodes of Thread Communication
* Message Queues::
* Signalling Threads::            
* Threads and Dynamic Predicates::   
@end menu

@node Message Queues, Signalling Threads, ,Thread Communication
@subsection Message Queues

Prolog threads can exchange data using dynamic predicates, database
records, and other globally shared data. These provide no suitable means
to wait for data or a condition as they can only be checked in an
expensive polling loop. @emph{Message queues} provide a means for
threads to wait for data or conditions without using the CPU.

Each thread has a message-queue attached to it that is identified
by the thread. Additional queues are created using
@code{message_queue_create/2}.

@table @code

@item thread_send_message(+@var{Term})
@findex thread_send_message/1
@snindex thread_send_message/1
@cnindex thread_send_message/1
Places @var{Term} in the message-queue of the thread running the goal. 
Any term can be placed in a message queue, but note that the term is 
copied to the receiving thread and variable-bindings are thus lost. 
This call returns immediately.

@item thread_send_message(+@var{QueueOrThreadId}, +@var{Term})
@findex thread_send_message/2
@snindex thread_send_message/2
@cnindex thread_send_message/2
Place @var{Term} in the given queue or default queue of the indicated
thread (which can even be the message queue of itself (see
@code{thread_self/1}). Any term can be placed in a message queue, but note that
the term is copied to the receiving thread and variable-bindings are
thus lost. This call returns immediately.

If more than one thread is waiting for messages on the given queue and
at least one of these is waiting with a partially instantiated
@var{Term}, the waiting threads are @emph{all} sent a wakeup signal,
starting a rush for the available messages in the queue.  This behaviour
can seriously harm performance with many threads waiting on the same
queue as all-but-the-winner perform a useless scan of the queue. If
there is only one waiting thread or all waiting threads wait with an
unbound variable an arbitrary thread is restarted to scan the queue.
@comment	\footnote{See the documentation for the POSIX thread functions
@comment		  pthread_cond_signal() v.s.\ pthread_cond_broadcastt()
@comment		  for background information.}

@item thread_get_message(?@var{Term})
@findex thread_get_message/1
@snindex thread_get_message/1
@cnindex thread_get_message/1
Examines the thread message-queue and if necessary blocks execution
until a term that unifies to @var{Term} arrives in the queue.  After
a term from the queue has been unified unified to @var{Term}, the
term is deleted from the queue and this predicate returns.

Please note that not-unifying messages remain in the queue.  After
the following has been executed, thread 1 has the term @code{gnu}
in its queue and continues execution using @var{A} is @code{gnat}.

@example
   <thread 1>
   thread_get_message(a(A)),

   <thread 2>
   thread_send_message(b(gnu)),
   thread_send_message(a(gnat)),
@end example

See also @code{thread_peek_message/1}.

@item message_queue_create(?@var{Queue})
@findex message_queue_create/1
@snindex message_queue_create/1
@cnindex message_queue_create/1
If @var{Queue} is an atom, create a named queue.  To avoid ambiguity
on @code{thread_send_message/2}, the name of a queue may not be in use
as a thread-name.  If @var{Queue} is unbound an anonymous queue is
created and @var{Queue} is unified to its identifier.

@item message_queue_destroy(+@var{Queue})
@findex message_queue_destroy/1
@snindex message_queue_destroy/1
@cnindex message_queue_destroy/1
Destroy a message queue created with @code{message_queue_create/1}.  It is
@emph{not} allows to destroy the queue of a thread.  Neither is it
allowed to destroy a queue other threads are waiting for or, for
anonymous message queues, may try to wait for later.

@item thread_get_message(+@var{Queue}, ?@var{Term})
@findex thread_get_message/2
@snindex thread_get_message/2
@cnindex thread_get_message/2
As @code{thread_get_message/1}, operating on a given queue. It is allowed to
peek into another thread's message queue, an operation that can be used
to check whether a thread has swallowed a message sent to it.

@item thread_peek_message(?@var{Term})
@findex thread_peek_message/1
@snindex thread_peek_message/1
@cnindex thread_peek_message/1
Examines the thread message-queue and compares the queued terms
with @var{Term} until one unifies or the end of the queue has been
reached.  In the first case the call succeeds (possibly instantiating
@var{Term}.  If no term from the queue unifies this call fails.

@item thread_peek_message(+@var{Queue}, ?@var{Term})
@findex thread_peek_message/2
@snindex thread_peek_message/2
@cnindex thread_peek_message/2
As @code{thread_peek_message/1}, operating on a given queue. It is allowed to
peek into another thread's message queue, an operation that can be used
to check whether a thread has swallowed a message sent to it.

@end table


Explicit message queues are designed with the @emph{worker-pool} model
in mind, where multiple threads wait on a single queue and pick up the
first goal to execute.  Below is a simple implementation where the
workers execute arbitrary Prolog goals.  Note that this example provides
no means to tell when all work is done. This must be realised using
additional synchronisation.

@example
%	create_workers(+Id, +N)
%	
%	Create a pool with given Id and number of workers.

create_workers(Id, N) :-
	message_queue_create(Id),
	forall(between(1, N, _),
	       thread_create(do_work(Id), _, [])).

do_work(Id) :-
	repeat,
	  thread_get_message(Id, Goal),
	  (   catch(Goal, E, print_message(error, E))
	  ->  true
	  ;   print_message(error, goal_failed(Goal, worker(Id)))
	  ),
	fail.

%	work(+Id, +Goal)
%	
%	Post work to be done by the pool

work(Id, Goal) :-
	thread_send_message(Id, Goal).
@end example

@node Signalling Threads, Threads and Dynamic Predicates,Message Queues, Thread Communication
@subsection Signalling Threads

These predicates provide a mechanism to make another thread execute some
goal as an @emph{interrupt}.  Signalling threads is safe as these
interrupts are only checked at safe points in the virtual machine.
Nevertheless, signalling in multi-threaded environments should be
handled with care as the receiving thread may hold a @emph{mutex}
(see @code{with_mutex/2}).  Signalling probably only makes sense to start
debugging threads and to cancel no-longer-needed threads with @code{throw/1},
where the receiving thread should be designed carefully do handle
exceptions at any point.

@table @code
@item thread_signal(+@var{ThreadId}, :@var{Goal})
@findex thread_signal/2
@snindex thread_signal/2
@cnindex thread_signal/2
Make thread @var{ThreadId} execute @var{Goal} at the first
opportunity.  In the current implementation, this implies at the first
pass through the @emph{Call-port}. The predicate @code{thread_signal/2}
itself places @var{Goal} into the signalled-thread's signal queue
and returns immediately.

Signals (interrupts) do not cooperate well with the world of
multi-threading, mainly because the status of mutexes cannot be
guaranteed easily.  At the call-port, the Prolog virtual machine
holds no locks and therefore the asynchronous execution is safe.

@var{Goal} can be any valid Prolog goal, including @code{throw/1} to make
the receiving thread generate an exception and @code{trace/0} to start
tracing the receiving thread.

@comment In the Windows version, the receiving thread immediately executes
@comment the signal if it reaches a Windows GetMessage() call, which generally
@comment happens of the thread is waiting for (user-)input.
@end table

@node Threads and Dynamic Predicates, , Signalling Threads, Thread Communication
@subsection Threads and Dynamic Predicates

Besides queues threads can share and exchange data using dynamic
predicates. The multi-threaded version knows about two types of
dynamic predicates. By default, a predicate declared @emph{dynamic}
(see @code{dynamic/1}) is shared by all threads. Each thread may
assert, retract and run the dynamic predicate. Synchronisation inside
Prolog guarantees the consistency of the predicate. Updates are
@emph{logical}: visible clauses are not affected by assert/retract
after a query started on the predicate. In many cases primitive from
thread synchronisation should be used to ensure application invariants on
the predicate are maintained.

Besides shared predicates, dynamic predicates can be declared with the
@code{thread_local/1} directive. Such predicates share their
attributes, but the clause-list is different in each thread.

@table @code
@item thread_local(@var{+Functor/Arity}) 
@findex thread_local/1 (directive)
@snindex thread_local/1 (directive)
@cnindex thread_local/1 (directive)
related to the dynamic/1 directive.  It tells the system that the
predicate may be modified using @code{assert/1}, @code{retract/1},
etc, during execution of the program.  Unlike normal shared dynamic
data however each thread has its own clause-list for the predicate.
As a thread starts, this clause list is empty.  If there are still
clauses as the thread terminates these are automatically reclaimed by
the system.  The @code{thread_local} property implies
the property @code{dynamic}.

Thread-local dynamic predicates are intended for maintaining
thread-specific state or intermediate results of a computation.

It is not recommended to put clauses for a thread-local predicate into
a file as in the example below as the clause is only visible from the
thread that loaded the source-file.  All other threads start with an
empty clause-list.

@example
:- thread_local
	foo/1.

foo(gnat).
@end example

@end table


@node Thread Synchronisation, , Thread Communication, Threads
@section Thread Synchronisation

All internal Prolog operations are thread-safe. This implies two Prolog
threads can operate on the same dynamic predicate without corrupting the
consistency of the predicate. This section deals with user-level
@emph{mutexes} (called @emph{monitors} in ADA or
@emph{critical-sections} by Microsoft).  A mutex is a
@emph{MUT}ual @emph{EX}clusive device, which implies at most one thread
can @emph{hold} a mutex.

Mutexes are used to realise related updates to the Prolog database.
With `related', we refer to the situation where a `transaction' implies
two or more changes to the Prolog database.  For example, we have a
predicate @code{address/2}, representing the address of a person and we want
to change the address by retracting the old and asserting the new
address.  Between these two operations the database is invalid: this
person has either no address or two addresses, depending on the
assert/retract order.

Here is how to realise a correct update:

@example
:- initialization
	mutex_create(addressbook).

change_address(Id, Address) :-
	mutex_lock(addressbook),
	retractall(address(Id, _)),
	asserta(address(Id, Address)),
	mutex_unlock(addressbook).
@end example


@table @code
@item mutex_create(?@var{MutexId})
@findex mutex_create/1
@snindex mutex_create/1
@cnindex mutex_create/1
Create a mutex.  if @var{MutexId} is an atom, a @emph{named} mutex is
created.  If it is a variable, an anonymous mutex reference is returned.
There is no limit to the number of mutexes that can be created.

@item mutex_destroy(+@var{MutexId})
@findex mutex_destroy/1
@snindex mutex_destroy/1
@cnindex mutex_destroy/1
Destroy a mutex.  After this call, @var{MutexId} becomes invalid and
further references yield an @code{existence_error} exception.

@item with_mutex(+@var{MutexId}, :@var{Goal})
@findex with_mutex/2
@snindex with_mutex/2
@cnindex with_mutex/2
Execute @var{Goal} while holding @var{MutexId}.  If @var{Goal} leaves
choicepoints, these are destroyed (as in @code{once/1}).  The mutex is unlocked
regardless of whether @var{Goal} succeeds, fails or raises an exception.
An exception thrown by @var{Goal} is re-thrown after the mutex has been
successfully unlocked.  See also @code{mutex_create/2}.

Although described in the thread-section, this predicate is also
available in the single-threaded version, where it behaves simply as
@code{once/1}.

@item mutex_lock(+@var{MutexId})
@findex mutex_lock/1
@snindex mutex_lock/1
@cnindex mutex_lock/1
Lock the mutex.  Prolog mutexes are @emph{recursive} mutexes: they
can be locked multiple times by the same thread.  Only after unlocking
it as many times as it is locked, the mutex becomes available for
locking by other threads. If another thread has locked the mutex the
calling thread is suspended until to mutex is unlocked.

If @var{MutexId} is an atom, and there is no current mutex with that
name, the mutex is created automatically using @code{mutex_create/1}.  This
implies named mutexes need not be declared explicitly.

Please note that locking and unlocking mutexes should be paired
carefully. Especially make sure to unlock mutexes even if the protected
code fails or raises an exception. For most common cases use
@code{with_mutex/2}, which provides a safer way for handling Prolog-level
mutexes.

@item mutex_trylock(+@var{MutexId})
@findex mutex_trylock/1
@snindex mutex_trylock/1
@cnindex mutex_trylock/1
As mutex_lock/1, but if the mutex is held by another thread, this
predicates fails immediately.

@item mutex_unlock(+@var{MutexId})
@findex mutex_unlock/1
@snindex mutex_unlock/1
@cnindex mutex_unlock/1
Unlock the mutex. This can only be called if the mutex is held by the
calling thread. If this is not the case, a @code{permission_error}
exception is raised.

@item mutex_unlock_all
@findex mutex_unlock_all/0
@snindex mutex_unlock_all/0
@cnindex mutex_unlock_all/0
Unlock all mutexes held by the current thread.  This call is especially
useful to handle thread-termination using @code{abort/0} or exceptions.  See
also @code{thread_signal/2}.

@item current_mutex(?@var{MutexId}, ?@var{ThreadId}, ?@var{Count})
@findex current_mutex/3
@snindex current_mutex/3
@cnindex current_mutex/3
Enumerates all existing mutexes.  If the mutex is held by some thread,
@var{ThreadId} is unified with the identifier of the holding thread and
@var{Count} with the recursive count of the mutex. Otherwise,
@var{ThreadId} is @code{[]} and @var{Count} is 0.
@end table


@node Parallelism, Tabling, Threads, Extensions
@chapter Parallelism

@cindex parallelism
@cindex or-parallelism
There has been a sizeable amount of work on an or-parallel
implementation for YAP, called @strong{YAPOr}. Most of this work has
been performed by Ricardo Rocha. In this system parallelism is exploited
implicitly by running several alternatives in or-parallel. This option
can be enabled from the @code{configure} script or by checking the
system's @code{Makefile}.

@strong{YAPOr} is still a very experimental system, going through rapid
development. The following restrictions are of note:

@itemize @bullet
@item @strong{YAPOr} currently only supports the Linux/X86 and SPARC/Solaris
platforms. Porting to other Unix-like platforms should be straightforward.

@item @strong{YAPOr} does not support parallel updates to the
data-base.

@item @strong{YAPOr} does not support opening or closing of streams during
parallel execution.

@item Garbage collection and stack shifting are not supported in
@strong{YAPOr}.  

@item Built-ins that cause side-effects can only be executed when
left-most in the search-tree. There are no primitives to provide
asynchronous or cavalier execution of these built-ins, as in Aurora or
Muse.

@item YAP does not support voluntary suspension of work.
@end itemize

We expect that some of these restrictions will be removed in future
releases.

@node Tabling, Low Level Tracing, Parallelism , Extensions
@chapter Tabling
@cindex tabling

@strong{YAPTab} is the tabling engine that extends YAP's execution
model to support tabled evaluation for definite programs. YAPTab was
implemented by Ricardo Rocha and its implementation is largely based
on the ground-breaking design of the XSB Prolog system, which
implements the SLG-WAM. Tables are implemented using tries and YAPTab
supports the dynamic intermixing of batched scheduling and local
scheduling at the subgoal level. Currently, the following restrictions
are of note:
@itemize @bullet
@item YAPTab does not handle tabled predicates with loops through negation (undefined behaviour).
@item YAPTab does not handle tabled predicates with cuts (undefined behaviour).
@item YAPTab does not support coroutining (configure error).
@item YAPTab does not support tabling dynamic predicates (permission error).
@end itemize

To experiment with YAPTab use @code{--enable-tabling} in the configure
script or add @code{-DTABLING} to @code{YAP_EXTRAS} in the system's
@code{Makefile}. We next describe the set of built-ins predicates
designed to interact with YAPTab and control tabled execution:

@table @code
@item table +@var{P}
@findex table/1
@snindex table/1
@cnindex table/1
Declares predicate @var{P} (or a list of predicates
@var{P1},...,@var{Pn} or [@var{P1},...,@var{Pn}]) as a tabled
predicate. @var{P} must be written in the form
@var{name/arity}. Examples:
@example
:- table son/3.
:- table father/2.
:- table mother/2.
@end example
@noindent or
@example
:- table son/3, father/2, mother/2.
@end example
@noindent or
@example
:- table [son/3, father/2, mother/2].
@end example

@item is_tabled(+@var{P})
@findex is_tabled/1
@snindex is_tabled/1
@cnindex is_tabled/1
Succeeds if the predicate @var{P} (or a list of predicates
@var{P1},...,@var{Pn} or [@var{P1},...,@var{Pn}]), of the form
@var{name/arity}, is a tabled predicate.

@item tabling_mode(+@var{P},?@var{Mode})
@findex tabling_mode/2
@snindex tabling_mode/2
@cnindex tabling_mode/2
Sets or reads the default tabling mode for a tabled predicate @var{P}
(or a list of predicates @var{P1},...,@var{Pn} or
[@var{P1},...,@var{Pn}]). The list of @var{Mode} options includes:
@table @code
@item batched
      Defines that, by default, batched scheduling is the scheduling
      strategy to be used to evaluated calls to predicate @var{P}.
@item local
      Defines that, by default, local scheduling is the scheduling
      strategy to be used to evaluated calls to predicate @var{P}.
@item exec_answers
      Defines that, by default, when a call to predicate @var{P} is
      already evaluated (completed), answers are obtained by executing
      compiled WAM-like code directly from the trie data
      structure. This reduces the loading time when backtracking, but
      the order in which answers are obtained is undefined.
@item load_answers
      Defines that, by default, when a call to predicate @var{P} is
      already evaluated (completed), answers are obtained (as a
      consumer) by loading them from the trie data structure. This
      guarantees that answers are obtained in the same order as they
      were found. Somewhat less efficient but creates less choice-points.
@end table
The default tabling mode for a new tabled predicate is @code{batched}
and @code{exec_answers}. To set the tabling mode for all predicates at
once you can use the @code{yap_flag/2} predicate as described next.

@item yap_flag(tabling_mode,?@var{Mode})
@findex tabling_mode (yap_flag/2 option)
Sets or reads the tabling mode for all tabled predicates. The list of
@var{Mode} options includes:
@table @code
@item default
      Defines that (i) all calls to tabled predicates are evaluated
      using the predicate default mode, and that (ii) answers for all
      completed calls are obtained by using the predicate default mode.
@item batched
      Defines that all calls to tabled predicates are evaluated using
      batched scheduling. This option ignores the default tabling mode
      of each predicate.
@item local
      Defines that all calls to tabled predicates are evaluated using
      local scheduling. This option ignores the default tabling mode
      of each predicate.
@item exec_answers
      Defines that answers for all completed calls are obtained by
      executing compiled WAM-like code directly from the trie data
      structure. This option ignores the default tabling mode
      of each predicate.
@item load_answers
      Defines that answers for all completed calls are obtained by
      loading them from the trie data structure. This option ignores
      the default tabling mode of each predicate.
@end table

@item abolish_table(+@var{P})
@findex abolish_table/1
@snindex abolish_table/1
@cnindex abolish_table/1
Removes all the entries from the table space for predicate @var{P} (or
a list of predicates @var{P1},...,@var{Pn} or
[@var{P1},...,@var{Pn}]). The predicate remains as a tabled predicate.

@item abolish_all_tables/0
@findex abolish_all_tables/0
@snindex abolish_all_tables/0
@cnindex abolish_all_tables/0
Removes all the entries from the table space for all tabled
predicates. The predicates remain as tabled predicates.

@item show_table(+@var{P})
@findex show_table/1
@snindex show_table/1
@cnindex show_table/1
Prints table contents (subgoals and answers) for predicate @var{P}
(or a list of predicates @var{P1},...,@var{Pn} or
[@var{P1},...,@var{Pn}]).

@item table_statistics(+@var{P})
@findex table_statistics/1
@snindex table_statistics/1
@cnindex table_statistics/1
Prints table statistics (subgoals and answers) for predicate @var{P}
(or a list of predicates @var{P1},...,@var{Pn} or
[@var{P1},...,@var{Pn}]).

@item tabling_statistics/0
@findex tabling_statistics/0
@snindex tabling_statistics/0
@cnindex tabling_statistics/0
Prints statistics on space used by all tables.
@end table


@node Low Level Tracing, Low Level Profiling, Tabling, Extensions
@chapter Tracing at Low Level

It is possible to follow the flow at abstract machine level if
YAP is compiled with the flag @code{LOW_LEVEL_TRACER}. Note
that this option is of most interest to implementers, as it quickly generates
an huge amount of information.

Low level tracing can be toggled from an interrupt handler by using the
option @code{T}. There are also two built-ins that activate and
deactivate low level tracing:

@table @code
@item start_low_level_trace
@findex start_low_level_trace/0
@snindex start_low_level_trace/0
@cnindex start_low_level_trace/0
Begin display of messages at procedure entry and retry.

@item stop_low_level_trace
@findex start_low_level_trace/0
@snindex start_low_level_trace/0
@cnindex start_low_level_trace/0
Stop display of messages at procedure entry and retry.
@end table

Note that this compile-time option will slow down execution.

@node Low Level Profiling, , Low Level Tracing, Extensions
@chapter Profiling the Abstract Machine

Implementors may be interested in detecting on which abstract machine
instructions are executed by a program. The @code{ANALYST} flag can give
WAM level information. Note that this option slows down execution very
substantially, and is only of interest to developers of the system
internals, or to system debuggers.

@table @code
@item reset_op_counters
@findex reset_op_counters/0
@snindex reset_op_counters/0
@cnindex reset_op_counters/0
Reinitialize all counters.

@item show_op_counters(+@var{A})
@findex show_op_counters/1
@snindex show_op_counters/1
@cnindex show_op_counters/1
Display the current value for the counters, using label @var{A}. The
label must be an atom.

@item show_ops_by_group(+@var{A})
@findex show_ops_by_group/1
@snindex show_ops_by_group/1
@cnindex show_ops_by_group/1
Display the current value for the counters, organized by groups, using
label @var{A}. The label must be an atom.

@end table

@node Debugging,Efficiency,Extensions,Top 
@chapter Debugging

@menu
* Deb Preds:: Debugging Predicates
* Deb Interaction:: Interacting with the debugger
@end menu

@node Deb Preds, Deb Interaction, , Debugging
@section Debugging Predicates

The following predicates are available to control the debugging of
programs:

@table @code
@item debug
@findex debug/0
@saindex debug/0
@cyindex debug/0
Switches the debugger on.

@item debugging
@findex debugging/0
@syindex debugging/0
@cyindex debugging/0
Outputs status information about the debugger which includes the leash
mode and the existing spy-points, when the debugger is on.

@item nodebug
@findex nodebug/0
@syindex nodebug/0
@cyindex nodebug/0
Switches the debugger off.

@item spy +@var{P}
@findex spy/1
@syindex spy/1
@cyindex spy/1
 Sets spy-points on all the predicates represented by
@var{P}. @var{P} can either be a single specification or a list of 
specifications. Each one must be of the form @var{Name/Arity} 
or @var{Name}. In the last case all predicates with the name 
@var{Name} will be spied. As in C-Prolog, system predicates and 
predicates written in C, cannot be spied.

@item nospy +@var{P}
@findex nospy/1
@syindex nospy/1
@cyindex nospy/1
 Removes spy-points from all predicates specified by @var{P}.
The possible forms for @var{P} are the same as in @code{spy P}.

@item nospyall
@findex nospyall/0
@syindex nospyall/0
@cnindex nospyall/0
Removes all existing spy-points.

@item notrace
Switches off the debugger and stops tracing.

@item leash(+@var{M})
@findex leash/1
@syindex leash/1
@cyindex leash/1
 Sets leashing mode to @var{M}.
The mode can be specified as:
@table @code
@item full
prompt on Call, Exit, Redo and Fail
@item tight
prompt on Call, Redo and Fail
@item half
prompt on Call and Redo
@item loose
prompt on Call
@item off
never prompt
@item none
never prompt, same as @code{off}
@end table
@noindent
The initial leashing mode is @code{full}.


@noindent
The user may also specify directly the debugger ports 
where he wants to be prompted. If the argument for leash 
is a number @var{N}, each of lower four bits of the number is used to
control prompting at one the ports of the box model. The debugger will 
prompt according to the following conditions:

@itemize @bullet
@item
if @code{N/\ 1 =\= 0}  prompt on fail 
@item
if @code{N/\ 2 =\= 0} prompt on redo
@item
if @code{N/\ 4 =\= 0} prompt on exit
@item
if @code{N/\ 8 =\= 0} prompt on call
@end itemize
@noindent
Therefore, @code{leash(15)} is equivalent to @code{leash(full)} and
@code{leash(0)} is equivalent to @code{leash(off)}.

@noindent
Another way of using @code{leash} is to give it a list with the names of
the ports where the debugger should stop. For example,
@code{leash([call,exit,redo,fail])} is the same as @code{leash(full)} or
@code{leash(15)} and @code{leash([fail])} might be used instead of
@code{leash(1)}.

@item spy_write(+@var{Stream},Term)
@findex spy_write/2
@snindex spy_write/2
@cnindex spy_write/2
If defined by the user, this predicate will be used to print goals by
the debugger instead of @code{write/2}.

@item trace
@findex trace/0
@syindex trace/0
@cyindex trace/0
Switches on the debugger and starts tracing.

@item notrace
@findex notrace/0
@syindex notrace/0
@cyindex notrace/0
Ends tracing and exits the debugger. This is the same as
@code{nodebug/0}.

@end table


@node Deb Interaction, , Deb Preds, Debugging
@section Interacting with the debugger

Debugging with YAP is similar to debugging with C-Prolog. Both systems
include a procedural debugger, based on Byrd's four port model. In this
model, execution is seen at the procedure level: each activation of a
procedure is seen as a box with control flowing into and out of that
box.

 In the four port model control is caught at four key points: before 
entering the procedure, after exiting the procedure (meaning successful 
evaluation of all queries activated by the procedure), after backtracking but 
before trying new alternative to the procedure and after failing the 
procedure. Each one of these points is named a port:

@smallexample
@group
           *--------------------------------------*
   Call    |                                      |    Exit
---------> +  descendant(X,Y) :- offspring(X,Y).  + --------->
           |                                      |
           |  descendant(X,Z) :-                  |
<--------- +     offspring(X,Y), descendant(Y,Z). + <---------
   Fail    |                                      |    Redo
           *--------------------------------------*
@end group
@end smallexample

@table @code

@item Call
The call port is activated before initial invocation of
procedure. Afterwards, execution will try to match the goal with the
head of existing clauses for the procedure.
@item Exit
This port is activated if the procedure succeeds.
Control will  now leave the procedure and return to its ancestor.
@item Redo
if the goal, or goals, activated after the call port
fail  then backtracking will eventually return control to this procedure
through  the redo port.
@item Fail
If all clauses for this predicate fail, then the
invocation fails,  and control will try to redo the ancestor of this
invocation.
@end table

To start debugging, the user will either call @code{trace} or spy the
relevant procedures, entering debug mode, and start execution of the
program. When finding the first spy-point, YAP's debugger will take
control and show a message of the form:

@example
* (1)  call:  quicksort([1,2,3],_38) ?
@end example

 The debugger message will be shown while creeping, or at spy-points,
and it includes four or five fields:

@itemize @bullet
@item
 The first three characters are used to point out special states of the
debugger. If the port is exit and the first character is '?', the
current call is non-deterministic, that is, it still has alternatives to
be tried. If the second character is a @code{*}, execution is at a
spy-point. If the third character is a @code{>}, execution has returned
either from a skip, a fail or a redo command.
@item
 The second field is the activation number, and uniquely identifies the
activation. The number will start from 1 and will be incremented for
each activation found by the debugger.
@item
 In the third field, the debugger shows the active port.
@item
 The fourth field is the goal. The goal is written by
 @code{write_term/3} on the standard error stream, using the options
 given by @code{debugger_print_options}.
@end itemize

 If the active port is leashed, the debugger will prompt the user with a
@code{?}, and wait for a command. A debugger command is just a
character, followed by a return. By default, only the call and redo
entries are leashed, but the @code{leash/1} predicate can be used in
order to make the debugger stop where needed.

 There are several commands available, but the user only needs to 
remember the help command, which is @code{h}. This command shows all the 
available options, which are:
@table @code
@item c - creep
this command makes YAP continue execution and stop at the next
leashed port.
@item return - creep
the same as c
@item l - leap
YAP will execute until it meets a port for a spied predicate; this mode
keeps all computation history for debugging purposes, so it is more
expensive than standard execution. Use @t{k} or @t{z} for fast execution.
@item k - quasi-leap
similar to leap but faster since the computation history is
not kept; useful when leap becomes too slow.
@item z - zip
same as @t{k}
@item s - skip
YAP will continue execution without showing any messages until
returning to the current activation. Spy-points will be  ignored in this
mode. Note that this command keeps all debugging history, use @t{t} for fast execution. This command is meaningless, and therefore illegal, in the fail
and exit ports.
@item t - fast-skip
similar to skip but faster since computation history is not
kept; useful if skip becomes slow.
@item f [@var{GoalId}] - fail
If given no argument, forces YAP to fail the goal, skipping the fail
port and backtracking to the parent. 
 If @t{f} receives a goal number as
the argument, the command fails all the way to the goal. If goal @var{GoalId} has completed execution, YAP fails until meeting the first active ancestor.
@item r [@var{GoalId}] - retry
This command forces YAP to jump back call to the port. Note that any
side effects of the goal cannot be undone. This command is not available
at the call port.  If @t{f} receives a goal number as the argument, the
command retries goal @var{GoalId} instead. If goal @var{GoalId} has
completed execution, YAP fails until meeting the first active ancestor.
    
@item a - abort
execution will be aborted, and the interpreter will return to the
top-level. YAP disactivates debug mode, but spypoints are not removed.
@item n - nodebug
stop debugging and continue execution. The command will not clear active
spy-points.
@item e - exit
leave YAP.
@item h - help
show the debugger commands.
@item ! Query
execute a query. YAP will not show the result of the query.
@item b - break
break active execution and launch a break level. This is  the same as @code{!
break}.
@item + - spy this goal
start spying the active goal. The same as @code{! spy  G} where @var{G}
is the active goal.
@item - - nospy this goal
stop spying the active goal. The same as @code{! nospy G} where @var{G} is
the active goal.
@item p - print
shows the active goal using print/1
@item d - display
shows the active goal using display/1
@item <Depth - debugger write depth
sets the maximum write depth, both for composite terms and lists, that
will be used by the debugger. For more
information about @code{write_depth/2} (@pxref{I/O Control}).
@item < - full term
resets to the default of ten the debugger's maximum write depth. For
more information about @code{write_depth/2} (@pxref{I/O Control}).
@item A - alternatives
 show the list of backtrack points in the current execution. 
@item g [@var{N}] 
 show the list of ancestors in the current debugging environment. If it
 receives @var{N}, show the first @var{N} ancestors.
@end table

The debugging information, when fast-skip @code{quasi-leap} is used, will
be lost.

@node Efficiency, C-Interface, Debugging, Top

@chapter Efficiency Considerations

We next discuss several issues on trying to make Prolog programs run
fast in YAP. We assume two different programming styles:

@table @bullet
@item Execution of @emph{deterministic} programs often
boils down to a recursive loop of the form:
@example
loop(Env) :-
        do_something(Env,NewEnv),
        loop(NewEnv).
@end example

@end table

@section Deterministic Programs

@section Non-Deterministic Programs

@section Data-Base Operations

@section Indexing

@section Profiling

The indexation mechanism restricts the set of clauses to be tried in a 
procedure by using information about the status of a selected argument of 
the goal (in YAP, as in most compilers, the first argument). 
This argument 
is then used as a key, selecting a restricted set of a clauses from all the 
clauses forming the procedure.

As an example, the two clauses for concatenate:

@example
concatenate([],L,L).
concatenate([H|T],A,[H|NT]) :- concatenate(T,A,NT).
@end example

If the first argument for the goal is a list, then only the second clause 
is of interest. If the first argument is the nil atom, the system needs to 
look only for the first clause. The indexation generates instructions that 
test the value of the first argument, and then proceed to a selected clause, 
or group of clauses.

Note that if the first argument was a free variable, then both clauses 
should be tried. In general, indexation will not be useful if the first 
argument is a free variable.

When activating a predicate, a Prolog system needs to store state 
information. This information, stored in a structure known as choice point 
or fail point, is necessary when backtracking to other clauses for the 
predicate. The operations of creating and using a choice point are very 
expensive, both in the terms of space used and time spent.
Creating a choice point is not necessary if there is only a clause for 
the predicate as there are no clauses to backtrack to. With indexation, this 
situation is extended: in the example, if the first argument was the atom 
nil, then only one clause would really be of interest, and it is pointless to 
create a choice point. This feature is even more useful if the first argument 
is a list: without indexation, execution would try the first clause, creating 
a choice point. The clause would fail, the choice point would then be used to 
restore the previous state of the computation and the second clause would 
be tried. The code generated by the indexation mechanism would behave 
much more efficiently: it would test the first argument and see whether it 
is a list, and then proceed directly to the second clause.

An important side effect concerns the use of "cut". In the above 
example, some programmers would use a "cut" in the first clause just to 
inform the system that the predicate is not backtrackable and force the 
removal the choice point just created. As a result, less space is needed but 
with a great loss in expressive power: the "cut" would prevent some uses of 
the procedure, like generating lists through backtracking. Of course, with 
indexation the "cut" becomes useless: the choice point is not even created.

Indexation is also very important for predicates with a large number 
of clauses that are used like tables:

@example
logician(aristoteles,greek).
logician(frege,german).
logician(russel,english).
logician(godel,german).
logician(whitehead,english).
@end example

An interpreter like C-Prolog, trying to answer the query:

@example
?- logician(godel,X).
@end example

@noindent
would blindly follow the standard Prolog strategy, trying first the
first clause, then the second, the third and finally finding the
relevant clause.  Also, as there are some more clauses after the
important one, a choice point has to be created, even if we know the
next clauses will certainly fail. A "cut" would be needed to prevent
some possible uses for the procedure, like generating all logicians.  In
this situation, the indexing mechanism generates instructions that
implement a search table. In this table, the value of the first argument
would be used as a key for fast search of possibly matching clauses. For
the query of the last example, the result of the search would be just
the fourth clause, and again there would be no need for a choice point.

 If the first argument is a complex term, indexation will select clauses
just by testing its main functor. However, there is an important
exception: if the first argument of a clause is a list, the algorithm
also uses the list's head if not a variable. For instance, with the
following clauses,

@example
rules([],B,B).
rules([n(N)|T],I,O) :- rules_for_noun(N,I,N), rules(T,N,O).
rules([v(V)|T],I,O) :- rules_for_verb(V,I,N), rules(T,N,O).
rules([q(Q)|T],I,O) :- rules_for_qualifier(Q,I,N), rules(T,N,O).
@end example
@noindent
if the first argument of the goal is a list, its head will be tested, and only 
the clauses matching it will be tried during execution.

Some advice on how to take a good advantage of this mechanism:

@itemize @bullet

@item
 Try to make the first argument an input argument.

@item
 Try to keep together all clauses whose first argument is not a 
variable, that will decrease the number of tests since the other clauses are 
always tried.

@item
 Try to avoid predicates having a lot of clauses with the same key. 
For instance, the procedure:

@end itemize

@example
type(n(mary),person).
type(n(john), person).
type(n(chair),object).
type(v(eat),active).
type(v(rest),passive).
@end example

@noindent
 becomes more efficient with:

@example
type(n(N),T) :- type_of_noun(N,T).
type(v(V),T) :- type_of_verb(V,T).

type_of_noun(mary,person).
type_of_noun(john,person).
type_of_noun(chair,object).

type_of_verb(eat,active).
type_of_verb(rest,passive).
@end example

@node C-Interface,YAPLibrary,Efficiency,Top
@chapter C Language interface to YAP

YAP provides the user with the necessary facilities for writing
predicates in a language other than Prolog. Since, under Unix systems,
most language implementations are link-able to C, we will describe here
only the YAP interface to the C language.

Before describing in full detail how to interface to C code, we will examine 
a brief example.

Assume the user requires a predicate @code{my_process_id(Id)} which succeeds
when @var{Id} unifies with the number of the process under which YAP is running.

In this case we will create a @code{my_process.c} file containing the
C-code described below.

@example
@cartouche
#include "YAP/YAPInterface.h"

static int my_process_id(void) 
@{
     YAP_Term pid = YAP_MkIntTerm(getpid());
     YAP_Term out = YAP_ARG1;
     return(YAP_Unify(out,pid));
@}

void init_my_predicates()
@{
     YAP_UserCPredicate("my_process_id",my_process_id,1);
@}
@end cartouche
@end example

The commands to compile the above file depend on the operating
system. Under Linux (i386 and Alpha) you should use:
@example
      gcc -c -shared -fPIC my_process.c
      ld -shared -o my_process.so my_process.o
@end example
@noindent
Under WIN32 in a MINGW/CYGWIN environment, using the standard
installation path you should use:
@example
      gcc -mno-cygwin  -I "c:/Yap/include" -c my_process.c
      gcc -mno-cygwin "c:/Yap/bin/yap.dll" --shared -o my_process.dll my_process.o
@end example
@noindent
Under WIN32 in a pure CYGWIN environment, using the standard
installation path, you should use:
@example
      gcc -I/usr/local -c my_process.c
      gcc -shared -o my_process.dll my_process.o /usr/local/bin/yap.dll
@end example
@noindent
Under Solaris2 it is sufficient to use:
@example
      gcc  -fPIC -c my_process.c
@end example
@noindent
Under SunOS it is sufficient to use:
@example
      gcc -c my_process.c
@end example
@noindent
Under Digital Unix you need to create a @code{so} file. Use:
@example
      gcc tst.c -c -fpic
      ld my_process.o -o my_process.so -shared -expect_unresolved '*'
@end example
@noindent
and replace my @code{process.so} for my @code{process.o} in the
remainder of the example.
@noindent
And could be loaded, under YAP, by executing the following Prolog goal
@example
      load_foreign_files(['my_process'],[],init_my_predicates).
@end example
Note that since YAP4.3.3 you should not give the suffix for object
files. YAP will deduce the correct suffix from the operating system it
is running under.

After loading that file the following Prolog goal
@example
       my_process_id(N)
@end example
@noindent
would unify N with the number of the process under which YAP is running.


Having presented a full example, we will now examine in more detail the
contents of the C source code file presented above.

The include statement is used to make available to the C source code the
macros for the handling of Prolog terms and also some YAP public
definitions.

The function @code{my_process_id} is the implementation, in C, of the
desired predicate.  Note that it returns an integer denoting the success
of failure of the goal and also that it has no arguments even though the
predicate being defined has one.
 In fact the arguments of a Prolog predicate written in C are accessed
through macros, defined in the include file, with names @var{YAP_ARG1},
@var{YAP_ARG2}, ..., @var{YAP_ARG16} or with @var{YAP_A}(@var{N})
where @var{N} is the argument number (starting with 1).  In the present
case the function uses just one local variable of type @code{YAP_Term}, the
type used for holding YAP terms, where the integer returned by the
standard unix function @code{getpid()} is stored as an integer term (the
conversion is done by @code{YAP_MkIntTerm(Int))}. Then it calls the
pre-defined routine @code{YAP_Unify(YAP_Term, YAP_Term)} which in turn returns an
integer denoting success or failure of the unification.

The role of the procedure @code{init_my_predicates} is to make known to
YAP, by calling @code{YAP_UserCPredicate}, the predicates being
defined in the file.  This is in fact why, in the example above,
@code{init_my_predicates} was passed as the third argument to
@code{load_foreign_files}.

The rest of this appendix describes exhaustively how to interface C to YAP.

@menu
* Manipulating Terms:: Primitives available to the C programmer
* Unifying Terms:: How to Unify Two Prolog Terms
* Manipulating Strings:: From character arrays to Lists of codes and back
* Memory Allocation:: Stealing Memory From YAP
* Controlling Streams:: Control How YAP sees Streams
* Utility Functions:: From character arrays to Lists of codes and back
* Calling YAP From C:: From C to YAP to C to YAP 
* Module Manipulation in C:: Create and Test Modules from within C
* Miscellaneous C-Functions:: Other Helpful Interface Functions
* Writing C:: Writing Predicates in C
* Loading Objects:: Loading Object Files
* Save&Rest:: Saving and Restoring
* YAP4 Notes:: Changes in Foreign Predicates Interface
@end menu

@node Manipulating Terms, Unifying Terms, , C-Interface
@section Terms

This section provides information about the primitives available to the C
programmer for manipulating Prolog terms.

Several C typedefs are included in the header file @code{yap/YAPInterface.h} to
describe, in a portable way, the C representation of Prolog terms.
The user should write is programs using this macros to ensure portability of
code across different versions of YAP.


The more important typedef is @var{YAP_Term} which is used to denote the
type of a Prolog term.

Terms, from a point of view of the C-programmer,  can be classified as
follows
@table @i
@item    uninstantiated variables
@item    instantiated variables
@item    integers
@item    floating-point numbers
@item    database references
@item    atoms
@item    pairs (lists)
@item    compound terms
@end table

@findex YAP_IsVarTerm (C-Interface function)
The primitive
@example
     YAP_Bool YAP_IsVarTerm(YAP_Term @var{t})
@end example
@noindent
@findex YAP_IsNonVarTerm (C-Interface function)
returns true iff its argument is an uninstantiated variable. Conversely the
primitive
@example
      YAP_Bool YAP_NonVarTerm(YAP_Term @var{t})
@end example
@noindent
returns true iff its argument is not a variable.


The user can create a new uninstantiated variable using the primitive
@example
      YAP_Term  YAP_MkVarTerm()
@end example


@findex YAP_IsIntTerm (C-Interface function)
@findex YAP_IsFloatTerm (C-Interface function)
@findex YAP_IsDBRefTerm (C-Interface function)
@findex YAP_IsAtomTerm (C-Interface function)
@findex YAP_IsPairTerm (C-Interface function)
@findex YAP_IsApplTerm (C-Interface function)
The following primitives can be used to discriminate among the different types
of non-variable terms:
@example
      YAP_Bool YAP_IsIntTerm(YAP_Term @var{t})
      YAP_Bool YAP_IsFloatTerm(YAP_Term @var{t})
      YAP_Bool YAP_IsDbRefTerm(YAP_Term @var{t})
      YAP_Bool YAP_IsAtomTerm(YAP_Term @var{t})
      YAP_Bool YAP_IsPairTerm(YAP_Term @var{t})
      YAP_Bool YAP_IsApplTerm(YAP_Term @var{t})
@end example

Next, we mention the primitives that allow one to destruct and construct
terms. All the above primitives ensure that their result is
@i{dereferenced}, i.e. that it is not a pointer to another term.

@findex YAP_MkIntTerm (C-Interface function)
@findex YAP_IntOfTerm (C-Interface function)
The following primitives are provided for creating an integer term from an
integer and to access the value of an integer term.
@example
      YAP_Term YAP_MkIntTerm(YAP_Int  @var{i})
      YAP_Int  YAP_IntOfTerm(YAP_Term @var{t})
@end example
@noindent
where @code{YAP_Int} is a typedef for the C integer type appropriate for
the machine or compiler in question (normally a long integer). The size
of the allowed integers is implementation dependent but is always
greater or equal to 24 bits: usually 32 bits on 32 bit machines, and 64
on 64 bit machines.

@findex YAP_MkFloatTerm (C-Interface function)
@findex YAP_FloatOfTerm (C-Interface function)
The two following primitives play a similar role for floating-point terms
@example
      YAP_Term YAP_MkFloatTerm(YAP_flt @var{double})
      YAP_flt  YAP_FloatOfTerm(YAP_Term @var{t})
@end example
@noindent
where @code{flt} is a typedef for the appropriate C floating point type,
nowadays a @code{double}

@findex YAP_IsBigNumTerm (C-Interface function)
@findex YAP_MkBigNumTerm (C-Interface function)
@findex YAP_BigNumOfTerm (C-Interface function)
The following primitives are provided for verifying whether a term is
a big int, creating a term from a big integer and to access the value
of a big int from a term.
@example
      YAP_Bool YAP_IsBigNumTerm(YAP_Term @var{t})
      YAP_Term YAP_MkBigNumTerm(void  *@var{b})
      void *YAP_BigNumOfTerm(YAP_Term @var{t}, void *@var{b})
@end example
@noindent
YAP must support bignum for the configuration you are using (check the
YAP configuration and setup). For now, YAP only supports the GNU GMP
library, and @code{void *} will be a cast for @code{mpz_t}. Notice
that @code{YAP_BigNumOfTerm} requires the number to be already
initialised. As an example, we show how to print a bignum:

@example
static int
p_print_bignum(void)
@{
  mpz_t mz;
  if (!YAP_IsBigNumTerm(YAP_ARG1))
    return FALSE;

  mpz_init(mz);
  YAP_BigNumOfTerm(YAP_ARG1, mz);
  gmp_printf("Shows up as %Zd\n", mz);
  mpz_clear(mz);
  return TRUE;
@}
@end example


Currently, no primitives are supplied to users for manipulating data base
references. 

@findex YAP_MkAtomTerm (C-Interface function)
@findex YAP_AtomOfTerm (C-Interface function)
A special typedef @code{YAP_Atom} is provided to describe Prolog
@i{atoms} (symbolic constants). The two following primitives can be used
to manipulate atom terms
@example
      YAP_Term YAP_MkAtomTerm(YAP_Atom at)
      YAP_Atom YAP_AtomOfTerm(YAP_Term @var{t})
@end example
@noindent
@findex YAP_LookupAtom (C-Interface function)
@findex YAP_FullLookupAtom (C-Interface function)
@findex YAP_AtomName (C-Interface function)
The following primitives are available for associating atoms with their
names 
@example
      YAP_Atom  YAP_LookupAtom(char * @var{s})
      YAP_Atom  YAP_FullLookupAtom(char * @var{s})
      char     *YAP_AtomName(YAP_Atom @var{t})
@end example
The function @code{YAP_LookupAtom} looks up an atom in the standard hash
table. The function @code{YAP_FullLookupAtom} will also search if the
atom had been "hidden": this is useful for system maintenance from C
code. The functor @code{YAP_AtomName} returns a pointer to the string
for the atom.

@noindent
@findex YAP_IsWideAtom (C-Interface function)
@findex YAP_LookupWideAtom (C-Interface function)
@findex YAP_WideAtomName (C-Interface function)
The following primitives handle constructing atoms from strings with
wide characters, and vice-versa:
@example
      YAP_Atom  YAP_LookupWideAtom(wchar_t * @var{s})
      wchar_t  *YAP_WideAtomName(YAP_Atom @var{t})
@end example

@noindent
@findex YAP_IsIsWideAtom (C-Interface function)
The following primitive tells whether an atom needs wide atoms in its
representation:
@example
      int  YAP_IsWideAtom(YAP_Atom @var{t})
@end example

@noindent
@findex YAP_AtomNameLength (C-Interface function)
The following primitive can be used to obtain the size of an atom in a
representation-independent way: 
@example
      int      YAP_AtomNameLength(YAP_Atom @var{t})
@end example

@findex YAP_AtomGetHold  (C-Interface function)
@findex YAP_AtomReleaseHold  (C-Interface function)
@findex YAP_AGCHook  (C-Interface function)
The next routines give users some control over  the atom
garbage collector. They allow the user to guarantee that an atom is not
to be garbage collected (this is important if the atom is hold
externally to the Prolog engine, allow it to be collected, and call a
hook on garbage collection:
@example
      int  YAP_AtomGetHold(YAP_Atom @var{at})
      int  YAP_AtomReleaseHold(YAP_Atom @var{at})
      int  YAP_AGCRegisterHook(YAP_AGC_hook @var{f})
      YAP_Term  YAP_TailOfTerm(YAP_Term @var{t})
@end example

@findex YAP_MkPairTerm (C-Interface function)
@findex YAP_MkNewPairTerm (C-Interface function)
@findex YAP_HeadOfTerm (C-Interface function)
@findex YAP_TailOfTerm (C-Interface function)
A @i{pair} is a Prolog term which consists of a tuple of two Prolog
terms designated as the @i{head} and the @i{tail} of the term. Pairs are
most often used to build @emph{lists}. The following primitives can be
used to manipulate pairs:
@example
      YAP_Term  YAP_MkPairTerm(YAP_Term @var{Head}, YAP_Term @var{Tail})
      YAP_Term  YAP_MkNewPairTerm(void)
      YAP_Term  YAP_HeadOfTerm(YAP_Term @var{t})
      YAP_Term  YAP_TailOfTerm(YAP_Term @var{t})
@end example
One can construct a new pair from two terms, or one can just build a
pair whose head and tail are new unbound variables. Finally, one can
fetch the head or the tail.

@findex YAP_MkApplTerm (C-Interface function)
@findex YAP_MkNewApplTerm (C-Interface function)
@findex YAP_ArgOfTerm (C-Interface function)
@findex YAP_ArgsOfTerm (C-Interface function)
@findex YAP_FunctorOfTerm (C-Interface function)
A @i{compound} term consists of a @i{functor} and a sequence of terms with
length equal to the @i{arity} of the functor. A functor, described in C by
the typedef @code{Functor}, consists of an atom and of an integer.
The following primitives were designed to manipulate compound terms and 
functors
@example
      YAP_Term     YAP_MkApplTerm(YAP_Functor @var{f}, unsigned long int @var{n}, YAP_Term[] @var{args})
      YAP_Term     YAP_MkNewApplTerm(YAP_Functor @var{f}, int @var{n})
      YAP_Term     YAP_ArgOfTerm(int argno,YAP_Term @var{ts})
      YAP_Term    *YAP_ArgsOfTerm(YAP_Term @var{ts})
      YAP_Functor  YAP_FunctorOfTerm(YAP_Term @var{ts})
@end example
@noindent
The @code{YAP_MkApplTerm} function constructs a new term, with functor
@var{f} (of arity @var{n}), and using an array @var{args} of @var{n}
terms with @var{n} equal to the arity of the
functor. @code{YAP_MkNewApplTerm} builds up a compound term whose
arguments are unbound variables. @code{YAP_ArgOfTerm} gives an argument
to a compound term. @code{argno} should be greater or equal to 1 and
less or equal to the arity of the functor.  @code{YAP_ArgsOfTerm}
returns a pointer to an array of arguments.

YAP allows one to manipulate the functors of compound term. The function
@code{YAP_FunctorOfTerm} allows one to obtain a variable of type
@code{YAP_Functor} with the functor to a term. The following functions
then allow one to construct functors, and to obtain their name and arity. 

@findex YAP_MkFunctor (C-Interface function)
@findex YAP_NameOfFunctor (C-Interface function)
@findex YAP_ArityOfFunctor (C-Interface function)
@example
      YAP_Functor  YAP_MkFunctor(YAP_Atom @var{a},unsigned long int @var{arity})
      YAP_Atom     YAP_NameOfFunctor(YAP_Functor @var{f})
      YAP_Int      YAP_ArityOfFunctor(YAP_Functor @var{f})
@end example
@noindent

Note that the functor is essentially a pair formed by an atom, and
arity.

@node Unifying Terms, Manipulating Strings, Manipulating Terms, C-Interface
@section Unification

@findex YAP_Unify (C-Interface function)
YAP provides a single routine to attempt the unification of two Prolog
terms. The routine may succeed or fail:
@example
      Int      YAP_Unify(YAP_Term @var{a}, YAP_Term @var{b})
@end example
@noindent
The routine attempts to unify the terms @var{a} and
@var{b} returning @code{TRUE} if the unification succeeds and @code{FALSE}
otherwise.

@node Manipulating Strings, Memory Allocation, Unifying Terms, C-Interface
@section Strings

@findex YAP_StringToBuffer (C-Interface function)
The YAP C-interface now includes an utility routine to copy a string
represented as a list of a character codes to a previously allocated buffer
@example
      int YAP_StringToBuffer(YAP_Term @var{String}, char *@var{buf}, unsigned int @var{bufsize})
@end example
@noindent
The routine copies the list of character codes @var{String} to a
previously allocated buffer @var{buf}. The string including a
terminating null character must fit in @var{bufsize} characters,
otherwise the routine will simply fail. The @var{StringToBuffer} routine
fails and generates an exception if @var{String} is not a valid string.

@findex YAP_BufferToString (C-Interface function)
@findex YAP_NBufferToString (C-Interface function)
@findex YAP_WideBufferToString (C-Interface function)
@findex YAP_NWideBufferToString (C-Interface function)
@findex YAP_BufferToAtomList (C-Interface function)
@findex YAP_NBufferToAtomList (C-Interface function)
@findex YAP_WideBufferToAtomList (C-Interface function)
@findex YAP_NWideBufferToAtomList (C-Interface function)
@findex YAP_BufferToDiffList (C-Interface function)
@findex YAP_NBufferToDiffList (C-Interface function)
@findex YAP_WideBufferToDiffList (C-Interface function)
@findex YAP_NWideBufferToDiffList (C-Interface function)
The C-interface also includes utility routines to do the reverse, that
is, to copy a from a buffer to a list of character codes, to a
difference list,  or to a list of
character atoms. The routines work either on strings of characters or
strings of wide characters:
@example
      YAP_Term YAP_BufferToString(char *@var{buf})
      YAP_Term YAP_NBufferToString(char *@var{buf}, size_t @var{len})
      YAP_Term YAP_WideBufferToString(wchar_t *@var{buf})
      YAP_Term YAP_NWideBufferToString(wchar_t *@var{buf}, size_t @var{len})
      YAP_Term YAP_BufferToAtomList(char *@var{buf})
      YAP_Term YAP_NBufferToAtomList(char *@var{buf}, size_t @var{len})
      YAP_Term YAP_WideBufferToAtomList(wchar_t *@var{buf})
      YAP_Term YAP_NWideBufferToAtomList(wchar_t *@var{buf}, size_t @var{len})
@end example
@noindent
Users are advised to use the @var{N} version of the routines. Otherwise,
the user-provided string must include a terminating null character.

@findex YAP_ReadBuffer (C-Interface function)
The C-interface function calls the parser on a sequence of characters
stored at @var{buf} and returns the resulting term.
@example
      YAP_Term YAP_ReadBuffer(char *@var{buf},YAP_Term *@var{error})
@end example
@noindent
The user-provided string must include a terminating null
character. Syntax errors will cause returning @code{FALSE} and binding
@var{error} to a Prolog term.

@node Memory Allocation, Controlling Streams, Manipulating Strings, C-Interface
@section Memory Allocation

@findex YAP_AllocSpaceFromYAP (C-Interface function)
The next routine can be used to ask space from the Prolog data-base:
@example
      void      *YAP_AllocSpaceFromYAP(int @var{size})
@end example
@noindent
The routine returns a pointer to a buffer allocated from the code area,
or @code{NULL} if sufficient space was not available.

@findex YAP_FreeSpaceFromYAP (C-Interface function)
The space allocated with @code{YAP_AllocSpaceFromYAP} can be released
back to YAP by using:
@example
      void      YAP_FreeSpaceFromYAP(void *@var{buf})
@end example
@noindent
The routine releases a buffer allocated from the code area. The system
may crash if @code{buf} is not a valid pointer to a buffer in the code
area.

@node Controlling Streams, Utility Functions, Memory Allocation, C-Interface
@section Controlling YAP Streams from @code{C}

@findex YAP_StreamToFileNo (C-Interface function)
The C-Interface also provides the C-application with a measure of
control over the YAP Input/Output system. The first routine allows one
to find a file number given a current stream:
@example
      int      YAP_StreamToFileNo(YAP_Term @var{stream})
@end example
@noindent
This function gives the file descriptor for a currently available
stream. Note that null streams and in memory streams do not have
corresponding open streams, so the routine will return a
negative. Moreover, YAP will not be aware of any direct operations on
this stream, so information on, say, current stream position, may become
stale.

@findex YAP_CloseAllOpenStreams (C-Interface function)
A second routine that is sometimes useful is:
@example
      void      YAP_CloseAllOpenStreams(void)
@end example
@noindent
This routine closes the YAP Input/Output system except for the first
three streams, that are always associated with the three standard Unix
streams. It is most useful if you are doing @code{fork()}.

@findex YAP_FlushAllStreams (C-Interface function)
Last, one may sometimes need to flush all streams:
@example
      void      YAP_CloseAllOpenStreams(void)
@end example
@noindent
It is also useful before you do a @code{fork()}, or otherwise you may
have trouble with unflushed output.

@findex YAP_OpenStream (C-Interface function)
The next routine allows a currently open file to become a stream. The
routine receives as arguments a file descriptor, the true file name as a
string, an atom with the user name, and a set of flags:
@example
      void      YAP_OpenStream(void *@var{FD}, char *@var{name}, YAP_Term @var{t}, int @var{flags})
@end example
@noindent
The available flags are @code{YAP_INPUT_STREAM},
@code{YAP_OUTPUT_STREAM}, @code{YAP_APPEND_STREAM},
@code{YAP_PIPE_STREAM}, @code{YAP_TTY_STREAM}, @code{YAP_POPEN_STREAM},
@code{YAP_BINARY_STREAM}, and @code{YAP_SEEKABLE_STREAM}. By default, the
stream is supposed to be at position 0. The argument @var{name} gives
the name by which YAP should know the new stream.

@node Utility Functions, Calling YAP From C, Controlling Streams, C-Interface
@section Utility Functions in @code{C}


The C-Interface  provides the C-application with a a number of utility
functions that are useful.


@findex YAP_Record (C-Interface function)
The first provides a way to insert a term into the data-base
@example
      void      *YAP_Record(YAP_Term @var{t})
@end example
@noindent
This function returns a pointer to a copy of the term in the database
(or to @t{NULL} if the operation fails.

@findex YAP_Recorded (C-Interface function)
The next functions provides a way to recover the term from the data-base:
@example
      YAP_Term      YAP_Recorded(void *@var{handle})
@end example
@noindent
Notice that the semantics are the same as for @code{recorded/3}: this
function creates a new copy of the term in the stack, with fresh
variables. The function returns @t{0L} if it cannot create a new term.

@findex YAP_Erase (C-Interface function)
Last, the next function allows one to recover space:
@example
      int      YAP_Erase(void *@var{handle})
@end example
@noindent
Notice that any accesses using @var{handle} after this operation may
lead to a crash.

The following functions are often required to compare terms.

@findex YAP_ExactlyEqual (C-Interface function)
The first function succeeds if two terms are actually the same term, as
@code{==/2}:
@example
      int      YAP_ExactlyEqual(YAP_Term t1, YAP_Term t2)
@end example
@noindent

The second function succeeds if two terms are variant terms, and returns
0 otherwise, as
@code{=@=/2}:
@example
      int      YAP_Variant(YAP_Term t1, YAP_Term t2)
@end example
@noindent

The second function computes a hash function for a term, as in
@code{term_hash/4}.
@example
     YAP_Int    YAP_TermHash(YAP_Term t, YAP_Int range, YAP_Int depth, int  ignore_variables));
@end example
@noindent
The first three arguments follow @code{term_has/4}. The last argument
indicates what to do if we find a variable: if @code{0} fail, otherwise
ignore the variable. 

@node Calling YAP From C, Module Manipulation in C, Utility Functions, C-Interface
@section From @code{C} back to Prolog

@findex YAP_RunGoal (C-Interface function)
There are several ways to call Prolog code from C-code. By default, the
@code{YAP_RunGoal()} should be used for this task. It assumes the engine
has been initialised before:

@example
  YAP_RunGoal(YAP_Term Goal)
@end example
Execute query @var{Goal} and return 1 if the query succeeds, and 0
otherwise. The predicate returns 0 if failure, otherwise it will return
an @var{YAP_Term}. 

Quite often, one wants to run a query once. In this case you should use
@var{Goal}:
@example
  YAP_RunGoalOnce(YAP_Term Goal)
@end example
The  @code{YAP_RunGoal()} function makes sure to recover stack space at
the end of execution.

Prolog terms are pointers: a problem users often find is that the term
@var{Goal} may actually @emph{be moved around} during the execution of
@code{YAP_RunGoal()}, due to garbage collection or stack shifting. If
this is possible, @var{Goal} will become invalid after executing
@code{YAP_RunGoal()}. In this case, it is a good idea to save @var{Goal}
@emph{slots}, as shown next:

@example
  long sl = YAP_InitSlot(scoreTerm);

  out = YAP_RunGoal(t);
  t = YAP_GetFromSlot(sl);
  YAP_RecoverSlots(1);
  if (out == 0) return FALSE;
@end example
Slots are safe houses in the stack, the garbage collector and the stack
shifter know about them and make sure they have correct values. In this
case, we use a slot to preserve @var{t} during the execution of
@code{YAP_RunGoal}. When the execution of @var{t} is over we read the
(possibly changed) value of @var{t} back from the slot @var{sl} and tell
YAP that the slot @var{sl} is not needed and can be given back to the
system. The slot functions are as follows:

@table @code
@item YAP_Int YAP_NewSlots(int @var{NumberOfSlots})
@findex YAP_NewSlots (C-Interface function)
Allocate @var{NumberOfSlots} from the stack and return an handle to the
last one. The other handle can be obtained by decrementing the handle.

@item YAP_Int YAP_CurrentSlot(void)
@findex YAP_CurrentSlot (C-Interface function)
Return a handle to the system's default slot.

@item YAP_Int YAP_InitSlot(YAP_Term @var{t})
@findex YAP_InitSlot (C-Interface function)
Create a new slot, initialise it with @var{t}, and return a handle to
this slot, that also becomes the current slot.

@item YAP_Term *YAP_AddressFromSlot(YAP_Int @var{slot})
@findex YAP_AddressFromSlot (C-Interface function)
Return the address of slot @var{slot}: please use with care.

@item void YAP_PutInSlot(YAP_Int @var{slot}, YAP_Term @var{t})
@findex YAP_PutInSlot (C-Interface function)
Set the contents of slot @var{slot} to @var{t}.

@item int YAP_RecoverSlots(int @var{HowMany})
@findex YAP_RecoverSlots (C-Interface function)
Recover the space for @var{HowMany} slots: these will include the
current default slot. Fails if no such slots exist.

@item YAP_Int YAP_ArgsToSlots(int @var{HowMany})
@findex YAP_ArgsToSlots (C-Interface function)
Store the current first  @var{HowMany} arguments in new slots.

@item void YAP_SlotsToArgs(int @var{HowMany}, YAP_Int @var{slot})
@findex YAP_SlotsToArgs (C-Interface function)
Set the first @var{HowMany} arguments to the @var{HowMany} slots
starting at @var{slot}.
@end table

The following functions complement @var{YAP_RunGoal}:
@table @code
@findex YAP_RestartGoal (C-Interface function)
Look for the next solution to the current query by forcing YAP to
backtrack to the latest goal. Notice that slots allocated since the last
@code{YAP_RunGoal} will become invalid.

@item  @code{int} YAP_Reset(@code{void})
@findex YAP_Reset (C-Interface function)
Reset execution environment (similar to the @code{abort/0}
built-in). This is useful when you want to start a new query before
asking all solutions to the previous query.

@item  @code{int} YAP_ShutdownGoal(@code{int backtrack})
@findex YAP_ShutdownGoal (C-Interface function)
Clean up the current goal. If
@code{backtrack} is true, stack space will be recovered and bindings
will be undone. In both cases, any slots allocated since the goal was
created will become invalid.

@item  @code{YAP_Bool} YAP_GoalHasException(@code{YAP_Term *tp})
@findex YAP_RestartGoal (C-Interface function)
Check if the last goal generated an exception, and if so copy it to the
space pointed to by @var{tp}

@item  @code{void} YAP_ClearExceptions(@code{void})
@findex YAP_ClearExceptions (C-Interface function)
Reset any exceptions left over by the system.
@end table

The @var{YAP_RunGoal} interface is designed to be very robust, but may
not be the most efficient when repeated calls to the same goal are made
and when there is no interest in processing exception. The
@var{YAP_EnterGoal} interface should have lower-overhead:
@table @code
@item  @code{YAP_PredEntryPtr} YAP_FunctorToPred(@code{YAP_Functor} @var{f},
@findex YAP_FunctorToPred (C-Interface function)
Return the predicate whose main functor is @var{f}.

@item  @code{YAP_PredEntryPtr} YAP_AtomToPred(@code{YAP_Atom} @var{at},
@findex YAP_AtomToPred (C-Interface function)
Return the arity 0 predicate whose name is @var{at}.

@item  @code{YAP_Bool} YAP_EnterGoal(@code{YAP_PredEntryPtr} @var{pe},
@code{YAP_Term *} @var{array}, @code{YAP_dogoalinfo *} @var{infop})
@findex YAP_EnterGoal (C-Interface function)
Execute a  query for predicate @var{pe}. The query is given as an
array of terms @var{Array}. @var{infop} is the address of a goal
handle that can be used to backtrack and to recover space. Succeeds if
a solution was found.

Notice that you cannot create new slots if an YAP_EnterGoal goal is open.

@item  @code{YAP_Bool} YAP_RetryGoal(@code{YAP_dogoalinfo *} @var{infop})

@findex YAP_RetryGoal (C-Interface function)
Backtrack to a query created by @code{YAP_EnterGoal}. The query is
given by the handle @var{infop}. Returns whether a new solution could
be be found.

@item  @code{YAP_Bool} YAP_LeaveGoal(@code{YAP_Bool} @var{backtrack},
@code{YAP_dogoalinfo *} @var{infop})
@findex YAP_LeaveGoal (C-Interface function)
Exit a query query created by @code{YAP_EnterGoal}. If
@code{backtrack} is @code{TRUE}, variable bindings are undone and Heap
space is recovered.  Otherwise, only stack space is recovered, ie,
@code{LeaveGoal} executes a cut.
@end table
Next, follows an example of how to use @code{YAP_EnterGoal}:
@example
void
runall(YAP_Term g)
@{
    YAP_dogoalinfo goalInfo;
    YAP_Term *goalArgs = YAP_ArraysOfTerm(g);
    YAP_Functor *goalFunctor = YAP_FunctorOfTerm(g);
    YAP_PredEntryPtr goalPred = YAP_FunctorToPred(goalFunctor);
    
    result = YAP_EnterGoal( goalPred, goalArgs, &goalInfo );
    while (result)
       result = YAP_RetryGoal( &goalInfo );
    YAP_LeaveGoal(TRUE, &goalInfo);
@}
@end example

@findex YAP_CallProlog (C-Interface function)
YAP allows calling a @strong{new} Prolog interpreter from @code{C}. One
way is to first construct a goal @code{G}, and then it is sufficient to
perform:
@example
      YAP_Bool      YAP_CallProlog(YAP_Term @var{G})
@end example
@noindent
the result will be @code{FALSE}, if the goal failed, or @code{TRUE}, if
the goal succeeded. In this case, the variables in @var{G} will store
the values they have been unified with. Execution only proceeds until
finding the first solution to the goal, but you can call
@code{findall/3} or friends if you need all the solutions.

Notice that during execution, garbage collection or stack shifting may
have moved the terms 

@node Module Manipulation in C, Miscellaneous C-Functions, Calling YAP From C, C-Interface
@section Module Manipulation in C

YAP allows one to create a new module from C-code. To create the new
code it is sufficient to call:
@example
      YAP_Module      YAP_CreateModule(YAP_Atom @var{ModuleName})
@end example
@noindent
Notice that the new module does not have any predicates associated and
that it is not the current module. To find the current module, you can call:
@example
      YAP_Module      YAP_CurrentModule()
@end example

Given a module, you may want to obtain the corresponding name. This is
possible by using:
@example
      YAP_Term      YAP_ModuleName(YAP_Module mod)
@end example
@noindent
Notice that this function returns a term, and not an atom. You can
@code{YAP_AtomOfTerm} to extract the corresponding Prolog atom.

@node Miscellaneous C-Functions, Writing C, Module Manipulation in C, C-Interface
@section Miscellaneous C Functions

@table @code
@item  @code{int} YAP_SetYAPFlag(@code{yap_flag_t flag, int value})
@findex YAP_SetYAPFlag (C-Interface function)

This function allows setting some YAP flags from @code{C} .Currently,
only two boolean flags are accepted: @code{YAPC_ENABLE_GC} and
@code{YAPC_ENABLE_AGC}.  The first enables/disables the standard garbage
collector, the second does the same for the atom garbage collector.`

@end table


@node Writing C, Loading Objects, Miscellaneous C-Functions, C-Interface
@section Writing predicates in C

We will distinguish two kinds of predicates:
@table @i
@item @i{deterministic} predicates which either fail or succeed but are not
backtrackable, like the one in the introduction;
@item @i{backtrackable}
predicates which can succeed more than once.
@end table

@findex YAP_UserCPredicate (C-Interface function)
The first kind of predicates should be implemented as a C function with
no arguments which should return zero if the predicate fails and a
non-zero value otherwise. The predicate should be declared to
YAP, in the initialization routine, with a call to
@example
      void YAP_UserCPredicate(char *@var{name}, YAP_Bool *@var{fn}(), unsigned long int @var{arity});
@end example
@noindent
where @var{name} is the name of the predicate, @var{fn} is the C function
implementing the predicate and @var{arity} is its arity.

@findex YAP_UserBackCPredicate (C-Interface function, deprecated)
@findex YAP_UserBackCutCPredicate (C-Interface function)
@findex YAP_PRESERVE_DATA (C-Interface function)
@findex YAP_PRESERVED_DATA (C-Interface function)
@findex YAP_cutsucceed (C-Interface function)
@findex YAP_cutfail (C-Interface function)
For the second kind of predicates we need three C functions. The first one
 is called when the predicate is first activated; the second one
is called on backtracking to provide (possibly) other solutions; the
 last one is called on pruning. Note
also that we normally also need to preserve some information to find out
the next solution.

In fact the role of the two functions can be better understood from the
following Prolog definition
@example
       p :- start.
       p :- repeat,
                continue.
@end example
@noindent
where @code{start} and @code{continue} correspond to the two C functions
described above.


As an example we will consider implementing in C a predicate @code{n100(N)}
which, when called with an instantiated argument should succeed if that
argument is a numeral less or equal to 100, and, when called with an
uninstantiated argument, should provide, by backtracking, all the positive
integers less or equal to 100.

   To do that we first declare a structure, which can only consist
of Prolog terms, containing the information to be preserved on backtracking
and a pointer variable to a structure of that type.

@example
#include "YAPInterface.h"

static int start_n100(void);
static int continue_n100(void);

typedef struct @{
    YAP_Term next_solution;  /* the next solution */
   @} n100_data_type;

n100_data_type *n100_data;
@end example

We now write the @code{C} function to handle the first call:

@example
static int start_n100(void)
@{
      YAP_Term t = YAP_ARG1;
      YAP_PRESERVE_DATA(n100_data,n100_data_type);
      if(YAP_IsVarTerm(t)) @{
          n100_data->next_solution = YAP_MkIntTerm(0);
          return continue_n100();
       @}
      if(!YAP_IsIntTerm(t) || YAP_IntOfTerm(t)<0 || YAP_IntOfTerm(t)>100) @{
          YAP_cut_fail();
      @} else @{
          YAP_cut_succeed();
      @}
@}

@end example

The routine starts by getting the dereference value of the argument.
The call to @code{YAP_PRESERVE_DATA} is used to initialize the memory which will
hold the information to be preserved across backtracking. The first
argument is the variable we shall use, and the second its type. Note
that we can only use @code{YAP_PRESERVE_DATA} once, so often we will
want the variable to be a structure.

If the argument of the predicate is a variable, the routine initializes the 
structure to be preserved across backtracking with the information
required to provide the next solution, and exits by calling @code{
continue_n100} to provide that solution.

If the argument was not a variable, the routine then checks if it was an
integer, and if so, if its value is positive and less than 100. In that
case it exits, denoting success, with @code{YAP_cut_succeed}, or
otherwise exits with @code{YAP_cut_fail} denoting failure.

The reason for using for using the functions @code{YAP_cut_succeed} and
@code{YAP_cut_fail} instead of just returning a non-zero value in the
first case, and zero in the second case, is that otherwise, if
backtracking occurred later, the routine @code{continue_n100} would be
called to provide additional solutions.

The code required for the second function is
@example
static int continue_n100(void)
@{
      int n;
      YAP_Term t;
      YAP_Term sol = YAP_ARG1;
      YAP_PRESERVED_DATA(n100_data,n100_data_type);
      n = YAP_IntOfTerm(n100_data->next_solution);
      if( n == 100) @{
           t = YAP_MkIntTerm(n);
           YAP_Unify(sol,t);
           YAP_cut_succeed();
        @}
       else @{
           YAP_Unify(sol,n100_data->next_solution);
           n100_data->next_solution = YAP_MkIntTerm(n+1);
           return(TRUE);
        @}
@}
@end example

Note that again the macro @code{YAP_PRESERVED_DATA} is used at the
beginning of the function to access the data preserved from the previous
solution.  Then it checks if the last solution was found and in that
case exits with @code{YAP_cut_succeed} in order to cut any further
backtracking.  If this is not the last solution then we save the value
for the next solution in the data structure and exit normally with 1
denoting success. Note also that in any of the two cases we use the
function @code{YAP_unify} to bind the argument of the call to the value
saved in @code{ n100_state->next_solution}.


Note also that the only correct way to signal failure in a backtrackable
predicate is to use the @code{YAP_cut_fail} macro.

Backtrackable predicates should be declared to YAP, in a way
similar to what happened with deterministic ones, but using instead a
call to
@example
      void YAP_UserBackCutCPredicate(char *@var{name},
                 int *@var{init}(), int *@var{cont}(), int *@var{cut}(),
                 unsigned long int @var{arity}, unsigned int @var{sizeof});
@end example
@noindent
where @var{name} is a string with the name of the predicate, @var{init},
@var{cont}, @var{cut} are the C functions used to start, continue and
when pruning the execution of the predicate, @var{arity} is the
predicate arity, and @var{sizeof} is the size of the data to be
preserved in the stack. In this example, we would have something like

@example
void
init_n100(void)
@{
  YAP_UserBackCutCPredicate("n100", start_n100, continue_n100, NULL, 1, 1);
@}
@end example
Notice that we do not actually need to do anything on receiving a cut in
this case.

@node Loading Objects, Save&Rest, Writing C, C-Interface
@section Loading Object Files

The primitive predicate
@example
      load_foreign_files(@var{Files},@var{Libs},@var{InitRoutine})
@end example
@noindent
should be used, from inside YAP, to load object files produced by the C
compiler. The argument @var{ObjectFiles} should be a list of atoms
specifying the object files to load, @var{Libs} is a list (possibly
empty) of libraries to be passed to the unix loader (@code{ld}) and
InitRoutine is the name of the C routine (to be called after the files
are loaded) to perform the necessary declarations to YAP of the
predicates defined in the files. 

YAP will search for @var{ObjectFiles} in the current directory first. If
it cannot find them it will search for the files using the environment
variable @code{YAPLIBDIR}, if defined, or in the default library.

YAP also supports the SWI-Prolog interface to loading foreign code:

@table @code
@item open_shared_object(+@var{File}, -@var{Handle})
@findex open_shared_object/2
@snindex open_shared_object/2
@cnindex open_shared_object/2
    File is the name of a shared object file (called dynamic load
    library in MS-Windows). This file is attached to the current process
    and @var{Handle} is unified with a handle to the library. Equivalent to
    @code{open_shared_object(File, [], Handle)}. See also
    load_foreign_library/[1,2].

    On errors, an exception @code{shared_object}(@var{Action},
    @var{Message}) is raised. @var{Message} is the return value from
    dlerror().

@item open_shared_object(+@var{File}, -@var{Handle}, +@var{Options})
@findex open_shared_object/3
@snindex open_shared_object/3
@cnindex open_shared_object/3
    As @code{open_shared_object/2}, but allows for additional flags to
    be passed. @var{Options} is a list of atoms. @code{now} implies the
    symbols are 
    resolved immediately rather than lazily (default). @code{global} implies
    symbols of the loaded object are visible while loading other shared
    objects (by default they are local). Note that these flags may not
    be supported by your operating system. Check the documentation of
    @code{dlopen()} or equivalent on your operating system. Unsupported
    flags  are silently ignored. 

@item close_shared_object(+@var{Handle})
@findex close_shared_object/1
@snindex close_shared_object/1
@cnindex close_shared_object/1
    Detach the shared object identified by @var{Handle}. 

@item call_shared_object_function(+@var{Handle}, +@var{Function})
@findex call_shared_object_function/2
@snindex call_shared_object_function/2
@cnindex call_shared_object_function/2
    Call the named function in the loaded shared library. The function
    is called without arguments and the return-value is
    ignored. In SWI-Prolog, normally this function installs foreign
    language predicates using calls to @code{PL_register_foreign()}.
@end table

@node Save&Rest, YAP4 Notes, Loading Objects, C-Interface
@section Saving and Restoring

@comment The primitive predicates @code{save} and @code{restore} will save and restore
@comment object code loaded with @code{load_foreign_files}. However, the values of
@comment any non-static data created by the C files loaded will not be saved nor
@comment restored.

YAP4 currently does not support @code{save} and @code{restore} for object code
loaded with @code{load_foreign_files}. We plan to support save and restore
in future releases of YAP.

@node YAP4 Notes, , Save&Rest, C-Interface
@section Changes to the C-Interface in YAP4

YAP4 includes several changes over the previous @code{load_foreign_files}
interface. These changes were required to support the new binary code
formats, such as ELF used in Solaris2 and Linux.
@itemize @bullet
@item All Names of YAP objects now start with @var{YAP_}. This is
designed to avoid clashes with other code. Use @code{YAPInterface.h} to
take advantage of the new interface. @code{c_interface.h} is still
available if you cannot port the code to the new interface.

@item Access to elements in the new interface always goes through
@emph{functions}. This includes access to the argument registers,
@code{YAP_ARG1} to @code{YAP_ARG16}. This change breaks code such as
@code{unify(&ARG1,&t)}, which is nowadays:
@example
@{
   YAP_Unify(ARG1, t);
@}
@end example

@item @code{cut_fail()} and @code{cut_succeed()} are now functions.

@item The use of @code{Deref} is deprecated. All functions that return
Prolog terms, including the ones that access arguments, already
dereference their arguments.

@item Space allocated with PRESERVE_DATA is ignored by garbage
collection and stack shifting. As a result, any pointers to a Prolog
stack object, including some terms, may be corrupted after garbage
collection or stack shifting. Prolog terms should instead be stored as
arguments to the backtrackable procedure.

@end itemize

@node YAPLibrary, Compatibility, C-Interface, Top
@chapter Using YAP as a Library

YAP can be used as a library to be called from other
programs. To do so, you must first create the YAP library:
@example
make library
make install_library
@end example
This will install a file @code{libyap.a} in @var{LIBDIR} and the Prolog
headers in @var{INCLUDEDIR}. The library contains all the functionality
available in YAP, except the foreign function loader and for
@code{YAP}'s startup routines.

To actually use this library you must follow a five step process:

@enumerate
@item
 You must initialize the YAP environment. A single function,
@code{YAP_FastInit} asks for a contiguous chunk in your memory space, fills
it in with the data-base, and sets up YAP's stacks and
execution registers. You can use a saved space from a standard system by
calling @code{save_program/1}.
     
@item You then have to prepare a query to give to
YAP. A query is a Prolog term, and you just have to use the same
functions that are available in the C-interface.

@item You can then use @code{YAP_RunGoal(query)} to actually evaluate your
query. The argument is the query term @code{query}, and the result is 1
if the query succeeded, and 0 if it failed.

@item You can use the term destructor functions to check how
arguments were instantiated.

@item If you want extra solutions, you can use
@code{YAP_RestartGoal()} to obtain the next solution.

@end enumerate

The next program shows how to use this system. We assume the saved
program contains two facts for the procedure @t{b}:

@example
@cartouche
#include <stdio.h>
#include "YAP/YAPInterface.h"


int
main(int argc, char *argv[]) @{
  if (YAP_FastInit("saved_state") == YAP_BOOT_ERROR)
    exit(1);
  if (YAP_RunGoal(YAP_MkAtomTerm(YAP_LookupAtom("do")))) @{
    printf("Success\n");
    while (YAP_RestartGoal())
      printf("Success\n");
  @}
  printf("NO\n");
@}
@end cartouche
@end example

The program first initializes YAP, calls the query for the
first time and succeeds, and then backtracks twice. The first time
backtracking succeeds, the second it fails and exits.

To compile this program it should be sufficient to do:

@example
cc -o exem -I../YAP4.3.0 test.c -lYAP -lreadline -lm
@end example

You may need to adjust the libraries and library paths depending on the
Operating System and your installation of YAP.

Note that YAP4.3.0 provides the first version of the interface. The
interface may change and improve in the future.

The following C-functions are available from YAP:

@itemize @bullet
@item  YAP_CompileClause(@code{YAP_Term} @var{Clause})
@findex  YAP_CompileClause/1
Compile the Prolog term @var{Clause} and assert it as the last clause
for the corresponding procedure.

@item  @code{int} YAP_ContinueGoal(@code{void})
@findex YAP_ContinueGoal/0
Continue execution from the point where it stopped.

@item  @code{void} YAP_Error(@code{int} @var{ID},@code{YAP_Term} @var{Cause},@code{char *} @var{error_description})
@findex YAP_Error/1
Generate an YAP System Error with description given by the string
@var{error_description}. @var{ID} is the error ID, if known, or
@code{0}. @var{Cause} is the term that caused the crash.

@item  @code{void} YAP_Exit(@code{int} @var{exit_code})
@findex YAP_Exit/1
Exit YAP immediately. The argument @var{exit_code} gives the error code
and is supposed to be 0 after successful execution in Unix and Unix-like
systems.

@item  @code{YAP_Term} YAP_GetValue(@code{Atom} @var{at})
@findex  YAP_GetValue/1
Return the term @var{value} associated with the atom @var{at}. If no
such term exists the function will return the empty list.

@item  YAP_FastInit(@code{char *} @var{SavedState})
@findex  YAP_FastInit/1
Initialize a copy of YAP from @var{SavedState}. The copy is
monolithic and currently must be loaded at the same address where it was
saved. @code{YAP_FastInit} is a simpler version of @code{YAP_Init}.

@item  YAP_Init(@var{InitInfo})
@findex  YAP_Init/1
Initialize YAP. The arguments are in a @code{C}
structure of type @code{YAP_init_args}.

The fields of @var{InitInfo} are @code{char *} @var{SavedState},
@code{int} @var{HeapSize}, @code{int} @var{StackSize}, @code{int}
@var{TrailSize}, @code{int} @var{NumberofWorkers}, @code{int}
@var{SchedulerLoop}, @code{int} @var{DelayedReleaseLoad}, @code{int}
@var{argc}, @code{char **} @var{argv}, @code{int} @var{ErrorNo}, and
@code{char *} @var{ErrorCause}. The function returns an integer, which
indicates the current status. If the result is @code{YAP_BOOT_ERROR}
booting failed.

If @var{SavedState} is not NULL, try to open and restore the file
@var{SavedState}. Initially YAP will search in the current directory. If
the saved state does not exist in the current directory YAP will use
either the default library directory or the directory given by the
environment variable @code{YAPLIBDIR}. Note that currently
the saved state must be loaded at the same address where it was saved.

If @var{HeapSize} is different from 0 use @var{HeapSize} as the minimum
size of the Heap (or code space). If @var{StackSize} is different from 0
use @var{HeapSize} as the minimum size for the Stacks. If
@var{TrailSize} is different from 0 use @var{TrailSize} as the minimum
size for the Trails.

The @var{NumberofWorkers}, @var{NumberofWorkers}, and
@var{DelayedReleaseLoad} are only of interest to the or-parallel system.

The argument count @var{argc} and string of arguments @var{argv}
arguments are to be passed to user programs as the arguments used to
call YAP.

If booting failed you may consult @code{ErrorNo} and @code{ErrorCause}
for the cause of the error, or call
@code{YAP_Error(ErrorNo,0L,ErrorCause)} to do default processing. 


@item  @code{void} YAP_PutValue(@code{Atom} @var{at}, @code{YAP_Term} @var{value})
@findex  YAP_PutValue/2
Associate the term @var{value} with the atom @var{at}. The term
@var{value} must be a constant. This functionality is used by YAP as a
simple way for controlling and communicating with the Prolog run-time.

@item  @code{YAP_Term} YAP_Read(@code{int (*)(void)} @var{GetC})
@findex  YAP_Read/1
Parse a Term  using the function @var{GetC} to input characters.

@item  @code{YAP_Term} YAP_Write(@code{YAP_Term} @var{t})
@findex  YAP_CopyTerm/1
Copy a Term @var{t} and all associated constraints. May call the garbage
collector and returns @code{0L} on error (such as no space being
available).

@item  @code{void} YAP_Write(@code{YAP_Term} @var{t}, @code{void (*)(int)}
@var{PutC}, @code{int} @var{flags})
@findex  YAP_Write/3
Write a Term @var{t} using the function @var{PutC} to output
characters. The term is written according to a mask of the following
flags in the @code{flag} argument: @code{YAP_WRITE_QUOTED},
@code{YAP_WRITE_HANDLE_VARS},  and @code{YAP_WRITE_IGNORE_OPS}.

@item  @code{void} YAP_WriteBuffer(@code{YAP_Term} @var{t}, @code{char *}
@var{buff}, @code{unsigned int}
@var{size}, @code{int} @var{flags})
@findex  YAP_WriteBuffer/4
Write a YAP_Term @var{t} to buffer @var{buff} with size @var{size}. The
term is written according to a mask of the following flags in the
@code{flag} argument: @code{YAP_WRITE_QUOTED},
@code{YAP_WRITE_HANDLE_VARS}, and @code{YAP_WRITE_IGNORE_OPS}.

@item  @code{void} YAP_InitConsult(@code{int} @var{mode}, @code{char *} @var{filename})
@findex YAP_InitConsult/2
Enter consult mode on file @var{filename}. This mode maintains a few
data-structures internally, for instance to know whether a predicate
before or not. It is still possible to execute goals in consult mode.

If @var{mode} is @code{TRUE} the file will be reconsulted, otherwise
just consulted. In practice, this function is most useful for
bootstrapping Prolog, as otherwise one may call the Prolog predicate
@code{compile/1} or @code{consult/1} to do compilation.

Note that it is up to the user to open the file @var{filename}. The
@code{YAP_InitConsult} function only uses the file name for internal
bookkeeping.

@item  @code{void} YAP_EndConsult(@code{void})
@findex YAP_EndConsult/0
Finish consult mode.

@end itemize

Some observations:

@itemize @bullet
@item The system will core dump if you try to load the saved state in a
different address from where it was made. This may be a problem if
your program uses @code{mmap}. This problem will be addressed in future
versions of YAP.

@item Currently, the YAP library will pollute the name
space for your program.

@item The initial library includes the complete YAP system. In
the future we plan to split this library into several smaller libraries
(e.g. if you do not want to perform I/O).

@item You can generate your own saved states. Look at  the
@code{boot.yap} and @code{init.yap} files.

@end itemize

@node Compatibility, Operators, YAPLibrary, Top
@chapter Compatibility with Other Prolog systems

YAP has been designed to be as compatible as possible with
other Prolog systems, and initially with C-Prolog. More recent work on
YAP has included features initially proposed for the Quintus
and SICStus Prolog systems.

Developments since @code{YAP4.1.6} we have striven at making
YAP compatible with the ISO-Prolog standard. 

@menu
* C-Prolog:: Compatibility with the C-Prolog interpreter
* SICStus Prolog:: Compatibility with the SICStus Prolog system
* ISO Prolog::  Compatibility with the ISO Prolog standard
@end menu

@node C-Prolog, SICStus Prolog, , Compatibility
@section Compatibility with the C-Prolog interpreter

@menu
C-Prolog Compatibility
* Major Differences with C-Prolog:: Major Differences between YAP and C-Prolog
* Fully C-Prolog Compatible:: YAP predicates fully compatible with
C-Prolog
* Not Strictly C-Prolog Compatible:: YAP predicates not strictly as C-Prolog
* Not in C-Prolog:: YAP predicates not available in C-Prolog
* Not in YAP:: C-Prolog predicates not available in YAP
@end menu

@node Major Differences with C-Prolog, Fully C-Prolog Compatible, , C-Prolog
@subsection Major Differences between YAP and C-Prolog.

YAP includes several extensions over the original C-Prolog system. Even
so, most C-Prolog programs should run under YAP without changes.

The most important difference between YAP and C-Prolog is that, being
YAP a compiler, some changes should be made if predicates such as
@code{assert}, @code{clause} and @code{retract} are used. First
predicates which will change during execution should be declared as
@code{dynamic} by using commands like:

@example
:- dynamic f/n.
@end example

@noindent where @code{f} is the predicate name and n is the arity of the
predicate. Note that  several such predicates can be declared in a
single command:
@example
 :- dynamic f/2, ..., g/1.
@end example

Primitive predicates such as @code{retract} apply only to dynamic
predicates.  Finally note that not all the C-Prolog primitive predicates
are implemented in YAP. They can easily be detected using the
@code{unknown} system predicate provided by YAP.

Last, by default YAP enables character escapes in strings. You can
disable the special interpretation for the escape character by using:
@example
@code{:- yap_flag(character_escapes,off).}
@end example
@noindent
or by using:
@example
@code{:- yap_flag(language,cprolog).}
@end example

@node Fully C-Prolog Compatible, Not Strictly C-Prolog Compatible, Major Differences with C-Prolog, C-Prolog
@subsection YAP predicates fully compatible with C-Prolog

These are the Prolog built-ins that are fully compatible in both
C-Prolog and YAP:

@printindex cy

@node Not Strictly C-Prolog Compatible, Not in C-Prolog, Fully C-Prolog Compatible, C-Prolog
@subsection YAP predicates not strictly compatible with C-Prolog

These are YAP built-ins that are also available in C-Prolog, but
that are not fully compatible:

@printindex ca

@node Not in C-Prolog, Not in YAP, Not Strictly C-Prolog Compatible, C-Prolog
@subsection YAP predicates not available in C-Prolog

These are YAP built-ins not available in C-Prolog.

@printindex cn

@node Not in YAP, , Not in C-Prolog, C-Prolog
@subsection YAP predicates not available in C-Prolog

These are C-Prolog built-ins not available in YAP:

@table @code
@item 'LC'
The following Prolog text uses lower case letters.

@item 'NOLC'
The following Prolog text uses upper case letters only.
@end table

@node SICStus Prolog, ISO Prolog, C-Prolog, Compatibility
@section Compatibility with the Quintus and SICStus Prolog systems

The Quintus Prolog system was the first Prolog compiler to use Warren's
Abstract Machine. This system was very influential in the Prolog
community. Quintus Prolog implemented compilation into an abstract
machine code, which was then emulated. Quintus Prolog also included
several new built-ins, an extensive library, and in later releases a
garbage collector. The SICStus Prolog system, developed at SICS (Swedish
Institute of Computer Science), is an emulator based Prolog system
largely compatible with Quintus Prolog. SICStus Prolog has evolved
through several versions. The current version includes several
extensions, such as an object implementation, co-routining, and
constraints.

Recent work in YAP has been influenced by work in Quintus and
SICStus Prolog. Wherever possible, we have tried to make YAP
compatible with recent versions of these systems, and specifically of
SICStus Prolog. You should use 
@example
:- yap_flag(language, sicstus).
@end example
@noindent
for maximum compatibility with SICStus Prolog.

@menu
SICStus Compatibility
* Major Differences with SICStus:: Major Differences between YAP and SICStus Prolog
* Fully SICStus Compatible:: YAP predicates fully compatible with
SICStus Prolog
* Not Strictly SICStus Compatible:: YAP predicates not strictly as
SICStus Prolog
* Not in SICStus Prolog:: YAP predicates not available in SICStus Prolog
@end menu

@node Major Differences with SICStus, Fully SICStus Compatible, , SICStus Prolog
@subsection Major Differences between YAP and SICStus Prolog.

Both YAP and SICStus Prolog obey the Edinburgh Syntax and are based on
the WAM. Even so, there are quite a few important differences:

@itemize @bullet
@item Differently from SICStus Prolog, YAP does not have a
notion of interpreted code. All code in YAP is compiled.

@item YAP does not support an intermediate byte-code
representation, so the @code{fcompile/1} and @code{load/1} built-ins are
not available in YAP.

@item YAP implements escape sequences as in the ISO standard. SICStus
Prolog implements Unix-like escape sequences.

@item YAP implements @code{initialization/1} as per the ISO
standard. Use @code{prolog_initialization/1} for the SICStus Prolog
compatible built-in.

@item Prolog flags are different in SICStus Prolog and in YAP.

@item The SICStus Prolog @code{on_exception/3} and
@code{raise_exception} built-ins correspond to the ISO built-ins
@code{catch/3} and @code{throw/1}.

@item The following SICStus Prolog v3 built-ins are not (currently)
implemented in YAP (note that this is only a partial list):
@code{file_search_path/2},
@code{stream_interrupt/3}, @code{reinitialize/0}, @code{help/0},
@code{help/1}, @code{trimcore/0}, @code{load_files/1},
@code{load_files/2}, and @code{require/1}.

      The previous list is incomplete. We also cannot guarantee full
compatibility for other built-ins (although we will try to address any
such incompatibilities). Last, SICStus Prolog is an evolving system, so
one can be expect new incompatibilities to be introduced in future
releases of SICStus Prolog.

@item YAP allows asserting and abolishing static code during
execution through the @code{assert_static/1} and @code{abolish/1}
built-ins. This is not allowed in Quintus Prolog or SICStus Prolog.

@item The socket predicates, although designed to be compatible with
SICStus Prolog, are built-ins, not library predicates, in YAP.

@item This list is incomplete.

@end itemize

The following differences only exist if the @code{language} flag is set
to @code{yap} (the default):

@itemize @bullet
@item The @code{consult/1} predicate in YAP follows C-Prolog
semantics. That is, it adds clauses to the data base, even for
preexisting procedures. This is different from @code{consult/1} in
SICStus Prolog.

@cindex update semantics
@item By default, the data-base in YAP follows "immediate update
semantics", instead of "logical update semantics", as Quintus Prolog or
SICStus Prolog do. The difference is depicted in the next example:

@example
:- dynamic a/1.

?- assert(a(1)).

?- retract(a(X)), X1 is X +1, assertz(a(X)).
@end example
With immediate semantics, new clauses or entries to the data base are
visible in backtracking. In this example, the first call to
@code{retract/1} will succeed. The call to @strong{assertz/1} will then
succeed. On backtracking, the system will retry
@code{retract/1}. Because the newly asserted goal is visible to
@code{retract/1}, it can be retracted from the data base, and
@code{retract(a(X))} will succeed again. The process will continue
generating integers for ever. Immediate semantics were used in C-Prolog.

With logical update semantics, any additions or deletions of clauses
for a goal @emph{will not affect previous activations of the
goal}. In the example, the call to @code{assertz/1} will not see the
update performed by the @code{assertz/1}, and the query will have a
single solution.

Calling @code{yap_flag(update_semantics,logical)} will switch
YAP to use logical update semantics.

@item @code{dynamic/1} is a built-in, not a directive, in YAP.

@item By default, YAP fails on undefined predicates. To follow default
SICStus Prolog use:
@example
:- yap_flag(unknown,error).
@end example

@item By default, directives in YAP can be called from the top level.

@end itemize

@node Fully SICStus Compatible, Not Strictly SICStus Compatible, Major Differences with SICStus, SICStus Prolog
@subsection YAP predicates fully compatible with SICStus Prolog

These are the Prolog built-ins that are fully compatible in both SICStus
Prolog and YAP:

@printindex sy

@node Not Strictly SICStus Compatible, Not in SICStus Prolog, Fully SICStus Compatible, SICStus Prolog
@subsection YAP predicates not strictly compatible with SICStus Prolog

These are YAP built-ins that are also available in SICStus Prolog, but
that are not fully compatible:

@printindex sa

@node Not in SICStus Prolog, , Not Strictly SICStus Compatible, SICStus Prolog
@subsection YAP predicates not available in SICStus Prolog

These are YAP built-ins not available in SICStus Prolog.

@printindex sn


@node ISO Prolog, , SICStus Prolog, Compatibility
@section Compatibility with the ISO Prolog standard

The Prolog standard was developed by ISO/IEC JTC1/SC22/WG17, the
international standardization working group for the programming language
Prolog. The book "Prolog: The Standard" by Deransart, Ed-Dbali and
Cervoni gives a complete description of this standard. Development in
YAP from YAP4.1.6 onwards have striven at making YAP
compatible with ISO Prolog. As such:

@itemize @bullet
@item   YAP now supports all of the built-ins required by the
ISO-standard, and,
@item   Error-handling is as required by the standard.
@end itemize

YAP by default is not fully ISO standard compliant. You can set the 
@code{language} flag to @code{iso} to obtain very good
compatibility. Setting this flag changes the following:

@itemize @bullet
@item By default, YAP uses "immediate update semantics" for its
database, and not "logical update semantics", as per the standard,
(@pxref{SICStus Prolog}). This affects @code{assert/1},
@code{retract/1}, and friends.

Calling @code{set_prolog_flag(update_semantics,logical)} will switch
YAP to use logical update semantics.

@item By default, YAP implements the @code{atom_chars/2}
(@pxref{Testing Terms}), and @code{number_chars/2}, (@pxref{Testing
Terms}), built-ins as per the original Quintus Prolog definition, and
not as per the ISO definition.

Calling @code{set_prolog_flag(to_chars_mode,iso)} will switch
YAP to use the ISO definition for
@code{atom_chars/2} and @code{number_chars/2}.

@item By default, YAP fails on undefined predicates. To follow the ISO
Prolog standard use:
@example
:- set_prolog_flag(unknown,error).
@end example

@item By default, YAP allows executable goals in directives. In ISO mode
most directives can only be called from top level (the exceptions are
@code{set_prolog_flag/2} and @code{op/3}).

@item Error checking for meta-calls under ISO Prolog mode is stricter
than by default.

@item The @code{strict_iso} flag automatically enables the ISO Prolog
standard. This feature should disable all features not present in the
standard.

@end itemize

The following incompatibilities between YAP and the ISO standard are
known to still exist:

@itemize @bullet

@item Currently, YAP does not handle overflow errors in integer
operations, and handles floating-point errors only in some
architectures. Otherwise, YAP follows IEEE arithmetic.

@end itemize

Please inform the authors on other incompatibilities that may still
exist.

@node Operators, Predicate Index, Compatibility, Top
@appendix Summary of YAP Predefined Operators


 The Prolog syntax caters for operators of three main kinds:

@itemize @bullet
@item
prefix;
@item
infix;
@item
postfix.
@end itemize

 Each operator has precedence in the range 1 to 1200, and this 
precedence is used to disambiguate expressions where the structure of the 
term denoted is not made explicit using brackets. The operator of higher 
precedence is the main functor.

 If there are two operators with the highest precedence, the ambiguity 
is solved analyzing the types of the operators. The possible infix types are: 
@var{xfx}, @var{xfy}, and @var{yfx}.

 With an operator of type @var{xfx} both sub-expressions must have lower 
precedence than the operator itself, unless they are bracketed (which 
assigns to them zero precedence). With an operator type @var{xfy} only the  
left-hand sub-expression must have lower precedence. The opposite happens 
for @var{yfx} type.

 A prefix operator can be of type @var{fx} or @var{fy}. 
A postfix operator can be of type @var{xf} or @var{yf}. 
The meaning of the notation is analogous to the above.
@example
a + b * c
@end example
@noindent
means
@example
a + (b * c)
@end example
@noindent
as + and * have the following types and precedences:
@example
:-op(500,yfx,'+').
:-op(400,yfx,'*').
@end example

Now defining
@example
:-op(700,xfy,'++').
:-op(700,xfx,'=:=').
a ++ b =:= c
@end example
@noindent means
@example  
a ++ (b =:= c)
@end example
 

The following is the list of the declarations of the predefined operators:

@example
:-op(1200,fx,['?-', ':-']).
:-op(1200,xfx,[':-','-->']).
:-op(1150,fx,[block,dynamic,mode,public,multifile,meta_predicate,
              sequential,table,initialization]).
:-op(1100,xfy,[';','|']).
:-op(1050,xfy,->).
:-op(1000,xfy,',').
:-op(999,xfy,'.').
:-op(900,fy,['\+', not]).
:-op(900,fx,[nospy, spy]).
:-op(700,xfx,[@@>=,@@=<,@@<,@@>,<,=,>,=:=,=\=,\==,>=,=<,==,\=,=..,is]).
:-op(500,yfx,['\/','/\','+','-']).
:-op(500,fx,['+','-']).
:-op(400,yfx,['<<','>>','//','*','/']).
:-op(300,xfx,mod).
:-op(200,xfy,['^','**']).
:-op(50,xfx,same).
@end example

@node Predicate Index, Concept Index, Operators, Top
@unnumbered Predicate Index
@printindex fn

@node Concept Index, , Predicate Index, Top
@unnumbered Concept Index
@printindex cp

@contents

@bye
