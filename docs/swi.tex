@chapter SWI-Prolog Emulation

This library provides a number of SWI-Prolog builtins that are not by
default in YAP. This library is loaded with the
@code{use_module(library(swi))} command.

@table @code

@item append(?@var{List1},?@var{List2},?@var{List3})
@findex append/3
@snindex append/3
@cnindex append/3
Succeeds when @var{List3} unifies with the concatenation of @var{List1}
and @var{List2}. The predicate can be used with any instantiation
pattern (even three variables).

@item between(+@var{Low},+@var{High},?@var{Value})
@findex between/3
@snindex between/3
@cnindex between/3

@var{Low} and @var{High} are integers, @var{High} less or equal than
@var{Low}. If @var{Value} is an integer, @var{Low} less or equal than
@var{Value} less or equal than @var{High}.  When @var{Value} is a
variable it is successively bound to all integers between @var{Low} and
@var{High}.  If @var{High} is @code{inf}, @code{between/3} is true iff
@var{Value} less or equal than @var{Low}, a feature that is particularly
interesting for generating integers from a certain value.

@item chdir(+@var{Dir})
@findex chdir/1
@snindex chdir/1
@cnindex chdir/1

Compatibility predicate.  New code should use @code{working_directory/2}.

@item concat_atom(+@var{List},-@var{Atom})
@findex concat_atom/2
@snindex concat_atom/2
@cnindex concat_atom/2

@var{List} is a list of atoms, integers or floating point numbers. Succeeds
if @var{Atom} can be unified with the concatenated elements of @var{List}. If
@var{List} has exactly 2 elements it is equivalent to @code{atom_concat/3},
allowing for variables in the list.

@item concat_atom(?@var{List},+@var{Separator},?@var{Atom})
@findex concat_atom/3
@snindex concat_atom/3
@cnindex concat_atom/3

Creates an atom just like concat_atom/2, but inserts @var{Separator}
between each pair of atoms.  For example:
\@example
?- concat_atom([gnu, gnat], ', ', A).

A = 'gnu, gnat'
@end example

(Unimplemented) This predicate can also be used to split atoms by
instantiating @var{Separator} and @var{Atom}:

@example
?- concat_atom(L, -, 'gnu-gnat').

L = [gnu, gnat]
@end example

@item nth1(+@var{Index},?@var{List},?@var{Elem})
@findex nth1/3
@snindex nth1/3
@cnindex nth1/3
Succeeds when the @var{Index}-th element of @var{List} unifies with
@var{Elem}. Counting starts at 1.

Set environment variable.  @var{Name} and @var{Value} should be
instantiated to atoms or integers.  The environment variable will be
passed to @code{shell/[0-2]} and can be requested using @code{getenv/2}.
They also influence @code{expand_file_name/2}.

@item setenv(+@var{Name},+@var{Value})
@findex setenv/2
@snindex setenv/2
@cnindex setenv/2
Set environment variable.  @var{Name} and @var{Value} should be
instantiated to atoms or integers.  The environment variable will be
passed to @code{shell/[0-2]} and can be requested using @code{getenv/2}.
They also influence @code{expand_file_name/2}.

@item term_to_atom(?@var{Term},?@var{Atom})
@findex term_to_atom/2
@snindex term_to_atom/2
@cnindex term_to_atom/2
Succeeds if @var{Atom} describes a term that unifies with @var{Term}. When
@var{Atom} is instantiated @var{Atom} is converted and then unified with
@var{Term}.  If @var{Atom} has no valid syntax, a @code{syntax_error}
exception is raised. Otherwise @var{Term} is ``written'' on @var{Atom}
using @code{write/1}.

@item working_directory(-@var{Old},+@var{New})
@findex working_directory/2
@snindex working_directory/2
@cnindex working_directory/2

Unify @var{Old} with an absolute path to the current working directory
and change working directory to @var{New}.  Use the pattern
@code{working_directory(CWD, CWD)} to get the current directory.  See
also @code{absolute_file_name/2} and @code{chdir/1}.

@end table

@node Invoking Predicates on all Members of a List,Forall, , SWI-Prolog
@section Invoking Predicates on all Members of a List
@c \label{sec:applylist}

All the predicates in this section call a predicate on all members of a
list or until the predicate called fails.  The predicate is called via
@code{call/[2..]}, which implies common arguments can be put in
front of the arguments obtained from the list(s). For example:

@example
?- maplist(plus(1), [0, 1, 2], X).

X = [1, 2, 3]
@end example

we will phrase this as ``@var{Predicate} is applied on ...''

@table @code

@item maplist(+@var{Pred},+@var{List})
@findex maplist/2
@snindex maplist/2
@cnindex maplist/2
@var{Pred} is applied successively on each element of @var{List} until
the end of the list or @var{Pred} fails. In the latter case 
@code{maplist/2} fails.

@item maplist(+@var{Pred},+@var{List1},+@var{List2})
@findex maplist/3
@snindex maplist/3
@cnindex maplist/3
Apply @var{Pred} on all successive triples of elements from
@var{List1} and
@var{List2}. Fails if @var{Pred} can not be applied to a
pair. See the example above.

@item maplist(+@var{Pred},+@var{List1},+@var{List2},+@var{List4})
@findex maplist/4
@snindex maplist/4
@cnindex maplist/4
Apply @var{Pred} on all successive triples of elements from @var{List1},
@var{List2} and @var{List3}. Fails if @var{Pred} can not be applied to a
triple. See the example above.

@c @item findlist(+@var{Pred},+@var{List1},?@var{List2})
@c @findex findlist/3
@c @snindex findlist/3
@c @cnindex findlist/3
@c Unify @var{List2} with a list of all elements of @var{List1} to which
@c @var{Pred} applies.
@end table

@node Forall,hProlog and SWI-Prolog Attributed Variables,Invoking Predicates on all Members of a List, SWI-Prolog
@section Forall			
@c \label{sec:forall2}

@table @code
@item forall(+@var{Cond},+@var{Action})
@findex forall/2
@snindex forall/2
@cnindex forall/2

For all alternative bindings of @var{Cond} @var{Action} can be proven.
The next example verifies that all arithmetic statements in the list
@var{L} are correct. It does not say which is wrong if one proves wrong.

@example
?- forall(member(Result = Formula, [2 = 1 + 1, 4 = 2 * 2]),
                 Result =:= Formula).
@end example

@end table

@node hProlog and SWI-Prolog Attributed Variables, SWI-Prolog Global Variables, Forall,SWI-Prolog
@section hProlog and SWI-Prolog Attributed Variables
@cindex hProlog Attributed Variables

Attributed variables
@c @ref{Attributed variables}
provide a technique for extending the
Prolog unification algorithm by hooking the binding of attributed
variables. There is little consensus in the Prolog community on the
exact definition and interface to attributed variables.  Yap Prolog
traditionally implements a SICStus-like interface, but to enable
SWI-compatibility we have implemented the SWI-Prolog interface,
identical to the one realised by Bart Demoen for hProlog.

Binding an attributed variable schedules a goal to be executed at the
first possible opportunity. In the current implementation the hooks are
executed immediately after a successful unification of the clause-head
or successful completion of a foreign language (builtin) predicate. Each
attribute is associated to a module and the hook (attr_unify_hook/2) is
executed in this module.  The example below realises a very simple and
incomplete finite domain reasoner.

@example
:- module(domain,
	  [ domain/2			% Var, ?Domain
	  ]).
:- use_module(library(oset)).

domain(X, Dom) :-
	var(Dom), !,
	get_attr(X, domain, Dom).
domain(X, List) :-
	sort(List, Domain),
	put_attr(Y, domain, Domain),
	X = Y.

%	An attributed variable with attribute value Domain has been
%	assigned the value Y

attr_unify_hook(Domain, Y) :-
	(   get_attr(Y, domain, Dom2)
	->  oset_int(Domain, Dom2, NewDomain),
	    (   NewDomain == []
	    ->	fail
	    ;	NewDomain = [Value]
	    ->	Y = Value
	    ;	put_attr(Y, domain, NewDomain)
	    )
	;   var(Y)
	->  put_attr( Y, domain, Domain )
	;   memberchk(Y, Domain)
	).
@end example


Before explaining the code we give some example queries:

@table @code
@item ?- domain(X, [a,b]), X = c		     
no
@item ?- domain(X, [a,b]), domain(X, [a,c]).   
      X = a
@item ?- domain(X, [a,b,c]), domain(X, [a,c]). 
  X = _D0
@end table

The predicate @code{domain/2} fetches (first clause) or assigns
(second clause) the variable a @emph{domain}, a set of values it can
be unified with.  In the second clause first associates the domain
with a fresh variable and then unifies X to this variable to deal
with the possibility that X already has a domain. The
predicate @code{attr_unify_hook/2} is a hook called after a variable with
a domain is assigned a value.  In the simple case where the variable
is bound to a concrete value we simply check whether this value is in
the domain. Otherwise we take the intersection of the domains and either
fail if the intersection is empty (first example), simply assign the
value if there is only one value in the intersection (second example) or
assign the intersection as the new domain of the variable (third
example).


@table @code

@item put_attr(+@var{Var},+@var{Module},+@var{Value})
@findex put_attr/3
@snindex put_attr/3
@cnindex put_attr/3
If @var{Var} is a variable or attributed variable, set the value for the
attribute named @var{Module} to @var{Value}. If an attribute with this
name is already associated with @var{Var}, the old value is replaced.
Backtracking will restore the old value (i.e. an attribute is a mutable
term. See also @code{setarg/3}). This predicate raises a type error if
@var{Var} is not a variable or @var{Module} is not an atom.

@item get_attr(+@var{Var},+@var{Module},+@var{Value})
@findex get_attr/3
@snindex get_attr/3
@cnindex get_attr/3
Request the current @var{value} for the attribute named @var{Module}.  If
@var{Var} is not an attributed variable or the named attribute is not
associated to @var{Var} this predicate fails silently.  If @var{Module}
is not an atom, a type error is raised.

@item del_attr(+@var{Var},+@var{Module})
@findex del_attr/2
@snindex del_attr/2
@cnindex del_attr/2
Delete the named attribute.  If @var{Var} loses its last attribute it
is transformed back into a traditional Prolog variable.  If @var{Module}
is not an atom, a type error is raised. In all other cases this
predicate succeeds regarless whether or not the named attribute is
present.

@item attr_unify_hook(+@var{AttValue},+@var{VarValue})
@findex attr_unify_hook/2
@snindex attr_unify_hook/2
@cnindex attr_unify_hook/2
Hook that must be defined in the module an attributed variable refers
to. Is is called @emph{after} the attributed variable has been
unified with a non-var term, possibly another attributed variable.
@var{AttValue} is the attribute that was associated to the variable
in this module and @var{VarValue} is the new value of the variable.
Normally this predicate fails to veto binding the variable to
@var{VarValue}, forcing backtracking to undo the binding.  If
@var{VarValue} is another attributed variable the hook often combines
the two attribute and associates the combined attribute with 
@var{VarValue} using @code{put_attr/3}.

@c     \predicate{attr_portray_hook}{2}{+AttValue, +Var}
@c Called by write_term/2 and friends for each attribute if the option
@c \term{attributes}{portray} is in effect.  If the hook succeeds the
@c attribute is considered printed.  Otherwise \exam{Module = ...} is
@c printed to indicate the existence of a variable.
@end table

@subsection Special Purpose SWI Predicates for Attributes

Normal user code should deal with @code{put_attr/3}, @code{get_attr/3}
and @code{del_attr/2}.  The routines in this section fetch or set the
entire attribute list of a variables. Use of these predicates is
anticipated to be restricted to printing and other special purpose
operations.

@table @code
@item get_attrs(+@var{Var},-@var{Attributes})
@findex get_attrs/2
@snindex get_attrs/2
@cnindex get_attrs/2
Get all attributes of @var{Var}. @var{Attributes} is a term of the form
@code{att(Module, Value, MoreAttributes)}, where @var{MoreAttributes} is
@code{[]} for the last attribute.

@item put_attrs(+@var{Var},+@var{Attributes})
@findex put_attrs/2
@snindex put_attrs/2
@cnindex put_attrs/2
Set all attributes of @var{Var}.  See get_attrs/2 for a description of
@var{Attributes}.

@item copy_term_nat(?@var{TI},-@var{TF})
@findex copy_term_nat/2
@snindex copy_term_nat/2
@cnindex copy_term_nat/2
As @code{copy_term/2}.  Attributes however, are @emph{not} copied but replaced
by fresh variables.
@end table


@node SWI-Prolog Global Variables,  ,hProlog and SWI-Prolog Attributed Variables,SWI-Prolog
@section SWI Global variables
@c		\label{sec:gvar}

SWI-Prolog global variables are associations between names (atoms) and
terms.  They differ in various ways from storing information using
@code{assert/1} or @code{recorda/3}.

@itemize @bullet
@item The value lives on the Prolog (global) stack.  This implies 
          that lookup time is independent from the size of the term.
	  This is particulary interesting for large data structures
	  such as parsed XML documents or the CHR global constraint
	  store.

@item They support both global assignment using @code{nb_setval/2} and
          backtrackable assignment using @code{b_setval/2}.

@item Only one value (which can be an arbitrary complex Prolog
   	  term) can be associated to a variable at a time.

@item Their value cannot be shared among threads.  Each thread
          has its own namespace and values for global variables.

@item Currently global variables are scoped globally.  We may
          consider module scoping in future versions.
@end itemize

Both @code{b_setval/2} and @code{nb_setval/2} implicitely create a variable if the
referenced name does not already refer to a variable.

Global variables may be initialised from directives to make them
available during the program lifetime, but some considerations are
necessary for saved-states and threads. Saved-states to not store global
variables, which implies they have to be declared with @code{initialization/1}
to recreate them after loading the saved state.  Each thread has
its own set of global variables, starting with an empty set.  Using
@code{thread_inititialization/1} to define a global variable it will be
defined, restored after reloading a saved state and created in all
threads that are created @emph{after} the registration.


@table @code
@item b_setval(+@var{Name},+@var{Value})
@findex b_setval/2
@snindex b_setval/2
@cnindex b_setval/2
Associate the term @var{Value} with the atom @var{Name} or replaces
the currently associated value with @var{Value}.  If @var{Name} does
not refer to an existing global variable a variable with initial value
@code{[]} is created (the empty list).  On backtracking the
assignment is reversed.

@item b_getval(+@var{Name},-@var{Value})
@findex b_getval/2
@snindex b_getval/2
@cnindex b_getval/2
Get the value associated with the global variable @var{Name} and unify
it with @var{Value}. Note that this unification may further instantiate
the value of the global variable. If this is undesirable the normal
precautions (double negation or @code{copy_term/2}) must be taken. The
@code{b_getval/2} predicate generates errors if @var{Name} is not an atom or
the requested variable does not exist.
@end table

@table @code

@item nb_setval(+@var{Name},+@var{Value})
@findex nb_setval/2
@snindex nb_setval/2
@cnindex nb_setval/2
Associates a copy of @var{Value} created with @code{duplicate_term/2}
with the atom @var{Name}.  Note that this can be used to set an
initial value other than @code{[]} prior to backtrackable assignment.

@item nb_getval(+@var{Name},-@var{Value})
@findex nb_getval/2
@snindex nb_getval/2
@cnindex nb_getval/2
The @code{nb_getval/2} predicate is a synonym for b_getval/2, introduced for
compatibility and symetry.  As most scenarios will use a particular
global variable either using non-backtracable or backtrackable
assignment, using @code{nb_getval/2} can be used to document that the 
variable is used non-backtracable.

@c     \predicate{nb_linkval}{2}{+Name, +Value}
@c Associates the term @var{Value} with the atom @var{Name} without copying
@c it. This is a fast special-purpose variation of nb_setval/2 intended for
@c expert users only because the semantics on backtracking to a point
@c before creating the link are poorly defined for compound terms. The
@c principal term is always left untouched, but backtracking behaviour on
@c arguments is undone if the orginal assignment was \jargon{trailed} and
@c left alone otherwise, which implies that the history that created the
@c term affects the behaviour on backtracking. Please consider the
@c following example:

@c \begin{code}
@c demo_nb_linkval :-
@c 	T = nice(N),
@c 	(   N = world,
@c 	    nb_linkval(myvar, T),
@c 	    fail
@c 	;   nb_getval(myvar, V),
@c 	    writeln(V)
@c 	).
@c \end{code}

@item nb_current(?@var{Name},?@var{Value})
@findex nb_current/2
@snindex nb_current/2
@cnindex nb_current/2
Enumerate all defined variables with their value. The order of
enumeration is undefined.

@item nb_delete(?@var{Name})
@findex nb_delete/1
@snindex nb_delete/1
@cnindex nb_delete/1
Delete the named global variable.
@end table

@subsection Compatibility of SWI-Prolog Global Variables

Global variables have been introduced by various Prolog
implementations recently.  The implementation of them in SWI-Prolog is
based on hProlog by Bart Demoen. In discussion with Bart it was
decided that the semantics if hProlog @code{nb_setval/2}, which is
equivalent to @code{nb_linkval/2} is not acceptable for normal Prolog
users as the behaviour is influenced by how builtin predicates
constructing terms (@code{read/1}, @code{=../2}, etc.) are implemented.

GNU-Prolog provides a rich set of global variables, including arrays.
Arrays can be implemented easily in SWI-Prolog using @code{functor/3} and
@code{setarg/3} due to the unrestricted arity of compound terms.


@node Extensions,Debugging,SWI-Prolog,Top 
@chapter Extensions to Prolog

YAP includes several extensions that are not enabled by
default, but that can be used to extend the functionality of the
system. These options can be set at compilation time by enabling the
related compilation flag, as explained in the @code{Makefile}
