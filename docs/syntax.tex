@c -*- mode: texinfo; coding: utf-8; -*-


@node Syntax, Loading Programs, Run, Top
@chapter Syntax

We will describe the syntax of YAP at two levels. We first will
describe the syntax for Prolog terms. In a second level we describe
the @i{tokens} from which Prolog @i{terms} are
built.

@menu
* Formal Syntax:: Syntax of terms 
* Tokens:: Syntax of Prolog tokens
* Encoding:: How characters are encoded and Wide Character Support
@end menu

@node Formal Syntax, Tokens, ,Syntax
@section Syntax of Terms
@cindex syntax

Below, we describe the syntax of YAP terms from the different
classes of tokens defined above. The formalism used will be @emph{BNF},
extended where necessary with attributes denoting integer precedence or
operator type.

@example
 term       ---->     subterm(1200)   end_of_term_marker

 subterm(N) ---->     term(M)         [M <= N]

 term(N)    ---->     op(N, fx) subterm(N-1)
             |        op(N, fy) subterm(N)
             |        subterm(N-1) op(N, xfx) subterm(N-1)
             |        subterm(N-1) op(N, xfy) subterm(N)
             |        subterm(N) op(N, yfx) subterm(N-1)
             |        subterm(N-1) op(N, xf)
             |        subterm(N) op(N, yf)

 term(0)   ---->      atom '(' arguments ')'
             |        '(' subterm(1200)  ')'
             |        '@{' subterm(1200)  '@}'
             |        list
             |        string
             |        number
             |        atom
             |        variable

 arguments ---->      subterm(999)
             |        subterm(999) ',' arguments

 list      ---->      '[]'
             |        '[' list_expr ']'

 list_expr ---->      subterm(999)
             |        subterm(999) list_tail

 list_tail ---->      ',' list_expr
             |        ',..' subterm(999)
             |        '|' subterm(999)
@end example

@noindent
Notes:

@itemize @bullet

@item
@i{op(N,T)} denotes an atom which has been previously declared with type
@i{T} and base precedence @i{N}.

@item
Since ',' is itself a pre-declared operator with type @i{xfy} and
precedence 1000, is @i{subterm} starts with a '(', @i{op} must be
followed by a space to avoid ambiguity with the case of a functor
followed by arguments, e.g.:

@example
+ (a,b)        [the same as '+'(','(a,b)) of arity one]
@end example
versus
@example
+(a,b)         [the same as '+'(a,b) of arity two]
@end example

@item
In the first rule for term(0) no blank space should exist between
@i{atom} and '('.

@item
@cindex end of term
Each term to be read by the YAP parser must end with a single
dot, followed by a blank (in the sense mentioned in the previous
paragraph). When a name consisting of a single dot could be taken for
the end of term marker, the ambiguity should be avoided by surrounding the
dot with single quotes.

@end itemize

@node Tokens, Encoding, Formal Syntax, Syntax
@section Prolog Tokens
@cindex token

Prolog tokens are grouped into the following categories:

@menu
* Numbers:: Integer and Floating-Point Numbers
* Strings:: Sequences of Characters
* Atoms:: Atomic Constants
* Variables:: Logical Variables
* Punctuation Tokens:: Tokens that separate other tokens
* Layout:: Comments and Other Layout Rules
@end menu

@node Numbers, Strings, ,Tokens
@subsection Numbers
@cindex number

Numbers can be further subdivided into integer and floating-point numbers.

@menu
* Integers:: How Integers are read and represented
* Floats:: Floating Point Numbers
@end menu

@node Integers, Floats, ,Numbers
@subsubsection Integers
@cindex integer

Integer numbers
are described by the following regular expression:

@example
<integer> := @{<digit>+<single-quote>|0@{xXo@}@}<alpha_numeric_char>+
@end example
@noindent
where @{...@} stands for optionality, @i{+} optional repetition (one or
more times), @i{<digit>} denotes one of the characters 0 ... 9, @i{|}
denotes or, and @i{<single-quote>} denotes the character "'". The digits
before the @i{<single-quote>} character, when present, form the number
basis, that can go from 0, 1 and up to 36. Letters from @code{A} to
@code{Z} are used when the basis is larger than 10.

Note that if no basis is specified then base 10 is assumed. Note also
that the last digit of an integer token can not be immediately followed
by one of the characters 'e', 'E', or '.'.

Following the ISO standard, YAP also accepts directives of the
form @code{0x} to represent numbers in hexadecimal base and of the form
@code{0o} to represent numbers in octal base. For usefulness,
YAP also accepts directives of the form @code{0X} to represent
numbers in hexadecimal base.

Example:
the following tokens all denote the same integer
@example
10  2'1010  3'101  8'12  16'a  36'a  0xa  0o12
@end example

Numbers of the form @code{0'a} are used to represent character
constants. So, the following tokens denote the same integer:
@example
0'd  100
@end example

YAP (version @value{VERSION}) supports integers that can fit
the word size of the machine. This is 32 bits in most current machines,
but 64 in some others, such as the Alpha running Linux or Digital
Unix. The scanner will read larger or smaller integers erroneously.

@node Floats, , Integers,Numbers
@subsubsection Floating-point Numbers
@cindex floating-point number

Floating-point numbers are described by:

@example
   <float> := <digit>+@{<dot><digit>+@}
               <exponent-marker>@{<sign>@}<digit>+
            |<digit>+<dot><digit>+
               @{<exponent-marker>@{<sign>@}<digit>+@}
@end example

@noindent
where @i{<dot>} denotes the decimal-point character '.',
@i{<exponent-marker>} denotes one of 'e' or 'E', and @i{<sign>} denotes
one of '+' or '-'.

Examples:
@example
10.0   10e3   10e-3   3.1415e+3
@end example

Floating-point numbers are represented as a double in the target
machine. This is usually a 64-bit number.

@node Strings, Atoms, Numbers,Tokens
@subsection Character Strings
@cindex string

Strings are described by the following rules:
@example
  string --> '"' string_quoted_characters '"'

  string_quoted_characters --> '"' '"' string_quoted_characters
  string_quoted_characters --> '\'
                          escape_sequence string_quoted_characters
  string_quoted_characters -->
                          string_character string_quoted_characters

  escape_sequence --> 'a' | 'b' | 'r' | 'f' | 't' | 'n' | 'v'
  escape_sequence --> '\' | '"' | ''' | '`'
  escape_sequence --> at_most_3_octal_digit_seq_char '\'
  escape_sequence --> 'x' at_most_2_hexa_digit_seq_char '\'
@end example
where @code{string_character} in any character except the double quote
and escape characters.

Examples:
@example
""   "a string"   "a double-quote:""" 
@end example

The first string is an empty string, the last string shows the use of
double-quoting. The implementation of YAP represents strings as
lists of integers. Since YAP 4.3.0 there is no static limit on string
size.

Escape sequences can be used to include the non-printable characters
@code{a} (alert), @code{b} (backspace), @code{r} (carriage return),
@code{f} (form feed), @code{t} (horizontal tabulation), @code{n} (new
line), and @code{v} (vertical tabulation). Escape sequences also be
include the meta-characters @code{\}, @code{"}, @code{'}, and
@code{`}. Last, one can use escape sequences to include the characters
either as an octal or hexadecimal number.

The next examples demonstrates the use of escape sequences in YAP:

@example
"\x0c\" "\01\" "\f" "\\" 
@end example

The first three examples return a list including only character 12 (form
feed). The last example escapes the escape character.

Escape sequences were not available in C-Prolog and in original
versions of YAP up to 4.2.0. Escape sequences can be disable by using:
@example
:- yap_flag(character_escapes,false).
@end example


@node Atoms, Variables, Strings, Tokens
@subsection Atoms
@cindex atom

Atoms are defined by one of the following rules:
@example
   atom --> solo-character
   atom --> lower-case-letter name-character*
   atom --> symbol-character+
   atom --> single-quote  single-quote
   atom --> ''' atom_quoted_characters '''


  atom_quoted_characters --> ''' ''' atom_quoted_characters
  atom_quoted_characters --> '\' atom_sequence string_quoted_characters
  atom_quoted_characters --> character string_quoted_characters
@end example
where:
@example
   <solo-character>     denotes one of:    ! ;
   <symbol-character>   denotes one of:    # & * + - . / : < 
                                           = > ? @@ \ ^ ~ `
   <lower-case-letter>  denotes one of:    a...z
   <name-character>     denotes one of:    _ a...z A...Z 0....9
   <single-quote>       denotes:           '
@end example

and @code{string_character} denotes any character except the double quote
and escape characters. Note that escape sequences in strings and atoms
follow the same rules.

Examples:
@example
a   a12x   '$a'   !   =>  '1 2'
@end example

Version @code{4.2.0} of YAP removed the previous limit of 256
characters on an atom. Size of an atom is now only limited by the space
available in the system.

@node Variables, Punctuation Tokens, Atoms, Tokens
@subsection Variables
@cindex variable

Variables are described by:
@example
   <variable-starter><variable-character>+
@end example
where
@example
  <variable-starter>   denotes one of:    _ A...Z
  <variable-character> denotes one of:    _ a...z A...Z
@end example

@cindex anonymous variable
If a variable is referred only once in a term, it needs not to be named
and one can use the character @code{_} to represent the variable. These
variables are known as anonymous variables. Note that different
occurrences of @code{_} on the same term represent @emph{different}
anonymous variables. 

@node Punctuation Tokens, Layout, Variables, Tokens
@subsection Punctuation Tokens
@cindex punctuation token

Punctuation tokens consist of one of the following characters:
@example
( ) , [ ] @{ @} |
@end example

These characters are used to group terms.

@node Layout, ,Punctuation Tokens, Tokens
@subsection Layout
@cindex comment
Any characters with ASCII code less than or equal to 32 appearing before
a token are ignored.

All the text appearing in a line after the character @i).  Comments can also be
inserted by using the sequence @code{/*} to start the comment and
@code{*} followed by @code{/} to finish it. In the presence of any sequence of comments or
layout characters, the YAP parser behaves as if it had found a
single blank character. The end of a file also counts as a blank
character for this purpose.

@node Encoding, , Tokens, Syntax
@section Wide Character Support

@cindex encodings

@menu
* Stream Encoding:: How Prolog Streams can be coded
* BOM:: The Byte Order Mark
@end menu

@cindex UTF-8
@cindex Unicode
@cindex UCS
@cindex internationalization

YAP now implements a SWI-Prolog compatible interface to wide
characters and the Universal Character Set (UCS). The following text
was adapted from the SWI-Prolog manual.

YAP now  supports wide characters, characters with character
codes above 255 that cannot be represented in a single byte.
@emph{Universal Character Set} (UCS) is the ISO/IEC 10646 standard
that specifies a unique 31-bits unsigned integer for any character in
any language.  It is a superset of 16-bit Unicode, which in turn is
a superset of ISO 8859-1 (ISO Latin-1), a superset of US-ASCII.  UCS
can handle strings holding characters from multiple languages and
character classification (uppercase, lowercase, digit, etc.) and
operations such as case-conversion are unambiguously defined.

For this reason YAP, following SWI-Prolog, has two representations for
atoms. If the text fits in ISO Latin-1, it is represented as an array
of 8-bit characters.  Otherwise the text is represented as an array of
wide chars, which may take 16 or 32 bits.  This representational issue
is completely transparent to the Prolog user.  Users of the foreign
language interface sometimes need to be aware of these issues though.

Character coding comes into view when characters of strings need to be
read from or written to file or when they have to be communicated to
other software components using the foreign language interface. In this
section we only deal with I/O through streams, which includes file I/O
as well as I/O through network sockets.


@node Stream Encoding, , BOM, Encoding
@subsection Wide character encodings on streams



Although characters are uniquely coded using the UCS standard
internally, streams and files are byte (8-bit) oriented and there are a
variety of ways to represent the larger UCS codes in an 8-bit octet
stream. The most popular one, especially in the context of the web, is
UTF-8. Bytes 0...127 represent simply the corresponding US-ASCII
character, while bytes 128...255 are used for multi-byte
encoding of characters placed higher in the UCS space. Especially on
MS-Windows the 16-bit Unicode standard, represented by pairs of bytes is
also popular.

Prolog I/O streams have a property called @emph{encoding} which
specifies the used encoding that influence @code{get_code/2} and
@code{put_code/2} as well as all the other text I/O predicates.

The default encoding for files is derived from the Prolog flag
@code{encoding}, which is initialised from the environment.  If the
environment variable @env{LANG} ends in "UTF-8", this encoding is
assumed. Otherwise the default is @code{text} and the translation is
left to the wide-character functions of the C-library (note that the
Prolog native UTF-8 mode is considerably faster than the generic
@code{mbrtowc()} one).  The encoding can be specified explicitly in
@code{load_files/2} for loading Prolog source with an alternative
encoding, @code{open/4} when opening files or using @code{set_stream/2} on
any open stream (not yet implemented). For Prolog source files we also
provide the @code{encoding/1} directive that can be used to switch
between encodings that are compatible to US-ASCII (@code{ascii},
@code{iso_latin_1}, @code{utf8} and many locales).  
@c See also
@c \secref{intsrcfile} for writing Prolog files with non-US-ASCII
@c characters and \secref{unicodesyntax} for syntax issues. 
For
additional information and Unicode resources, please visit
@uref{http://www.unicode.org/}.

YAP currently defines and supports the following encodings:

@itemize @bullet
@item  octet
Default encoding for @emph{binary} streams.  This causes
the stream to be read and written fully untranslated.

@item  ascii
7-bit encoding in 8-bit bytes.  Equivalent to @code{iso_latin_1},
but generates errors and warnings on encountering values above
127.

@item  iso_latin_1
8-bit encoding supporting many western languages.  This causes
the stream to be read and written fully untranslated.

@item  text
C-library default locale encoding for text files.  Files are read and
written using the C-library functions @code{mbrtowc()} and
@code{wcrtomb()}.  This may be the same as one of the other locales,
notably it may be the same as @code{iso_latin_1} for western
languages and @code{utf8} in a UTF-8 context.

@item  utf8
Multi-byte encoding of full UCS, compatible to @code{ascii}.
See above.

@item  unicode_be
Unicode Big Endian.  Reads input in pairs of bytes, most
significant byte first.  Can only represent 16-bit characters.

@item  unicode_le
Unicode Little Endian.  Reads input in pairs of bytes, least
significant byte first.  Can only represent 16-bit characters.
@end itemize 

Note that not all encodings can represent all characters. This implies
that writing text to a stream may cause errors because the stream
cannot represent these characters. The behaviour of a stream on these
errors can be controlled using @code{open/4} or @code{set_stream/2} (not
implemented). Initially the terminal stream write the characters using
Prolog escape sequences while other streams generate an I/O exception.


@node BOM, Stream Encoding, , Encoding
@subsection BOM: Byte Order Mark

@cindex BOM
@cindex Byte Order Mark
From @ref{Stream Encoding}, you may have got the impression text-files are
complicated. This section deals with a related topic, making live often
easier for the user, but providing another worry to the programmer.
@strong{BOM} or @emph{Byte Order Marker} is a technique for
identifying Unicode text-files as well as the encoding they use. Such
files start with the Unicode character @code{0xFEFF}, a non-breaking,
zero-width space character. This is a pretty unique sequence that is not
likely to be the start of a non-Unicode file and uniquely distinguishes
the various Unicode file formats. As it is a zero-width blank, it even
doesn't produce any output. This solves all problems, or ...

Some formats start of as US-ASCII and may contain some encoding mark to
switch to UTF-8, such as the @code{encoding="UTF-8"} in an XML header.
Such formats often explicitly forbid the the use of a UTF-8 BOM. In
other cases there is additional information telling the encoding making
the use of a BOM redundant or even illegal.

The BOM is handled by the @code{open/4} predicate. By default, text-files are
probed for the BOM when opened for reading. If a BOM is found, the
encoding is set accordingly and the property @code{bom(true)} is
available through @code{stream_property/2}. When opening a file for
writing, writing a BOM can be requested using the option
@code{bom(true)} with @code{open/4}.

