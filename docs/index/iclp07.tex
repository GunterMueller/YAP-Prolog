%==============================================================================
\documentclass{llncs} 
%------------------------------------------------------------------------------
%\usepackage{a4wide}
\usepackage{float}
\usepackage{xspace}
\usepackage{epsfig}
\usepackage{wrapfig}
\usepackage{subfigure}

\renewcommand{\rmdefault}{ptm}
%------------------------------------------------------------------------------
\floatstyle{ruled}
\newfloat{Algorithm}{ht}{lop}
%------------------------------------------------------------------------------
\newcommand{\wamcodesize}{scriptsize}
\newcommand{\code}[1]{\texttt{#1}}
\newcommand{\instr}[1]{\textsf{#1}}
\newcommand{\try}{\instr{try}\xspace}
\newcommand{\retry}{\mbox{\instr{retry}}\xspace}
\newcommand{\trust}{\instr{trust}\xspace}
\newcommand{\TryRetryTrust}{\mbox{\instr{try-retry-trust}}\xspace}
\newcommand{\fail}{\instr{fail}\xspace}
\newcommand{\jump}{\instr{jump}\xspace}
\newcommand{\jitiSTAR}{\mbox{\instr{dindex\_on\_*}}\xspace}
\newcommand{\switchSTAR}{\mbox{\instr{switch\_on\_*}}\xspace}
\newcommand{\jitiONterm}{\mbox{\instr{dindex\_on\_term}}\xspace}
\newcommand{\jitiONconstant}{\mbox{\instr{dindex\_on\_constant}}\xspace}
\newcommand{\jitiONstructure}{\mbox{\instr{dindex\_on\_structure}}\xspace}
\newcommand{\switchONterm}{\mbox{\instr{switch\_on\_term}}\xspace}
\newcommand{\switchONconstant}{\mbox{\instr{switch\_on\_constant}}\xspace}
\newcommand{\switchONstructure}{\mbox{\instr{switch\_on\_structure}}\xspace}
\newcommand{\getcon}{\mbox{\instr{get\_constant}}\xspace}
\newcommand{\proceed}{\instr{proceed}\xspace}
\newcommand{\Cline}{\cline{2-3}}
\newcommand{\JITI}{demand-driven indexing\xspace}
%------------------------------------------------------------------------------
\newenvironment{SmallProg}{\begin{tt}\begin{small}\begin{tabular}[b]{l}}{\end{tabular}\end{small}\end{tt}}
\newenvironment{ScriptProg}{\begin{tt}\begin{scriptsize}\begin{tabular}[b]{l}}{\end{tabular}\end{scriptsize}\end{tt}}
\newenvironment{FootProg}{\begin{tt}\begin{footnotesize}\begin{tabular}[c]{l}}{\end{tabular}\end{footnotesize}\end{tt}}

\newcommand{\TODOcomment}[2]{%
  \stepcounter{TODOcounter#1}%
  {\scriptsize\bf$^{(\arabic{TODOcounter#1})}$}%
  \marginpar[\fbox{
    \parbox{2cm}{\raggedleft
      \scriptsize$^{({\bf{\arabic{TODOcounter#1}{#1}}})}$%
      \scriptsize #2}}]%
  {\fbox{\parbox{2cm}{\raggedright
      \scriptsize$^{({\bf{\arabic{TODOcounter#1}{#1}}})}$%
      \scriptsize #2}}}
}%
\newcounter{TODOcounter}
\newcommand{\TODO}[1]{\TODOcomment{}{#1}}
%------------------------------------------------------------------------------

\title{Demand-Driven Indexing of Prolog Clauses}
\titlerunning{Demand-Driven Indexing of Prolog Clauses}

\author{V\'{\i}tor Santos Costa\inst{1} \and Konstantinos
  Sagonas\inst{2} \and Ricardo Lopes\inst{1}}
\authorrunning{V. Santos Costa, K. Sagonas and R. Lopes}

\institute{
  University of Porto, Portugal
  \and
  National Technical University of Athens, Greece
}

\begin{document}
\maketitle

\begin{abstract}
  As logic programming applications grow in size, Prolog systems need
  to efficiently access larger and larger data sets and the need for
  any- and multi-argument indexing becomes more and more profound.
  Static generation of multi-argument indexing is one alternative, but
  applications often rely on features that are inherently dynamic
  (e.g., generating hypotheses for ILP data sets during runtime) which
  makes static techniques inapplicable or inaccurate. Another
  alternative, which has not been investigated so far, is to employ
  dynamic schemes for flexible demand-driven indexing of Prolog
  clauses. We propose such schemes and discuss issues that need to be
  addressed for their efficient implementation in the context of
  WAM-based Prolog systems. We have implemented demand-driven indexing
  in two different Prolog systems and have been able to obtain
  non-negligible performance speedups: from a few percent up to orders
  of magnitude. Given these results, we see very little reason for
  Prolog systems not to incorporate some form of dynamic indexing
  based on actual demand. In fact, we see demand-driven indexing as
  the first step towards effective runtime optimization of Prolog
  programs.
\end{abstract}


\section{Introduction}
%=====================
The WAM~\cite{Warren83} has mostly been a blessing but occasionally
also a curse for Prolog systems. Its ingenious design has allowed
implementors to get byte code compilers with decent performance --- it
is not a fluke that most Prolog systems are still based on the WAM. On
the other hand, \emph{because} the WAM gives good performance in many
cases, implementors have not incorporated in their systems many
features that drastically depart from WAM's basic characteristics.
%
For example, first argument indexing is sufficient for many Prolog
applications. However, it is clearly sub-optimal for applications
accessing large databases; for a long time now, the database community
has recognized that good indexing is the basis for fast query
processing~\cite{}.

As logic programming applications grow in size, Prolog systems need to
efficiently access larger and larger data sets and the need for any-
and multi-argument indexing becomes more and more profound. Static
generation of multi-argument indexing is one alternative. The problem
is that this alternative is often unattractive because it may
drastically increase the size of the generated byte code and do so
unnecessarily. Static analysis can partly address this concern, but in
applications that rely on features which are inherently dynamic (e.g.,
generating hypotheses for inductive logic programming data sets during
runtime) static analysis is inapplicable or grossly inaccurate.
Another alternative, which has not been investigated so far, is to do
flexible indexing on demand during program execution.

This is precisely what we advocate with this paper. More specifically,
we present a small extension to the WAM that allows for flexible
indexing of Prolog clauses during runtime based on actual demand. For
static predicates, the scheme we propose is partly guided by the
compiler; for dynamic code, besides being demand-driven by queries,
the method needs to cater for code updates during runtime. Where our
schemes radically depart from current practice is that they generate
new byte code during runtime, in effect doing a form of just-in-time
compilation. In our experience these schemes pay off. We have
implemented \JITI in two different Prolog systems (Yap and XXX) and
have obtained non-trivial speedups, ranging from a few percent to
orders of magnitude, across a wide range of applications. Given these
results, we see very little reason for Prolog systems not to
incorporate some form of indexing based on actual demand from queries.
In fact, we see \JITI as only the first step towards effective runtime
optimization of Prolog programs.

This paper is structured as follows. After commenting on the state of
the art and related work concerning indexing in Prolog systems
(Sect.~\ref{sec:related}) we briefly review indexing in the WAM
(Sect.~\ref{sec:prelims}). We then present \JITI schemes for static
(Sect.~\ref{sec:static}) and dynamic (Sect.~\ref{sec:dynamic})
predicates, their implementation in two Prolog systems
(Sect.~\ref{sec:impl}) and the performance benefits they bring
(Sect.~\ref{sec:perf}). The paper ends with some concluding remarks.


\section{State of the Art and Related Work} \label{sec:related}
%==============================================================
% Indexing in Prolog systems:
To the best of our knowledge, many Prolog systems still only support
indexing on the main functor symbol of the first argument. Some
others, like YAP version 4~\cite{YAP}, can look inside some compound
terms. SICStus Prolog supports \emph{shallow
  backtracking}~\cite{ShallowBacktracking@ICLP-89}; choice points are
fully populated only when it is certain that execution will enter the
clause body. While shallow backtracking avoids some of the performance
problems of unnecessary choice point creation, it does not offer the
full benefits that indexing can provide. Other systems like
BIM-Prolog~\cite{IndexingProlog@NACLP-89}, SWI-Prolog~\cite{SWI} and
XSB~\cite{XSB} allow for user-controlled multi-argument indexing (via
an \code{:-~index} directive). Notably, ilProlog~\cite{ilProlog} uses
compile-time heuristics and generates code for multi-argument indexing
automatically. In all these systems, this support comes with various
implementation restrictions. For example, in SWI-Prolog at most four
arguments can be indexed; in XSB the compiler does not offer
multi-argument indexing and the predicates need to be asserted
instead; we know of no system where multi-argument indexing looks
inside compound terms. More importantly, requiring users to specify
arguments to index on is neither user-friendly nor guarantees good
performance results.

% Trees, tries and unification factoring:
Recognizing the need for better indexing, researchers have proposed
more flexible index mechanisms for Prolog. For example, Hickey and
Mudambi proposed \emph{switching trees}~\cite{HickeyMudambi@JLP-89},
which rely on the presence of mode information. Similar proposals were
put forward by Van Roy, Demoen and Willems who investigated indexing
on several arguments in the form of a \emph{selection tree}~\cite{VRDW87}
and by Zhou et al.\ who implemented a \emph{matching tree} oriented
abstract machine for Prolog~\cite{TOAM@ICLP-90}. For static
predicates, the XSB compiler offers support for \emph{unification
factoring}~\cite{UnifFact@POPL-95}; for asserted code, XSB can
represent databases of facts using \emph{tries}~\cite{Tries@JLP-99}
which provide left-to-right multi-argument indexing. However, in XSB
none of these mechanisms is used automatically; instead the user has
to specify appropriate directives.

% Comparison with static analysis techniques and Mercury:
Long ago, Kliger and Shapiro argued that such tree-based indexing
schemes are not cost effective for the compilation of Prolog
programs~\cite{KligerShapiro@ICLP-88}. Some of their arguments make
sense for certain applications, but, as we shall show, in general 
they underestimate the benefits of indexing on EDB predicates.
Nevertheless, it is true that unless the modes of
predicates are known we run the risk of doing indexing on output
arguments, whose only effect is an unnecessary increase in compilation
times and, more importantly, in code size. In a programming language
like Mercury~\cite{Mercury@JLP-96} where modes are known the compiler
can of course avoid this risk; indeed in Mercury modes (and types) are
used to guide the compiler generate good indexing tables. However, the
situation is different for a language like Prolog. Getting accurate
information about the set of all possible modes of predicates requires
a global static analyzer in the compiler --- and most Prolog systems
do not come with one. More importantly, it requires a lot of
discipline from the programmer (e.g., that applications use the module
system religiously and never bypass it). As a result, most Prolog
systems currently do not provide the type of indexing that
applications require. Even in systems like Ciao~\cite{Ciao@SCP-05},
which do come with built-in static analysis and more or less force
such a discipline on the programmer, mode information is not used for
multi-argument indexing!

% The grand finale:
The situation is actually worse for certain types of Prolog
applications. For example, consider applications in the area of
inductive logic programming. These applications on the one hand have
big demands for effective indexing since they need to efficiently
access big datasets and on the other they are very unfit for static
analysis since queries are often ad hoc and generated only during
runtime as new hypotheses are formed or refined.
%
Our thesis is that the Prolog abstract machine should be able to adapt
automatically to the runtime requirements of such or, even better, of
all applications by employing increasingly aggressive forms of dynamic
compilation. As a concrete example of what this means in practice, in
this paper we will attack the problem of providing effective indexing
during runtime. Naturally, we will base our technique on the existing
support for indexing that the WAM provides, but we will extend this
support with the technique of \JITI that we describe in the next
sections.


\section{Indexing in the WAM} \label{sec:prelims}
%================================================
To make the paper relatively self-contained we briefly review the
indexing instructions of the WAM and their use. In the WAM, the first
level of dispatching involves a test on the type of the argument. The
\switchONterm instruction checks the tag of the dereferenced value in
the first argument register and implements a four-way branch where one
branch is for the dereferenced register being an unbound variable, one
for being atomic, one for (non-empty) list, and one for structure. In
any case, control goes to a (possibly empty) bucket of clauses. In the
buckets for constants and structures the second level of dispatching
involves the value of the register. The \switchONconstant and
\switchONstructure instructions implement this dispatching: typically
with a \fail instruction when the bucket is empty, with a \jump
instruction for only one clause, with a sequential scan when the
number of clauses is small, and with a hash lookup when the number of
clauses exceeds a threshold. For this reason the \switchONconstant and
\switchONstructure instructions take as arguments the hash table
\instr{T} and the number of clauses \instr{N} the table contains (or
equivalently, \instr{N} is the size of the hash table). In each bucket
of this hash table and also in the bucket for the variable case of
\switchONterm the code performs a sequential backtracking search of
the clauses using a \TryRetryTrust chain of instructions. The \try
instruction sets up a choice point, the \retry instructions (if any)
update certain fields of this choice point, and the \trust instruction
removes it.

The WAM has additional indexing instructions (\instr{try\_me\_else}
and friends) that allow indexing to be interspersed with the code of
clauses. For simplicity of presentation we will not consider them
here. This is not a problem since the above scheme handles all cases.
Also, we will feel free to do some minor modifications and
optimizations when this simplifies things.

We present an example. Consider the Prolog code shown in
Fig.~\ref{fig:carc:facts}. It is a fragment of the well-known machine
learning dataset \textit{Carcinogenesis}~\cite{Carcinogenesis@ILP-97}.
The five clauses get compiled to the WAM code shown in
Fig.~\ref{fig:carc:clauses}. The first argument indexing indexing code
that a Prolog compiler generates is shown in
Fig.~\ref{fig:carc:index}. This code is typically placed before the
code for the clauses and the \switchONconstant instruction is the
entry point of predicate. Note that compared with vanilla WAM this
instruction has an extra argument: the register on the value of which
we will index ($r_1$). The extra argument will allow us to go beyond
first argument indexing. Another departure from the WAM is that if
this argument register contains an unbound variable instead of a
constant then execution will continue with the next instruction; in
effect we have merged part of the functionality of \switchONterm into
the \switchONconstant instruction. This small change in the behavior
of \switchONconstant will allow us to get \JITI. Let's see how.

%------------------------------------------------------------------------------
\begin{figure}[t]
\centering
\subfigure[Some Prolog clauses\label{fig:carc:facts}]{%
  \begin{ScriptProg}
    has\_property(d1,salmonella,p).\\
    has\_property(d1,salmonella\_n,p).\\
    has\_property(d2,salmonella,p). \\
    has\_property(d2,cytogen\_ca,n).\\
    has\_property(d3,cytogen\_ca,p).
  \end{ScriptProg}
}%
\subfigure[WAM indexing\label{fig:carc:index}]{%
  \begin{sf}
    \begin{\wamcodesize}
      \begin{tabular}[b]{l}
        \switchONconstant $r_1$ 5 $T_1$  \\
        \try   $L_1$ \\
        \retry $L_2$ \\
        \retry $L_3$ \\
        \retry $L_4$ \\
        \trust $L_5$ \\
	\\
	\begin{tabular}[b]{r|c@{\ }|l|}
	  \Cline
	  $T_1$: & \multicolumn{2}{c|}{Hash Table Info}\\ \Cline\Cline
	  \      & d1 & \try   $L_1$ \\
	  \      &    & \trust $L_2$ \\ \Cline
          \      & d2 & \try   $L_3$ \\
	  \      &    & \trust $L_4$ \\ \Cline
	  \      & d3 & \jump  $L_5$ \\
	  \Cline
	\end{tabular}
      \end{tabular}
    \end{\wamcodesize}
  \end{sf}
}%
\subfigure[Code for the clauses\label{fig:carc:clauses}]{%
  \begin{sf}
    \begin{\wamcodesize}
      \begin{tabular}[b]{rl}
	$L_1$: & \getcon $r_1$ d1            \\
	\      & \getcon $r_2$ salmonella    \\
	\      & \getcon $r_3$ p             \\
        \      & \proceed                    \\
	$L_2$: & \getcon $r_1$ d1            \\
        \      & \getcon $r_2$ salmonella\_n \\
        \      & \getcon $r_3$ p             \\
        \      & \proceed                    \\
	$L_3$: & \getcon $r_1$ d2            \\
        \      & \getcon $r_2$ salmonella    \\
        \      & \getcon $r_3$ p             \\
        \      & \proceed                    \\
	$L_4$: & \getcon $r_1$ d2            \\
	\      & \getcon $r_2$ cytogen\_ca   \\
	\      & \getcon $r_3$ n             \\
	\      & \proceed                    \\
	$L_5$: & \getcon $r_1$ d3            \\
	\      & \getcon $r_2$ cytogen\_ca   \\
	\      & \getcon $r_3$ p             \\
	\      & \proceed
      \end{tabular}
    \end{\wamcodesize}
  \end{sf}
}%
\subfigure[Any arg indexing\label{fig:carc:jiti_single:before}]{%
  \begin{sf}
    \begin{\wamcodesize}
      \begin{tabular}[b]{l}
        \switchONconstant $r_1$ 5 $T_1$  \\
        \jitiONconstant $r_2$   5 3    \\
        \jitiONconstant $r_3$   5 3    \\
        \try   $L_1$ \\
        \retry $L_2$ \\
        \retry $L_3$ \\
        \retry $L_4$ \\
        \trust $L_5$ \\
	\\
	\begin{tabular}[b]{r|c@{\ }|l|}
	  \Cline
	  $T_1$: & \multicolumn{2}{c|}{Hash Table Info}\\ \Cline\Cline
	  \      & \code{d1} & \try   $L_1$ \\
	  \      &           & \trust $L_2$ \\ \Cline
          \      & \code{d2} & \try   $L_3$ \\
	  \      &           & \trust $L_4$ \\ \Cline
	  \      & \code{d3} & \jump  $L_5$ \\
	  \Cline
	\end{tabular}
      \end{tabular}
    \end{\wamcodesize}
  \end{sf}
}%
\caption{Part of the Carcinogenesis dataset and WAM code that a byte
  code compiler generates}
\label{fig:carc}
\end{figure}
%------------------------------------------------------------------------------


\section{Demand-Driven Indexing of Static Predicates} \label{sec:static}
%=======================================================================
For static predicates the compiler has complete information about all
clauses and shapes of their head arguments. It is both desirable and
possible to take advantage of this information at compile time and so
we treat the case of static predicates separately.
%
We will do so with schemes of increasing effectiveness and
implementation complexity.

\subsection{A simple WAM extension for any argument indexing}
%------------------------------------------------------------
Let us initially consider the case where the predicates to index
consist only of Datalog facts. This is commonly the case for all
extensional database predicates where indexing is most effective and
called for.

Refer to the example in Fig.~\ref{fig:carc}.
%
The indexing code of Fig.~\ref{fig:carc:index} incurs a small cost for
a call where the first argument is a variable (namely, executing the
\switchONconstant instruction) but the instruction pays off for calls
where the first argument is bound. On the other hand, for calls where
the first argument is a free variable and some other argument is
bound, a choice point will be created, the \TryRetryTrust chain will
be used, and execution will go through the code of all clauses. This
is clearly inefficient, more so for larger data sets.
%
We can do much better with the relatively simple scheme shown in
Fig.~\ref{fig:carc:jiti_single:before}. Immediately after the
\switchONconstant instruction, we can statically generate
\jitiONconstant (demand indexing) instructions, one for each remaining
argument. Recall that the entry point of the predicate is the
\switchONconstant instruction. The \jitiONconstant $r_i$ \instr{N A}
instruction works as follows:
\begin{itemize}
\item if the argument register $r_i$ is a free variable, then
  execution continues with the next instruction;
\item otherwise, \JITI kicks in as follows. The abstract machine will
  scan the WAM code of the clauses and create an index table for the
  values of the corresponding argument. It can do so because the
  instruction takes as arguments the number of clauses \instr{N} to
  index and the arity \instr{A} of the predicate. (In our example, the
  numbers 5 and 3.) For Datalog facts, this information is sufficient.
  Also, because the WAM byte code for the clauses has a very regular
  structure, the index table can be created very quickly. Upon its
  creation, the \jitiONconstant instruction will get transformed to a
  \switchONconstant. Again this is straightforward because of the two
  instructions have similar layouts in memory. Execution of the
  abstract machine will continue with the \switchONconstant
  instruction.
\end{itemize}
Figure~\ref{fig:carg:jiti_single:after} shows the index table $T_2$
which is created for our example and how the indexing code looks after
the execution of a call with mode \code{(out,in,?)}. Note that the
\jitiONconstant instruction for argument register $r_2$ has been
appropriately patched. The call that triggered \JITI and subsequent
calls of the same mode will use table $T_2$. The index for the second
argument has been created.
%------------------------------------------------------------------------------
\begin{figure}
  \centering
  \begin{sf}
    \begin{\wamcodesize}
      \begin{tabular}{c@{\hspace*{2em}}c@{\hspace*{2em}}c}
	\begin{tabular}{l}
          \switchONconstant $r_1$ 5 $T_1$ \\
          \switchONconstant $r_2$ 5 $T_2$ \\
          \jitiONconstant $r_3$   5 3     \\
          \try $L_1$   \\
          \retry $L_2$ \\
          \retry $L_3$ \\
          \retry $L_4$ \\
          \trust $L_5$ \\
	\end{tabular}
	&
	\begin{tabular}{r|c@{\ }|l|}
	  \Cline
	  $T_1$: & \multicolumn{2}{c|}{Hash Table Info}\\ \Cline\Cline
	  \      & \code{d1} & \try   $L_1$ \\
	  \      &           & \trust $L_2$ \\ \Cline
          \      & \code{d2} & \try   $L_3$ \\
	  \      &           & \trust $L_4$ \\ \Cline
	  \      & \code{d3} & \jump  $L_5$ \\
	  \Cline
	\end{tabular}
	&
	\begin{tabular}{r|c@{\ }|l|}
	  \Cline
	  $T_2$: & \multicolumn{2}{|c|}{Hash Table Info}\\ \Cline\Cline
	  \      & \code{salmonella}    & \try $L_1$   \\
	  \      &                      & \trust $L_3$ \\ \Cline
	  \      & \code{salmonella\_n} & \jump $L_2$  \\ \Cline
	  \      & \code{cytrogen\_ca}  & \try $L_4$   \\
	  \      &                      & \trust $L_5$ \\
	  \Cline
	\end{tabular}
      \end{tabular}
    \end{\wamcodesize}
  \end{sf}
  \caption{WAM code after demand-driven indexing for argument 2;
    table $T_2$ is generated dynamically}
  \label{fig:carg:jiti_single:after}
\end{figure}
%------------------------------------------------------------------------------

The main advantage of this scheme is its simplicity. The compiled code
(Fig.~\ref{fig:carc:jiti_single:before}) is not significantly bigger
than the code which a WAM-based compiler would generate
(Fig.~\ref{fig:carc:index}) and, even if \JITI turns out unnecessary
during runtime (e.g. execution encounters only open calls or with only
the first argument bound), the extra overhead is minimal: the
execution of some \jitiONconstant instructions for the open call only.
%
In short, this is a simple scheme that allows for \JITI on \emph{any
single} argument. At least for big sets of Datalog facts, we see
little reason not to use this indexing scheme.

\paragraph*{Optimizations.}
Because we are dealing with static code, there are opportunities for
some easy optimizations. Suppose we statically determine that there
will never be any calls with \code{in} mode for some arguments or that
these arguments are not discriminating enough.\footnote{In our example,
suppose the third argument of \code{has\_property/3} had the atom
\code{p} as value throughout.} Then we can avoid generating
\jitiONconstant instructions for them. Also, suppose we detect or
heuristically decide that some arguments are most likely than others
to be used in the \code{in} mode. Then we can simply place the
\jitiONconstant instructions for these arguments \emph{before} the
instructions for other arguments. This is possible since all indexing
instructions take the argument register number as an argument.

\subsection{From any argument indexing to multi-argument indexing}
%-----------------------------------------------------------------
The scheme of the previous section gives us only single argument
indexing. However, all the infrastructure we need is already in place.
We can use it to obtain (fixed-order) multi-argument \JITI in a
straightforward way.

Note that the compiler knows exactly the set of clauses that need to
be tried for each query with a specific symbol in the first argument.
This information is needed in order to construct, at compile time, the
hash table $T_1$ of Fig.~\ref{fig:carc:index}. For multi-argument
\JITI, instead of generating for each hash bucket only \TryRetryTrust
instructions, the compiler can prepend appropriate demand indexing
instructions. We illustrate this on our running example. The table
$T_1$ contains four \jitiONconstant instructions: two for each of the
remaining two arguments of hash buckets with more than one
alternative. For hash buckets with none or only one alternative (e.g.,
for \code{d3}'s bucket) there is obviously no need to resort to \JITI
for the remaining arguments. Figure~\ref{fig:carc:jiti_multi} shows
the state of the hash tables after the execution of queries
\code{has\_property(C,salmonella,T)}, which creates table $T_2$, and
\code{has\_property(d2,P,n)} which creates the $T_3$ table and
transforms the \jitiONconstant instruction for \code{d2} and register
$r_3$ to the appropriate \switchONconstant instruction.

%------------------------------------------------------------------------------
\begin{figure}[t]
  \centering
  \begin{sf}
    \begin{\wamcodesize}
      \begin{tabular}{@{}cccc@{}}
	\begin{tabular}{l}
          \switchONconstant $r_1$ 5 $T_1$ \\
          \switchONconstant $r_2$ 5 $T_2$ \\
          \jitiONconstant $r_3$   5 3     \\
          \try $L_1$   \\
          \retry $L_2$ \\
          \retry $L_3$ \\
          \retry $L_4$ \\
          \trust $L_5$ \\
	\end{tabular}
	&
	\begin{tabular}{r|c@{\ }|l|}
	  \Cline
	  $T_1$: & \multicolumn{2}{c|}{Hash Table Info}\\ \Cline\Cline
	  \      & \code{d1} & \jitiONconstant $r_2$ 2 3 \\
	  \      &           & \jitiONconstant $r_3$ 2 3 \\
	  \      &           & \try   $L_1$ \\
	  \      &           & \trust $L_2$ \\ \Cline
          \      & \code{d2} & \jitiONconstant $r_2$ 2 3 \\
	  \      &           & \switchONconstant $r_3$ 2 $T_3$ \\
	  \      &           & \try   $L_3$ \\
	  \      &           & \trust $L_4$ \\ \Cline
	  \      & \code{d3} & \jump  $L_5$ \\
	  \Cline
	\end{tabular}
	&
	\begin{tabular}{r|c@{\ }|l|}
	  \Cline
	  $T_2$: & \multicolumn{2}{|c|}{Hash Table Info}\\ \Cline\Cline
	  \      & \code{salmonella}    & \jitiONconstant $r_3$ 2 3 \\
	  \      &                      & \try $L_1$   \\
	  \      &                      & \trust $L_3$ \\ \Cline
	  \      & \code{salmonella\_n} & \jump $L_2$  \\ \Cline
	  \      & \code{cytrogen\_ca}  & \jitiONconstant $r_3$ 2 3 \\
	  \      &                      & \try $L_4$   \\
	  \      &                      & \trust $L_5$ \\
	  \Cline
	\end{tabular}
	&
	\begin{tabular}{r|c@{\ }|l|}
	  \Cline
	  $T_3$: & \multicolumn{2}{|c|}{Hash Table Info}\\ \Cline\Cline
	  \      & \code{p} & \jump $L_3$ \\ \Cline
	  \      & \code{n} & \jump $L_4$ \\
	  \Cline
	\end{tabular}
      \end{tabular}
    \end{\wamcodesize}
  \end{sf}
  \caption{\JITI for all argument combinations;
    table $T_1$ is static; $T_2$ and $T_3$ are generated dynamically}
  \label{fig:carc:jiti_multi}
\end{figure}
%------------------------------------------------------------------------------

\paragraph{Implementation issues.}
In the \jitiONconstant instructions of Fig.~\ref{fig:carc:jiti_multi}
notice the integer 2 which denotes the number of clauses that the
instruction will index. Using this number an index table of
appropriate size will be created, such as $T_3$. To fill this table we
need information about the clauses to index and the symbols to hash
on. The clauses can be obtained by scanning the labels of the
\TryRetryTrust instructions following \jitiONconstant; the symbols by
looking at appropriate byte code offsets (based on the argument
register number) from these labels. In our running example, the
symbols can be obtained by looking at the second argument of the
\getcon instruction whose argument register is $r_2$. In the loaded
bytecode, assuming the argument register is represented in one byte,
these symbols are found $sizeof(\getcon) + sizeof(opcode) + 1$ bytes
away from the clause label; see Fig.~\ref{fig:carc:clauses}. Thus,
multi-argument \JITI is easy to get and the creation of index tables
can be extremely fast when indexing Datalog facts.

\subsection{Beyond Datalog and other implementation issues}
%----------------------------------------------------------
Indexing on demand clauses with function symbols is not significantly
more difficult. The scheme we have described is applicable but
requires the following extensions:
\begin{enumerate}
\item Besides \jitiONconstant we also need \jitiONterm and
  \jitiONstructure instructions. These are the \JITI counterparts of
  the WAM's \switchONterm and \switchONstructure.
\item Because the byte code for the clause heads does not necessarily
  have a regular structure, the abstract machine needs to be able to
  ``walk'' the byte code instructions and recover the symbols on which
  indexing will be based. Writing such a code walking procedure is not
  hard.\footnote{In many Prolog systems, a procedure with similar
  functionality often exists for the disassembler, the debugger, etc.}
\item Indexing on an argument that contains unconstrained variables
  for some clauses is tricky. The WAM needs to group clauses in this
  case and without special treatment creates two choice points for
  this argument (one for the variables and one per each group of
  clauses). However, this issue and how to deal with it is well-known
  by now. Possible solutions to it are described in a 1987 paper by
  Carlsson~\cite{FreezeIndexing@ICLP-87} and can be readily adapted to
  \JITI. Alternatively, in a simple implementation, we can skip \JITI
  for arguments with variables in some clauses.
\end{enumerate}
Before describing \JITI more formally, we remark on the following
design decisions whose rationale may not be immediately obvious:
\begin{itemize}
\item By default, only table $T_1$ is generated at compile time (as in
  the WAM) and the additional index tables $T_2, T_3, \ldots$ are
  generated dynamically. This is because we do not want to increase
  compiled code size unnecessarily (i.e., when there is no demand for
  these indices).
\item On the other hand, we generate \jitiSTAR instructions at compile
  time for the head arguments.\footnote{The \jitiSTAR instructions for
  the $T_1$ table can be generated either by the compiler or by the
  loader.} This does not noticeably increase the generated byte code
  but it greatly simplifies code loading. Notice that a nice property
  of the scheme we have described is that the loaded byte code can be
  patched \emph{without} the need to move any instructions.
% The indexing tables are typically not intersperced with the byte code.
\item Finally, one may wonder why the \jitiSTAR instructions create
  the dynamic index tables with an additional code walking pass
  instead of piggy-backing on the pass which examines all clauses via
  the main \TryRetryTrust chain. Main reasons are: 1) in many cases
  the code walking can be selective and guided by offsets and 2) by
  first creating the index table and then using it we speed up the
  execution of the queries encountered during runtime and often avoid
  unnecessary choice point creations.
\end{itemize}
This is \JITI as we have implemented it.
% in one of our Prolog systems.
However, we note that these decisions are orthogonal to the main idea
and are under compiler control. If, for example, analysis determines
that some argument sequences will never demand indexing we can simply
avoid generation of \jitiSTAR instructions for these. Similarly, if we
determine that some argument sequences will definitely demand indexing
we can speed up execution by generating the appropriate index tables
at compile time instead of at runtime.

\subsection{Demand-driven index construction and its properties}
%---------------------------------------------------------------
The idea behind \JITI can be captured in a single sentence: \emph{we
can generate every index we need during program execution when this
index is demanded}. Subsequent uses of these indices can speed up
execution considerably more than the time it takes to construct them
(more on this below) so this runtime action makes sense.\footnote{In
fact, because choice points are expensive in the WAM, \JITI can speed
up even the execution of the query that triggers the process, not only
subsequent queries.}
%
We describe the process of demand-driven index construction.

% \subsubsection{Demand-driven index construction}
%-------------------------------------------------
Let $p/k$ be a predicate with $n$ clauses.
%
At a high level, its indices form a tree whose root is the entry point
of the predicate. For simplicity, we assume that the root node of the
tree and the interior nodes corresponding to the index table for the
first argument have been constructed at compile time. Leaves of this
tree are the nodes containing the code for the clauses of the
predicate and each clause is identified by a unique label \mbox{$L_i,
1 \leq i \leq n$}. Execution always starts at the first instruction of
the root node and follows Algorithm~\ref{alg:construction}. The
algorithm might look complicated but is actually quite simple.
%
Each non-leaf node contains a sequence of byte code instructions with
groups of the form \mbox{$\langle I_1, \ldots, I_m, T_1, \ldots, T_l
\rangle, 0 \leq m \leq k, 1 \leq l \leq n$} where each of the $I$
instructions, if any, is either a \switchSTAR or a \jitiSTAR
instruction and the $T$ instructions are either a sequence of
\TryRetryTrust instructions (if $l > 1$) or a \jump instruction (if
\mbox{$l = 1$}). Step~2.2 dynamically constructs an index table $\cal
T$ whose buckets are the newly created interior nodes in the tree.
Each bucket associated with a single clause contains a \jump
instruction to the label of that clause. Each bucket associated with
many clauses starts with the $I$ instructions which are yet to be
visited and continues with a \TryRetryTrust chain pointing to the
clauses. When the index construction is done, the instruction mutates
to a \switchSTAR WAM instruction.
%-------------------------------------------------------------------------
\begin{Algorithm}[t]
  \caption{Actions of the abstract machine with \JITI}
  \label{alg:construction}
  \begin{enumerate}
  \item if the current instruction $I$ is a \switchSTAR, \try, \retry,
    \trust or \jump, the action is an in the WAM;
  \item if the current instruction $I$ is a \jitiSTAR with arguments $r,
    l$, and $k$ where $r$ is a register then
    \begin{enumerate}
    \item[2.1] if register $r$ contains a variable, the action is simply to
      \instr{goto} the next instruction in the node;
    \item[2.2] if register $r$ contains a value $v$, the action is to
      dynamically construct the index as follows:
      \begin{itemize}
      \item[2.2.1] collect the subsequent instructions in a list $\cal I$
	until the next instruction is a \try;\footnote{Note that there
	will always be a \try following a \jitiSTAR instruction.}
      \item[2.2.2] for each label $L$ in the \TryRetryTrust chain
	inspect the code of the clause with label $L$ to find the
	symbol~$c$ associated with register $r$ in the clause; (This
	step creates a list of $\langle c, L \rangle$ pairs.)
      \item[2.2.3] create an index table $\cal T$ out of these pairs as
	follows:
	\begin{itemize}
	\item if $I$ is a \jitiONconstant or a \jitiONstructure then
	  create an index table for the symbols in the list of pairs;
	  each entry of the table is identified by a symbol $c$ and
	  contains:
	  \begin{itemize}
	  \item the instruction \jump $L_c$ if $L_c$ is the only label
	    associated with $c$;
	  \item the sequence of instructions obtained by appending to
	    $\cal I$ a \TryRetryTrust chain for the sequence of labels
	    $L'_1, \ldots, L'_l$ that are associated with $c$
	  \end{itemize}
	\item if $I$ is a \jitiONterm then
	  \begin{itemize}
	  \item partition the sequence of labels $\cal L$ in the list
	    of pairs into sequences of labels ${\cal L}_c, {\cal L}_l$
	    and ${\cal L}_s$ for constants, lists and structures,
	    respectively;
	  \item for each of the four sequences ${\cal L}, {\cal L}_c,
	    {\cal L}_l, {\cal L}_s$ of labels create code as follows:
	    \begin{itemize}
	    \item the instruction \fail if the sequence is empty;
	    \item the instruction \jump $L$ if $L$ is the only label in
	      the sequence;
	    \item the sequence of instructions obtained by appending to
	      $\cal I$ a \TryRetryTrust chain for the current sequence
	      of labels;
	    \end{itemize}
	  \end{itemize}
	\end{itemize}
      \item[2.2.4] transform the \jitiSTAR $r, l, k$ instruction to
	a \switchSTAR $r, l, \&{\cal T}$ instruction; and
      \item[2.2.5] continue execution with this instruction.
      \end{itemize}
    \end{enumerate}
  \end{enumerate}
\end{Algorithm}
%-------------------------------------------------------------------------

\paragraph*{Complexity properties.}
Complexity-wise, dynamic index construction does not add any overhead
to program execution. First, note that each demanded index table will
be constructed at most once. Also, a \jitiSTAR instruction will be
encountered only in cases where execution would examine all clauses in
the \TryRetryTrust chain.\footnote{This statement is possibly not
valid the presence of Prolog cuts.} The construction visits these
clauses \emph{once} and then creates the index table in time linear in
the number of clauses. One pass over the list of $\langle c, L
\rangle$ pairs suffices. After index construction, execution will
visit only a subset of these clauses as the index table will be
consulted.
%% Finally, note that the maximum number of \jitiSTAR instructions
%% that will be visited for each query is bounded by the maximum
%% number of index positions (symbols) in the clause heads of the
%% predicate.
Thus, in cases where \JITI is not effective, execution of a query will
at most double due to dynamic index construction. In fact, this worst
case is extremely unlikely in practice. On the other hand, \JITI can
change the complexity of evaluating a predicate call from $O(n)$ to
$O(1)$ where $n$ is the number of clauses.

\subsection{More implementation choices}
%---------------------------------------
The observant reader has no doubt noticed that
Algorithm~\ref{alg:construction} provides multi-argument indexing but
only for the main functor symbol of arguments. For clauses with
compound terms that require indexing in their sub-terms we can either
employ a program transformation like \emph{unification
factoring}~\cite{UnifFact@POPL-95} at compile time or modify the
algorithm to consider index positions inside compound terms. This is
relatively easy to do but requires support from the register allocator
(passing the sub-terms of compound terms in appropriate argument
registers) and/or a new set of instructions. Due to space limitations
we omit further details.

Algorithm~\ref{alg:construction} relies on a procedure that inspects
the code of a clause and collects the symbols associated with some
particular index position (step~2.2.2). If we are satisfied with
looking only at clause heads, this procedure needs to understand only
the structure of \instr{get} and \instr{unify} instructions. Thus, it
is easy to write. At the cost of increased implementation complexity,
this step can of course take into account other information that may
exist in the body of the clause (e.g., type tests such as
\code{var(X)}, \code{atom(X)}, aliasing constraints such as \code{X =
Y}, numeric constraints such as \code{X > 0}, etc).

A reasonable concern for \JITI is increased memory consumption during
runtime due to the index tables. In our experience, this does not seem
to be a problem in practice since most applications do not have demand
for indexing on many argument combinations. In applications where it
does become a problem or when running in an environment with limited
memory, we can easily put a bound on the size of index tables, either
globally or for each predicate separately. For example, the \jitiSTAR
instructions can either become inactive when this limit is reached, or
better yet we can recover the space of some tables. To do so, we can
employ any standard recycling algorithm (e.g., least recently used)
and reclaim the of index tables that are no longer in use. This is
easy to do by reverting the corresponding \jitiSTAR instructions back
to \switchSTAR instructions. If the indices are needed again, they can
simply be regenerated.


\section{Demand-Driven Indexing of Dynamic Predicates} \label{sec:dynamic}
%=========================================================================
We have so far lived in the comfortable world of static predicates,
where the set of clauses to index is fixed and the compiler can take
advantage of this knowledge. Dynamic code introduces several
complications:
\begin{itemize}
\item We need mechanisms to update multiple indices when new clauses
  are asserted or retracted. In particular, we need the ability to
  expand and possibly shrink multiple code chunks after code updates.
\item We do not know a priori which are the best index positions and
  cannot determine whether indexing on some arguments is avoidable.
\item Supporting the so-called logical update (LU) semantics of the
  ISO Prolog standard becomes harder.
\end{itemize}
We will briefly discuss possible ways of handling these complications.
However, we note that Prolog systems typically provide indexing for
dynamic predicates and thus already deal in some way or another with
these issues. It's just that \JITI makes these problems harder.


\section{Implementation in XXX and in YAP} \label{sec:impl}
%==========================================================
The implementation of \JITI in XXX follows a variant of the scheme
presented in Sect.~\ref{sec:static}. The compiler uses heuristics to
determine the best argument to index on (i.e., this argument is not
necessarily the first) and employs \switchSTAR instructions for this
task. It also statically generates \jitiONconstant instructions for
other argument positions that are good candidates for \JITI.
Currently, an argument is considered a good candidate if it has only
constants or only structure symbols in all clauses. Thus, XXX uses
only \jitiONconstant and \jitiONstructure instructions, never a
\jitiONterm. Also, XXX does not perform \JITI inside structure
symbols.\footnote{Instead, it prompts its user to request unification
factoring for predicates that look likely to benefit from indexing
inside compound terms. The user can then use the appropriate compiler
directive for these predicates.} For dynamic predicates \JITI is
employed only if they consist of Datalog facts; if a clause which is
not a Datalog fact is asserted, all dynamically created index tables
for the predicate are simply dropped and the \jitiONconstant
instruction becomes a \instr{noop}. All these are done automatically,
but the user can disable \JITI in compiled code using an appropriate
compiler option.

YAP implements \JITI since version 5. The current implementation
supports static code, dynamic code, and the internal database. It
differs from the algorithm presented in Sect.~\ref{sec:static} in that
\emph{all indexing code is generated on demand}. Thus, YAP cannot
assume that a \jitiSTAR instruction is followed by a \TryRetryTrust
chain. Instead, by default YAP has to search the whole predicate for
clauses that match the current position in the indexing code. Doing so
for every index expansion was found to be very inefficient for larger
relations: in such cases YAP will maintain a list of matching clauses
at each \jitiSTAR node. Indexing dynamic predicates in YAP follows
very much the same algorithm as static indexing: the key idea is that
most nodes in the index tree must be allocated separately so that they
can grow or contract independently. YAP can index arguments where some
clauses have unconstrained variables, but only for static predicates,
as it would complicate updates.

YAP uses the term JITI (Just-In-Time Indexing) to refer to \JITI. In
the next section we will take the liberty to use this term as a
convenient abbreviation.


\section{Performance Evaluation} \label{sec:perf}
%================================================
Next, we evaluate \JITI on a set of benchmarks and on real life
applications. For the benchmarks of Sect.~\ref{sec:perf:overhead}
and~\ref{sec:perf:speedups}, which involve both systems, we used a 2.4
GHz P4-based laptop with 512~MB of memory running Linux and report
times in milliseconds. For the benchmarks of Sect.~\ref{sec:perf:ILP},
which involve YAP only, we used a 8-node cluster, where each node is a
dual-core AMD 2600+ machine with 2GB of memory
%
%VITOR PLEASE ADD
%
and report times in seconds.

\subsection{JITI Overhead} \label{sec:perf:overhead}
%---------------------------------------------------
   6.2 JITI overhead (show the "bad" cases first)
       present Prolog/tabled benchmarks that do NOT benefit from JITI
       and measure the time overhead -- hopefully this is low

\subsection{JITI Speedups} \label{sec:perf:speedups}
%---------------------------------------------------
% Our experience with the indexing algorithm described here shows a
% significant performance improvement over the previous indexing code in
% our system. Quite often, this has allowed us to tackle applications
% which previously would not have been feasible. We next present some
% results that show how useful the algorithms can be.

       Here I already have "compress", "mutagenesis" and "sg\_cyl"
       The "sg\_cyl" has a really impressive speedup (2 orders of
       magnitude).  We should keep the explanation in your text.
       Then we should add "pta" and "tea" from your PLDI paper.
       If time permits, we should also add some FSA benchmarks
       (e.g. "k963", "dg5" and "tl3" from PLDI)

Next, we present performance results for demand-driven indexing on a
number of benchmarks and real-life applications. Throughout, we
compare performance with single argument indexing. We use YAP-5.1.2
and XXX in our comparisons.

As a base reference, our first dataset is a set of well known small
tabling benchmarks from the XSB Prolog benchmark collection. We chose
these datasets first because they are relatively small and easy to
understand. The benchmarks are: \texttt{cylinder}, computes which
nodes in a cylinder are ..., the well-known \texttt{fibonacci}
function, \texttt{first} that computes the first $k$ terminal symbols
in a grammar, a version of the \texttt{knap-sack} problem, and path
reachability benchmarks in two \texttt{cycle} graphs: a \texttt{chain}
graph, and a \texttt{tree} graph. The \texttt{path} benchmarks use a
right-recursive with base clause first (\texttt{LRB)} definition of
\texttt{path/3}. The YAP results were obtained on an AMD-64 4600+
machine running Ubuntu 6.10.

\begin{table}[ht]
%\vspace{-\intextsep}
%\begin{table}[htbp] 
%\centering
  \centering
  \begin {tabular}{|l|r|r||r|r|} \hline %\cline{1-3}
    &  \multicolumn{2}{|c|}{\bf YAP}  & \multicolumn{2}{||c|}{\bf XXX} \\
    {\bf Benchs.}  & JITI & \bf WAM   & \bf JITI  &  \bf WAM  \\
    \hline
    \texttt{cyl}      & 3    & 48 &  &\\
    \texttt{9queens}      & 67    & 74 &  &\\
    \texttt{cubes}    & 24    & 24  & &\\
    \texttt{fib\_atoms}    & 8    & 8  & &\\
    \texttt{fib\_list}     13      & 12    &  & &  \\
    \texttt{first}    & 5     & 6    &    & \\
    \texttt{ks2} &  49      & 44    & &    \\
    \hline 
    \texttt{cycle}              & 26        & 28      & &      \\
    \texttt{tree}              & 25        & 31      & &      \\
    \hline
\end{tabular}
\caption{Tabling Benchmarks: Time is measured in msecs in all cases.}
\label{tab:aleph}
\end{table}

Notice that these are very fast benchmarks: we ran the results 10
times and present the average. We then used a standard unpaired t-test
to verify whether the results are significantly different. Our results
do not show significant variations between JITI and WAM indexing on
\texttt{fibonacci}, \texttt{first} and \texttt{ks2} benchmarks. Both
\texttt{fibonaccis} are small core recursive programs, most effort is
spent in constructing lists or manipulating atoms. The \texttt{first}
and \texttt{ks2} manipulate small amounts of data that is well indexed
through the first argument.

The JITI brings a significant benefit in the \texttt{cyl} dataset.Most
work in the dataset consists of calling \texttt{cyl/2} facts.
Inspecting the program shows three different call modes for
\texttt{cyl/2}: both arguments are unbound; the first argument is
bound; or the \emph{only the second argument is bound}. The JITI
improves performance in the latter case only, but this does make a
large difference, as the WAM code has to visit all thousand clauses if
the second argument is unbound.

\subsection{JITI in ILP} \label{sec:perf:ILP}
%--------------------------------------------
The need for just-in-time indexing was originally motivated by ILP
applications.  Table~\ref{tab:aleph} shows JITI performance on some
learning tasks using the ALEPH system~\cite{}. The dataset
\texttt{Krki} tries to learn rules from a small database of chess
end-games; \texttt{GeneExpression} learns rules for yeast gene
activity given a database of genes, their interactions, and
micro-array gene expression data; \texttt{BreastCancer} processes
real-life patient reports towards predicting whether an abnormality
may be malignant; \texttt{IE-Protein\_Extraction} processes
information extraction from paper abstracts to search proteins;
\texttt{Susi} learns from shopping patterns; and \texttt{Mesh} learns
rules for finite-methods mesh design. The datasets
\texttt{Carcinogenesis}, \texttt{Choline}, \texttt{Mutagenesis},
\texttt{Pyrimidines}, and \texttt{Thermolysin} are about predicting
chemical properties of compounds. The first three datasets store
properties of interest as tables, but \texttt{Thermolysin} learns from
the 3D-structure of a molecule's conformations.  Several of these
datasets are standard across Machine Learning literature.
\texttt{GeneExpression}~\cite{} and \texttt{BreastCancer}~\cite{} were
partly developed by some of the authors.  Most datasets perform simple
queries in an extensional database. The exception is
\texttt{Mutagenesis} where several predicates are defined
intensionally, requiring extensive computation.


\begin{table}[ht]
%\vspace{-\intextsep}
%\begin{table}[htbp] 
%\centering
  \centering
  \begin {tabular}{|l|r|r|r|r|} \hline %\cline{1-3}                             
    &  \multicolumn{2}{|c|}{\bf Time in sec.}  & \bf \JITI \\
    {\bf Benchs.}  & \bf $A1$   & \bf JITI & \bf Ratio \\
    \hline
    \texttt{BreastCancer}      & 1450    & 88 & 16\\
    \texttt{Carcinogenesis}    & 17,705    & 192  &92\\
    \texttt{Choline}           & 14,766    & 1,397  & 11  \\
    \texttt{GeneExpression}    & 193,283     & 7,483    & 26    \\
    \texttt{IE-Protein\_Extraction} &  1,677,146      & 2,909    & 577    \\
    \texttt{Krki}              & 0.3        & 0.3      & 1      \\
    \texttt{Krki II}           & 1.3     & 1.3     & 1     \\
    \texttt{Mesh}              & 4    & 3  & 1.3  \\
    \texttt{Mutagenesis}       & 51,775  & 27,746 & 1.9\\
    \texttt{Pyrimidines}       & 487,545     & 253,235  & 1.9    \\
    \texttt{Susi}              & 105,091    & 307    & 342  \\
    \texttt{Thermolysin}       & 50,279      &  5,213     & 10      \\
    \hline
\end{tabular}
\caption{Machine Learning (ILP) Datasets: Times are given in Seconds,
  we give time for standard indexing with no indexing on dynamic
  predicates versus the \JITI implementation}
\label{tab:aleph}
\end{table}

We compare times for 10 runs of the saturation/refinement cycle of the
ILP system.  Table~\ref{tab:aleph} shows results. The \texttt{Krki}
datasets have small search spaces and small databases, so they
essentially achieve the same performance under both versions: there is
no slowdown. The \texttt{Mesh}, \texttt{Mutagenesis}, and
\texttt{Pyrimides} applications do not benefit much from indexing in
the database, but they do benefit from indexing in the dynamic
representation of the search space, as their running times halve.

The \texttt{BreastCancer} and \texttt{GeneExpression} applications use
1NF data (that is, unstructured data). The benefit here is mostly from
multiple-argument indexing.  \texttt{BreastCancer} is particularly
interesting. It consists of 40 binary relations with 65k elements
each, where the first argument is the key, like in
\texttt{sg\_cyl}. We know that most calls have the first argument
bound, hence indexing was not expected to matter very much. Instead,
the results show \JITI running time to improve by an order of
magnitude. Like in \texttt{sg\_cyl}, this suggests that even a small
percentage of badly indexed calls can come to dominate running time.

\texttt{IE-Protein\_Extraction} and \texttt{Thermolysin} are example
applications that manipulate structured data.
\texttt{IE-Protein\_Extraction} is the largest dataset we consider,
and indexing is simply critical: it is not possible to run the
application in reasonable time with one argument
indexing. \texttt{Thermolysin} is smaller and performs some
computation per query: even so, indexing improves performance by an
order of magnitude.

\begin{table*}[ht]
  \centering
  \begin {tabular}{|l|r|r||r|r|} \hline %\cline{1-3}
    &  \multicolumn{2}{|c||}{\bf Static Code}  & \multicolumn{2}{|c|}{\bf Dynamic Code} \\
Benchmarks    &  \textbf{Clause} & {\bf Index}  & \textbf{Clause} & {\bf Index} \\
%    \textbf{Benchmarks} &   & Total & T & W & S &  & Total & T & C & W & S  \\
    \hline
    \texttt{BreastCancer}
    & 60940 & 46887 
    % & 46242 & 3126  & 125
    & 630  & 14
    % &42 & 18& 57 &6
    \\

    \texttt{Carcinogenesis} 
    & 1801 & 2678
    % &1225 & 587 & 865
    & 13512 & 942
    %& 291 & 91 & 457 & 102
    \\

    \texttt{Choline}  & 666 & 174
    % &67 & 48 & 58
    & 3172 & 174
    % & 76 & 4 & 48 & 45
 \\
    \texttt{GeneExpression}    &  46726 & 22629
    % &6780 & 6473 & 9375
    & 116463 & 9015
    %& 2703 & 932 & 3910 & 1469
 \\

    \texttt{IE-Protein\_Extraction}    &146033 & 129333
    %&39279 & 24322 & 65732
    & 53423 & 1531
    %& 467 & 108 & 868 & 86
 \\

    \texttt{Krki}              & 678 & 117
    %&52 & 24 & 40
    & 2047 & 24
    %& 10 & 2 & 10 & 1
 \\

    \texttt{Krki II}           & 1866 & 715
    %&180 & 233    & 301
    & 2055 & 26
    %& 11 & 2 & 11 & 1
 \\

    \texttt{Mesh}              & 802 & 161
    %&49 & 18 & 93
    & 2149 & 109
    %& 46 & 4 & 35 & 22
 \\

    \texttt{Mutagenesis}       & 1412 & 1848
    %&1045 & 291 & 510
    & 4302 & 595
    %& 156 & 114 & 264 & 61
 \\

    \texttt{Pyrimidines}       & 774 & 218
    %&76 & 63 & 77
    & 25840 & 12291
    %& 4847 & 43 & 3510 & 3888
 \\

    \texttt{Susi}              & 5007 & 2509
    %&855 & 578 & 1076
    & 4497 & 759
    %& 324 & 58 & 256 & 120
 \\

    \texttt{Thermolysin}       & 2317 & 929
    %&429 & 184 & 315
    & 116129 & 7064
    %& 3295 & 1438 & 2160 & 170
 \\

    \hline
\end{tabular}
\caption{Memory Performance on Machine Learning (ILP) Datasets: memory
  usage is given in KB}
\label{tab:ilpmem}
\end{table*}


Table~\ref{tab:ilpmem} discusses the memory cost paid in using
\JITI. The table presents data obtained at a point near the end of
execution.  Because dynamic memory expands and contracts, we chose a
point where memory usage should be at a maximum. The first two numbers
show data usage on \emph{static} predicates. Static data-base sizes
range from 146MB (\texttt{IE-Protein|_Extraction} to less than a MB
(\texttt{Choline}, \texttt{Krki}, \texttt{Mesh}). Indexing code can be
more than the original code, as in \texttt{Mutagenesis}, or almost as
much, eg, \texttt{IE-Protein\_Extraction}. In most cases the YAP \JITI
adds at least a third and often a half to the original data-base. A
more detailed analysis shows the source of overhead to be very
different from dataset to dataset. In \texttt{IE-Protein|_Extraction}
the problem is that hash tables are very large. Hash tables are also
where most space is spent in \texttt{Susi}. In \texttt{BreastCancer}
hash tables are actually small, so most space is spent in
\TryRetryTrust chains. \texttt{Mutagenesis} is similar: even though
YAP spends a large effort in indexing it still generates long
\TryRetryTrust chains. Storing sets of matching clauses at \jitiSTAR
nodes takes usually over 10\% of total memory usage, but is never dominant.

This version of ALEPH uses the internal data-base to store the IDB.
The size of reflects the search space, and is to some extent
independent of the program's static data, although small applications
such as \texttt{Krki} do tend to have a small search space. ALEPH's
author very carefully designed the system to work around overheads in
accessing the data-base, so indexing should not be as important.  In
fact, indexing has a much lower space overhead in this case,
suggesting it is not so critical. A more detailed analysis shows tha
indexing is working well: most space is spent on hashes tables and on
internal nodes of tree, and relatively little space is spent on
\TryRetryTrust chains.



\section{Concluding Remarks}
%===========================
\begin{itemize}
\item Mention the non-trivial speedups in actual applications; also
  that it is important to realize that certain applications have ad
  hoc query patterns (e.g., ILP) are not amenable to static analyses
\end{itemize}

%==============================================================================
\bibliographystyle{splncs}
\bibliography{lp}
%==============================================================================

\end{document}
